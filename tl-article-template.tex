% !TEX TS-program = XeLaTeX
% all document files must be coded in UTF-8
\documentclass[anonymous=true]{textolivre}

% used to create dummy text for the template file
\definecolor{dark-gray}{gray}{0.35} % color used to display dummy texts
\usepackage{lipsum}
\SetLipsumParListSurrounders{\colorlet{oldcolor}{.}\color{dark-gray}}{\color{oldcolor}}

% used here only to provide the XeLaTeX and BibTeX logos
\usepackage{hologo}

% used in this example to provide source code environment
%\crefname{lstlisting}{lista}{listas}
%\Crefname{lstlisting}{Lista}{Listas}
%\usepackage{listings}
%\renewcommand\lstlistingname{Lista}
%\lstset{language=bash,
%	breaklines=true,
%	basicstyle=\linespread{1}\small\ttfamily,
%	numbers=none,xleftmargin=0.5cm,
%	frame=none,
%	framexleftmargin=0.5em,
%	framexrightmargin=0.5em,
%	showstringspaces=false,
%	upquote=true,
%	commentstyle=\color{gray},
%	literate=%
%           {á}{{\'a}}1 {é}{{\'e}}1 {í}{{\'i}}1 {ó}{{\'o}}1 {ú}{{\'u}}1 
%	   {à}{{\`a}}1 {è}{{\`e}}1 {ì}{{\`i}}1 {ò}{{\`o}}1 {ù}{{\`u}}1
%           {ã}{{\~a}}1 {ẽ}{{\~e}}1 {ĩ}{{\~i}}1 {õ}{{\~o}}1 {ũ}{{\~u}}1
%           {â}{{\^a}}1 {ê}{{\^e}}1 {î}{{\^i}}1 {ô}{{\^o}}1 {û}{{\^u}}1
%	   {ä}{{\"a}}1 {ë}{{\"e}}1 {ï}{{\"i}}1 {ö}{{\"o}}1 {ü}{{\"u}}1
%	   {Á}{{\'A}}1 {É}{{\'E}}1 {Í}{{\'I}}1 {Ó}{{\'O}}1 {Ú}{{\'U}}1
%	   {À}{{\`A}}1 {È}{{\`E}}1 {Ì}{{\`I}}1 {Ò}{{\`O}}1 {Ù}{{\`U}}1
%	   {Ã}{{\~A}}1 {Ẽ}{{\~E}}1 {Ũ}{{\~u}}1 {Õ}{{\~O}}1 {Ũ}{{\~U}}1
%	   {Â}{{\^A}}1 {Ê}{{\^E}}1 {Î}{{\^I}}1 {Ô}{{\^O}}1 {Û}{{\^U}}1
%	   {Ä}{{\"A}}1 {Ë}{{\"E}}1 {Ï}{{\"I}}1 {Ö}{{\"O}}1 {Ü}{{\"U}}1
%           {ç}{{\c{c}}}1 {Ç}{{\c{C}}}1
%}

\journalname{Texto Livre: Linguagem e Tecnologia}
\thevolume{14}
\thenumber{1}
\theyear{2020}
\receiveddate{\DTMdisplaydate{2020}{10}{22}{-1}} % YYYY MM DD
\accepteddate{\DTMdisplaydate{2020}{9}{3}{-1}}
\publisheddate{\today}

\title{Modelo de artigo para submissão à revista Texto Livre}
\othertitle{Article template for submitting to the Texto Livre Journal}
% if there is a third language title, add here:
%\othertitle{Artikelvorlage zur Einreichung beim Texto Livre Journal}

\author[1]{Primeiro Autor}
\author[2]{Segundo Autor}
\author[3]{Terceiro Autor}
\author[4]{Quarto Autor}

\affil[1]{Departamento Um, Instituição Um, Endereço Um. Email: \url{abc@exemple.edu.br}. \orcid{0000-0000-0000-0000}}
\affil[2]{Departamento Dois, Instituição Dois, Endereço Dois. Email: \url{autor2@example.com.br}. \orcid{0000-0000-0000-0000}}
\affil[3]{Departamento Três, Instituição Três, Endereço Três. \orcid{0000-0000-0000-0000}}

% Corresponding author
\corrauthor{Primeiro Autor}

% DOI
\articledoi{10.35699/1983-3652.yyyy.nnnnn}

% Abbreviated author list for the running footer
\runningauthor{Autor et al.}

%\usepackage[backend=biber,style=abnt, ittitles]{biblatex}
%\DeclareLanguageMapping{brazil}{brazil-apa}
\addbibresource{tl-article-template.bib}     
% use biber instead of bibtex
% $ biber tl-article-template
% $ pdflatex tl-article-template.tex

% set language of the article
\setdefaultlanguage[variant=brazilian]{portuguese}
\setotherlanguage{english}
% for langues that use special fonts, you must provide the typeface that will be used
% \setotherlanguage{arabic}
% \newfontfamily\arabicfont[Script=Arabic]{Amiri}
% \newfontfamily\arabicfontsf[Script=Arabic]{Amiri}
% \newfontfamily\arabicfonttt[Script=Arabic]{Amiri}
%
% in the article, to add arabic text use: \textlang{arabic}{ ... }


\begin{document}
% set language of the article
\maketitle

\begin{poliabstract}
\begin{abstract}
Este artigo é um modelo que visa orientar os autores que submeterão seus textos à revista \textit{Texto Livre} a fazê-lo no padrão determinado pela revista a fim de facilitar sua empreitada, visualizando e configurando o texto no formato já configurado nesse modelo. Para fazê-lo, basta escrever ou colar seus textos no lugar aqui designado, substituindo os textos e figuras aqui existentes, verificando se eles mantêm a formatação aqui descrita. O resumo deverá ter entre 150 e 200 palavras.

\keywords{palavra1 \sep palavra dois \sep palavra-três \sep palavra4.}
\end{abstract}

\begin{english}
\begin{abstract}
This article is a template article which aims to guide authors who will submit their papers to the \textit{Texto Livre} journal. This template should facilitate their endeavor, viewing and configuring the text in the format already configured accordingly. Just write or paste your content in the desired places, replacing texts and figures, checking if they keep up the format. The abstract should have between 150 and 200 words.

\keywords{word1 \sep word two \sep word-three \sep word4.}
\end{abstract}
\end{english}

% if there is another abstract, insert it here using the same scheme
\end{poliabstract}


\section{Introdução}\label{sec-intro}
Este artigo é o modelo de documento que visa orientar os autores que submeterão seus textos à revista \textit{Texto Livre}.
Este documento fornece o padrão determinado pela revista, a fim de facilitar sua empreitada, visualizando e configurando 
o texto no formato já configurado no modelo para a submissão. Para redigir seu texto neste modelo, basta escrever ou colar 
seus textos no lugar aqui designado, substituindo os textos e figuras aqui existentes, verificando se eles mantêm a formatação aqui descrita.

A \Cref{sec-organizacao} apresenta as normas para a submissão. Informações sobre a utilização do modelo
estão presentes na \Cref{sec-modelo}. A \Cref{sec-organizacao} apresenta as considerações gerais sobre a organização do artigo, 
apresenta a utilização de listas numeradas e não-numeradas (\Cref{sec-listas}), 
ilustra como inserir figuras e tabelas no documento (\Cref{sec-figuras-tabelas}), assim como citações 
e notas de rodapé (\Cref{sec-quotesandfootnotes}). A \Cref{sec-equacao} apresenta exemplos de equações 
e a \Cref{sec-codigos} apresenta como inserir algoritmos e códigos. 
Na página \pageref{sec-bib} encontram-se as referências e, em seguida, o \Cref{apx-longtable} que
apresenta a inclusão de uma tabela longa, que se entende por várias páginas. 

\lipsum[1-5]

\section{Normas para submissão}\label{sec-normas}
A revista Texto Livre tem como foco publicações originais em área interdisciplinar entre 
Estudos da Linguagem e Tecnologias digitais, com foco na produção textual e na produção de 
documentação para software livre e aceita submissões de textos inéditos (artigos, resenhas, ensaios e traduções). 
As resenhas devem tratar de livros publicados nos últimos 24 meses no Brasil ou no Exterior, 
em primeira edição ou tradução, abrangendo áreas contempladas neste periódico.

O texto deverá vir devidamente revisado pelo autor. A comissão editorial reserva-se o direito de 
fazer nova revisão e de fazer as alterações necessárias. Textos que apresentem problemas de forma, 
estilo e/ou adequação aos padrões da revista serão rejeitados.
Todos os artigos na revista \textit{Texto Livre} são publicados sob 
a \href{https://creativecommons.org/}{Licença Creative Commons CC-BY}. 
A reprodução de textos ou trechos devem seguir atribuições dada pela licença. 
Para qualquer finalidade, solicitamos a comunicação prévia aos editores da revista.

O formato de página deverá ser A4. O limite de palavras é de 5 a 10 mil palavras (do título às referências) 
para artigos e de 800 a 1.000 palavras para resenhas.
Serão aceitos textos em português, inglês, francês ou espanhol. 
Para textos com mais de um autor, todos os autores devem ser indicados na página de registro da submissão.
Os autores devem ter registro no \href{https://orcid.org/}{ORCID}. No que se refere à autoria,
\begin{quote} 
``Não se considera co-autor quem simplesmente auxiliou o autor na produção da obra literária, artística ou científica, 
revendo-a, atualizando-a, bem como fiscalizando ou dirigindo sua edição ou apresentação por qualquer meio'' (\citefield[Art. 15,  § 1º]{lei9610}{title}).
\end{quote}
Em caso de artigos em coautoria, todos os autores devem ser indicados na página de registro da submissão pelo autor 
responsável, na seção de metadados da submissão e em documento suplementar. Além disso, no documento suplementar, 
o responsável pela submissão deve declarar qual foi a contribuição específica de cada autor para a produção do artigo.

O artigo submetido não poderá ter sido publicado anteriormente, nem estar sob avaliação em outro periódico. 
A originalidade e o ineditismo são cruciais para o trâmite de textos no periódico 
em questão (um programa anti-plágio será utilizado pelo editor para verificação da originalidade do artigo). 


\subsection{Código de Conduta e Boas Práticas}\label{sec-conduta}
A revista \textit{Texto Livre} segue as diretrizes do Código de Conduta e Boas Práticas do 
\href{http://publicationethics.org/}{COPE} (Committee on Publication Ethics) 
e as submissões devem atender a essas diretrizes: para conhecimento do Código, 
consulte o texto original em \href{http://publicationethics.org/files/Code_of_conduct_for_journal_editors_1.pdf}{inglês} 
ou sua tradução para o \href{http://www.periodicos.letras.ufmg.br/CCBP-COPE.pdf}{português}.

Todas as submissões que contiverem relatos de pesquisas feitas com seres humanos deverão apresentar análise do COEP (apresentação do número CAAE) ou de outros órgãos de ética em pequisa.


\subsection{Formato de arquivo}\label{sec-formato}
O manuscrito deverá ser enviado em formato \TeX{}, utilizando o modelo da revista (veja a \Cref{sec-modelo}),
ou em formato ODT (\textit{Open Document Format}). Não serão aceitos formatos proprietários, como .doc, .docx ou .rtf
(conforme Comunicado de 13 de setembro de 2011 \url{http://www.periodicos.letras.ufmg.br/index.php/textolivre/about/editorialPolicies#custom-0}). 
As imagens deverão ser enviadas separadamente em arquivos (veja as orientações na \Cref{sec-figuras-tabelas}).
Deverá ser enviado também o arquivo \texttt{.bib} contendo a bibliografia utilizada, no formato \Hologo{BibTeX}.
O arquivo \texttt{.bib} deverá ser enviado mesmo que os autores optem por enviar o manuscrito no formato ODT.
Veja mais sobre o formato no tutorial do \href{https://www.overleaf.com/learn/latex/Bibliography_management_in_LaTeX#The_bibliography_file}{Overleaf},
no \href{https://en.wikibooks.org/wiki/LaTeX/Bibliography_Management}{Wikibooks} ou em \textcite{araujo2020}.



\section{Utilização do modelo}\label{sec-modelo}
Recomendamos a utilização do \Hologo{XeLaTeX}\footnote{
Para ter suporte a várias línguas, decidiu-se adotar o \Hologo{XeLaTeX} ao invés do \LaTeX{}.
O \Hologo{XeLaTeX} processa documentos \TeX{} codificados em URF-8, isto possibilita 
a utilização de qualquer caractere do Unicode diretamente, o que não é possível com o \Hologo{pdfTeX}.  
} para a confecção do manuscrito.
Toda a formatação do texto é definida no arquivo de classe \texttt{textolivre.cls}. 
Este arquivo não deverá ser editado. 
Utilize o arquivo \texttt{tl-article-template.tex} como base para redigir o manuscrito.
Caso haja necessidade de incluir novos pacotes e criar novas definições específicas para um artigo, 
estas deverão ser feitas no preâmbulo do arquivo \texttt{.tex}. 
Em caso de dúvida sobre a utilização do estilo, entre em contato com a equipe editorial da revista.

A bibliografia deverá ser elaborara separadamente, em um arquivo \texttt{.bib}, utilizando as 
convenções do \Hologo{BibTeX}. Para gerar a bibliografia deverá ser utilizada a ferramenta \texttt{Biber}.
A \Cref{lst-compiledocument} apresenta a sequência de comandos para compilar o documento final.
\begin{lstlisting}[language=bash, label=lst-compiledocument, caption={Sequência para gerar o documento final.}]
xelatex tl-article-template.tex 
biber tl-article-template 
xelatex tl-article-template.tex
\end{lstlisting} %stopzone

\section{Organização do Texto}\label{sec-organizacao}
O artigo deverá ter título na língua da manuscrito, título em português e título em inglês, nesta ordem.
Se o manuscrito for redigido em português, teremos apenas dois títulos: em português e em inglês.
Se for redigido em inglês, teremos a ordem inversa: inglês e português.
A mesma regra vale para o resumo que deverá ser apresentado nestas três ou duas línguas.

O artigo poderá ser organizado de forma a conter seções, subseções e subsubseções. 
Não há limites para o tamanho de cada uma. Entretanto, deve-se evitar secções que não possuam texto. 
Ainda nesta secção, iremos apresentar a utilização de listas (\Cref{sec-listas}), 
tabelas e figuras (\Cref{sec-figuras-tabelas}), citações e notas de rodapé (\Cref{sec-quotesandfootnotes}). 
Na \Cref{sec-equacao} iremos apresentar a utilização de equações e na \Cref{sec-codigos}
apresentaremos a forma de utilização de códigos fonte.


\lipsum

\subsection{Listas numeradas e não numeradas}\label{sec-listas}
Os artigos poderão conter listas numeradas ou não numeradas.  
Quando não houver numeração, deverão preferencialmente serem marcadas com ponto, conforme o exemplo abaixo: 
\begin{itemize}
\item exemplo 1
\item exemplo 2
\item exemplo 3
\end{itemize}
\lipsum[2]

Quando for necessário uma lista ordenada, dever-se-á, preferencialmente, seguir a numeração arábica:
\begin{enumerate}
\item exemplo 1
\item exemplo 2
\item exemplo 3
\end{enumerate}
\lipsum[3-4]


\subsection{Inserindo figuras e tabelas}\label{sec-figuras-tabelas}
Nesta seção iremos descrever como inserir figuras no documento em conformidade com o padrão da revista. 
As figuras serão numeradas na ordem em que são mencionadas no texto. Os arquivos de imagem deverão
ser enviados separadamente com nomenclatura que siga a numeração das figuras (não é necessário para envios em \LaTeX{}).
As figuras deverão ficar centralizadas, com legenda abaixo e informação de fonte, caso necessário.
O título, fonte e legenda de figuras devem ser parte do manuscrito, e não parte do arquivo de figura.
As figuras devem ser apropriadamente recortadas para exibir apenas o que é de interesse dos autores,
reduzindo assim os espaços em branco que circundam as figuras.
Os autores devem enviar a figura no formato adequado. 
Sempre que possível, utilize figuras vetoriais para manter uma melhor qualidade e propiciar uma boa apresentação do texto. 
Use os formatos PDF ou EPS para as imagens vetoriais e os formatos PNG ou JPEG para as imagens rasterizadas. 

Não faça referência às figuras dizendo `a figura abaixo/acima' pois com a diagramação do texto a figura poderá não ficar exatamente no logo abaixo/acima do texto que a referencia.
Desta forma, prefira sempre referenciar as figuras pelo seu nome, por exemplo, veja a \Cref{fig-img-a}. O mesmo é válido para tabelas ou qualquer outro elemento flutuante 
que seja inserido no texto (figuras, tabelas, equações, códigos e outros).

\begin{figure}[htbp]
 \centering
 \includegraphics[width=0.5\textwidth]{example-image-a}
 \caption{Imagem de teste.}
 \label{fig-img-a}
 \source{Forneça aqui a fonte da imagem.}
\end{figure}

\lipsum[12]

\begin{table}[htpb]
\caption{Legenda da tabela do documento modelo da revista \textit{Texto Livre}.}
\label{tbl-tabela-01}
\begin{tabular}{llp{4.3cm}}
\toprule 
Tabela                      & Coluna 1                       & Coluna 2                                                               \\ 
\midrule
Lorem ipsum                 & Non consectetur                & Leo vel fringilla                                                      \\ 
\midrule
Lorem ipsum dolor sit amet. & Non consectetur a erat nam at. & Leo vel fringilla est ullamcorper eget nulla facilisi etiam dignissim. \\ 
\bottomrule
\end{tabular}
\source{Fonte da tabela.}
\end{table}

\lipsum[3]

Ainda outros exemplos de utilização de figuras e tabelas são apresentados nas \Cref{fig:example,fig:twosubs,fig:landscape,tab:example}. Em especial, observe os exemplos para criar subfiguras (\Cref{fig:twosubs}) e inserir uma figura grande com orientação paisagem (\Cref{fig:landscape}).

\begin{figure}[htbp]
\centering
\includegraphics[width=0.6\textwidth]{example-image}
\caption{Esta é a legenda da figura. Esta pode ser breve ou longa e conter referências se necessário.}
\label{fig:example}
\source{Referência ao autor e à publicação original da figura. Se a figura foi de autoria própria, apenas indicar ``autoria própria'' como fonte.}
\end{figure}

\lipsum[30-35]

\begin{table}[htbp]
\caption{Recorde de velocidade de automóveis (GR 5-10).}
\label{tab:example}
\centering
\begin{tabular}{l l l l r}
\headrow \thead{Speed (mph)} & \thead{Driver} & \thead{Car} & \thead{Engine} & \thead{Date} \\
407.447     & Craig Breedlove & Spirit of America          & GE J47    & 8/5/63   \\
413.199     & Tom Green       & Wingfoot Express           & WE J46    & 10/2/64  \\
434.22      & Art Arfons      & Green Monster              & GE J79    & 10/5/64  \\
468.719     & Craig Breedlove & Spirit of America          & GE J79    & 10/13/64 \\
526.277     & Craig Breedlove & Spirit of America          & GE J79    & 10/15/65 \\
536.712     & Art Arfons      & Green Monster              & GE J79    & 10/27/65 \\
555.127     & Craig Breedlove & Spirit of America, Sonic 1 & GE J79    & 11/2/65  \\
576.553     & Art Arfons      & Green Monster              & GE J79    & 11/7/65  \\
600.601     & Craig Breedlove & Spirit of America, Sonic 1 & GE J79    & 11/15/65 \\
622.407     & Gary Gabelich   & Blue Flame                 & Rocket    & 10/23/70 \\
633.468     & Richard Noble   & Thrust 2                   & RR RG 146 & 10/4/83  \\
763.035     & Andy Green      & Thrust SSC                 & RR Spey   & 10/15/97\\
\end{tabular}
\source{\url{https://www.sedl.org/afterschool/toolkits/science/pdf/ast_sci_data_tables_sample.pdf}}
\end{table}


\lipsum[2-4]

\begin{figure}[htbp]
\begin{minipage}{0.47\textwidth}
\includegraphics[width=\linewidth]{example-image}
\subcaption{Esta é uma subfigura.}
\source{Autoria própria.}
\end{minipage}
\hfill
\begin{minipage}{0.47\textwidth}
\includegraphics[width=\linewidth]{example-image}
\subcaption{Esta é outra subfigura.}
\end{minipage}

\caption{Esta é a legenda geral aplicada a ambas figuras.}
\label{fig:twosubs}
\source{Caso ambas figuras tenham a mesma autoria, basta especificar a fonte uma única vez.}
\end{figure}

\begin{sidewaysfigure}
\centering
\includegraphics[width=0.85\textwidth]{example-image}
\caption{Esta é a legenda para uma figura grande que ocupa toda a página. Para melhor apresentação desta figura, ela é rotacionada utilizando o ambiente \texttt{sidewaysfigure}, sendo então exibida no formato paisagem.}
\label{fig:landscape}
\source{Figura de autoria própria.}
\end{sidewaysfigure}


Se possível, evite tabelas muito longas. Se elas foram necessários, utilize-as preferencialmente na sessão de apêndice. 
Veja como exemplo a \Cref{longtbl-01} em \Cref{apx-longtable}.



\subsection{Citações e notas de rodapé}\label{sec-quotesandfootnotes}
Para inserir citações no texto utilize a formatação descrita nesta seção. 

\begin{quote}
Book printing differs significantly from ordinary typing with respect to dashes, hyphens, and minus signs.
In good math books, these symbols are all different; in fact there usually are at least four different symbols (...)
Hyphens are used for compound words like `daughter-in-law' and `X-rated'. En-dashes are used for number ranges like 
`pages 13--34', and also in contexts like `exercise 1.2.6--52'. 
Em-dashes are used for punctuation in sentences---they are what we often call simply dashes. 
And minus signs are used in formulas. A conscientious user of \TeX{} will be careful to distinguish these four usages 
\cite[p. 4]{donaldknuth1984}.
\end{quote}

Elementum pulvinar etiam non quam lacus. Posuere urna nec tincidunt praesent semper feugiat nibh. 
Nunc sed id semper risus in hendrerit. Fermentum odio eu feugiat pretium. 
Ac tortor dignissim convallis aenean et tortor. 
Elementum integer enim neque volutpat ac tincidunt vitae\footnote{
Notas de rodapé devem aparecer no final da página onde a nota foi utilizada. A fonte utilizada é  Liberation Sans, corpo 10, alinhamento justificado.
}.

Para realizar citações devemos utilizar os comandos do \texttt{biblatex}. Por exemplo, para realizar uma
citação textual, utilize comando \texttt{textcite} (equivalente ao comando \texttt{citet}) para gerar o
resultado como aqui exemplificado: \textcite{donaldknuth1984}. 
Para criar uma citação entre parênteses, utilize o comando \texttt{parencite} (equivalente ao comando \texttt{citep})
para gerar um resultado como o do exemplo a seguir: \parencite{donaldknuth1984}.



\subsection{Exemplo de utilização de equações}\label{sec-equacao}
Nesta secção iremos apresentar a forma de utilização de equações. A \Cref{eq-poisson} apresenta um
exemplo de equação no modelo da revista \textit{Texto Livre}.
\begin{equation}
l(\Lambda)=\sum_{i=1}^{n} \sum_{w=1}^{q} (z_{i w} \ln (\lambda_{i w}) - \lambda_{i w} - \ln (z_{i w}!))
\label{eq-poisson}
\end{equation}
As equações deverão ser numeradas, para que seja possível realizar referência a elas ao longo do texto.
Outro exemplo é apresentado na \Cref{eq-frac}.
\begin{equation}
  x = a_0 + \frac{1}{\displaystyle a_1 
          + \frac{1}{\displaystyle a_2 
          + \frac{1}{\displaystyle a_3 + a_4}}}
\label{eq-frac}
\end{equation}

Apresentamos nas \Cref{eq-align-ex1,eq-align-ex2,eq-align-ex3,eq-align-ex4,eq-align-ex5,eq-align-ex6} um exemplo de utilização de equações com mais de uma linha. Para este tipo de equação devemos utilizar o ambiente \texttt{aling}. As equações que compõem a sequência poderão ser enumeradas ou não. Case sejam enumeradas, opte por inseri-las dentro do contexto do ambiente \texttt{subsequations} para que a numeração seja conforme o exemplo apresentado.

\begin{subequations}
\begin{align}
H(Z|X) &= \sum_x p(x) H(Z|X=x) \label{eq-align-ex1} \\
       &= - \sum_x p(x) \sum_z p(Z=z|X=x) \log p(Z=z|X=x) \label{eq-align-ex2}\\
       &= - \sum_x p(x) \sum_y p(Y=z-x|X=x) \log p(Y=z-x|X=x) \label{eq-align-ex3} \\
       &= - \sum_x p(x) \sum_y p(Y=y|X=x) \log p(Y=y|X=x) \label{eq-align-ex4} \\
       &= \sum_x p(x) H(Y|X=x) \label{eq-align-ex5} \\
       &= H(Y|X) \label{eq-align-ex6}
\end{align}
\end{subequations}

Figure \ref{fig:example} shows a normal figure, while figure \ref{fig:twosubs} show one made up of two sub-figures. Figure \ref{fig:landscape} is an example of a landscaped figure. You can use the \verb|\subcaption{...}| command from the \texttt{subcaption} package to add captions for subfigures and subtables, but do not use the \texttt{subfigure} package: it is incompatible with this template.


\lipsum[10-14]



\subsection{Inserindo códigos no manuscrito}\label{sec-codigos}
Para inserir códigos no texto utilize o ambiente \texttt{lstlisting}. 
O modelo da revista permite a inserção da fonte utilizada como referência, confome 
pode ser visto no exemplo apresentado na \Cref{lst-code}.
 
\begin{lstlisting}[language=python, label=lst-code, caption={\textit{Bubble sort}, ou ordenação por flutuação.}, source={Rosetta Code (\url{https://rosettacode.org/wiki/Sorting_algorithms/Bubble_sort}).}]
def bubble_sort(seq):
    """Inefficiently sort the mutable sequence (list) in place.
       seq MUST BE A MUTABLE SEQUENCE.
 
       As with list.sort() and random.shuffle this does NOT return 
    """
    changed = True
    while changed:
        changed = False
        for i in xrange(len(seq) - 1):
            if seq[i] > seq[i+1]:
                seq[i], seq[i+1] = seq[i+1], seq[i]
                changed = True
    return seq
\end{lstlisting} %stopzone



\lipsum[20-21]


\section{Conclusão}\label{sec-conclusao}
Para elaborar um texto com melhor qualidade \cite{donaldknuth1984,leslielamport1994,araujo2020}, busque utilizar o modelo para \LaTeX{} 
disponível no sítio da revista.

\lipsum[17-19]



\printbibliography\label{sec-bib}
% if the text is not in Portuguese, it might be necessary to use the code below instead to print the correct ABNT abbreviations [s.n.], [s.l.] 
%\begin{portuguese}
%\printbibliography[title={Bibliography}]
%\end{portuguese}

\appendix 
\section{Tabela longa}\label{apx-longtable}
Apresentamos aqui um exemplo de uma tabela longa. Para este tipo de tabela utilzie o ambiente \texttt{longtable}.

\setlength\LTleft{-1in}
\setlength\LTright{-1in}
\begin{small}
\begin{longtable}{
    >{\raggedright\arraybackslash}p{0.3\textwidth}
    p{0.4\textwidth}
    p{0.4\textwidth}
    }
\caption{Exemplo de tabela longa.}
\label{longtbl-01}
\\
\toprule
coluna 1 & coluna 2 & coluna 3 \\
\midrule
\lipsum[2] & \lipsum[3] & \lipsum[4] \\
\midrule
\lipsum[5] & \lipsum[6] & \lipsum[7] \\
\midrule
\lipsum[8] & \lipsum[9] & \lipsum[10] \\
\midrule
\lipsum[11] & \lipsum[12] & \lipsum[13] \\
\bottomrule
\source{Texto de preenchimento gerado pelo pacote lipsum.}
\end{longtable}
\end{small}


\end{document}
