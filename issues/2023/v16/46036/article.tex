% !TEX TS-program = XeLaTeX
% use the following command:
% all document files must be coded in UTF-8
\documentclass[english]{textolivre}
% build HTML with: make4ht -e build.lua -c textolivre.cfg -x -u article "fn-in,svg,pic-align"

\journalname{Texto Livre}
\thevolume{16}
%\thenumber{1} % old template
\theyear{2023}
\receiveddate{\DTMdisplaydate{2023}{5}{1}{-1}} % YYYY MM DD
\accepteddate{\DTMdisplaydate{2023}{6}{14}{-1}}
\publisheddate{\DTMdisplaydate{2023}{8}{18}{-1}}
\corrauthor{Yousef Methkal Abd Algani}
\articledoi{10.1590/1983-3652.2023.46036}
%\articleid{NNNN} % if the article ID is not the last 5 numbers of its DOI, provide it using \articleid{} commmand 
% list of available sesscions in the journal: articles, dossier, reports, essays, reviews, interviews, editorial
\articlesessionname{articles}
\runningauthor{Algani and Gross} 
%\editorname{Leonardo Araújo} % old template
\sectioneditorname{Daniervelin Pereira}
\layouteditorname{Thaís Courtinho}

\title{Lecturers-based evaluation on the role of technological advancements in teaching}
\othertitle{Avaliação por docentes sobre o papel dos avanços tecnológicos no ensino}
% if there is a third language title, add here:
%\othertitle{Artikelvorlage zur Einreichung beim Texto Livre Journal}

\author[1]{Yousef Methkal Abd Algani~\orcid{0000-0003-2801-5880}\thanks{Email: \href{yosefabdalgani@gmail.com}{yosefabdalgani@gmail.com}}}
\author[1]{Zehavit Gross~\orcid{0000-0002-8758-2036}\thanks{Email: \href{zehavit.gross@biu.ac.il}{zehavit.gross@biu.ac.il}}}
\affil[1]{Bar-Ilan University, Faculty of Education, Tel- Aviv, Israel.}

\addbibresource{article.bib}
% use biber instead of bibtex
% $ biber article

% used to create dummy text for the template file
\definecolor{dark-gray}{gray}{0.35} % color used to display dummy texts
\usepackage{lipsum}
\SetLipsumParListSurrounders{\colorlet{oldcolor}{.}\color{dark-gray}}{\color{oldcolor}}

% used here only to provide the XeLaTeX and BibTeX logos
\usepackage{hologo}

% if you use multirows in a table, include the multirow package
\usepackage{multirow}

% provides sidewaysfigure environment
\usepackage{rotating}

% CUSTOM EPIGRAPH - BEGIN 
%%% https://tex.stackexchange.com/questions/193178/specific-epigraph-style
\usepackage{epigraph}
\renewcommand\textflush{flushright}
\makeatletter
\newlength\epitextskip
\pretocmd{\@epitext}{\em}{}{}
\apptocmd{\@epitext}{\em}{}{}
\patchcmd{\epigraph}{\@epitext{#1}\\}{\@epitext{#1}\\[\epitextskip]}{}{}
\makeatother
\setlength\epigraphrule{0pt}
\setlength\epitextskip{0.5ex}
\setlength\epigraphwidth{.7\textwidth}
% CUSTOM EPIGRAPH - END

% LANGUAGE - BEGIN
% ARABIC
% for languages that use special fonts, you must provide the typeface that will be used
% \setotherlanguage{arabic}
% \newfontfamily\arabicfont[Script=Arabic]{Amiri}
% \newfontfamily\arabicfontsf[Script=Arabic]{Amiri}
% \newfontfamily\arabicfonttt[Script=Arabic]{Amiri}
%
% in the article, to add arabic text use: \textlang{arabic}{ ... }
%
% RUSSIAN
% for russian text we also need to define fonts with support for Cyrillic script
% \usepackage{fontspec}
% \setotherlanguage{russian}
% \newfontfamily\cyrillicfont{Times New Roman}
% \newfontfamily\cyrillicfontsf{Times New Roman}[Script=Cyrillic]
% \newfontfamily\cyrillicfonttt{Times New Roman}[Script=Cyrillic]
%
% in the text use \begin{russian} ... \end{russian}
% LANGUAGE - END

% EMOJIS - BEGIN
% to use emoticons in your manuscript
% https://stackoverflow.com/questions/190145/how-to-insert-emoticons-in-latex/57076064
% using font Symbola, which has full support
% the font may be downloaded at:
% https://dn-works.com/ufas/
% add to preamble:
% \newfontfamily\Symbola{Symbola}
% in the text use:
% {\Symbola }
% EMOJIS - END

% LABEL REFERENCE TO DESCRIPTIVE LIST - BEGIN
% reference itens in a descriptive list using their labels instead of numbers
% insert the code below in the preambule:
%\makeatletter
%\let\orgdescriptionlabel\descriptionlabel
%\renewcommand*{\descriptionlabel}[1]{%
%  \let\orglabel\label
%  \let\label\@gobble
%  \phantomsection
%  \edef\@currentlabel{#1\unskip}%
%  \let\label\orglabel
%  \orgdescriptionlabel{#1}%
%}
%\makeatother
%
% in your document, use as illustraded here:
%\begin{description}
%  \item[first\label{itm1}] this is only an example;
%  % ...  add more items
%\end{description}
% LABEL REFERENCE TO DESCRIPTIVE LIST - END


% add line numbers for submission
%\usepackage{lineno}
%\linenumbers

\begin{document}
\maketitle

\begin{polyabstract}
\begin{abstract}
The impact of technological advancements on the duties of lecturers in the twenty-first century has been significant. The rapid integration of technological advancements in the academic setting, particularly among lecturers, necessitates a proficient comprehension of the function that technology serves in the instruction of foreign languages as a fundamental component of their professional skill set. This article seeks to unveil the views of university lecturers concerning the implementation of technologically assisted instruments for the instruction of foreign languages, while also taking into account the lecturers’ perspectives regarding the application of such technologies. The analysis follows the dimension of the “TPACK” model. This model acknowledges the awareness and proficiency of linguistics lecturers in incorporating technological advancements within the educational system. The incorporation of digital tools is achieved through an expansion of the “pedagogical content knowledge (PCK)” model originally proposed by Lee Shulman. Using this model, the purpose of this research is to conduct an analysis of three significant areas (subject, pedagogy, and technology) of the use of technology in the instruction of foreign languages by lecturers. Three hundred university lecturers of foreign languages are represented in this research, and they were drawn from different universities. There was no focus on any one specific foreign language, and the lecturers’ participation consisted only of providing information in the form of questionnaire responses. According to the results of the research, a significant number of university lecturers have already embraced the use of some of these technologies in teaching foreign languages. Despite this, it is nevertheless necessary for them to get familiarized with advanced technological tools. 

\keywords{Language Pedagogy \sep Lecturers \sep TPACK \sep Technology}
\end{abstract}

\begin{portuguese}
\begin{abstract}
O impacto dos avanços tecnológicos nas funções dos professores no século XXI tem sido significativo. A rápida integração dos avanços tecnológicos no ambiente acadêmico, particularmente entre os professores, exige uma compreensão proficiente da função que a tecnologia desempenha no ensino de línguas estrangeiras como um componente fundamental de seu conjunto de habilidades profissionais. Este artigo busca desvendar a visão de professores universitários sobre a implementação de instrumentos tecnologicamente assistidos para o ensino de línguas estrangeiras, levando em consideração também as perspectivas dos professores sobre a aplicação de tais tecnologias. A análise segue a dimensão do modelo “TPACK”. Esse modelo reconhece a consciência e a proficiência dos professores de linguística na incorporação de avanços tecnológicos no sistema educacional. A incorporação de ferramentas digitais é conseguida através de uma expansão do modelo de "conhecimento pedagógico do conteúdo (PCK)" originalmente proposto por Lee Shulman. Usando esse modelo, o objetivo desta pesquisa é conduzir a análise em três áreas significativas (disciplina, pedagogia e tecnologia) do uso da tecnologia no ensino de línguas estrangeiras por professores. Trezentos professores universitários de línguas estrangeiras estão representados nesta pesquisa, e foram sorteados em diferentes universidades. Não houve foco em nenhuma língua estrangeira específica, e a participação dos palestrantes consistiu apenas em fornecer informações na forma de respostas a questionários. De acordo com os resultados da pesquisa, um número significativo de docentes universitários já aderiu ao uso de algumas dessas tecnologias no ensino de línguas estrangeiras. Apesar disso, é necessário que eles se familiarizem com ferramentas tecnológicas avançadas.

\keywords{Pedagogia de Línguas \sep Docentes \sep TPACK \sep Tecnologia}
\end{abstract}
\end{portuguese}
% if there is another abstract, insert it here using the same scheme
\end{polyabstract}

\section{Introduction}

For decades, scholars in the field of linguistics have dug into the question of how language instruction can be improved in a variety of ways \cite{al-awawdeh_foreign_2022}. Various approaches to language instruction have developed in response to the unique difficulties presented by actual classroom settings. Integrating technological advancements as components of pedagogical tools in university lecture halls has been adopted by different universities around the world \cite{hartono_language:_2021}. \textcite{kumar_analyzing_2021} conducted an investigation on the efficacy of technology in enhancing language teaching. The investigation revealed that technology was employed in nearly all domains of language instruction. The utilization of technology has proven to be beneficial in enhancing the standard of input, ensuring genuine communication, and delivering prompt and pertinent feedback.

The incorporation of digital resources in the context of learning has resulted in a transformation of language pedagogy. These devices are now utilized to supplement the conventional resources employed in teaching. \textcite{hartono_language:_2021} and \textcite{hall_first_2020} supported the view that language teaching method has been changed due to technology. \textcite{lyublinskaya_analysis_2022} indicate that the significance of utilizing technology in education, particularly in virtual learning settings, has become increasingly imperative due to the global shift towards online instruction following the outbreak of the pandemic. Following this, many lecturers have been adjusting and acquiring the necessary skills to effectively utilize these technological tools in the instructional process.

In light of the incorporation of technological devices in the realm of education, scholars have conducted thorough investigations into the technologies suitable for contemporary pedagogy, as posited by \textcite{hartono_language:_2021}. The aforementioned circumstance facilitated the emergence “from the technological pedagogical content knowledge (TPACK) model”. The model’s implementation enables the evaluation of linguistics lecturers’ proficiency to incorporate technology during the pedagogical practices.

\section{Literature Review}

When emerging technologies are considered together with the issues they provide to lecturers, using technology to teach becomes even more challenging. In light of this obstacle, what are some ways that lecturers, particularly those who instruct students in a foreign language, might use technology in their lessons? There is a need for an approach that views instruction as a dialogue between what the lecturer knows and how he or she applies that knowledge in the specific conditions or settings of the lecture room.

%\begin{itemize}
%    \item \textbf{Exploring the Premise of “TPACK” Model}
%\end{itemize}
\subsection{Exploring the Premise of “TPACK” Model}

The TPACK model was developed following the conceptualization of “Pedagogical Content Knowledge (PCK)” \cite{wittrock_paradigms_1986}, to elucidate the interplay between educators’ understanding of technological advances in education and PCK, which synergistically contribute to effective technology-enhanced instruction. \textcite{lyublinskaya_analysis_2022} posited that the emergence of the TPACK model can be attributed to the heightened focus among scholars on the readiness of language educators to incorporate technology into their instructional practices. According to \textcite{saubern_describing_2020}, this paradigm was first presented for the field of mathematics education. Subsequently, it was applied to a variety of other topics and situations by other writers.

This particular approach necessitates that educators must possess a comprehensive understanding of “pedagogical knowledge (PK)”, “Subject-Matter knowledge (CK)”, as well as “technology knowledge (TK)”, which is commonly known as extensive imparting information. Moreover, the model underscores the paramount importance of the simultaneous interaction among the three components, namely “CK”, “PK”, as well as “TK”, for language teaching procedures. The TPACK model recognizes the inevitability of ongoing technological progress. In light of this emerging paradigm, it is imperative for educators to receive instruction in the utilization of diverse technologies, as well as develop the requisite expertise to handle the vicissitudes engendered by the implementation of novel software and hardware.

The TPACK model offers an innovative perspective on deploying digital tools in lecture halls via concentrating not only on the instruction of instrumental abilities but also on the interrelationship between those competencies and the instructional component. Cognitive, methodological, emotional, and behavioral skills should be stressed in first lecturer education, according to \textcite{vygotskii_mind_1978}. These are the abilities that should be addressed in initial lecturer education. Because of their expertise and comprehension, university lecturers may employ technological tools in lecture halls. The model is explicated in intricate detail as an intricate interplay among three distinct domains of knowledge, namely content, pedagogy, and technology, in order to attain the desired objective. The goal of this interaction is to ensure that lecturers have a well-rounded understanding of how to effectively incorporate technology into their classrooms. The interplay of various “bodies of information, both theoretically and in practice, provides the forms of flexible knowledge that are necessary to effectively incorporate technology usage into teaching” \cite[p. 38]{andriany2022analysis}.

“Content knowledge” implies the understanding lecturers have concerning an area of study which will form a component of the class lecture. According to the research conducted by \textcite{gomez-trigueros_digital_2021}, the curriculum content suitable for “a high school history or science class differs from that of an art appreciation undergraduate course or an astrophysics doctoral seminar”. A fundamental requirement for educators is to possess a comprehensive understanding of the subject matter they impart. Moreover, Technological Knowledge (TK) pertains to the proficiency necessary for the operation of computers and their corresponding software. The notion of Technological Knowledge (TK) as employed in the Technological Pedagogical Content Knowledge (TPACK) model bears a striking resemblance to the description of Fluency of Information Technology (FITness) as introduced by the Committee on Information Technology Literacy of the National Research Council in 1999 \cite[p. 78]{national_research_council_read_1999}. \textcite{hannah_inclusive_2020} argue that the concept of FITness surpasses conventional understandings of computer literacy. It necessitates a comprehensive comprehension of technological innovation that enables individuals to effectively apply it in their private and business domains, identify situations where information technology can facilitate or impede goal attainment, and continuously adapt to technological advancements.

In addition, the use of various teaching strategies in an efficient manner is what is referred to as Pedagogical Knowledge (PK). This form of knowledge covers not only the techniques that are used in teaching but also knowledge about the nature of the audience that is being taught to, as well as ways for assessing student comprehension. This type of knowledge extends beyond the simple application of teaching methods. According to \textcite{yeh_toward_2021}, a language lecturer that has an in-depth pedagogical understanding knows how students generate information and gain abilities, as well as how they cultivate habits of mind and favorable attitudes toward learning.

%\begin{itemize}
%    \item \textbf{“Pedagogical Content Knowledge”}
%\end{itemize}
\subsection{“Pedagogical Content Knowledge”}

The notion of PCK centers around the concept of modifying subject matter for instructional purposes. \textcite[p. 83]{wittrock_paradigms_1986} asserted that the “transformation occurs when the educator scrutinizes the subject matter, identifies multiple approaches to represent it, and modifies and customizes the instructional resources to accommodate diverse perspectives and the learners’ prior understanding”. This phenomenon occurs when the educator employs diverse methods to illustrate the content. \textcite{gomez-trigueros_digital_2021} posit that the central focus of educational instruction, program of study, evaluation, and reporting is the acquisition of comprehensive knowledge. This comprehensive understanding encompasses the factors that facilitate the acquisition of knowledge, in addition to the interconnections among teaching methods, educational content, and evaluation.

Thus, an experienced lecturer deals differently in such circumstances, he/she is distinct from one who is just starting out in the profession since the former has a greater understanding of how to implement a variety of instructional models and tactics, as well as how to improve student interaction inside the classroom, \textcite{oner_virtual_2020}.

%\begin{itemize}
%    \item \textbf{Technology Pedagogical Knowledge} 
%\end{itemize}
\subsection{Technology Pedagogical Knowledge}

The term “Technological Pedagogical Knowledge”, represents the claim that technology may help make educational methods more effective. One example of this would be the use of asynchronous discussion forums to enhance the social creation of knowledge. This domain of inquiry pertains to the examination of potential modifications in pedagogy and cognition resulting from the implementation of particular technological tools in distinct manners. This involves having a solid understanding of the pedagogical affordances and restrictions posed by a variety of technology tools, particularly in relation to the pedagogical designs and techniques that are developmentally and disciplinary-appropriate \cite{lyublinskaya_analysis_2022}.

In the academic industry, the distinction between experienced and poor educators is largely dependent upon whether or not the lecturer has a strong grasp of how technology may be employed in teaching tactics. This is especially true when technology is incorporated into the pedagogical expertise of lecturers. According to \textcite{koehler_what_2009} TPK is a type of knowledge that is concerned with how lecturers use a range of tools...for a particular task, the ability to choose a tool based on its fitness, strategies for using the tool’s affordances, and knowledge of pedagogical strategies and the ability to apply it. TPK is beneficial for a number of reasons, one of which is that it enables language lecturers to broaden their expertise in the use of various technological tools, particularly those that are suitable for the classroom, given that the majority of today’s most popular software applications were not developed with instruction in mind. This, however, demands reorientating lecturers on the value and application of digital tools in lecture halls, in such a way that the technologies in question may be reconfigured and tailored for pedagogical reasons. Transformative pedagogical practices entail a progressive, inventive, and receptive approach to utilizing technological devices, not solely for the sake of technology, but rather to enhance student learning and understanding \cite{hartono_language:_2021}.

\subsection{Technological Content Knowledge}

The understanding of how different forms of technology and content connect at an interface is referred to as TCK. The term “Technological Content Knowledge” (TCK) pertains to an understanding of the interplay between technological advances and information, including their mutual influence and limitations. One example of TCK is the use of computer simulations to portray and analyze the motions of the earth’s crust. When teaching a foreign language, the target language itself constitutes the subject matter or topic knowledge. The effect of this component of the TPACK model is that the obligation of understanding the right technology that is suited for the pedagogy of a given topic is passed on to the language lecturers. This also implies that language lecturers are able to pick, change, apply, and integrate certain technological components that are the most suitable for the material that will be taught. EFL lecturers in China, for instance, are now teaching English based on the content of TED talks (technology, entertainment, and design), and they support students’ conversation as a result of the ease with which they can access internet video resources such as TED video lectures \cite{andriany2022analysis}.

%\begin{enumerate}[label={\alph*)}]
%    \item \textbf{Problem of the Study}
\subsubsection{Problem of the Study}
    
    It is increasingly vital for lecturers of foreign languages to have a solid awareness of the function of digital tools in the teaching of foreign languages as part of their professional repertoire. This is especially important given the rapid rate at which modern technology is being used in the classroom, particularly by lecturers. With the incorporation of the TPACK model, this article aims to evaluate the perspectives of university lectures concerning using technologically assisted tools for the teaching of foreign languages, taking into account their perspectives on the application of such technologies.

%    \item \textbf{Research Questions}
\subsubsection{Research Questions}

    The present investigation is grounded on the following questions for the study:
    \begin{enumerate}[label={\arabic*.}]
    \item What are the attitudes of university lecturers regarding the integration of technological tools in foreign language instruction?
    \item What is the perspective of linguistics and foreign language lecturers regarding the efficacy of incorporating technological tools in their teaching of foreign languages?
    \end{enumerate}

%    \item \textbf{Research Objective}
\subsubsection{Research Objective}

    The primary goal of this study is to critically evaluate the views of university lecturers on the integration of technological tools for foreign languages at the university level. This primary will be explored through the prism of the TPACK model. The following specific objectives are pursued in the study:
    \begin{enumerate}[label={\roman*.}]
    \item To expound on the attitudes of university lecturers regarding the integration of technological tools in foreign language teaching;
    \item To analyze the perspectives of linguistics and foreign language lecturers concerning the effectiveness of incorporating technological tools in their teaching of foreign languages.
    \end{enumerate}
%\end{enumerate}








\section{Methodology and Procedure}

%\begin{enumerate}[label={\Alph*.}]
%    \item \textbf{Research Design}
\subsection{Research Design}

    The research employed a descriptive-analytical approach due to the specialized nature of the subject matter. Additionally, a quantitative methodology was utilized to assess the gathered data. This study involved a random selection of 300 university professors of foreign languages who utilize digital communication networks.

%    \item \textbf{Study Community}
\subsection{Study Community}

    The study’s sample comprises 300 foreign language lecturers who were selected from various locations using digital communication platforms. The foreign language lecturers who were involved in the study were concurrently teaching various foreign languages at different universities during the timeframe of January 2023 to April 2023, which coincided with the duration of the study.

%    \item \textbf{The Study Sample}
\subsection{The Study Sample}

    In this research, a random sampling technique was adopted in selecting from the study population. Consequently, three hundred lecturers (one-hundred and seventy-five male and one-hundred and twenty-five female) who teach foreign languages in different universities actively participated in this research. However, a web-based questionnaire was utilized via Google Forms to gather data. The statistical software package SPSS was utilized for both the entry and analysis of the data. The tabular representation depicts the attributes of the selected populace while considering various pertinent variables.

\begin{table}[h!]
\centering
\begin{threeparttable}
\caption{Demographic Variables}
\label{tab1}
\begin{tabular}{llll}
\toprule
Categories & Sub-categories & Frequency & Percentage \\ \midrule
\arrayrulecolor[gray]{.7}
\multirow{2}{*}{Gender} & Male & 175 & 58.33\% \\ %\cline{2-4} 
 & Female & 125 & 41.64\% \\ \midrule %\hline
\multirow{2}{*}{Educational Qualification} & Masters & 147 & 49\% \\ %\cline{2-4} 
 & PhD & 153 & 51\% \\ \midrule %\hline
\multirow{2}{*}{Years of Experience} & 1-10 years & 160 & 84\% \\ %\cline{2-4} 
 & 11 years or more & 140 & 16\% \\ 
\arrayrulecolor{black}
\bottomrule
\end{tabular}
\end{threeparttable}
\end{table}

%    \item \textbf{Study Tools}
\subsection{Study Tools}

    Each participant in the study was administered a Likert Scale questionnaire consisting of five points. The survey comprises two fundamental components. The initial segment is allocated to the examination of significant demographic variables. The last section aligns with the viewpoints espoused by lecturers of foreign languages regarding the incorporation of technological apparatuses within the lecture room environment. Also, it takes into account the evaluation of the effects of the use of technology in foreign language instruction, as determined by the implementation of the TPACK model.

%    \item \textbf{Method of Data Analysis}
\subsection{Method of Data Analysis}

    Statistical tools, such as graphs and tables, were utilized to assess the values provided by the respondents across the various inputs of the questionnaire. The responses obtained from the questionnaire were utilized to ascertain frequencies and percentages, which were subsequently inputted into the system.
%\end{enumerate}




\section{Data Presentation}

Based on the quantitative approach adopted by this research, participants’ responses, including the questions on the questionnaires are presented in the table below. However, these questions were based on the main objectives of this study.

%\begin{enumerate}[label={\Alph*.},series=questions]
%    \item \textbf{What are the opinions and attitudes of foreign language lecturers on the use of technology in teaching foreign language?}
%\end{enumerate}
\subsection{What are the opinions and attitudes of foreign language lecturers on the use of technology in teaching foreign language?}

This section seeks to expound on the views of university lecturers on using digital tools in lecture halls. \Cref{tab2} presents the result of the collated data.

\begin{table}[h!]
\centering
\begin{threeparttable}
\caption{Result of Usage of Technology in Foreign Language Teaching, Including Attitudes of Lecturers}
\label{tab2}
\begin{tabular}{p{5cm}lllllll}
\toprule
Question Variables & SA & A & N & SD & D & Mean & \multicolumn{1}{>{\raggedright\arraybackslash}p{1.6cm}}{Std. Deviation} \\ 
\midrule
I use frequently some digital tools in teaching foreign language. & 17.32 & 8.26 & 3.76 & 40.91 & 29.75 & 2.74 & 1.78 \\ 
I have received training on the proper integration of technology while teaching language. & 18.48 & 16.33 & 4.68 & 32.39 & 28.12 & 27.82 & 1.80 \\ 
Foreign language teaching should be primarily technology-based. & 20.65 & 18.39 & 3.62 & 30.17 & 27.17 & 2.63 & 1.91 \\ 
I encounter some problems while using technological tools to teach. & 33.74 & 30.39 & 5.72 & 16.19 & 14.41 & 3.42 & 1.63 \\ 
Foreign language students learn language easily with the use of technological tools. & 43.77 & 35.93 & 8.61 & 6.53 & 5.16 & 4.01 & 1.19 \\ 
\bottomrule
\end{tabular}
\end{threeparttable}
\end{table}

\Cref{tab2} represents the views of university lecturers, particularly those that teach foreign languages with the integration of technology in teaching foreign language. The following findings are evident from the data presented above:

\begin{enumerate}
    \item Over 69\% of the participants claim that they do not use technology-based tools in teaching, about 25\% affirm using these tools, while 3.76\% remained neutral. This shows that despite the effectiveness of the use of technological tools to teach, there are underlying factors that can deter its integration into the educational system. The 2.74 mean value is indicative of the lower number of study participants that accept that they frequently use various technological tools in teaching foreign languages in the lecture halls;
    \item While approximately 35\% of the participant affirmed that they have received training on the use of technology in language teaching, more than 60\% of the lecturers that participated in the study refute this claim. While there is advocacy on the literacy of language pedagogy using technology, it is necessary to note that adequate provision such as training teaches on how to make use of technological tools is required for effective integration of technology in language pedagogy;
    \item The third question item from the table above indicate that language lecturers still appreciate the use of traditional teaching tools in the classroom. The majority of the participants, which is approximately 58\% rejected the idea that technological tools should only be used in teaching foreign language, while more than 38\% accepted the idea. This indicates that the integration of technological tools in foreign language may need to be supported by other conventional tools that also facilitate the teaching of foreign languages in the lecture halls;
    \item Foreign language lecturers majorly face challenges in the use of technology in teaching if they lack proper training on the usage of these tools. About 62\% of the participants affirm that they encounter challenges in the usage of these tools, while 32\% of them rejected this claim;
    \item The last responses reflect the effectiveness of the integration of technology in teaching foreign language. However, this effectiveness is not only seen in the attitude of language lecturers but also in the performances of the students. From the responses, 80\% of the participants affirmed the effectiveness of technology usage, while 11\% refuted this claim.
\end{enumerate}

Overall, the lecturers showed a positive attitude towards the integration of technological tools in the teaching of foreign languages at the university level. The data indicates that despite the observation that a greater number of lecturers do not use these technological tools in teaching foreign language in the lecture halls, they acknowledge the effectiveness.

%\begin{enumerate}[label={\Alph*.},resume=questions]
%    \item \textbf{What are the opinions of language lecturers on the effectiveness of the use of technology in foreign language teaching?}
%\end{enumerate}
\subsection{What are the opinions of language lecturers on the effectiveness of the use of technology in foreign language teaching?}

This section explores the result of the lecturers’ opinions based on the TPACK model. It is divided into different tables for each of the components of the TPACK model.

\begin{table}[h!]
\centering
\begin{threeparttable}
\caption{Result of data on Technological Knowledge (TK)}
\label{tab3}
\begin{tabular}{p{5cm}lllllll}
\toprule
Question Variables & SA & A & N & SD & D & Mean & \multicolumn{1}{>{\raggedright\arraybackslash}p{1.6cm}}{Std. Deviation} \\ 
\midrule
As a lecturer, I am conversant with all the relevant technological tools used in teaching foreign language in my school & 16.36 & 12.64 & 2.21 & 38.34 & 30. 45 & 2.47 & 1.86 \\ 
Use of digital tools complements traditional resources used in foreign language teaching? & 43.65 & 33.19 & 2.52 & 12.15 & 8.49 & 3.89 & 1.17 \\ 
I frequently attend professional development opportunities focused on knowledge of the integration of technology in teaching foreign language? & 15.32 & 13.18 & 4.19 & 35.39 & 31.92 & 2.49 & 1.87 \\ 
\bottomrule
\end{tabular}
\end{threeparttable}
\end{table}

\Cref{tab3} reveals that the effectiveness of the integration of technology in teaching foreign language is highly determined by lecturers’ knowledge of these tools. The data in the table are thus summarized below:

\begin{enumerate}
    \item In the first response, the majority of the participants affirmed that they are not conversant with all the relevant technology used in teaching language. About 29\% of the participant accepted that they know and are conversant with how to operate these tools, while 69\% of the participants refuted this idea. The implication of this finding is that the percentage of foreign language lecturers who are constantly catching up with lasts technological tools for language pedagogy is on the low side;
    \item The second responses affirm that the technology plays the role of complementing the traditional teaching method and tools used in teaching foreign language. More than 50\% of the participants confirm that technology complements regular language pedagogy, while 22\% refuted this claim;
    \item The third responses from the research participant reflect on the steps foreign language lecturers take to ensure that they have the necessary technological knowledge to effectively integrate technology into their teaching. The majority of the participants (68\%) indicated that they do not attend professional development sessions or workshops in order to improve their knowledge of the use of technological tools, while 28\% rejected this claim.
\end{enumerate}

\begin{table}[h!]
\centering
\begin{threeparttable}
\caption{Result of the data on the Pedagogical Knowledge (PK)}
\label{tab4}
\begin{tabular}{p{5cm}lllllll}
\toprule
Question Variables & SA & A & N & SD & D & Mean & Std. Deviation \\ 
\midrule
I integrate technology in a way that aligns with my pedagogical approach & 17.87 & 16.46 & 5.66 & 31.83 & 28.18 & 2.97 & 1.57 \\
I use digital materials in evaluating students’ understanding of the concept taught. & 12.17 & 13.59 & 7.09 & 36.17 & 30.98 & 2.43 & 1.89 \\
Technological methods should replace traditional methods of teaching foreign language? & 12.28 & 10.86 & 2.63 & 40.29 & 33.94 & 2.31 & 1.91 \\
\midrule
\end{tabular}
\end{threeparttable}
\end{table}

The above table reflects on the effectiveness of the integration of technology in language teaching through the evaluation of how language lecturers align pedagogical principles with the technology they use in teaching language. The findings of the table are summarized below:

\begin{enumerate}
    \item In the first question, more than 60\% of the language lecturers used for the study affirmed that they do not use technology in a way that aligns with my pedagogical approach, while 33\% of the participants rejected this claim. This finding suggests that despite the availability of the latest technological tools which suit different pedagogical methods, university lecturers only make use of the ones at their disposal;
    \item The second question in the above table is highly important to the integration of technology in teaching. It shows that even though that language lecturers use technological tools to teach, it is necessary for them to use these tools to assess students’ understanding of a concept that has been taught. While 24\% of the participant affirmed to this idea, 69\% of them refuted this claim;
    \item The response to the third question is an indication that some lecturers prefer that the technology method of language teaching exists with the traditional method. Out of the 300 participants, more than 50\% confirm that the technological method of foreign language teaching should not completely replace the traditional method.
\end{enumerate}

\begin{table}[h!]
\caption{Result of the data on the Content Knowledge}
\label{tab5}
\begin{tabular}{|p{5cm}|l|l|l|l|l|l|l|}
\hline
Question Variables & SA & A & N & SD & D & Mean & Std. Deviation \\ \hline
I align technology with the learning objectives and content of the foreign language course. & 33.69 & 31.75 & 4.83 & 16.28 & 13.45 & 4.05 & 1.14 \\ \hline
I consider students’ needs, interests, and learning styles when selecting and using technology & 13.38 & 12.95 & 6.28 & 34.64 & 32.75 & 2.63 & 1.79 \\ \hline
Students show positive attitude when the lecturer aligns technology with the content of the language course & 43.59 & 33.36 & 2.25 & 12.88 & 7.92 & 4.95 & 0.94 \\ \hline
\end{tabular}
\end{table}

The above table reveals the effectiveness of the integration of technology by university lecturers through the evaluation of how these language lecturers align technology with the learning objectives and content of the language course. The findings are summarized as this:

\begin{enumerate}
    \item The responses from the first question show that university lecturers who make use of technology to teach, integrate appropriate technology with the appropriate content they want to teach. From the table above, over 63\% of the participants affirm that they align technology with the learning objectives and content of the language course, while 30\% of the participants refuted this claim. This conclusion has the consequence that language lecturers need to be aware of the particular technologies that are most effective for addressing subject-matter learning in their domains and how the technology may be changed or dictated by the content, or vice versa;
    \item The second question addresses how university lecturers select the technological tools they use in teaching foreign language. From the responses, more than 63\% of the participants accepted the claim that they do not put into consideration students’ interests while selecting technology to be used in teaching a subject matter. Just over 25\% accepted the claim that they consider students’ needs, interests, and learning styles when selecting and using technology.
    \item Lastly, a greater percentage of university lecturers affirmed that students show a positive attitude when technology is used to teach them a foreign language. Over 77\% of the participants align with this idea while about 20\% refuted this idea. This finding suggests that as the language lecturer is integrated, the interest and needs of students should be prioritized.
\end{enumerate}

%\begin{enumerate}[label={\Alph*.},resume=questions]
%    \item \textbf{Discussions and Implications}
%\end{enumerate}
\subsection{Discussions and Implications}

The analysis has presented different views of the lecturers drawn across different universities and also analyzed the findings. It is pertinent to further reiterate what has been established in the presented result. The lecturers disclosed their utilization of technology in language instruction, and the findings revealed that a significant proportion of them refrain from incorporating technology in their teaching practices. More than half of the participants assert that they do not utilize technology-based instruments in their pedagogical practices, whereas about a quarter of them acknowledge employing such tools. This demonstrates that despite the efficacy of utilizing technological resources for pedagogical purposes, there exist underlying impediments that may hinder its assimilation within the academic model.

Furthermore, certain language lecturers have attested to the infrequency of their training on the appropriate incorporation of said tools during their teaching practices. According to the data, 35\% of the participants acknowledged having received instruction on the implementation of technology in language instruction, whereas 55\% of the respondents denied having undergone such training. The promotion of technology-based language pedagogy has been advocated for; however, it is imperative to acknowledge that the successful integration of technology in language pedagogy necessitates adequate provision of training to equip educators with the skills to effectively utilize technological tools. Approximately 58\% of the participants demonstrated a rejection of the notion that technological tools should be exclusively utilized in the instruction of foreign language, whereas 38\% of the participants exhibited acceptance of this concept. Lack of adequate training on the usage of technological tools poses a significant challenge to lecturers in their teaching endeavors. Approximately 62\% of the participants reported encountering challenges in utilizing these tools, whereas 32\% of the participants refuted this assertion.

Moreover, a significant proportion of the respondents acknowledged their lack of familiarity with the various technologies employed in language instruction. Approximately 29\% of the participants acknowledged possessing knowledge and proficiency in operating the aforementioned tools, whereas 69\% of the participants contested this notion. This finding suggests that a relatively small proportion of foreign language lecturers are consistently keeping up with the latest technological advancements in language pedagogy.

The lecturers have also confirmed that technology serves as a complement to the conventional teaching methods and tools employed in foreign language instruction. Over half of the participants affirmed that technology serves as a complementary tool to conventional language pedagogy, whereas 22\% of the participants contradicted this assertion. The findings of the study are in agreement with the findings of other studies, including the research conducted by \textcite{abdulrahaman_multimedia_2020}, \textcite{ahmadi_use_2018}, \textcite{hartono_language:_2021}, \textcite{lyublinskaya_analysis_2022}, among others.



\section{Conclusion}

According to the results of the research, there seems to be an upsurge in the variety of investigations that are being undertaken on the topic of the incorporation of digital tools in foreign language lecture halls.  It is possible to draw the conclusion from this growth that more technological tools are employed by university lecturers in teaching foreign languages. The results of the survey reveal, in addition, that lecturers do not routinely use a variety of digital tools in the lecture halls to teach a foreign language; nonetheless, lecturers admit that the use of digital tools will enhance the teaching and learning of a foreign language in lecture halls. 

Further studies are recommended to explore this integration of technology in foreign language teaching with a larger study population. It is suggested that future research can be carried out using a variety of various sample groups that are much bigger and diverse, using more sophisticated statistical measures. Furthermore, there is a need to conduct experimental studies on the integration of technological tools in the teaching of foreign languages. Experimental studies with experimental groups and control groups may unveil how these technological tools enhance the teaching and learning of foreign languages.


\printbibliography\label{sec-bib}
% if the text is not in Portuguese, it might be necessary to use the code below instead to print the correct ABNT abbreviations [s.n.], [s.l.]
%\begin{portuguese}
%\printbibliography[title={Bibliography}]
%\end{portuguese}


%full list: conceptualization,datacuration,formalanalysis,funding,investigation,methodology,projadm,resources,software,supervision,validation,visualization,writing,review
\begin{contributors}[sec-contributors]
\authorcontribution{Yousef Methkal Abd Algani}[conceptualization,datacuration,formalanalysis,investigation,methodology,writing,review]
\authorcontribution{Zehavit Gross}[formalanalysis,supervision]
\end{contributors}


\end{document}

