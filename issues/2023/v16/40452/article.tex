% !TEX TS-program = XeLaTeX
% use the following command:
% all document files must be coded in UTF-8
\documentclass[spanish]{textolivre}
% build HTML with: make4ht -e build.lua -c textolivre.cfg -x -u article "fn-in,svg,pic-align"

\journalname{Texto Livre}
\thevolume{16}
%\thenumber{1} % old template
\theyear{2023}
\receiveddate{\DTMdisplaydate{2022}{7}{13}{-1}} % YYYY MM DD
\accepteddate{\DTMdisplaydate{2022}{9}{8}{-1}}
\publisheddate{\DTMdisplaydate{2022}{11}{28}{-1}}
\corrauthor{Álvaro Pérez García}
\articledoi{10.1590/1983-3652.2023.40452}
%\articleid{NNNN} % if the article ID is not the last 5 numbers of its DOI, provide it using \articleid{} commmand 
% list of available sesscions in the journal: articles, dossier, reports, essays, reviews, interviews, editorial
\articlesessionname{dossier}
\runningauthor{Pérez García y Sacaluga Rodríguez} 
%\editorname{Leonardo Araújo} % old template
\sectioneditorname{Hugo Heredia Ponce}
\layouteditorname{Daniervelin Pereira}

\title{El \textit{storytelling} como recurso didáctico-comunicativo para fomentar la lectura}
\othertitle{A contação de histórias como recurso didático-comunicativo para promover a leitura}
\othertitle{Storytelling as a didactic-communicative resource to encourage reading}
% if there is a third language title, add here:
%\othertitle{Artikelvorlage zur Einreichung beim Texto Livre Journal}

\author[1]{Álvaro Pérez García~\orcid{0000-0001-9624-5202}\thanks{Email: \href{mailto:alvaro.perezgarcia@unir.net}{alvaro.perezgarcia@unir.net}}}
\author[2]{Ignacio Sacaluga Rodríguez~\orcid{0000-0002-2923-819X}\thanks{Email: \href{mailto:ignacio.sacaluga@universidadeuropea.es}{ignacio.sacaluga@universidadeuropea.es}}}
\affil[1]{Universidad Internacional de La Rioja, Facultad de Educación,Departamento de Didáctica y Organización Escolar, Logroño, España.}
\affil[2]{Universidad Europea de Madrid, Facultad de Ciencias Sociales y de la Comunicación, Departamento de Comunicación Audiovisual y Publicidad, Madrid, España.}

\addbibresource{article.bib}
% use biber instead of bibtex
% $ biber article

% used to create dummy text for the template file
\definecolor{dark-gray}{gray}{0.35} % color used to display dummy texts
\usepackage{lipsum}
\SetLipsumParListSurrounders{\colorlet{oldcolor}{.}\color{dark-gray}}{\color{oldcolor}}

% used here only to provide the XeLaTeX and BibTeX logos
\usepackage{hologo}

% if you use multirows in a table, include the multirow package
\usepackage{multirow}

% provides sidewaysfigure environment
\usepackage{rotating}

% CUSTOM EPIGRAPH - BEGIN 
%%% https://tex.stackexchange.com/questions/193178/specific-epigraph-style
\usepackage{epigraph}
\renewcommand\textflush{flushright}
\makeatletter
\newlength\epitextskip
\pretocmd{\@epitext}{\em}{}{}
\apptocmd{\@epitext}{\em}{}{}
\patchcmd{\epigraph}{\@epitext{#1}\\}{\@epitext{#1}\\[\epitextskip]}{}{}
\makeatother
\setlength\epigraphrule{0pt}
\setlength\epitextskip{0.5ex}
\setlength\epigraphwidth{.7\textwidth}
% CUSTOM EPIGRAPH - END

% LANGUAGE - BEGIN
% ARABIC
% for languages that use special fonts, you must provide the typeface that will be used
% \setotherlanguage{arabic}
% \newfontfamily\arabicfont[Script=Arabic]{Amiri}
% \newfontfamily\arabicfontsf[Script=Arabic]{Amiri}
% \newfontfamily\arabicfonttt[Script=Arabic]{Amiri}
%
% in the article, to add arabic text use: \textlang{arabic}{ ... }
%
% RUSSIAN
% for russian text we also need to define fonts with support for Cyrillic script
% \usepackage{fontspec}
% \setotherlanguage{russian}
% \newfontfamily\cyrillicfont{Times New Roman}
% \newfontfamily\cyrillicfontsf{Times New Roman}[Script=Cyrillic]
% \newfontfamily\cyrillicfonttt{Times New Roman}[Script=Cyrillic]
%
% in the text use \begin{russian} ... \end{russian}
% LANGUAGE - END

% EMOJIS - BEGIN
% to use emoticons in your manuscript
% https://stackoverflow.com/questions/190145/how-to-insert-emoticons-in-latex/57076064
% using font Symbola, which has full support
% the font may be downloaded at:
% https://dn-works.com/ufas/
% add to preamble:
% \newfontfamily\Symbola{Symbola}
% in the text use:
% {\Symbola }
% EMOJIS - END

% LABEL REFERENCE TO DESCRIPTIVE LIST - BEGIN
% reference itens in a descriptive list using their labels instead of numbers
% insert the code below in the preambule:
%\makeatletter
%\let\orgdescriptionlabel\descriptionlabel
%\renewcommand*{\descriptionlabel}[1]{%
%  \let\orglabel\label
%  \let\label\@gobble
%  \phantomsection
%  \edef\@currentlabel{#1\unskip}%
%  \let\label\orglabel
%  \orgdescriptionlabel{#1}%
%}
%\makeatother
%
% in your document, use as illustraded here:
%\begin{description}
%  \item[first\label{itm1}] this is only an example;
%  % ...  add more items
%\end{description}
% LABEL REFERENCE TO DESCRIPTIVE LIST - END


% add line numbers for submission
%\usepackage{lineno}
%\linenumbers

\begin{document}
\maketitle

\begin{polyabstract}
\begin{abstract}
La importancia de la lectura en niños y adolescentes como actividad que les permite estimular fácilmente su imaginación y su función cerebral es, en la actualidad, un caballo de batalla tanto en la escuela como en la familia. Buscar recursos y herramientas motivadoras para fomentar el hábito lector es fundamental, y el \textit{storytelling} es un interesante recurso. Nuestro objetivo es analizar las experiencias y buenas prácticas de utilización del \textit{storytelling} para fomentar el hábito lector en niños y adolescentes que se están llevando a cabo en España y otros países, para valorar su implantación como estrategia didáctica. Se utilizó una metodología metodología cualitativa, basada en el método analítico-descriptivo. A pesar de que el \textit{storytelling} es un recurso cada vez más utilizado en el ámbito educativo, aún son escasas las experiencias encontradas sobre su uso para fomentar la lectura en niños y adolescentes. Concluimos que el \textit{storytelling} puede ser un recurso didáctico y creativo muy motivador para fomentar el hábito de la lectura, tanto en niños como en adolescentes, dado su componente multimedia y su capacidad de contar historias.

\keywords{Storytelling \sep Recurso didáctico-comunicativo \sep Lectura \sep Niños \sep Adolescentes}
\end{abstract}

\begin{portuguese}
\begin{abstract}
A importância da leitura nas crianças e adolescentes como uma atividade que lhes permite estimular facilmente a sua imaginação e o seu funcionamento cerebral é atualmente um cavalo de batalha tanto na escola como na família. A busca por recursos e ferramentas motivadoras para fomentar o hábito da leitura é essencial, e a contação de histórias é um recurso interessante. Nosso objetivo é analisar as experiências e boas práticas de uso da contação de histórias para promover o hábito da leitura em crianças e adolescentes que estão sendo realizadas na Espanha e em outros países, para avaliar sua implementação como estratégia didática. Foi utilizada uma metodologia qualitativa, baseada no método analítico-descritivo. Embora a contação de histórias seja um recurso cada vez mais utilizado no campo educacional, ainda são poucas as experiências encontradas sobre sua utilização para o incentivo à leitura em crianças e adolescentes. Concluímos que a contação de histórias pode ser um recurso didático e criativo muito motivador para incentivar o hábito da leitura, tanto em crianças como em adolescentes, dada a sua componente multimédia e a sua capacidade de contar histórias.

\keywords{Contação de histórias \sep Recurso didático-comunicativo \sep Leitura \sep Crianças \sep Adolescentes}
\end{abstract}
\end{portuguese}

\begin{english}
\begin{abstract}
The importance of reading in children and adolescents as an activity that allows them to easily stimulate their imagination and their brain function is, at present, a workhorse both at school and in the family. Finding resources and motivational tools to encourage the reading habit is fundamental, and storytelling is an interesting resource. Our goal is to analyse the experiences and good practices in the use of storytelling to encourage the reading habit in children and adolescents that are being carried out in Spain, in order to make a didactic proposal. A qualitative methodology was used, based on the analytical-descriptive method. Despite the fact that storytelling is a resource that is increasingly used in the educational field, there are still few experiences of its use to encourage reading among children and adolescents. We concluded that storytelling can be a very motivating didactic and creative resource to encourage the habit of reading, both in children and adolescents, given its multimedia component and its ability to tell stories.

\keywords{Storytelling \sep Didactic-communicative resource \sep Reading \sep Children \sep Adolescents}
\end{abstract}
\end{english}
% if there is another abstract, insert it here using the same scheme
\end{polyabstract}

\section{Introducción}\label{sec-intro}
Existe un interés creciente por la generación y aplicación de nuevas metodologías que contribuyan al desarrollo de las competencias educativas que la sociedad del conocimiento, en el contexto actual, exige. Las nuevas alfabetizaciones deben estar orientadas a unas necesidades formativas que permitan a los individuos desenvolverse con destreza en los nuevos entornos complejos y cambiantes de la globalización. Así, se antoja imprescindible “saber lidiar con nuevas situaciones, nuevas tecnologías, interpretar y presentar la información de forma acertada” \cite[p.~163-164]{rosales_statkus_relato_2017}. 

Para ello, se constata la utilidad del \textit{storytelling} como un recurso didáctico y comunicativo en servicio de la transmisión de valores y conocimientos, ya sea en entornos de aprendizaje reglados como no reglados. En este sentido, tal y como apuntan \textcite{gregori-signes_digital_2012,robin_digital_2008}, se dispone también como una herramienta eficaz para la generación de las competencias del siglo XXI.

La capacidad de contar historias, además de acompañar a los humanos casi desde el origen de su existencia, ha permitido dotar de sentido la propia experiencia vital y la noción de la realidad, muy probablemente porque existe una necesidad comunicativa que impacta también en el propio concepto del relato \cite{ricoeur_tiempo_1987}. Por tanto el \textit{storytelling} permite que, mediante las TIC, se siga promoviendo uno de los modos de comunicación más antiguos y útiles, facilitando la capacidad de recordar los mensajes que recibe un receptor y uno de los medios de persuasión más efectivos que se conocen \cite{delgado-ballester_once_2016}.

Así, en el ámbito educativo, las posibilidades y las oportunidades derivadas del \textit{storytelling} en el contexto del aula son incontables. De hecho, ha sido objeto de estudio de multitud de investigadores, tanto en el contexto internacional \cite{abrahamson_storytelling_1998,coulter_storytelling_2007,ferreira_motivating_2021} como en el nacional \cite{illera_relatos_2009,villalustre_martinez_digital_2014}.

Si se focaliza la utilización del \textit{storytelling} en el fomento de la lectura en niños y adolescentes, autores como \textcite{ferreira_motivating_2021} apuestan por la utilidad de los aprendizajes inmersivos como contextos metodológicos útiles para promover y afianzar las habilidades lectoras en los estudiantes. En este sentido, la motivación y creatividad que lleva asociadas la utilización del \textit{storytelling}, va a favorecer la predisposición del alumnado hacia la lectura, gracias al componente narrativo y multimedia, que lo hace muy atractivo.

\section{Aproximación conceptual al \textit{storytelling}}\label{sec-normas}
El \textit{storytelling} es entendido, en su concepción generalista, como el arte de contar historias o relatos \cite{lopez-hermida_nuevas_2013}. Otros autores han procurado dotar el significado del término de una mayor precisión, redimensionándolo como “una herramienta de comunicación estructurada en una secuencia de acontecimientos que apelan a nuestros sentidos y emociones” \cite[p. 17]{nunez_sera_2007}. \textcite{lopez-hermida_nuevas_2013} aluden a los elementos o recursos en los que se sustenta el \textit{storytelling}: palabras, imágenes y sonidos; pero inciden también en que, precisamente a merced de esos recursos, se ha conformado como un colaborador necesario en la construcción de la historia de la humanidad, con un papel preponderante en la evolución de esta \cite{fog_storytelling_2010}. Además, \textcite{garcia_montero_aportaciones_2009} le otorga cierto protagonismo en el mérito de encontrar sentido a sus propias acciones y a la vida en su sentido más amplio.

\textcite{perez_storytelling_2013} propone que el relato de sostenerse en elementos cuidadosamente combinados, a saber: el ethos (credibilidad), el phatos (emoción) y el logos (lógica). La suma, estratégicamente ponderada de la emoción, la lógica y la credibilidad facilita la penetración eficaz del mensaje en los receptores, pues “una historia es una verdad universal cargada de emociones y sensaciones tiene mucho más poder que un montón de argumentos y mucha más persuasión que un sinnúmero de datos” \cite[p. 78]{seguel_ramos_brandstory:_2014}. 

Esta realidad, para \textcite{keysers_empathic_2011}, tiene su fundamento en las multifuncionales neuronas espejo y su importante capacidad asociativa al conectar el sistema perceptivo con el sistema motor, el sistema emotivo y el sistema cognitivo. Este es el sistema por el cual los humanos han ejecutado y reforzado los aprendizajes básicos y complejos de la propia evolución humana.

Para \textcite[p. 59]{ferres_communication_2017}, en el aprendizaje “era similar la experiencia del adolescente que acompañaba a los adultos a buscar alimento que la del niño que por la noche escuchaba en torno al fuego los relatos de estas aventuras”. En ambos casos la multilateralidad de la reacción del receptor, desde su prisma no solo racional, sino también perceptivo, emocional y motor, se entiende determinante en el aprendizaje experiencial y la propia cognición del mensaje. De este modo, el \textit{storytelling} adquiere una mayor relevancia como recurso comunicativo multifuncional, al basarse en la creación y narración de relatos, orientados a la consecución de un objetivo un objetivo específico. Para \textcite[p. 12]{ramos_villagrasa_storytelling:_2019}, su principal ventaja radica en que la narración y su funcionamiento genera la cercanía de todas las personas en torno al relato, además, facilita la asimilación del aprendizaje derivado de las propias historias.

\textcite[p. 18]{nunez_sera_2007}, advierte en el relato, “algo de sagrado, porque es una verdad universal cargada de emociones y de sensaciones”. No de extrañar que, con estas características operativas, el \textit{storytelling} se haya consolidado como herramienta comunicativa con una gran capacidad de influencia. Para \textcite[p. 1060]{gill_building_2009}, le otorga cierta universalidad, pues el relato genera impacto en público objetivo con independencia de su origen, edad, género o cultura. 

Por tanto, el término \textit{storytelling} engloba necesariamente componentes informativos y emocionales, ambos necesarios y ambos responsables de que el mensaje cale con mayor intensidad en los receptores, propiciando una aprehensión profunda del sentido del relato. Precisamente, por ese nivel de consciencia en el que impacta el \textit{storytelling}, las historias, como señalan \textcite[p. 127]{atamara-rojas_storytelling_2020}, pueden conllevar, y a menudo conllevan, una estrategia de comunicación con finalidad persuasiva. De este modo, el \textit{storytelling} se caracteriza por su multifuncionalidad, pues encuentra un terreno fértil en ámbitos tan dispares como el publicitario, el comunicativo, el empresarial o el educativo. 

\section{El \textit{storytelling} como herramienta comunicativa de utilidad plurifuncional}\label{sec-conduta}
Aunque es cierto que la técnica de contar historias tiene, como se ha señalado con anterioridad, orígenes remotos; no es menos cierto que el \textit{storytelling} ha experimentado durante las últimas décadas cierto resurgir. Ámbitos como el periodístico, la política, la publicidad, e incluso el propio contexto empresarial, se han visto seducidos por técnicas y estrategias basadas en el \textit{storytelling} para la obtención de objetivos de distinta índole \cite{salmon_storytelling:_2008}. Además, en plena sociedad del conocimiento, los dispositivos móviles han sumido el rol de instrumentos imprescindibles para crear y consumir contenidos digitales. \cite{jenkins_transmedia_2003,pratten_getting_2011,scolari_narrativas_2013,galindo_rubio_alisis_2015}.

Precisamente por ese marcado carácter plurifuncional del \textit{storytelling}, se antoja relevante para esta investigación, observar de manera disgregada el impacto de este modo de narración en los diferentes ámbitos.


\subsection{Contexto comunicacional y empresarial }\label{sec-fmt-manuscrito}
El papel que se ha reservado a los medios de comunicación, como creadores de historias destinadas a un público masivo, se aplica a toda persona u organización capaz de crear \textit{storytelling} y \textit{digital storytelling} de manera única o seriada y en cualquier formato (textual, sonoro, audiovisual o de cualquier otra índole), a través de cualquier soporte y plataforma; ya sea con autoría individual o colaborativa; y con ramificaciones argumentales más o menos complejas \cite{perez_garcia_digital_2020}.

No obstante, en comunicación social conviene tener en cuenta la complejidad que implica una comunicación fluida, los distintos actores y elementos de cualquier proceso básico de comunicación, exige un armónico ordenamiento que garantice la eficacia del mensaje.

En este sentido, como apuntan \textcite{ferres_communication_2017}, no debería extrañar que el \textit{storytelling} se haya catapultado como una forma de comunicación casi hegemónica en todos los ámbitos de la comunicación persuasiva, al menos en aquellos en los que resulte “imprescindible crear demanda: desde la comunicación publicitaria hasta la política, pasando por el liderazgo, la economía, el derecho, el management y la gestión empresarial” \cite[p. 59]{ferres_communication_2017}. Así las cosas, el \textit{storytelling} contiene innumerables ventajas que le permiten generar un mayor impacto cuando se transforma en un formato publicitario, precisamente por su capacidad de interpelar a la emoción durante el proceso comunicativo con la audiencia \cite{Herrero_claves_2015}.

No obstante, no deben valorarse los beneficios del \textit{storytelling}, tan solo en términos de la demanda derivada del consumo de bienes y servicios, sino también en cuanto a los intangibles que componen el valor de una empresa.

El reconocimiento del valor de las historias en el ámbito empresarial está aceptado por la comunidad científica y profesional, así que no es necesario redundar en las historias que una empresa cuenta, sino que corresponde reflexionar acerca de la historia que la empresa es. Esto podría traer grandes ventajas y aplicaciones para el mundo empresarial. A partir del análisis de cómo se aplican las historias en el mundo corporativo se puede proponer una clasificación de las dimensiones en las que se puede desarrollar el \textit{storytelling} corporativo \cite[p. 197]{atamara-rojas_storytelling_2020}.

\textcite{green_storytelling_2004} se refiere a las historias y a su versatilidad para proyectar para conceptos complejos, algo que genera una experiencia didáctica mucho más amplia. \textcite{lugmayr_serious_2017} se adhieren a este razonamiento cuando aseguran que con el uso de la narratividad es más sencillo llegar a la emoción del lector o del espectador. Al tiempo, \textcite{reinsborough_re:_2017} apuestan por la capacidad del \textit{storytelling} vinculado al concepto de cambio social. Por su parte, \textcite{salmon_storytelling:_2008} advierte también de sus riesgos refiriéndose al contexto de la política y lo habitual que ha resultado su uso casi desde tiempos inmemoriales. \textcite{vizcaino_alcantud_storytelling_2017} subraya el uso instrumental del \textit{storytelling} en función de la naturaleza de los intereses a los que responde.

\subsection{Contexto educativo}\label{sec-formato}
La competencia comunicativa, entendida como la capacidad de expresarse y comunicarse de manera eficaz en una lengua específica, con independencia de los diferentes contextos, adquiere una enorme relevancia, no solo en el contexto educativo, sino también para cualquier ámbito de la vida. Así se recoge entre las competencias básicas propuestas en 2006 por el Consejo y el Parlamento Europeo y, en España, recogida en la Orden ECD/65/2015, de 21 de enero \cite{parlamento_europeo_y_consejo_recomendacion_2006,ministerio_de_educacion_cultura_y_deporte_orden_2015}. En dicha orden se alude a la vertebración de las competencias, los criterios de evaluación y los contenidos en los distintos ciclos del sistema educativo. 

La competencia digital es un instrumento de gran utilidad para, como señalan \textcite{hatlevik_digital_2013}, para el procesamiento de la información y su conversión en conocimiento. Por su lado, \textcite{krumsvik_teacher_2014}, la asocia también a la habilidad en el manejo de los lenguajes, las técnicas y las estrategias en los distintos soportes, para hacer un uso responsable de las mismas y generar el espíritu crítico necesario para la resolución de problemas \cite{calvani_digital_2010}. 

La importancia de la competencia digital ha sido vista por los órganos competentes en la materia tanto a nivel nacional como internacional, y, en esa línea, trabaja en nuestro país el Instituto Nacional de Tecnologías Educativas y de Formación del Profesorado \cite{intef_marco_2017} (\url{https://intef.es/}) que, en el año 2012, creó el proyecto Marco Común de Competencia Digital Docente, buscando ser y trabajar para ser una referencia descriptiva que pueda usarse para la identificación de necesidades de formación, evaluación y acreditación de las competencias digitales del profesorado en colaboración con las diferentes Comunidades Autónomas. Posteriormente, en 2017 aparece una nueva versión después de una revisión en profundidad, y queda organizado en cinco áreas competenciales (información y alfabetización informacional, comunicación y colaboración, creación de contenido digital, seguridad y resolución de problemas, que, a su vez, se subdividen en 21 competencias, que refuerzan la propuesta del Marco Europeo para la Competencia Digital de los Educadores (DigCompEdu), que se divide en seis áreas competenciales y subdivide en 22 competencias, centrándose en el uso didáctico y creativo de la Tecnología. 

Igualmente, la Organización de las Naciones Unidas para la Educación, la Ciencia y la Cultura (UNESCO), viene trabajando en el Marco de Competencias de los Docentes en materia de TIC de la UNESCO (ICT-CFT), cuyo objetivo es asesorar a los diferentes países a desarrollar reglamentos propios para cada país que subrayen la importancia de la competencia digital en los docentes.

Por tanto, y para su mejor aprovechamiento, la presencia del \textit{storytelling} en el ámbito educativo viene de lejos. En ese sentido ha existido un interés por parte de educadores y comunidad educativa por generar y explorar diferentes metodologías de aprendizaje activo, siendo el \textit{storytelling} una de las que mayor interés ha generado en los últimos años \cite{flanagan_how_2015}. Por su lado, Ramos-Villagrasa et al. comprobaron en un interesante estudio si, por un lado, esta aplicación resulta útil en los procesos activos de aprendizaje, concretamente en ámbito de la psicología social. Por otro, si los estudiantes la percibían como una metodología más útil que otras, “en comparación con aproximaciones convencionales” \cite[p. 12]{ramos_villagrasa_storytelling:_2019}. El resultado del citado estudio revela claramente la utilidad diferencial del \textit{storytelling}.

Al igual que existen evidencias de la eficacia del \textit{storytelling} en el sistema educativo \cite{bautista-garcia-vera_audiovisual_2009}, otros autores \cite{robin_digital_2008,lambert_digital_2013} ponen el foco en la perspectiva digital del \textit{storytelling}, que trazan narrativas multimedia y transmedia en el aula, ya sea a través del uso de fotografías \cite{sadik_digital_2008}, de contenido audiovisual \cite{ohler_world_2006} e incluso, aprovechando los beneficios expositivos de la web colaborativa, con la utilización de blogs \cite{abdel-hack_using_2014}. Por tanto, es patente las numerosas ventajas que ofrecen las TIC en la interactuación con el alumnado, no solo para evaluarlo, sino también para implicarlo en el aprendizaje \cite{dreon_digital_2011}.

Pero en este trabajo nos vamos a centrar en el fomento de la lectura a través de la utilización didáctico-creativa del \textit{storytelling}. A continuación, se muestran una serie de experiencias y buenas prácticas realizadas, que nos marcan el camino para trabajar en este campo.


\section{Objetivos}\label{sec-modelo}
El objetivo general de este trabajo es analizar las experiencias y buenas prácticas de utilización del \textit{storytelling} para fomentar el hábito lector en niños y adolescentes que se están llevando a cabo en España y otros países, para valorar su implantación como estrategia didáctica.

Este objetivo general se descompondrá en otros objetivos de carácter más específico:

\begin{itemize}
    \item Investigar aquellas experiencias y buenas prácticas que estén publicadas acerca del uso del \textit{storytelling} para fomentar la lectura.
    \item Elaborar un repositorio con aquellas experiencias nacionales e internacionales de utilización del \textit{storytelling} en el fomento de la lectura. 
\end{itemize}

\section{Metodología}\label{sec-organizacao}
Se utilizará una metodología cualitativa, basada en el método analítico-descriptivo, que según \textcite[p. 198]{abreu_metodo_2014} “busca un conocimiento inicial de la realidad que se produce de la observación directa del investigador y del conocimiento que se obtiene mediante la lectura o estudio de las informaciones aportadas por otros autores”. En este caso nos vamos a centrar en el análisis de experiencias y la evaluación de programas de formación que encontramos tras una revisión bibliográfica exhaustiva. Este método se descompone en las siguientes fases \cite{gomez-luna_metodologipara_2014}:  definición del problema o necesidad, búsqueda de la información, organización de la información y análisis de la información.

En el caso de nuestra investigación, partimos de la necesidad de buscar, analizar y organizar aquellas experiencias e investigaciones que haya sobre el fomento del hábito lector en niños y adolescentes a través del \textit{storytelling} como recurso didáctico-creativo. Para ello, se ha realizado una revisión bibliográfica exhaustiva, comenzando por aquellas publicaciones que están recogidas en las bases de datos de la Web of Science y Scopus. Tras realizar una primera búsqueda y comprobar la escasez de artículos referidos a esta temática, se amplió la búsqueda a publicaciones que estuvieran indexadas en Latindex.

Tras el análisis, han sido 11 las experiencias seleccionadas, estableciendo dos categorías: aquellas experiencias que utilizan el \textit{storytelling} para el fomento de la lectura en general y aquellas que exponen la utilización del \textit{storytelling} para facilitar la adquisición de destrezas en lectura y escritura de lenguas diferentes a la natal. 

\section{Experiencias y buenas prácticas de utilización del \textit{storytelling} para el fomento de la lectura}\label{sec-organizacao-latex}
A pesar de que el \textit{storytelling} es un recurso cada vez más utilizado en el ámbito educativo, aún son escasas las experiencias encontradas sobre su uso para fomentar la lectura en niños y adolescentes. En ese sentido, se ha querido destacar algunas de estas experiencias por su diseño innovador y los resultados obtenidos en su implementación. Se han dividido en dos categorías: aquellas experiencias que utilizan el \textit{storytelling} para el fomento de la lectura en general y aquellas que exponen la utilización del \textit{storytelling} para facilitar la adquisición de destrezas en lectura y escritura de lenguas diferentes a la natal.

Con respecto a la primera de las categorías, aquellas experiencias que utilizan el \textit{storytelling} para el fomento de la lectura en general:

Con respecto a las más actuales, destaca el proyecto TRANSFORMA: estrategia didáctica, basada en \textit{storytelling} digital para el fortalecimiento de la lectura crítica en estudiantes de Educación Superior \cite{bonillapardotrasforma:2021}, llevado a cabo en la Universidad Autónoma de Occidente de Cali (Colombia) y que tenía como objetivo realizar una investigación para determinar si la implementación de una estrategia didáctica centrada en el \textit{storytelling} podía favorecer la capacidad crítica del alumnado universitario a la hora de afrontar la lectura. Además, con su implementación “se procura que los estudiantes mejoren sus niveles de lectura crítica, se conviertan en autores de sus propios contenidos, y asuman un rol activo, comunicativo, reflexivo y productivo en su proceso de enseñanza-aprendizaje” \cite[p. 1]{bonillapardotrasforma:2021}.

\textcite{barwasser_storytelling_2021} utilizaban el \textit{storytelling} para mejorar el vocabulario, la lectura y la fluidez de los sonidos de las letras en alumnado de primer grado con dificultades para aprender alemán como segunda lengua.

Otro proyecto destacado es el de \textcite{ferreira_motivating_2021}, que realizaron una investigación basada en la metodología de estudio de casos. En ella estudiaron al alumnado de una institución portuguesa de la región del Médio Tajo, en el que investigaban si la realización de actividades educativas en contextos de enseñanza inmersivos, cómo el transmedia \textit{storytelling} podía incentivar el diseño e implementación de propuestas que mejoren las habilidades lectoras en las instituciones educativas. Los datos obtenidos en el estudio indican el gran potencial del uso de rutas transmedia en el contexto de la lectura, gracias a la combinación de experiencias diferentes, utilizando distintos medios, lo que despertó el interés y el compromiso de los estudiantes que participaron en el estudio. También se concluía que complementar la metodología existente con la integración de narrativas transmedia puede aportar numerosas ventajas, algo que se pudo comprobar a lo largo de las sesiones, especialmente cuando el alumnado leía la historia en el libro impreso y utilizaban la tableta para leer los códigos QR. Además, el profesorado participante destacó que, al utilizar estas herramientas, el alumnado se sentía más motivado e interesado en lo que estaba leyendo.

El desarrollo de habilidades básicas de lectura y escucha en estudiantes de segundo grado a través del \textit{storytelling} \cite{lopez_tangarife_desarrollo_2013} era el objetivo de una investigación realizada en la Universidad de Manizales (Colombia), que concluía que la utilización del \textit{storytelling} para desarrollar la lectura tenía grandes beneficios como estimular la creatividad y la imaginación, ampliar el vocabulario del alumnado, pronunciar de forma adecuada o trabajar las cuatro habilidades comunicativas escuchar, hablar, leer y escribir, entre otras.

El estudio de \textcite{satriani_storytelling_2019} en Indonesia, tenía como objetivo descubrir los beneficios y los desafíos de la aplicación del \textit{storytelling} en la enseñanza de la alfabetización, concretamente en el desarrollo de la capacidad de lectoescritura del alumnado, despertar su imaginación, enriquecer su vocabulario y desarrollar su conciencia e interés por la lectura. 

En Brasil, \textcite{prieto_propostas_2018} destacaban en un estudio que las actividades de lectura y \textit{storytelling} se enfatizan como motivadoras del aprendizaje que promueven el desarrollo humano, para contribuir a la mejora y sofisticación de las habilidades psicológicas en los niveles superiores.

En Irán, \textcite{rahimi_effects_2017} investigaron los efectos del \textit{storytelling} digital offline frente al online en el desarrollo de las habilidades de lectoescritura de los estudiantes de EFL (lectura y escritura). Los resultados del análisis revelaron principalmente que las habilidades de alfabetización lectora de quienes produjeron sus historias con la plataforma en línea mejoraron significativamente en comparación con el grupo de control, que había trabajado con el software fuera de línea.

Con respecto a las experiencias enmarcadas en la segunda categoría, es decir, aquellas que exponen la utilización del \textit{storytelling} para facilitar la adquisición de destrezas en lectura y escritura de lenguas diferentes a la natal, encontramos las siguientes:

Por su parte, \textcite{cedeno_damil_lectura_2019} plantea una interesante experiencia en educación superior para la enseñanza del inglés a través del \textit{Total Physical Response Storytelling} o Método de Enseñanza de Competencias a través de Lecturas y Narración de Cuentos, algo muy novedoso hasta el momento. Los resultados, tras la aplicación de esta modalidad, fueron muy positivos. Los estudiantes desarrollaron capacidades en el uso de la lengua extranjera, en este caso, el inglés. Tras las primeras semanas de la aplicación del método, la interactuación y participación del alumnado creció y el interés fue cada vez mayor, para conocer más sobre este método, siendo, por tanto, una alternativa para solucionar el problema del aprendizaje del inglés en educación superior.

En la misma línea investiga \textcite{arenas_efecto_2018}, cuyo objetivo era comprobar la efectividad del método “\textit{Storytelling and Interactive Reading} (SIR) en el desarrollo de destrezas lingüísticas en inglés”. Para ello, se realizó una investigación con estudiantes de sexto grado de una escuela peruana en la población de Juliaca. El método seguía las siguientes fases: iniciación al vocabulario, creación de un cuento, lectura y tareas de repaso. Se implementaron 12 narrativas de lectura interactiva a las que llamaron “Reading My Best Friend”, que estaban adecuadas a un lenguaje actualizado y al contexto del alumnado participante en este estudio. La investigación mostro que 7 de cada 10 tuvieron mejoras en sus destrezas lingüísticas en inglés, gracias a su capacidad de contar historias breves a través de imágenes, siendo capaces, también, de comprender mensajes hablados, responder a cuestiones sencillas y entender las instrucciones en inglés.

Algunas investigaciones anteriores nos llevan a la de \textcite{gomez_palacio_strategies_2010}, que realizó un estudio en Carolina del Norte (EEUU) para buscar estrategias metodológicas eficientes para mejorar las destrezas del alumnado con respecto a la oralidad en inglés, otorgando especial importancia a la lectura. Entre las principales conclusiones del estudio, señalaba que una estrategia como el \textit{storytelling} puede influir favorablemente en las actitudes de los estudiantes hacia la lectura, cambiar las prácticas de enseñanza, impactar positivamente en los resultados de los exámenes y tener un efecto ventajoso en los presupuestos escolares (\Cref{tab01} y \Cref{tab02}). 

\begin{table}[h!]
\begin{threeparttable}
\caption{Repositorio experiencias y buenas prácticas de utilización del \textit{storytelling} para el fomento de la lectura en general.}
\label{tab01}
\centering
\begin{tabular}{p{2,5cm} p{1cm} p{2cm} p{7cm}}
\toprule 
\textbf{Autor} & \textbf{Año} & \textbf{País} & \textbf{Experiencia}  \\
\midrule
Rahimi \& Yadollahi & 2017 & Irán & Efectos del \textit{storytelling} digital offline frente al online en el desarrollo de las habilidades de lectoescritura de los estudiantes de EFL (lectura y escritura).  \\
Prieto \textit{et al}. & 2018 & Brasil & La narración de historias (\textit{storytelling}) en educación infantil para el desarrollo de la lectura y la motivación hacia el aprendizaje. \\
Satriani & 2019 & Indonesia & Aplicación del \textit{storytelling} en la enseñanza de la alfabetización. \\
López Tangarife & 2020 & Colombia & Desarrollo de habilidades básicas de lectura y escucha en estudiantes de segundo grado a través del \textit{storytelling}. \\
Faria Ferreira et al. & 2021 & Portugal & Realización de actividades educativas en contextos de enseñanza inmersivos, como el Transmedia \textit{Storytelling} podían incentivar el diseño e implementación de propuestas que mejoren las habilidades lectoras en las instituciones educativas. \\
Barwasser et al. & 2021 & Alemania & El \textit{digital storytelling} para mejorar el vocabulario, la lectura y la fluidez de los sonidos de las letras en alumnado de primer grado con dificultades para aprender alemán como segunda lengua. \\
Bonilla Pardo & 2021 & Colombia & Proyecto TRANSFORMA: estrategia didáctica, basada en \textit{storytelling} digital para el fortalecimiento de la lectura crítica en estudiantes de Educación Superior. \\
Syam & 2022 & Indonesia & Utilizar el \textit{digital storytelling} como técnica para ayudar a la capacidad de lectura y escritura de los alumnos. \\
\bottomrule
\end{tabular}
\source{Elaboración Propia.}
\end{threeparttable}
\end{table}

\begin{table}[h!]
\begin{threeparttable}
\caption{Repositorio experiencias y buenas prácticas de utilización del \textit{storytelling} para facilitar la adquisición de destrezas en lectura y escritura de lenguas diferentes a la natal.}
\label{tab02}
\centering
\begin{tabular}{p{2,5cm} p{1cm} p{2cm} p{7cm}}
\toprule 
\textbf{Autor} & \textbf{Año} & \textbf{País} & \textbf{Experiencia}  \\
\midrule
Gómez Palacios & 2010 & EEUU & Utilización del \textit{storytelling} para mejorar las destrezas del alumnado con respecto a la oralidad en inglés. \\
Arenas & 2018 & Perú & “\textit{Storytelling and Interactive Reading} (SIR) en el desarrollo de destrezas lingüísticas en inglés”. \\
Cedeño & 2019 & España & \textit{Total Physical Response Storytelling} o Método de Enseñanza de Competencias a través de Lecturas y Narración de Cuentos. \\
\bottomrule
\end{tabular}
\source{Elaboración Propia.}
\end{threeparttable}
\end{table}

Aunque en este trabajo se han seleccionado solo algunas experiencias, no hay muchas más que recojan las posibilidades didáctico-creativas que el método del \textit{storytelling} posee para fortalecer la lectura en el alumnado de las diferentes etapas educativas.

\section{Conclusiones}\label{sec-idioma}
La utilidad del \textit{storytelling} en los procesos comunicativos ha sido objeto, como se reseña en esta investigación, de especial interés para la comunidad científica y, en consecuencia, de multitud de estudios. Queda acreditado el impacto transversal de componentes informativos y, especialmente, emocionales, en la intensidad de la recepción del mensaje y, sobre todo, en la aprehensión profunda del sentido del relato. Por ello, se detectan algunas características propias del \textit{storytelling} que extienden sus beneficios a los procesos educativos en su conjunto. Nos referimos, no solo a la intencionalidad persuasiva intrínseca al relato, sino también a su multifuncionalidad, pues como se ha expuesto, se hallan conexiones de pertinencia en ámbitos tan dispares como el comunicativo, el empresarial o el educativo. 

Estas ventajas que el \textit{storytelling} ofrece en el ámbito educativo no están siendo totalmente aprovechadas en el fomento de una de las habilidades básicas como es la lectura. Son escasos los estudios realizados sobre esta temática y las experiencias son limitadas, aunque como se puede comprobar en el último apartado, todas han sido muy exitosas, lo que sirve para proyectar este método. 

Se ha podido comprobar con el análisis de las diferentes experiencias e investigaciones que el \textit{storytelling} se utiliza sobre todo para facilitar la adquisición de destrezas en lectura y escritura de lenguas diferentes a la natal \cite{arenas_efecto_2018,cedeno_damil_lectura_2019,gomez_palacio_strategies_2010}. 

Se cumplen, por tanto, los objetivos propuestos en este trabajo, ya que se han analizado las experiencias y buenas prácticas de utilización del \textit{storytelling} para fomentar el hábito lector en niños y adolescentes que se están llevando a cabo en España y otros países, para valorar su implantación como estrategia didáctica; elaborando un repositorio con aquellas experiencias nacionales e internacionales de utilización del \textit{storytelling} en el fomento de la lectura.

Se concluye que las narrativas transmedia, además de despertar la creatividad y la imaginación, se convierten en una magnífica herramienta para desarrollar el pensamiento crítico hacia lo que el alumnado lee. El \textit{storytelling} favorece, por tanto, una lectura crítica y razonada, bases del aprendizaje, lo que refuerza las ventajas de su utilización educativa.
El alumnado, hoy en día, ha dejado de ser un mero consumidor de aprendizaje a ser también productor. Ese alumno \textit{prosumer} necesita, pues, de metodologías que le permitan crear sus propios contenidos, y el \textit{storytelling} les va a permitir crear historias que favorecerán su amor por la lectura y la escritura.

Como principales limitaciones nos hemos encontrado, sin duda, con la falta de investigaciones y experiencias sobre la temática, algo que se ha tenido que solventar ampliando la búsqueda en bases de datos con un índice de impacto menor.

Como propuestas de futuro se quiere trabajar en un documento más amplio que recoja este repositorio ampliado que ayude a otros investigadores a la hora de consultar experiencias y buenas prácticas relacionadas con la utilización del \textit{storytelling} para fomentar el hábito lector en niños y adolescentes. Además, se pretende seguir publicando artículos sobre esta temática.

%italicos


\printbibliography\label{sec-bib}
% if the text is not in Portuguese, it might be necessary to use the code below instead to print the correct ABNT abbreviations [s.n.], [s.l.]
%\begin{portuguese}
%\printbibliography[title={Bibliography}]
%\end{portuguese}


%full list: conceptualization,datacuration,formalanalysis,funding,investigation,methodology,projadm,resources,software,supervision,validation,visualization,writing,review
\begin{contributors}[sec-contributors]
\authorcontribution{Álvaro Pérez García}[conceptualization,investigation,methodology,visualization,writing,review]
\authorcontribution{Ignacio Sacaluga Rodríguez}[conceptualization,investigation,methodology,writing,review]
\end{contributors}

\end{document}

