% !TEX TS-program = XeLaTeX
% use the following command:
% all document files must be coded in UTF-8
\documentclass[portuguese]{textolivre}
% build HTML with: make4ht -e build.lua -c textolivre.cfg -x -u article "fn-in,svg,pic-align"

\journalname{Texto Livre}
\thevolume{16}
%\thenumber{1} % old template
\theyear{2023}
\receiveddate{\DTMdisplaydate{2023}{1}{21}{-1}} % YYYY MM DD
\accepteddate{\DTMdisplaydate{2023}{5}{29}{-1}}
\publisheddate{\DTMdisplaydate{2023}{6}{30}{-1}}
\corrauthor{Daniel Guillermo Gordillo Sánchez}
\articledoi{10.1590/1983-3652.2023.42638}
%\articleid{NNNN} % if the article ID is not the last 5 numbers of its DOI, provide it using \articleid{} commmand 
% list of available sesscions in the journal: articles, dossier, reports, essays, reviews, interviews, editorial
\articlesessionname{reports}
\runningauthor{Sousa et al.} 
%\editorname{Leonardo Araújo} % old template
\sectioneditorname{Daniervelin Pereira}
\layouteditorname{Thaís Coutinho}

\title{Diálogos latino-americanos e aprendizagem de espanhol: relato de experiência de Teletandem na UFCG}
\othertitle{Latin American dialogues and Spanish language learning: report of Teletandem experience at UFCG}
% if there is a third language title, add here:
%\othertitle{Artikelvorlage zur Einreichung beim Texto Livre Journal}

\author[1]{Gabrielle Alves de Sousa~\orcid{0000-0001-6472-9384}\thanks{Email: \href{mailto:gabrielle.alves@estudante.ufcg.edu.br}{gabrielle.alves@estudante.ufcg.edu.br}}}
\author[1]{Ruth Lins da Silva~\orcid{0000-0001-5273-1102}\thanks{Email: \href{mailto:ruth.lins@estudante.ufcg.edu.br}{ruth.lins@estudante.ufcg.edu.br}}}
\author[1]{Raissa Layane de Deus Souza~\orcid{0000-0001-8376-8143}\thanks{Email: \href{mailto:raissalecampo2016@gmail.com}{raissalecampo2016@gmail.com}}}
\author[2,3]{Daniel Guillermo Gordillo Sanchez ~\orcid{0000-0001-6725-0627}\thanks{Email: \href{mailto:danielgordillo65@gmail.com}{danielgordillo65@gmail.com}}}
\affil[1]{Universidade Federal de Campina Grande, Centro de Desenvolvimento Sustentável do Semiárido, Unidade Acadêmica de Educação do Campo, Sumé, PB, Brasil.}
\affil[2]{Universidade Federal da Paraíba, João Pessoa, PB, Brasil.}
\affil[3]{Universidade de Hamburgo, Alemanha.}


\addbibresource{article.bib}
% use biber instead of bibtex
% $ biber article

% used to create dummy text for the template file
\definecolor{dark-gray}{gray}{0.35} % color used to display dummy texts
\usepackage{lipsum}
\SetLipsumParListSurrounders{\colorlet{oldcolor}{.}\color{dark-gray}}{\color{oldcolor}}

% used here only to provide the XeLaTeX and BibTeX logos
\usepackage{hologo}

% if you use multirows in a table, include the multirow package
\usepackage{multirow}

% provides sidewaysfigure environment
\usepackage{rotating}

\usepackage{epigraph}

% CUSTOM EPIGRAPH - BEGIN 
%%% https://tex.stackexchange.com/questions/193178/specific-epigraph-style
\usepackage{epigraph}
\renewcommand\textflush{flushright}
\makeatletter
\newlength\epitextskip
\pretocmd{\@epitext}{\em}{}{}
\apptocmd{\@epitext}{\em}{}{}
\patchcmd{\epigraph}{\@epitext{#1}\\}{\@epitext{#1}\\[\epitextskip]}{}{}
\makeatother
\setlength\epigraphrule{0pt}
\setlength\epitextskip{0.5ex}
\setlength\epigraphwidth{.7\textwidth}
% CUSTOM EPIGRAPH - END

% LANGUAGE - BEGIN
% ARABIC
% for languages that use special fonts, you must provide the typeface that will be used
% \setotherlanguage{arabic}
% \newfontfamily\arabicfont[Script=Arabic]{Amiri}
% \newfontfamily\arabicfontsf[Script=Arabic]{Amiri}
% \newfontfamily\arabicfonttt[Script=Arabic]{Amiri}
%
% in the article, to add arabic text use: \textlang{arabic}{ ... }
%
% RUSSIAN
% for russian text we also need to define fonts with support for Cyrillic script
% \usepackage{fontspec}
% \setotherlanguage{russian}
% \newfontfamily\cyrillicfont{Times New Roman}
% \newfontfamily\cyrillicfontsf{Times New Roman}[Script=Cyrillic]
% \newfontfamily\cyrillicfonttt{Times New Roman}[Script=Cyrillic]
%
% in the text use \begin{russian} ... \end{russian}
% LANGUAGE - END

% EMOJIS - BEGIN
% to use emoticons in your manuscript
% https://stackoverflow.com/questions/190145/how-to-insert-emoticons-in-latex/57076064
% using font Symbola, which has full support
% the font may be downloaded at:
% https://dn-works.com/ufas/
% add to preamble:
% \newfontfamily\Symbola{Symbola}
% in the text use:
% {\Symbola }
% EMOJIS - END

% LABEL REFERENCE TO DESCRIPTIVE LIST - BEGIN
% reference itens in a descriptive list using their labels instead of numbers
% insert the code below in the preambule:
%\makeatletter
%\let\orgdescriptionlabel\descriptionlabel
%\renewcommand*{\descriptionlabel}[1]{%
%  \let\orglabel\label
%  \let\label\@gobble
%  \phantomsection
%  \edef\@currentlabel{#1\unskip}%
%  \let\label\orglabel
%  \orgdescriptionlabel{#1}%
%}
%\makeatother
%
% in your document, use as illustraded here:
%\begin{description}
%  \item[first\label{itm1}] this is only an example;
%  % ...  add more items
%\end{description}
% LABEL REFERENCE TO DESCRIPTIVE LIST - END


% add line numbers for submission
%\usepackage{lineno}
%\linenumbers

\begin{document}
\maketitle

\begin{polyabstract}
\begin{abstract}
Este artigo visa apresentar a experiência intercultural do Teletandem envolvendo estudantes de duas universidades federais brasileiras. Seguindo a abordagem metodológica do relato de experiência \cite{mussi_pressupostos_2021}, elaborou-se de forma colaborativa um ensaio acadêmico-reflexivo que emerge de uma prática virtual de línguas estrangeiras dentro do curso de Espanhol I na Universidade Federal de Campina Grande (UFCG). Inspirados pela filosofia de Paulo Freire, discutimos como essa ferramenta pode fomentar o ensino e a aprendizagem de línguas estrangeiras, criando cenários autênticos de interação entre os/as alunos/as e estimulando a autonomia, a reciprocidade e a colaboração entre pares. Simultaneamente, identificamos alguns desafios e obstáculos potenciais que podem interferir na atividade. Concluímos que o Teletandem se configura como uma poderosa atividade intercultural, capaz de impulsionar a troca e a reflexão sobre as demandas e os desafios dos povos rurais da América Latina.

\keywords{Aprendizagem de línguas \sep Tecnologia e educação \sep Experiências de aprendizagem \sep Educação intercultural}
\end{abstract}

\begin{english}
\begin{abstract}
This article aims to present the intercultural experience of Teletandem involving students from two Brazilian federal universities. Following the methodological approach of the experience report \cite{mussi_pressupostos_2021}, we developed a collaborative academic-reflexive essay that emerges from a virtual foreign language practice within the Spanish I course at the Federal University of Campina Grande (UFCG). Inspired by the philosophy of Paulo Freire, we discuss how this tool can foster the teaching and learning of foreign languages, creating authentic interaction scenarios among students and promoting autonomy, reciprocity, and peer collaboration. Simultaneously, we identify some challenges and potential obstacles that may interfere with the activity. We conclude that Teletandem emerges as a powerful intercultural activity capable of fostering exchange and reflection on the demands and challenges faced by rural populations in Latin America.

\keywords{Language instruction \sep Technology and education \sep Learning Experiences \sep Intercultural education}
\end{abstract}
\end{english}
% if there is another abstract, insert it here using the same scheme
\end{polyabstract}

\section{Introdução}\label{sec-intro}

\epigraph{No soy de aquí ni soy de allá \\ 
No tengo edad ni porvenir \\ 
Y ser feliz es mi color de identidad}{Facundo Cabral}

No Brasil, um país que mantém fronteira com vários países de língua espanhola, é fundamental o desenvolvimento de projetos, atividades e políticas públicas que valorizem a cultura, a história e a língua do universo hispânico circundante, promovendo a integração regional, a dinamização dos mercados, a troca de experiências, ciência e saberes, e a reflexão coletiva sobre os desafios regionais e globais contemporâneos. Certamente, esse movimento requer uma problematização da ideia totalizante de que no Brasil se fala uma única língua, o português, concepção que ignora a pluralidade de culturas e línguas que fazem contato e de povos que atravessaram as fronteiras nacionais e influenciam as realidades locais. Ao problematizar, desde uma perspectiva historiográfica, a hegemonia da língua portuguesa, em detrimento de outras línguas provenientes da migração, Müller de Oliveira indica que é fundamental “redefinir o conceito de nacionalidade, tornando-o plural e aberto à diversidade: seria mais democrático e culturalmente mais enriquecedor, menos violento e discricionário” \cite[p.9]{oliveira2008}.

Acreditamos que a educação pode ter um papel central nesse processo crítico, pontualmente a educação linguística, entendida desde a ótica intercultural, a qual promove espaços sociais comunicacionais que reconhecem, o “outro”, favorecendo o diálogo entre as diferentes culturas-línguas, assim como a valorização das identidades e diferenças dos indivíduos \cite{fleuri2001desafios}. O desenvolvimento de uma educação linguística intercultural é fundamental para a construção de uma sociedade sensível ao diálogo entre culturas, povos, saberes e epistemes. \textcite[p.24-25]{takaki_repensando_2019} ressaltam a importância da educação linguística e seu papel para promover uma cidadania ativa na contemporaneidade:

\begin{quote}
 Observa-se que o acesso a uma infinidade de informações não significa alteração nos nossos mundos (...) já que para houver um deslocamento de significado a informação precisa ser incorporada nos processos de construção de significados do sujeito e assim, tornar-se conhecimento que pode ser mobilizado nos processos de interpretação. 
\end{quote}

Nesse sentido, no cenário atual um dos desafios no ensino de línguas estrangeiras consiste na criação de ambientes e situações reais e autênticas que favoreçam a interação, a criação de novos significados, a troca de experiências, o pensamento crítico sobre desafios comuns e a promoção da alteridade. Desse modo, entendemos que dentro de um país, região ou comunidade as relações entre seus membros não se restringem apenas à esfera linguística; elas operam em um contexto cultural maior, que exprime relações sociais, geográficas, culturais, tensões, encontros e desencontros, tradições e formas de entender e representar o mundo. Seguindo as reflexões de \textcite{freire_pedagogia_2004} sobre a Pedagogia da Autonomia, consideramos que o ensino de uma língua estrangeira não se limita à transferência de um vocabulário ou de estruturas gramaticais, mas diz respeito à criação de possibilidades para a exploração desse conhecimento, estimulando uma autonomia intelectual no/a aluno/a que lhe permita descobrir novas experiências estéticas, políticas e linguísticas, agindo de forma crítica sobre seu entorno. 

Isto torna cada vez mais necessária a implementação de propostas e estratégias que ofereçam aos/às alunos/as diferentes experiências linguísticas e interculturais, buscando, por um lado, a melhoria nas habilidades formais do aprendizado de uma língua estrangeira, mas também o desenvolvimento de outras competências linguísticas, sociais e culturais, essenciais no mundo globalizado. Os recursos tecnológicos constituem um cenário muito fértil nessa empreitada, já que possibilitam 

\begin{quote}
 (...) uma comunicação eficiente entre pessoas distantes geograficamente, bem como tem contribuído para o ensino-aprendizagem de línguas estrangeiras na medida em que permite que falantes de países, línguas e culturas diferentes se comuniquem a fim de aprender ou aprimorar a fluência em outra língua. \cite[p.17]{franco2016}.
\end{quote}

Uma dessas possibilidades de encontro intercultural é o Tandem, conceito com raízes do latim que originalmente se refere às bicicletas para duas pessoas, e que, no âmbito da aprendizagem de línguas estrangeiras se apresenta como uma poderosa analogia que permite que os indivíduos, ora aprendizes, ora mestres, pratiquem e vivenciem novas línguas e culturas de forma simultânea e dinâmica. Nessa perspectiva, o Tandem posiciona o/a aprendiz num panorama mais amplo das dinâmicas regionais e mundiais, visando alcançar novas dimensões de vida, baseadas na autonomia, na reciprocidade e na solidariedade para a construção de conhecimento e o aprimoramento linguístico \cite{vassallo2006foreign}. Em palavras de \textcite[p.84-85]{ramme_tandem:_2015}:

\begin{quote}
 Sempre com a ideia de colaboração entre pessoas de diferentes culturas em mente, esse recurso de aprendizagem [o Tandem] consiste em conectar dois estudantes que queiram aprimorar seus conhecimentos na língua estrangeira do outro, ambos ajudando-se mutuamente a se aproximarem da língua e cultura de interesse. O princípio dessa parceria é, então, o mesmo do de um passeio de bicicleta tandem: ambos devem estar pedalando em direção ao mesmo objetivo para não cair.
\end{quote}

É importante destacar que, na maioria dos centros de idiomas, universidades e escolas de línguas, a abordagem do tandem é comumente aplicada em um ambiente presencial. No entanto, a evolução tecnológica e o próprio cenário desencadeado pela pandemia de COVID-19 revelaram a viabilidade de construir conhecimento à distância, assim como a riqueza dos encontros culturais, profissionais e acadêmicos em um ambiente virtual. Diversas plataformas têm possibilitado novos formatos para a metodologia tandem, dando origem ao conceito de Teletandem. Deve-se mencionar, no entanto, que os fundamentos teóricos e práticos do tandem virtual ou Teletandem (tandem que faz uso da internet e de ferramentas virtuais de comunicação) já haviam sido discutidos por \textcite{vassallo2006foreign}. Este conceito foi validado através de um estudo de caso realizado em um contexto universitário no Brasil, no qual foram demonstradas suas vantagens para a aprendizagem de línguas estrangeiras \cite{vassallo2006foreign}.

À luz das anteriores considerações, o presente artigo apresenta uma experiência de Teletandem desenvolvida entre as alunas da Licenciatura em Educação do Campo da Universidade Federal de Campina Grande (UFCG) – Campus Sumé, Paraíba, e os/as estudantes de origem hispana da Universidade Federal da Integração Latino-americana (UNILA), localizada na cidade fronteiriça de Foz do Iguaçu, Paraná. Inspirados/as na abordagem metodológica e na proposta de sistematização do relato de experiência \cite{mussi_pressupostos_2021}, estruturamos o texto em quatro seções, além desta introdução. Na seção "Contexto da experiência: aproximando a UNILA e a UFCG", fornecemos uma descrição sucinta das universidades participantes do projeto; em "Materiais e Procedimentos Metodológicos", detalhamos a metodologia e as ferramentas empregadas na experiência; "Conexões além das fronteiras: os povos do campo na América Latina", estabelecemos um diálogo teórico e compartilhamos impressões e reflexões subjetivas sobre as lições aprendidas no Teletandem; e finalmente, "Considerações Finais", onde fazemos um resumo dos principais resultados e descrevemos alguns desafios e obstáculos potenciais que podem interferir neste tipo de atividades.

\section{Contexto da experiência: aproximando a UNILA e a UFCG}

A UNILA é uma instituição pública de educação superior, localizada no município paranaense de Foz do Iguaçu, vizinho das cidades de Puerto Iguazú (Argentina) e Ciudad del Este (Paraguai). Essa região é conhecida como a Tríplice Fronteira Sul e se destaca por sua diversidade cultural, linguística e religiosa. Fundada no ano de 2010, a UNILA tem uma vocação internacional e integracionista, promovendo o intercâmbio cultural, científico, tecnológico e linguístico na América Latina e no Caribe. Segundo o Plano de Desenvolvimento Institucional desta universidade \cite{Universidade2019}, 50\% das vagas para estudantes são destinadas a candidatos/as de outras nacionalidades latino-americanas e caribenhas, sendo que a outra metade é composta por discentes brasileiros/as. Consequentemente, o bilinguismo (espanhol-português) nesta instituição é assegurado pela lei de sua criação, se refletindo na estrutura institucional e burocrática da universidade, favorecendo os diálogos regionais e internacionais e a promoção da diversidade cultural na região. Neste cenário, as salas de aula da UNILA são frequentadas por alunos/as e professores/as de países como Argentina, Colômbia, Venezuela, Cuba, Paraguai, Bolívia, Equador, dentre outros. 

Atualmente a UNILA oferta 29 cursos de graduação, divididos em quatro institutos: o Instituto Latino-Americano de Arte, Cultura e História (ILAACH), o Instituto Latino-Americano de Ciências da Vida e da Natureza (ILACVN), o Instituto Latino-Americano de Economia, Sociedade e Política (ILAESP) e o Instituto Latino-Americano de Tecnologia, Infraestrutura e Território (ILATIT). Uma das características mais inovadoras da UNILA é a composição curricular dos cursos de graduação, uma vez que todos possuem um Ciclo Comum de Estudos, dividido em três eixos: 1º. Estudo Compreensivo sobre a América Latina e Caribe, 2º. Epistemologia e Metodologia, 3º. Línguas Portuguesa e Espanhola. Este último essencial para o objetivo de consolidar uma universidade bilíngue. Para os/as estudantes oriundos/as de países hispânicos é ofertada a língua portuguesa nos primeiros semestres, preparando os/as discentes para todas as atividades de ensino, pesquisa e extensão \cite{Universidade2019}. Na experiência de Teletandem aqui descrita participaram dois estudantes do Chile e um estudante da Colômbia, do segundo semestre do curso de Ciências Econômicas – Economia, Integração e Desenvolvimento, matriculados na disciplina de Língua Portuguesa. A professora, brasileira, que ministrou a referida matéria na UNILA e o professor de espanhol da UFCG, de nacionalidade colombiana, coordenaram esta iniciativa de Teletandem entre ambas as instituições, após reuniões virtuais onde se exploraram as possibilidades para sua efetivação. 

Por sua vez, a UFCG é uma instituição de ensino superior pública localizada no estado da Paraíba, no nordeste brasileiro. Seu principal campus encontra-se na cidade de Campina Grande, a segunda cidade mais populosa do Estado, tendo outros nas cidades de Pombal, Patos, Sousa, Cajazeiras, Cuité e Sumé. Este último é denominado Centro de Desenvolvimento Sustentável do Semiárido (CDSA), beneficiando os municípios interioranos do Cariri da Paraíba, até então com limitado acesso à educação superior. O campus de Sumé foi criado no âmbito do Plano de Expansão Institucional da UFCG (PLANEXP) e iniciou suas atividades em 2009. Entre os seus propósitos estavam o de possibilitar aos jovens da faixa etária entre 18 e 24 anos uma formação acadêmica profissional de nível superior, de qualidade e pública. Atividades estas que continuam sendo realizadas e possibilitam que muitos/as jovens das cidades circunvizinhas possam acessar o ensino superior. Ao mesmo tempo, o CDSA promove o desenvolvimento sustentável na região, abrindo novas perspectivas econômicas produtivas e educacionais. O CDSA atua na formação de professores/as para o campo e no campo e em projetos de capacitação de profissionais na gestão de projetos em políticas públicas e no setor de produção. 

O curso de Licenciatura em Educação do Campo é o resultado de um longo processo de debates, mobilização e articulação entre movimentos sociais do campo, o Ministério da Educação e as universidades públicas brasileiras, dentre as quais a UFCG teve um lugar de destaque. É preciso evidenciar que a concepção da educação do campo era muito vinculada à ideia conservadora de “modernização para o campo”. A partir de um movimento crítico, a criação da licenciatura teve embasamento em premissas relacionadas com a valorização da cultura, identidade e educação para os povos do campo, bem como nos processos de sustentabilidade e agroecologia. \textcite{silva2011construccao} destaca a necessidade de uma educação do campo contextualizada, que vise a formação de professores/as para a Educação básica em diálogo “com a realidade social e cultural específica das populações que trabalham e vivem no e do campo, na diversidade de ações pedagógicas necessárias para concretizá-la como direito humano e como ferramenta e desenvolvimento social.” \cite[p.415]{silva2011construccao}

Dentre das áreas que compõem a grade da Licenciatura na Educação do Campo na UFCG se destaca a área de Linguagens e Códigos, que conta com 19 componentes curriculares relacionados com disciplinas de linguística, literatura, língua estrangeira, artes visuais, educação física, dentre outras. No componente curricular de Língua Estrangeira, além do inglês e do francês, cada uma com dois níveis, se encontram as disciplinas de Espanhol I e Espanhol II, ambas obrigatórias, com uma carga horária de 60 horas-aula no semestre e 04 horas-aula por semana. Nesta reflexão relataremos a experiência de Teletandem na disciplina de Espanhol I, onde estavam matriculadas três estudantes oriundas do interior do estado da Paraíba, dos últimos semestres do curso.



\section{Materiais e procedimentos metodológicos}

Durante o primeiro semestre do ano 2022, os/as participantes da UNILA e da UFCG realizaram encontros virtuais semanais de 60 minutos, dividindo este tempo igualmente entre a prática do português e do espanhol. As sessões ocorreram através da plataforma \emph{Google Meet}, com o auxílio de recursos como microfone e câmera de computador ou celular. A funcionalidade de compartilhamento de tela, somada ao envio de links e mensagens via chat, possibilitou a troca de informações multimídia sobre os costumes, conhecimentos e tradições tanto do Brasil quanto dos países hispanofalantes. Esta ferramenta demonstrou ser fundamental ao facilitar a disseminação desses conteúdos durante os encontros, uma vez que proporcionou a oportunidade de apresentação de slides, imagens e vídeos. Assim, ficou evidente a importância dos recursos multimodais para fomentar essa aproximação intercultural.

Ao todo, foram realizados quatro encontros durante o semestre, sendo que os/as participantes tinham um nível iniciante da língua estrangeira. Para a mediação das sessões foram usados materiais didáticos disponibilizados no projeto de extensão \emph{tándem na tríplice fronteira}\footnote{Disponível em: \url{https://tandemunila.wixsite.com/tandem/publicaciones} Acesso em 29/05/2023.} coordenado pela professora Dra. Larissa Tirloni da UNILA, que versam sobre assuntos cotidianos, científicos e acadêmicos, abordando-os de forma descontraída e sempre com perguntas para impulsionar a participação. Assim, por exemplo, foram abordadas unidades didáticas sobre o conceito de paisagem linguística, sobre os povos indígenas no Brasil, sobre a crise migratória nos Estados Unidos e sobre ensino remoto na pandemia. Este material foi fundamental para “quebrar o gelo”, instigar a fala, promover o debate e organizar as intervenções, que também podiam tratar sobre outros assuntos e temáticas que não faziam parte das mencionadas unidades didáticas. 

É relevante destacar que o docente responsável pela disciplina, naquela ocasião atuando como professor substituto na Licenciatura em Educação do Campo da UFCG, e também coautor deste artigo, não participou diretamente das sessões síncronas das alunas. Além de atuar na coordenação da experiência em conjunto com a professora de língua portuguesa da UNILA, sua contribuição concentrou-se principalmente na apresentação da metodologia do Teletandem e do material didático mencionado às estudantes da UFCG. Além disso, o professor facilitou discussões em sala de aula sobre os encontros virtuais. 

No final de cada encontro, as/os estudantes deviam elaborar um diário de aprendizagem. Esta ferramenta tinha como objetivo a sistematização dos conhecimentos adquiridos, o registro de dúvidas, curiosidades e elementos que despertassem interesse nas conversas. Este documento também possibilitava a comprovação dos encontros entre as duplas, por meio de capturas de tela que evidenciavam a realização das chamadas, as quais eram feitas diretamente das residências dos/as estudantes. Assim, a elaboração do diário de aprendizagem constituiu uma atividade avaliativa dentro da disciplina de Espanhol I, sendo de grande relevância para concretizar os aprendizados adquiridos nos encontros. Esses momentos representavam uma oportunidade valiosa para (re)considerar novas abordagens de engajamento, reciprocidade dos/as parceiros/as e estratégias didáticas para enriquecer os encontros virtuais. Após cada sessão de Teletandem, eram realizados debates que avaliavam os encontros, permitiam compartilhar os conhecimentos adquiridos e externalizar dúvidas e incertezas relacionadas ao Teletandem.

Portanto, é crucial enfatizar a formação prévia sobre o Teletandem, sua história e conceitos teóricos, foi fundamental para consolidar a metodologia das atividades. Acreditamos que o êxito de uma experiência de Teletandem depende diretamente do conhecimento que os participantes possuem sobre seus princípios, características e as responsabilidades inerentes ao aprendizado de uma língua estrangeira. Nesse contexto, o papel do professor como mediador revelou-se de suma importância, sendo seu acompanhamento ao longo do processo absolutamente indispensável.


\section{Conexões além das fronteiras: os povos do campo na América Latina}

O primeiro encontro destinava-se à apresentação dos/as participantes, que deveriam expor brevemente suas trajetórias acadêmicas e pessoais, além de descrever suas experiências prévias com línguas estrangeiras. Como era de esperar, essa primeira sessão foi marcada por certa insegurança e timidez ao se comunicar com desconhecidos/as em um idioma que estava sendo aprendido pela primeira vez. Contudo, rapidamente foi percebido que todos/as os/as participantes compartilhavam a posição de aprendizes e que as adversidades eram pequenas diante dos benefícios proporcionados pelo Teletandem. Gradativamente, foi construído um laço de confiança e os objetivos de aprendizagem foram explicitados, baseando-se nos repertórios e vivências subjetivas dos/as estudantes. Como observou \textcite[p.21]{brammerts2002aprender}, os parceiros do tandem “nunca ensinam, mas ajudam a aprender”, assim, foi fundamental ouvir a trajetória dos/as interlocutores/as, e identificar seus propósitos para o aprendizado da língua estrangeira. 

Essa consciência sobre as necessidades do outro foi indispensável para manter o espírito coletivo ao longo do processo de aprendizagem. Por isso, concordamos com \textcite[p.9]{freire_pedagogia_1981} quando aponta que: “Ninguém educa ninguém, como tampouco ninguém se educa a si mesmo: os homens se educam em comunhão, mediatizados pelo mundo”. 

É importante mencionar que, antes das aulas de espanhol na UFCG, a interação dos/as participantes brasileiros/as com a língua espanhola restringia-se a programas televisivos, tais como as novelas "Maria do Bairro", "Marimar" e "Rebelde", transmitidas pelos canais locais, além de músicas de artistas como Shakira, Enrique Iglesias e da notória canção "Despacito" de Luis Fonsi. Portanto, havia um entendimento embrionário acerca da geopolítica regional, geografia e história da América Latina. Assim sendo, foi extremamente enriquecedor ouvir as descrições e representações de outros lugares por parte de indivíduos que optaram por migrar para o Brasil para estudar na UNILA, enquanto ainda preservam suas tradições, costumes e vínculos com seus locais de origem e com a cultura hispana. Portanto, mais do que simplesmente decodificar ou ler palavras em espanhol, foi possível empreender uma leitura da 'palavra-mundo', fazendo um paralelo com o conceito proposto por \textcite{freire_importancia_1991} . 

A experiência do Teletandem possibilitou a exploração de aspectos macroestruturais da América Latina, abrangendo temas como economia e política. Da mesma forma, se abordaram dimensões mais subjetivas da vida universitária, do nosso dia a dia durante a pandemia, da família, dentre outros temas. Este processo favoreceu o conhecimento de experiências, narrativas e expressões de um universo distinto, que revelava características variadas: aspectos positivos e oportunidades, mas também tensões e problemáticas sociais. Durante as sessões tivemos a oportunidade de compartilhar informações sobre nossos países de origem (Brasil, Colômbia e Chile) e discutir as características específicas das regiões onde vivíamos. Questões climáticas foram abordadas, como o rigoroso inverno no meio do ano em Foz do Iguaçu e as secas intensas no Nordeste brasileiro. Essas discussões proporcionaram uma compreensão mais aprofundada de questões históricas, políticas e ambientais. Nossos/as parceiros/a também expressaram preocupações com o avanço dos monocultivos transgênicos, como açúcar, soja e milho, que frequentemente ocupam terras ancestrais de comunidades tradicionais no Paraná, estado onde eles/as moram. 

Isto nos permitiu dimensionar questões importantes sobre as lutas e demandas das comunidades rurais na América Latina, especialmente no que diz respeito à importância da terra. Através do Teletandem, pudemos compreender melhor as problemáticas relacionadas a questões fundiárias, conflitos agrários e seus impactos nas comunidades indígenas e camponesas. Um exemplo marcante é a situação enfrentada pela comunidade Wayuu, na Colômbia, que sofre com a exploração de recursos naturais em seus territórios, resultando em graves consequências sociais. 

Foi possível então explorar a interconexão entre a terra, a identidade cultural e as demandas por justiça social. Nossas conversas frequentemente tratavam das desigualdades socioeconômicas presentes na região e a importância de buscar soluções que promovam uma distribuição mais equitativa dos recursos naturais e um maior respeito pelos direitos territoriais das comunidades rurais. Portanto, ao mesmo tempo em que aprimoramos nosso conhecimento da língua estrangeira, pudemos enriquecer nosso entendimento dos desafios enfrentados pelas comunidades rurais na América Latina e fortalecer nossa conexão com estudantes de outros países, compartilhando experiências e aprendendo coletivamente. Nossa intenção foi ir além da abordagem educacional tradicional, conforme notado por \textcite{freire_pedagogia_1981}, buscando vincular a prática da língua estrangeira à realidade que nos cerca e assim poder desenvolver uma consciência crítica.

Deve-se destacar a oportunidade inestimável de aprofundar o conhecimento sobre aspectos culturais do Chile e da Colômbia, incluindo festivais, gastronomia e gêneros musicais. Tal aprendizado contribuiu significativamente para ampliar a compreensão acerca das dinâmicas sociais e culturais latino-americanas. Em uma das sessões, por exemplo, o foco foi o Carnaval de Barranquilla, festividade de grande relevância na América Latina e patrimônio mundial da UNESCO, que era até então pouco conhecida. Embora o tema já tivesse sido explorado em uma atividade do material didático durante a disciplina de Espanhol I, o relato autêntico de pessoas nativas do país trouxe uma perspectiva singular e profundamente enriquecedora. Além disso, a interação com falantes nativos/as de espanhol proporcionou a prática efetiva de estruturas gramaticais previamente estudadas, bem como o aperfeiçoamento da pronúncia de sons desafiadores para lusófonos, como o 'R'. Com a ajuda e compreensão dos demais participantes do tandem, palavras como "rojo", "ratón" e "risa" foram trabalhadas em um ambiente de aprendizagem seguro e, frequentemente, divertido. Essa experiência, além de consolidar competências linguísticas, fortaleceu laços de amizade que perduram. A conexão com os/as parceiros/as de tandem é mantida até hoje, principalmente por meio de plataformas digitais, como WhatsApp e Instagram.

Os encontros virtuais configuraram-se como espaços fundamentais para a prática linguística livre do medo de cometer erros, tornando o processo de aprendizado de um idioma mais prazeroso e propiciando o conhecimento de experiências profissionais e indivíduos de outras localidades. A prática do Teletandem reforçou o exercício da paciência e da disposição para ouvir e ser ouvido, uma vez que sua dinâmica exige que as pessoas envolvidas sejam igualmente participantes ativos/as e ouvintes atentos/as, alternando o tempo de fala e proporcionando espaço para que cada um possa expressar suas ideias e pensamentos. Paralelamente, desenvolveram-se habilidades para avaliar e corrigir o português dos/as parceiros/as. A presença de "erros" e dificuldades por parte deles/as contribuiu para a criação de um ambiente de igualdade, compreensão mútua e solidariedade.

\begin{quote}
 [...] é possível inferir que a conscientização da linguagem também apresenta implicações para o teletandem e para o contato intercultural que ele promove, na medida em que participantes conscientes tenderiam a respeitar, não somente a língua e a cultura de seus parceiros, mas as diferenças existentes em seu próprio sistema linguístico e as diferenças culturais de seu próprio país. \textcite[p.63]{franco2016} 
\end{quote}

Revelou-se muito satisfatória a possibilidade de estudar com pessoas de outros países, à semelhança do que ocorre com os/as estudantes da UNILA em Foz do Iguaçu (PR). Tal descoberta incentivou a busca por futuras oportunidades de intercâmbio e a interação com indivíduos de diferentes nacionalidades por meio de aplicativos e sites de aprendizagem de línguas estrangeiras. A aprendizagem de um novo idioma, cultura e costumes com uma pessoa nativa do país em foco é uma experiência inegavelmente enriquecedora e gratificante. Os parceiros hispano-falantes mostraram-se sempre solícitos e frequentemente preparavam jogos de trava-línguas que aguçaram a curiosidade acerca das palavras em espanhol.

O Teletandem proporcionou a exploração do universo cultural e gastronômico do mundo hispânico, assim como a oportunidade de apresentar os pratos e tradições do Brasil. Foi muito interessante, por exemplo, descobrir que uma receita transmitida por gerações na família dos/as participantes brasileiros/as correspondia a um prato típico da Colômbia: os \emph{Huevos pericos}, equivalentes aos “ovos refogados" conhecidos no Brasil e consumidos frequentemente com cuscuz, especialmente no Nordeste. Da mesma forma, notaram-se semelhanças no consumo de alimentos como o milho (maíz) e a macaxeira (\emph{yuca}), e a importância da \emph{arepa} para diversos países do continente, todos alimentos vinculados a uma matriz indígena. Essas descobertas assumem uma importância fundamental para os/as estudantes de Licenciatura em Educação do Campo, uma vez que promovem um estreitamento de laços com as identidades e culturas rurais de outras regiões.


\section{Considerações finais}

O aprendizado de línguas estrangeiras é um processo intrincadamente complexo, abrangendo diversas abordagens didáticas, teóricas e metodologias além do âmbito da educação formal. A prática de Teletandem surgiu como uma alternativa viável e altamente relevante, dadas as condições de ensino e as relações profissionais e de pesquisa estabelecidas pelos professores coordenadores na UFCG e UNILA. A adaptação necessária durante a pandemia também abriu caminho para este espaço pedagógico, ilustrando as amplas possibilidades oferecidas pela tecnologia no fomento de espaços educacionais. A experiência descrita neste texto ressalta a importância de implementar ações que estimulem o pensamento crítico, a curiosidade intelectual e a utilização de ferramentas inovadoras para o desenvolvimento de habilidades orais e escritas em línguas estrangeiras. Durante esse processo, foi feito um esforço para criar condições favoráveis ao conhecimento de outras culturas, geografias, tradições e idiomas, com o objetivo de estabelecer os fundamentos de uma sociedade que valorize a diversidade linguística, social e cultural. Portanto, este projeto se inseriu em um paradigma pedagógico que enfatiza a vivência prática do conhecimento e impulsiona a participação ativa dos/as alunos/as em uma sociedade onde a diversidade linguística e cultural é valorizada e reconhecida. O presente artigo buscou apresentar os múltiplos sentidos imersos desta prática educativa a partir das próprias representações e interesses dos/as estudantes.

Ancorados em princípios freireanos, a visão de educação linguística aqui defendida se afasta de abordagens meramente conteudistas, reconhecendo os alunos como entidades dinâmicas, engajadas em investigação e descoberta, aptas a traçar e examinar independentemente os caminhos do aprendizado de uma língua estrangeira, aplicando esses conhecimentos ao seu contexto imediato. Conforme indicado por \textcite[p.464]{tirloni_tandem_2015}, a metodologia de tandem propicia "um ambiente colaborativo e reflexivo, no qual os alunos são protagonistas de seu próprio aprendizado, sendo capazes de avaliar seu próprio progresso". 

Acreditamos que o processo de aprendizagem de um idioma está intrinsecamente ligado a variáveis cognitivas e didático-pedagógicas, assim como a questões sociais, políticas e culturais que permeiam as relações de poder em nossa sociedade. Essas dimensões amplas e contextualizadas nos permitem compreender de forma mais abrangente a forma como um indivíduo adquire proficiência em uma língua específica. Nesta reflexão, destacamos a relevância do Teletandem como uma poderosa ferramenta para o aprendizado da língua espanhola, especialmente em contextos como o do Cariri Paraibano. O Teletandem oferece oportunidades culturais únicas para jovens que, até recentemente, não tinham acesso à educação universitária nem a experiências de intercâmbio intercultural internacional. Essa experiência pode ser replicada em outras instituições de ensino superior, integrando novos atores e alianças institucionais de diferentes regiões do Brasil.

O Teletandem entre a UFCG e a UNILA proporcionou uma oportunidade única para expandir nossos horizontes, permitindo-nos compreender, comparar e analisar criticamente novas realidades, tornando-as menos distantes ou exóticas. Durante esse processo, pudemos problematizar as questões de racismo e discriminação enfrentadas por migrantes tanto externos (como pessoas de outros países que vêm estudar e/ou trabalhar no Brasil) quanto internos (referindo-nos aos processos de migração de pessoas do nordeste para o sudeste do país). Nesse sentido, é de extrema importância compreender as comunidades como "partes de uma totalidade maior (área, região, etc.), que por sua vez é uma parte de uma totalidade ainda maior", como colocado por \textcite[p.166]{freire_pedagogia_1981}. O encontro com a alteridade nos proporcionou novas perspectivas para refletir sobre nossa própria subjetividade, meio ambiente e coletividade.

Em relação aos desafios e possíveis dificuldades, observamos que a conectividade representa um obstáculo significativo a ser enfrentado, uma vez que é necessário contar com uma conexão estável à internet para realizar as videochamadas. A instabilidade internet, fenômeno comum durante o período de chuvas na Paraíba, causou certos entraves durante os encontros. Da mesma forma, vale ressaltar que é essencial que as câmeras e microfones dos/as participantes estejam em pleno funcionamento, pois isso torna os encontros virtuais menos impessoais e mais intimistas. 

É imprescindível assegurar o acesso à internet e aos dispositivos tecnológicos nas áreas rurais e em regiões mais remotas do Brasil, uma responsabilidade que deve ser garantida pelo Estado no cumprimento de seu papel de democratizar a educação para toda a população, especialmente em locais com menos presença governamental. Ao mesmo tempo, é importante reconhecer que a participação efetiva no Teletandem requer um domínio básico da informática, a fim de aproveitar ao máximo os recursos disponíveis nas plataformas para o compartilhamento de materiais virtuais disponíveis na internet. Nesse sentido, acreditamos que a universidade pode desempenhar um papel central na formação dessas competências. A disponibilidade de horários também é um fator a ser considerado, especialmente quando os/as participantes se encontram em fusos horários diferentes. Essa atividade requer organização e comprometimento de ambas as partes. Além disso, enfatiza-se a importância da reciprocidade durante os encontros, onde o conhecimento deve circular de maneira democrática e participativa, promovendo um ambiente colaborativo e de troca mútua.

Por fim, com o relato desta vivência pretendemos contribuir com a discussão teórico-metodológica sobre o Teletandem, refletindo também sobre seus desafios e oportunidades. Esperamos contribuir para pesquisadores/as da área, servindo como potencial exemplo para indicar novos caminhos, estudos e vivências que oportunizem o Teletandem em diversos contextos, regiões e países. 

\printbibliography\label{sec-bib}
% if the text is not in Portuguese, it might be necessary to use the code below instead to print the correct ABNT abbreviations [s.n.], [s.l.]
%\begin{portuguese}
%\printbibliography[title={Bibliography}]
%\end{portuguese}


%full list: conceptualization,datacuration,formalanalysis,funding,investigation,methodology,projadm,resources,software,supervision,validation,visualization,writing,review
\begin{contributors}[sec-contributors]
\authorcontribution{Gabrielle Alves de Sousa}[writing]
\authorcontribution{Ruth Lins da Silva}[writing]
\authorcontribution{Raissa Layane de Deus Souza}[writing,review]
\authorcontribution{Daniel Guillermo Gordillo Sanchez}[conceptualization,supervision,writing,review]
\end{contributors}



\end{document}
