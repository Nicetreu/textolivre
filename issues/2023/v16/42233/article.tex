% !TEX TS-program = XeLaTeX
% use the following command:
% all document files must be coded in UTF-8
\documentclass[spanish]{textolivre}
% build HTML with: make4ht -e build.lua -c textolivre.cfg -x -u article "fn-in,svg,pic-align"

\journalname{Texto Livre}
\thevolume{16}
%\thenumber{1} % old template
\theyear{2023}
\receiveddate{\DTMdisplaydate{2022}{12}{23}{-1}} % YYYY MM DD
\accepteddate{\DTMdisplaydate{2023}{1}{10}{-1}}
\publisheddate{\DTMdisplaydate{2023}{1}{27}{-1}}
\corrauthor{Carmen Llorente-Cejudo}
\articledoi{10.1590/1983-3652.2023.42233}
%\articleid{NNNN} % if the article ID is not the last 5 numbers of its DOI, provide it using \articleid{} commmand 
% list of available sesscions in the journal: articles, dossier, reports, essays, reviews, interviews, editorial
\articlesessionname{articles}
\runningauthor{Cabero Almenara et al.} 
%\editorname{Leonardo Araújo} % old template
\sectioneditorname{Hugo Heredia Ponce}
\layouteditorname{Thaís Coutinho}

\title{Nativos e inmigrantes digitales en el contexto de la COVID-19: las contradicciones de una diversidad de mitos}
\othertitle{Nativos e imigrantes digitais no contexto da COVID-19: as contradições de uma diversidade de mitos}
\othertitle{Digital natives and immigrants in the context of COVID-19: the contradictions of a diversity of myths}
% if there is a third language title, add here:
%\othertitle{Artikelvorlage zur Einreichung beim Texto Livre Journal}

\author[1]{Julio Cabero-Almenara~\orcid{0000-0002-1133-6031}\thanks{Email: \href{mailto:cabero@us.es}{cabero@us.es}}}
\author[2]{Rubicelia Valencia-Ortiz~\orcid{0000-0003-4656-5456}\thanks{Email: \href{mailto:rubicelia.valencia@macmillaneducation.com}{rubicelia.valencia@macmillaneducation.com}}}
\author[1]{Carmen Llorente-Cejudo~\orcid{0000-0002-4281-928X}\thanks{Email: \href{mailto:karen@us.es}{karen@us.es}}}
\author[1]{Antonio de Padua Palacios-Rodríguez ~\orcid{0000-0002-0689-6317}\thanks{Email: \href{mailto:aprodriguez@us.es}{aprodriguez@us.es}}}
\affil[1]{Universidad de Sevilla, Facultad de Ciencias de la Educación, Departamento de Didáctica y Organización Educativa, España.}
\affil[2]{Macmillan Education, Mexico.}

\addbibresource{article.bib}
% use biber instead of bibtex
% $ biber article

% used to create dummy text for the template file
\definecolor{dark-gray}{gray}{0.35} % color used to display dummy texts
\usepackage{lipsum}
\SetLipsumParListSurrounders{\colorlet{oldcolor}{.}\color{dark-gray}}{\color{oldcolor}}

% used here only to provide the XeLaTeX and BibTeX logos
\usepackage{hologo}

% if you use multirows in a table, include the multirow package
\usepackage{multirow}

% provides sidewaysfigure environment
\usepackage{rotating}

% CUSTOM EPIGRAPH - BEGIN 
%%% https://tex.stackexchange.com/questions/193178/specific-epigraph-style
\usepackage{epigraph}
\renewcommand\textflush{flushright}
\makeatletter
\newlength\epitextskip
\pretocmd{\@epitext}{\em}{}{}
\apptocmd{\@epitext}{\em}{}{}
\patchcmd{\epigraph}{\@epitext{#1}\\}{\@epitext{#1}\\[\epitextskip]}{}{}
\makeatother
\setlength\epigraphrule{0pt}
\setlength\epitextskip{0.5ex}
\setlength\epigraphwidth{.7\textwidth}
% CUSTOM EPIGRAPH - END

% LANGUAGE - BEGIN
% ARABIC
% for languages that use special fonts, you must provide the typeface that will be used
% \setotherlanguage{arabic}
% \newfontfamily\arabicfont[Script=Arabic]{Amiri}
% \newfontfamily\arabicfontsf[Script=Arabic]{Amiri}
% \newfontfamily\arabicfonttt[Script=Arabic]{Amiri}
%
% in the article, to add arabic text use: \textlang{arabic}{ ... }
%
% RUSSIAN
% for russian text we also need to define fonts with support for Cyrillic script
% \usepackage{fontspec}
% \setotherlanguage{russian}
% \newfontfamily\cyrillicfont{Times New Roman}
% \newfontfamily\cyrillicfontsf{Times New Roman}[Script=Cyrillic]
% \newfontfamily\cyrillicfonttt{Times New Roman}[Script=Cyrillic]
%
% in the text use \begin{russian} ... \end{russian}
% LANGUAGE - END

% EMOJIS - BEGIN
% to use emoticons in your manuscript
% https://stackoverflow.com/questions/190145/how-to-insert-emoticons-in-latex/57076064
% using font Symbola, which has full support
% the font may be downloaded at:
% https://dn-works.com/ufas/
% add to preamble:
% \newfontfamily\Symbola{Symbola}
% in the text use:
% {\Symbola }
% EMOJIS - END

% LABEL REFERENCE TO DESCRIPTIVE LIST - BEGIN
% reference itens in a descriptive list using their labels instead of numbers
% insert the code below in the preambule:
%\makeatletter
%\let\orgdescriptionlabel\descriptionlabel
%\renewcommand*{\descriptionlabel}[1]{%
%  \let\orglabel\label
%  \let\label\@gobble
%  \phantomsection
%  \edef\@currentlabel{#1\unskip}%
%  \let\label\orglabel
%  \orgdescriptionlabel{#1}%
%}
%\makeatother
%
% in your document, use as illustraded here:
%\begin{description}
%  \item[first\label{itm1}] this is only an example;
%  % ...  add more items
%\end{description}
% LABEL REFERENCE TO DESCRIPTIVE LIST - END


% add line numbers for submission
%\usepackage{lineno}
%\linenumbers

\begin{document}
\maketitle

\begin{polyabstract}
\begin{abstract}
En el contexto de la COVID-19 han ocurrido diferentes cambios y tensiones en el Sistema Educativo producto de varios factores. La rápida y fuerte transición hacia una formación a distancia apoyada eminentemente en la tecnología, aflorando debilidades en el Sistema Educativo derivadas de la formación del profesorado, la falta de tecnología, carencia de recursos educativos para ser utilizados en la formación virtual a distancia y falta de credibilidad sobre la eficacia de dicha modalidad formativa. Este artículo pone de manifiesto la inexistencia de diferencias en el dominio tecnológico entre los denominados “nativos” e “inmigrantes” digitales, entre estudiantes y docentes. En general, es problemático asumir que los estudiantes son competentes digitales, mientras que los docentes no han desarrollado dicho dominio. Esta ha ocasionado consecuencias negativas que han llevado a las instituciones educativas a desatender la formación de los estudiantes en competencias digitales. No debe confundirse la mera exposición a la tecnología con tener una alta capacidad para utilizarla. Se concluye defendiendo el término estudiantes digitales, en lugar de nativos digitales. Por ello, se discute que la alfabetización digital no implica solo manejar las herramientas tecnológicas, sino también pensar y solucionar problemas en una sociedad digital.

\keywords{Tecnología educativa \sep Formación del profesorado \sep Competencia digital}
\end{abstract}

\begin{portuguese}
\begin{abstract}
No contexto da COVID-19, ocorreram diferentes mudanças e tensões no sistema educativo como resultado de vários fatores. A rápida e forte transição para um sistema de ensino a distância baseado principalmente na tecnologia trouxe à superfície as deficiências do sistema educativo derivadas da formação de professores, a falta de tecnologia, a falta de recursos educativos a serem utilizados no ensino virtual a distância e a falta de credibilidade no que diz respeito à eficácia dessa modalidade de formação. Este artigo destaca a falta de diferenças no domínio tecnológico entre os chamados "nativos" e "imigrantes" digitais, entre estudantes e professores. Em geral, é problemático assumir que os estudantes são digitalmente competentes, enquanto os professores não desenvolveram tal proficiência. Isso ocasionou consequências negativas que levaram as instituições educativas a negligenciar a formação dos estudantes em competências digitais. A mera exposição à tecnologia não deve ser confundida com ter uma elevada capacidade de utilização. Conclui-se defendendo o termo aprendizes digitais, em vez de nativos digitais. Assim, argumenta-se que o letramento digital não se trata apenas de lidar com ferramentas tecnológicas, mas também de pensar e resolver problemas numa sociedade digital.

\keywords{Tecnologia educacional \sep Formação de professores \sep Competência digital}
\end{abstract}
\end{portuguese}

\begin{english}
\begin{abstract}
In the context of COVID-19 different changes and tensions have occurred in the Education System as a result of several factors. The rapid and strong transition towards a distance learning system mainly based on technology has brought to the surface weaknesses in the Education System derived from teacher training, the lack of technology, the lack of educational resources to be used in virtual distance learning and the lack of credibility regarding the effectiveness of this training modality. This article highlights the lack of differences in technological proficiency between so-called digital "natives" and "immigrants", between students and teachers. In general, it is problematic to assume that students are digitally competent, while teachers have not developed such proficiency. This has led to negative consequences that, in turn, caused educational institutions to neglect the training of students in digital competencies. Mere exposure to technology should not be confused with having a high capacity to use it. It concludes by defending the term digital learners, rather than digital natives. Thus, it is argued that digital literacy is not only about handling technological tools, but also thinking and problem-solving in a digital society.

\keywords{Educational technology \sep Teacher training \sep Digital competence}
\end{abstract}
\end{english}
% if there is another abstract, insert it here using the same scheme
\end{polyabstract}

\section{Introducción}
Sobre la penetración de las Tecnologías de la Información y Comunicación (TIC) en la enseñanza, se han formulado una serie de mitos. Algunos, alrededor de su poder para transformar la educación y crear escenarios formativos de calidad donde los estudiantes aprenderían con un mínimo esfuerzo; otros, respecto a que su utilización potencia las capacidades cognitivas del alumnado, y repercute en que tengan nuevas habilidades como la capacidad multitarea; o el que se analiza en el presente artículo, referido a su capacidad inherente de dominio tecnológico por haber nacido en un momento temporal concreto para el manejo de las TIC. Esto último ha servido para acuñar los términos “nativos” e “inmigrantes digitales.”, a los que han seguido otros como “generación net”, “generación Google” o “generación Google”. Todos esos términos están recibiendo críticas por su falta de solidez científica y que se han puesto de nuevo en duda en estos momentos de pandemia.

\section{¿Nativos e inmigrantes digitales?}
En los comienzos del Siglo XXI, autores como \textcite{prensky_ensenar_2011} o \textcite{tapscott_grown_2009} Tapscott plantearon la idea de la existencia de una generación usuaria uniforme, muy competente y hábil en el manejo de las tecnologías digitales. Esta engloba a las personas nacidas entre 1980 y 1994. Así, se acuña el término de “nativos digitales” y, por oposición, los denominados “inmigrantes digitales”. Tales conceptos, trasladados al terreno educativo, suponen considerar a los estudiantes como habilidosos digitales y a los docentes como ineficaces tecnológicos. Ello se debe al paralelismo que plantean los autores citados entre el hecho de que estos estudiantes han pasado gran parte de su vida rodeados de tecnologías digitales, de ahí se supone su competencia innata para su manejo. 

En cierta medida, lo que se sugiere desde esta perspectiva es que estos niños, desde pequeños, ya presentan unas potencialidades psicomotrices y una coordinación óculo-manual más elevadas si se comparan a las que poseen las generaciones anteriores por su exposición a las tecnologías digitales. Como señala \textcite{granado_palma_educacion_2019}, esto los lleva a indicar que el manejo de las TIC por estos nativos digitales es recibido con naturalidad. Consecuencia de ello “es también un cambio en su actitud ante las pantallas, pasando, en teoría, de ser sujetos pasivos receptores de un mensaje (espectadores) a sujetos más activos y participativos, que generan y son fuente activa de información y comunicación, con capacidad decisoria sobre lo que ven y sobre cómo lo ven" \cite[p. 29]{granado_palma_educacion_2019}.

Esta supuesta exposición tecnológica lleva a algunos autores a sugerir que los “nativos digitales” poseen unas características específicas y diferentes a sus oponentes inmigrantes \cite{lai_technology_2015, underwood_learning_2015, desmurget_fabrica_2020}:

\begin{itemize}
    \item Están alfabetizados digitalmente: son capaces de utilizar intuitivamente una variedad de dispositivos de tecnologías digitales y navegar por Internet.
    \item Se encuentran cómodos con la tecnología: están visualmente alfabetizados y es más probable el uso de Internet para la investigación que una biblioteca.
    \item Tienden a la impaciencia: el dispositivo en particular puede cambiar, pero siempre están conectados.
    \item Son multitareas: pasan rápido de una actividad a otra, a veces realizando varias simultáneamente.
    \item Prefieren la inmediatez.
    \item Demandan respuestas rápidas: más valor a la velocidad que a la precisión. 
    \item Necesitan un aprendizaje experiencial: prefieren aprender haciendo y no diciéndoles qué hacer, y aprender por descubrimiento.
    \item Tienden a ser sociales: se inclinan por actividades que implican interacción social, abiertos a la diversidad y prefieren el trabajo en equipo.
    \item Prefieren la estructura en el aprendizaje: orientación a las metas y materiales estructurados más que ambiguos.
    \item Valoran más las imágenes que el texto: prefieren lo visual y no les gusta leer grandes cantidades de texto.
    \item Otorgan importancia a la comunidad: prefieren trabajar en "cosas que importan" y creen que la ciencia y la tecnología pueden ser utilizadas para resolver.
\end{itemize}

Frente a estas posiciones, diversos autores defienden la idea de que las conceptualizaciones son más bien términos utilizados con fines publicitarios y periodísticos, pero que no han podido ser refrendadas por la investigación científica \cite{ecdlfo, creighton2018digital, granado_palma_educacion_2019, desmurget_fabrica_2020}. Además, su redundancia ha sido lo que se ha repercutido para que sea incorporado en el imaginario de la sociedad digital \cite{masgarcia2018tejido}. Como señala \textcite[p. 134]{creighton2018digital}, “muchos de los estudios que apoyaron el concepto de nativos digitales y/o inmigrantes digitales se basaron únicamente en datos y opiniones anecdóticas”. Por su parte, \textcite{granado_palma_educacion_2019} critica la postura de estos autores al señalar que a sus planteamientos les falta rigor científico y padecen de cierto reduccionismo, al concebir a las “generaciones nativas digitales como un todo uniforme, sin otras consideraciones como las educativas, las culturales, las geográficas, las familiares o las económicas” \cite[p. 32]{granado_palma_educacion_2019}. Por el contrario, se da una gran variación de competencias digitales entre los jóvenes, lo que genera una diversa gradación de comportamientos de los nativos digitales \cite{acostasilva_tras_2017, humanante2019competencias}. Por otra parte, su dominio no es uniforme en jóvenes de la misma edad, sino que existen una serie de factores condicionantes, como el género, la edad, la situación familiar o su ubicación \cite{owens_influence_2017, padilla2016inclusion, romero-tena_challenge_2020, casillas-martin_modelos_2021}. En consecuencia, es incoherente

Todo ello ha llevado a que diferentes autores manifiesten su desacuerdo sobre la utilización del término “nativo digital” para referirse a los estudiantes nacidos en un momento temporal concreto \cite{prensky_ensenar_2011}. Basándose en la falta de evidencia empírica o características sustantivas, se ha propuesto la utilización de otros términos. De forma específica, \textcite{creighton2018digital}, tras revisar 127 artículos publicados entre 1991 y 2014, señala que la terminología nativose e inmigrantes digitales es problemática por dos motivos: “(a) hay poca (si alguna) evidencia empírica incluida para apoyar las afirmaciones hechas sobre los nativos digitales y los inmigrantes digitales en educación superior, y (b) no existe evidencia basada en investigaciones de que debamos centrarnos en la edad como factor determinante en la identificación de estudiantes competentes y experimentados en Tecnologías de la Información y la Comunicación (TIC)" \cite[p. 137]{creighton2018digital}.



\section{Edad}
La separación entre los supuestos nativos e inmigrantes se ha establecido en función de la edad de los sujetos, pero a tal variable de discriminación se le puede poner una serie de objeciones. Una de ellas es que, si fuera real tal competencia digital derivada de que una persona haya nacido en un período temporal concreto, todas esas personas mantendrían características de dominio tecnológico similar. La realidad es que esto no ocurre, como han puesto de manifiesto una diversidad de estudios \cite{romero-rodriguez_media_2019, desmurget_fabrica_2020}. Además, si los “nativos” se asumen competentes en el nivel de manejo instrumental de las tecnologías, en lo que se refiere al manejo para su uso educativo y formativo, su nivel es más bien elemental y reducido \cite{desmurget_fabrica_2020}.

De forma complementaria, en un estudio longitudinal de más de una década con entrevistas realizadas a más de 150 adolescentes, \textcite{boyd_its_2014} indica el perjuicio del discurso de los nativos digitales y de “dar las cosas por sabidas”. Al respecto, aclara que el estar rodeados de pantallas no los está convirtiendo en especiales, y poseen pocas capacidades para la utilización educativa de las tecnologías digitales. 

En esta misma línea, se ubica el trabajo realizado por \textcite{fernandez_enguita_larga_2017} sobre la utilización de las TIC por los españoles en su puesto de trabajo. Se apoyó en la encuesta PIAAC 2012 administrada a 6055 españoles que, a través de las categorias “nativos digitales” (16-32 años), “inmigrantes digitales” (33-45 años), y una tercera que denominan como “preinmigrantes digitales” (46-65 años), permiten afirmar que han tenido una incorporación tardía a las tecnologías. Los resultados encontrados remarcan que los “preinmigrantes digitales” hacen un menor uso de las TIC en el puesto de trabajo, comparado con los “inmigrantes digitales” (incorporación a las TIC mediante un proceso de adaptación en su adultez temprana o media), si bien hacen un mayor uso de estas que los “nativos digitales”. Del mismo modo, el reporte del informe Horizon del 2014, referido a Europa, indica que los niveles de competencia digital en niños y adolescentes europeos es insuficiente. Más recientemente, \textcite{vila-Couñago_2020} analizaron las competencias digitales en un estudio realizado en Galicia con estudiantes de sexto curso de primaria. Uno de sus hallazgos señala que existe un nivel medio de 5.96 en una escala de 0 a 10, con una importante diferencia entre los dos componentes fundamentales que analizaron: conocimientos y capacidades (M=5.10) y actitudes (M=8.04). Por todo lo comentado, será más apropiado denominarlos como “ignorantes digitales”, “estudiantes digitales”, “expertos rutinarios”, “usuarios tecnológicos avanzados”; pero no “nativos digitales”. O, como señala \textcite{granado_palma_educacion_2019}, pueden ser considerados como “nativos digitales”, pero no “nativos digitalizados”. Saben usar las tecnologías, pero posiblemente no todos saben usarlas de forma inteligente; se debe recordar que hay una gran diferencia entre navegar y saber procesar lo que localiza \cite{gazzaley_distracted_2016}.

\subsection{Empleo de dispositivos}

Otro de los mitos que se asocian a estos nativos digitales es el amplio volumen de tecnologías que supuestamente utilizan. Aunque es cierto que invierten un número amplio de horas con las tecnologías, el número que manejan es más bien limitado, así como la diversidad de funciones a las cuales las destinan \cite{lai_technology_2015, castellanos_sanchez_nuevos_2017, prendes_personal_2017, dabbagh_student_2019, lopez_belmonte_alisis_2019, prendes2019university}. En su investigación, \textcite{castellanos_sanchez_nuevos_2017} concluyen que los estudiantes navegan a diario, emplean el correo electrónico y manejan de manera básica ciertas tecnologías como el paquete de ofimática; ahora bien, no se encuentran tan familiarizados con el uso de herramientas actuales, como los blogs o marcadores sociales. De ahí que muchas veces es el própio alumnado el que demanda más formación en competencias digitales \cite{lopez2019perfil}.

Por otra parte, como han sugerido \textcite{ordonez2021competencias} en un estudio longitudinal realizado con estudiantes universitarios de los años académicos 2012/13 al 2018/19, en el caso de los hombres se produce una disminución en el tiempo que dedican a las actividades académicas y un aumento en el tiempo dedicado a jugar. De otro lado, \textcite{lim_preservice_2020} encuentran en su investigación que existe una separación entre la percepción de comodidad y familiaridad respecto a los usos de las herramientas de la web 2.0 y su disposición para ser diseñadores activos y exitosos de sus “entornos personales de aprendizaje”. Resultados similares son hallados en España por \textcite{prendes_personal_2017, prendes2019university}. 

También se ha observado que los resultados de las investigaciones varían según la herramienta empleada para recoger la información. Es decir, si en vez de utilizar cuestionarios se utilizan otros instrumentos como, por ejemplo, la entrevista en profundidad o la observación de la realización de tareas. Como han apuntado \textcite{lim_preservice_2020} han apuntado que existe una fuerte diferencia entre lo que los estudiantes dicen que hacen con la tecnología digital y lo que realmente hacen.

En consecuencia, las sociedades necesitan ciudadanos con un alto sentido crítico, sobre todo en lo referido a la información. Esto implica tener las competencias necesarias para analizar la información y, al mismo tiempo, poseer la competencia para saber validarla. Como señala \textcite[p. 21]{perez-escoda_competencia_2021}, "ser crítico es, por tanto, saber pensar, dominar y explotar las propias competencias para interpretar cualquier contenido y extraer conclusiones propias y no heredadas de otros, o lo que es lo mismo, no repetir opiniones o prejuicios sin comprenderlos".

\subsection{Multitarea}

Otro mito asociado a los “nativos digitales” es su supuesta capacidad multitarea. Este mito alude a “que una persona es capaz de realizar dos o más tareas de procesamiento de información (o pensamiento) simultáneamente y/o al mismo tiempo; es decir, una persona es capaz de realizar múltiples tareas, cada una de las cuales requiere cognición y/o procesamiento de información (p. ej., leer el correo electrónico o conversar con alguien en línea mientras escucha una conferencia en clase o participa en un grupo de trabajo)” \cite[p. 138]{kirschner_myths_2017}. Sin embargo, la realidad, como pone de manifiesto la neurociencia, es que nuestro cerebro no tiene dicha potencialidad y solo somos capaces de realizar dos tareas simultáneas, cuando alguna de ellas está dentro de lo que podríamos considerar como procesamiento automático o inconsciente \cite{sousa2014neurociencia}. Es cierto que sí podemos aprender con la práctica a pasar rápidamente de una tarea a otra, pero ello no significa que tal cambio sea beneficioso y permita la realización de las tareas de forma eficaz.

Por el contrario, un cambio de tarea perjudica la ejecución de esfuerzos cognitivos de alto nivel como las relacionadas con la memoria y el aprendizaje \cite{kirschner_myths_2017, kutscher_ninos_2018}, produciendo al mismo tiempo una disminución de la atención, que es una variable cognitiva de alta significación para el aprendizaje \cite{goleman2013focus}. Por otra parte, es necesario ser conscientes de que trasladarse de una actividad a otra, además de la posibilidad de que sean realizadas de forma ineficaz, conlleva una carga cognitiva adicional para el sujeto. En este sentido, se están desarrollando investigaciones que han revelado que el uso descontrolado en clase de las tabletas, smartphones y ordenadores repercute negativamente sobre el aprendizaje alcanzado por los estudiantes debido al traslado de la atención del profesor al elemento tecnológico \cite{carter2017impact}. Los intentos de hacer múltiples tareas durante el trabajo escolar presentan tres efectos negativos. El primero es el tiempo dedicado a responder a la interrupción; el segundo es la enorme pérdida de tiempo que se dedica a que el estudiante regrese a donde estaba antes de la interrupción y el tercero es que su cerebro se ralentiza por la energía y el estrés invertidos al cambiar de una actividad a otra \cite{kutscher_ninos_2018}.


\subsection{Uso de las TIC por profesorado y alumnado}
Si esta situación de nativos e inmigrantes digitales se traslada al terreno educativo, se alude a la posible existencia de una brecha digital entre estudiantes y docentes. Sin embargo, la realidad, como se verá a continuación por los datos obtenidos en diferentes investigaciones, es diferente al imaginario que se ha creado.

\textcite{gallardo-echenique_lets_2015}, en un metaanálisis de artículos publicados entre 1991 y 2014 respecto a investigaciones enfocadas en el dominio tecnológico de docentes y estudiantes, llegan a la conclusión de que, a pesar de la alta confianza digital y las habilidades digitales de los estudiantes, su competencia digital puede ser mucho menor que la de sus profesores digitales. Resultados equiparables alcanzaron \textcite{recio2020analisis} en un estudio similar.

Pero las investigaciones no solo han mostrado que no se dan tantas diferencias entre profesores y estudiantes, como cabría esperar por el imaginario digital creado \cite{flores2013nativos}, menos aún cuando se refiere a la utilización de las TIC en contextos de formación y no de ocio o en el espacio doméstico \cite{wang2014investigation}. Algunas investigaciones señalan que los docentes son más competentes en el uso de las tecnologías en los procesos de enseñanza-aprendizaje que sus estudiantes \cite{wang2014investigation} y que son competentes respecto al dominio que indican que poseen de sus competencias digitales \cite{perez-escoda_competencia_2021}.

Por otro lado, estudios como el de \textcite{humanante2019competencias} han hecho un diagnóstico sobre las competencias en las TIC de los estudiantes universitarios que ingresan en la Facultad de Ciencias de la Salud de la Universidad Nacional de Chimborazo. La conclusión ha sido que los estudiantes consultados no son completamente competentes en cuanto al conocimiento y uso de las herramientas informáticas, por lo cual recomiendan la formación en competencias digitales relacionadas con la gestión, generación de información y la difusión del conocimiento.

Al mismo tiempo, debe señalarse que otros estudios llaman la atención respecto a la inseguridad que los docentes muestran en su dominio de las TIC y su integración educativa \cite{romero-rodriguez_media_2019}. Por ende, las habilidades tecnológico-didácticas no son iguales en todos los recursos existentes \cite{casillas_martin_profesorado_2020}. 

Por otra parte, la misma heterogeneidad que se produce en los estudiantes, también se encuentra en los docentes \cite{lopez_belmonte_alisis_2019}, así como que el nivel de su capacitación depende de la tecnología a la cual nos estemos refiriendo \cite{rolando2019experiencias}. De igual forma, en este colectivo se produce una diferencia entre el uso que hace de las TIC para el ocio y en el espacio doméstico y el aula \cite{ibieta2017role}. De ahí que se reclame una necesidad de formación tanto para los profesores que están en activo \cite{cabero-almenara_development_2020, mercader_resistencias_2019, rolando2019experiencias, jimenez-hernandez_digital_2020}, como para aquellos que se encuentran en proceso de formación \cite{serrano2017diferencias, infante2021acquisition}.

Cuando se les pregunta a los docentes qué nivel de dominio de las tecnologías tienen sus estudiantes para fines educativos, indican resultados diversos. Así, \textcite{moreno-guerrero_competencia_2020}, en el estudio que realizan con estudiantes de magisterio, manifiestan que muestran un dominio avanzado en las áreas de información, comunicación y creación de contenidos audiovisuales. Sin embargo, presentan bajas competencias en temáticas como la seguridad y la innovación tecnológica. \textcite{dabbagh_student_2019} encuentran que los estudiantes están comenzando a utilizar estas herramientas, pero aún no están completamente integradas en las tecnologías disponibles para la educación. Además, su uso y dominio se centra en la utilización de los ordenadores, los paquetes de Office y los motores de búsqueda de Internet, así como las redes sociales.	

Este bajo dominio de las tecnologías por parte de los estudiantes, está repercutiendo para que se reclame el establecimiento de planes de formación, a fin de aprender más sobre el manejo de las TIC y la evaluación crítica de la información que se presenta en ellas \cite{vsorgo2017attributes}. La investigación “Computer and Information Literacy Study” realizada por la \textcite{international2014}, evalúa los conocimientos de informática y alfabetización informacional de 60.000 estudiantes de octavo grado de 21 sistemas educativos de todo el mundo. En ella se encuentra que, en promedio, el 17\% de los estudiantes no alcanzaron el nivel más bajo de la escala y solo un pequeño 2\% llegan al nivel más alto, que requiere la aplicación del pensamiento crítico mientras se busca información en línea. El estudio también revela que, de los nueve países participantes de la Unión Europea, a excepción de la República Checa y Dinamarca, el 25\% de los estudiantes demuestra bajos niveles de alfabetización informática e informacional. Según sus conclusiones, sería ingenuo esperar que los jóvenes puedan desarrollar las competencias digitales que necesitan sin educación e instrucción formal. Resultados similares encuentran \textcite{garcia-ruiz_alfabetizacion_2020}, en la investigación realizada en diferentes países latinoamericanos donde analizaron la competencia digital de 3.782 estudiantes de entre nueve y doce años de siete países. Por ello, \textcite{denholm} señala que esa creencia en los “nativos digitales” hizo que algunas Universidades de los EE.UU. cometieran el error de cancelar cursos de capacitación digital.

Recuperar la capacitación digital se hace cada vez más necesaria \cite{calatayud2018formacion}. Sobre todo, si tenemos en cuenta que las tecnologías digitales estarán en la base de los nuevos trabajos que se desarrollarán en la cuarta revolución industrial \cite{oppenheimer_salvese_2018}. Ante esta situación, la propia Comisión Europea llama la atención respecto al riesgo latente de que Europa se enfrente a una grave escasez de ciudadanos calificados en la era digital, obstaculizando así el crecimiento y la competitividad en la región \cite{EuropeanCommission}.


\subsection{Reflexiones finales y conclusiones}

La COVID-19 ha traído diferentes cambios y tensiones en el sistema educativo producto de varios factores y de la rápida y fuerte transición hacia una formación a distancia eminentemente apoyada en la tecnología \cite{garcia_aretio_covid-19_2020}. Han aflorado debilidades en el sistema educativo derivadas de la formación del profesorado, la falta de tecnología, la carencia de recursos educativos para ser utilizados en la formación virtual a distancia y la falta de credibilidad sobre la eficacia de dicha modalidad formativa. Pero también, y es crucial en este artículo, ha puesto de manifiesto la inexistencia de diferencias en el dominio tecnológico entre los denominados “nativos” e “inmigrantes” digitales, es decir, entre estudiantes y docentes.

En tiempos de pandemia, se desarrolla el proyecto de investigación “DIFPRORET” \cite{burgos2020difproret}, que analiza las ventajas y desventajas que ha tenido la situación de la COVID-19, los principales retos que se deben superar para abordar la educación en periodos de confinamiento y cuáles serían las propuestas de mejora más relevantes en estas circunstancias. Entre los resultados, se observa que dos de las principales dificultades educativas son la organización del tiempo y la preparación de actividades en contextos de confinamiento. Como propuestas, se sugiere aumentar la formación en competencias individuales tanto de docentes como de estudiantes, el acceso a recursos informáticos y la inversión para paliar la desigualdad social. 

Por tanto, tal falta de formación no se da solo en los inmigrantes docentes, sino también en los nativos estudiantes. Por ejemplo, \textcite{gonzalez2020docencia}, en su trabajo sobre la situación educativa en momentos de pandemia en Colombia, encuentran que, según el 75,60\% de los docentes, los estudiantes no tenían las competencias necesarias para la interacción virtual o remota en sus actividades académicas. Por su parte, \textcite{area2020ensenanza}, en un estudio realizado en el grado de maestros en Educación Primaria en la Universidad de La Laguna, observan que en esta transferencia a lo virtual, el alumnado muestra un alto nivel de satisfacción; se destacan como aspectos positivos los itinerarios flexibles ofertados, la propuesta de las tareas y proyectos para el trabajo autónomo, la tutorización online recibida, así como la adquisición de las competencias digitales necesarias para su futuro profesional. Por tanto, si se adquieren competencias digitales, significa que cuando comienzan no las habían desarrollado. A idénticos resultados llegan \textcite{snoussi2020distance} en los Emiratos Árabes.

Pero esta situación no es nueva. Ya en momentos anteriores, \textcite{parkes_student_2015} evidencian que los estudiantes no estaban lo suficientemente preparados para equilibrar su vida laboral, familiar y social con sus vidas de estudio en un entorno de aprendizaje en línea. También se concluye que los estudiantes están mal preparados para varias competencias de e-learning y competencias de tipo académico. Por su lado, \textcite{araujo-vila_digital_2020} observan poco conocimiento del alumnado universitario respecto a las herramientas en línea, que son completamente necesarias para la formación virtual.  

Todo lo comentado tiene unas fuertes repercusiones para que cambien los procesos de formación. Como señala \textcite{desmurget_fabrica_2020} con cierta ironía,

\begin{quote}
    si se repite con la frecuencia necesaria que, por su apabullante dominio de lo digital, las nuevas generaciones tienen un cerebro y unas formas de aprender diferentes, la gente acabará por creerlo. Una vez que lo crea, su visión de la infancia, de la adquisición de conocimientos y del sistema educativo cambiará por completo. \cite[p. 39]{desmurget_fabrica_2020}
\end{quote}

 Es problemático asumir el concepto de nativos digitales, pues entraña el riesgo de la inacción. Como sugieren \textcite[p. 153]{fernandez_enguita_larga_2017}, “los alumnos necesitan que la escuela fomente y oriente su alfabetización digital; los profesores tienen la obligación profesional de formarse para hacerlo".  De acuerdo a \textcite[p. 137]{kirschner_myths_2017} si el concepto de “nativo digital” no está tan claro y las cualidades conseguidas tampoco, debe tenerse “cuidado con las afirmaciones de cambiar la educación porque esta generación de jóvenes es fundamentalmente diferente de las generaciones anteriores de estudiantes en cómo aprenden/pueden aprender debido a su uso de los medios”. Por ello, en el Marco de Competencia Digital para Ciudadanos (y por tanto para estudiantes) “DigComp” se ha planteado que una de las áreas competenciales por adquirirse es la de “Seguridad”, lo cual implica que se dominen las siguientes competencias: protección de dispositivos, protección de datos personales e identidad digital, protección de la salud y protección del entorno \cite{carretero_digcomp_2017, vuorikari_digcomp_2022}. En la misma línea, \textcite{luri2020escuela} ha publicado un libro con un título sugerente: “La escuela no es un parque de atracciones”, reclamando una transformación de la escuela donde se fomente la adquisición del conocimiento y se potencie el esfuerzo, la concentración y la atención.

Todo esto no significa que la educación no deba cambiar: debe hacerlo, entre otros motivos, porque, como institución, debe centrarse en la formación de la ciudadanía para afrontar los retos que la sociedad le presenta y le pueda presentar. Además, no se debe olvidar que la Sociedad del Conocimiento supone retos diferentes a los surgidos en la sociedad industrial o posindustrial. Por ello, tales transformaciones deben hacerse apoyándonos en bases científicas, no en titulares publicitarios.

La falacia de los nativos digitales tiene también otras consecuencias negativas. Entre ellas, el hecho de que las instituciones educativas están olvidando la necesidad de facilitar la formación en competencias digitales de los estudiantes y profesorado; sobre todo, cuando la sociedad futura será más tecnológica en todas sus dimensiones, incluyendo la laboral. Saber manejar las tecnologías en este mundo requiere competencias más elevadas que el saber enviar un Whatsapp o subir una foto a Instagram; implica la necesidad de crear materiales en diferentes formatos, compartirlos, transformarlos y evaluarlos.

Esta falta de formación en competencias digitales se refleja no solo en usos centrados en el ocio, sino también en la adquisición de conductas negativas, como la adicción a diferentes tecnologías y a las redes sociales que empieza a ser preocupante en jóvenes y adolescentes \cite{valencia_ortiz_use_2019, bernabeu_brotons_adicciones_2020, gonzalez-cortes_semana_2020, orosco_fabian_adolescentes_2020, ruiz-palmero_estudio_2021}. Tal es la situación que se presenta en diferentes marcos competenciales avalados por organismos oficiales, como la UNIÓN EUROPEA \cite{cabero-almenara_evaluation_2020}. En el Marco DigCompEdu, de Competencia Digital Docente, se denomina “Uso responsable y bienestar”, de manera que el docente se centra en que el estudiante sepa cómo comportarse de manera segura y responsable en línea \cite{cabero-almenara_marco_2020}. También puede referirse en este marco el fuerte desconocimiento que presentan sobre las herramientas antiplagio, las medidas preventivas y sus competencias para evitarlo, como han sugerido \textcite{cebrian-robles_conocimiento_2020} en un estudio realizado con estudiantes universitarios españoles y portugueses. 

Finalmente, como señala \textcite[p. 33]{granado_palma_educacion_2019}, apuntar que “ser de la generación nativa digital no faculta para ser competente digital". No podemos confundir la mera exposición a la tecnología con tener una alta capacidad para utilizarla. La alfabetización digital no implica solo manejar las herramientas tecnológicas, sino pensar digitalmente.



\printbibliography\label{sec-bib}
% if the text is not in Portuguese, it might be necessary to use the code below instead to print the correct ABNT abbreviations [s.n.], [s.l.]
%\begin{portuguese}
%\printbibliography[title={Bibliography}]
%\end{portuguese}


%full list: conceptualization,datacuration,formalanalysis,funding,investigation,methodology,projadm,resources,software,supervision,validation,visualization,writing,review
\begin{contributors}[sec-contributors]
\authorcontribution{Julio Cabero-Almenara}[conceptualization,supervision,writing,review]
\authorcontribution{Rubicelia Valencia-Ortiz}[conceptualization,supervision,writing,review]
\authorcontribution{Carmen Llorente-Cejudo}[conceptualization,supervision,writing,review]
\authorcontribution{Antonio de Padua Palacios-Rodríguez}[conceptualization,supervision,writing,review]
\end{contributors}

\end{document}

