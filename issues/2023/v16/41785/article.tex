% !TEX TS-program = XeLaTeX
% use the following command:
% all document files must be coded in UTF-8
\documentclass[spanish]{textolivre}
% build HTML with: make4ht -e build.lua -c textolivre.cfg -x -u article "fn-in,svg,pic-align"

\journalname{Texto Livre}
\thevolume{16}
%\thenumber{1} % old template
\theyear{2023}
\receiveddate{\DTMdisplaydate{2022}{11}{14}{-1}} % YYYY MM DD
\accepteddate{\DTMdisplaydate{2023}{1}{29}{-1}}
\publisheddate{\DTMdisplaydate{2023}{2}{9}{-1}}
\corrauthor{Marco Lovón}
\articledoi{10.1590/1983-3652.2023.41785}
%\articleid{NNNN} % if the article ID is not the last 5 numbers of its DOI, provide it using \articleid{} commmand 
% list of available sesscions in the journal: articles, dossier, reports, essays, reviews, interviews, editorial
\articlesessionname{articles}
\runningauthor{Lovón} 
%\editorname{Leonardo Araújo} % old template
\sectioneditorname{Hugo Heredia Ponce}
\layouteditorname{Thaís Coutinho}

\title{La necesidad de decolonialismo lingüístico sobre el subtitulaje en inglés}
\othertitle{A necessidade de decolonialismo linguístico na legendagem em inglês}
\othertitle{The need for linguistic decolonialism on English subtitling}
% if there is a third language title, add here:
%\othertitle{Artikelvorlage zur Einreichung beim Texto Livre Journal}

\author[1]{Marco Lovón~\orcid{0000-0002-9182-6072}\thanks{Email: \href{mlovonc@unmsm.edu.pe }{mlovonc@unmsm.edu.pe}}}
\affil[1]{Universidad Nacional Mayor de San Marcos, Lima, Perú.}


\addbibresource{article.bib}
% use biber instead of bibtex
% $ biber article

% used to create dummy text for the template file
\definecolor{dark-gray}{gray}{0.35} % color used to display dummy texts
\usepackage{lipsum}
\SetLipsumParListSurrounders{\colorlet{oldcolor}{.}\color{dark-gray}}{\color{oldcolor}}

% used here only to provide the XeLaTeX and BibTeX logos
\usepackage{hologo}

% if you use multirows in a table, include the multirow package
\usepackage{multirow}

% provides sidewaysfigure environment
\usepackage{rotating}

% CUSTOM EPIGRAPH - BEGIN 
%%% https://tex.stackexchange.com/questions/193178/specific-epigraph-style
\usepackage{epigraph}
\renewcommand\textflush{flushright}
\makeatletter
\newlength\epitextskip
\pretocmd{\@epitext}{\em}{}{}
\apptocmd{\@epitext}{\em}{}{}
\patchcmd{\epigraph}{\@epitext{#1}\\}{\@epitext{#1}\\[\epitextskip]}{}{}
\makeatother
\setlength\epigraphrule{0pt}
\setlength\epitextskip{0.5ex}
\setlength\epigraphwidth{.7\textwidth}
% CUSTOM EPIGRAPH - END

% LANGUAGE - BEGIN
% ARABIC
% for languages that use special fonts, you must provide the typeface that will be used
% \setotherlanguage{arabic}
% \newfontfamily\arabicfont[Script=Arabic]{Amiri}
% \newfontfamily\arabicfontsf[Script=Arabic]{Amiri}
% \newfontfamily\arabicfonttt[Script=Arabic]{Amiri}
%
% in the article, to add arabic text use: \textlang{arabic}{ ... }
%
% RUSSIAN
% for russian text we also need to define fonts with support for Cyrillic script
% \usepackage{fontspec}
% \setotherlanguage{russian}
% \newfontfamily\cyrillicfont{Times New Roman}
% \newfontfamily\cyrillicfontsf{Times New Roman}[Script=Cyrillic]
% \newfontfamily\cyrillicfonttt{Times New Roman}[Script=Cyrillic]
%
% in the text use \begin{russian} ... \end{russian}
% LANGUAGE - END

% EMOJIS - BEGIN
% to use emoticons in your manuscript
% https://stackoverflow.com/questions/190145/how-to-insert-emoticons-in-latex/57076064
% using font Symbola, which has full support
% the font may be downloaded at:
% https://dn-works.com/ufas/
% add to preamble:
% \newfontfamily\Symbola{Symbola}
% in the text use:
% {\Symbola }
% EMOJIS - END

% LABEL REFERENCE TO DESCRIPTIVE LIST - BEGIN
% reference itens in a descriptive list using their labels instead of numbers
% insert the code below in the preambule:
%\makeatletter
%\let\orgdescriptionlabel\descriptionlabel
%\renewcommand*{\descriptionlabel}[1]{%
%  \let\orglabel\label
%  \let\label\@gobble
%  \phantomsection
%  \edef\@currentlabel{#1\unskip}%
%  \let\label\orglabel
%  \orgdescriptionlabel{#1}%
%}
%\makeatother
%
% in your document, use as illustraded here:
%\begin{description}
%  \item[first\label{itm1}] this is only an example;
%  % ...  add more items
%\end{description}
% LABEL REFERENCE TO DESCRIPTIVE LIST - END


% add line numbers for submission
%\usepackage{lineno}
%\linenumbers


\usepackage{enumitem}
\newlist{myQuoteEnumerate}{enumerate}{2}% Set max nesting depth
\setlist[myQuoteEnumerate,1]{label=(\arabic*)}% Use numbers for level 1
\setlist[myQuoteEnumerate,2]{label=(\alph*)}%   Use letters for level 2

\newenvironment{MyQuote}{%
    \begin{myQuoteEnumerate}[resume=*,series=MyQuoteSeries]%
    \item \begin{quote}%
}{%
    \end{quote}%
    \end{myQuoteEnumerate}%
}%


\begin{document}
\maketitle

\begin{polyabstract}
\begin{abstract}
Las series de televisión y las películas cinematográficas se transmiten cada vez en inglés por medio de audios y escritos subtitulados, con lo que se masifica el empleo, la selección y el aprendizaje de la lengua inglesa. El objetivo de este artículo es desentramar desde el decolonialismo lingüístico la manera en que se impone socioculturalmente el inglés como una opción natural. Metodológicamente, se trata de un estudio cualitativo, que recurre a entrevistas realizadas a traductores, profesores de idiomas y estudiantes peruanos. El artículo concluye que la demanda de subtítulos facilita la enseñanza del inglés como práctica colonial por lo que se proponen alternativas, algunas básicas y otras profundas, frente a su masiva difusión.

\keywords{Ideologías lingüísticas \sep Decolonialismo \sep Subtitulaje \sep Doblaje \sep Inglés}
\end{abstract}

\begin{portuguese}
\begin{abstract}
    Séries de televisão e filmes cinematográficos são cada vez mais transmitidos em inglês por meio de áudios e escritos legendados, aumentando assim massivamente o uso, seleção e aprendizagem da língua inglesa. O objetivo deste artigo é desvendar, a partir da perspectiva do decolonialismo linguístico, a forma como o inglês é imposto socioculturalmente como uma escolha natural. Metodologicamente, este é um estudo qualitativo, utilizando entrevistas com tradutores, professores de idiomas e estudantes peruanos. O artigo conclui que a demanda por legendas facilita o ensino do inglês como uma prática colonial e propõe alternativas, algumas básicas e outras profundas, para sua difusão massiva.
    
\keywords{Ideologias linguísticas \sep Decolonialidade \sep Legendagem \sep Dublagem \sep Inglês}
\end{abstract}
\end{portuguese}

\begin{english}
\begin{abstract}
Television series and movies are increasingly being broadcast in English through subtitled audio and written material, thus increasing the use, selection and learning of the English language. The objective of this article is to unravel, from the perspective of linguistic decolonialism, how English is socioculturally imposed as a natural choice. Methodologically, this is a qualitative study, using interviews with Peruvian translators, language teachers and students. The article concludes that the demand for subtitles facilitates teaching English as a colonial practice and proposes decolonizing alternatives, some descriptive and other critical, in the face of its massive diffusion.

\keywords{Linguistic ideologies \sep Decolonialism \sep Subtitling \sep Dubbing \sep English}
\end{abstract}
\end{english}
% if there is another abstract, insert it here using the same scheme
\end{polyabstract}

\section{Introducción}
Este estudio cualitativo tiene como objetivo desentramar desde el decolonialismo lingüístico el posicionamiento que cobra el inglés en torno a la subtitulación de películas y series. Los estudios de la sociolingüística crítica consideran que es necesario cuestionar los espacios que mantienen estructuras sociales dominantes. Con este trabajo se evalúa un discurso que gira en torno a elementos de la modernidad y el régimen global del conocimiento \cite{heiss_pluriversalizar_2018}. En el mundo, muchas personas se inclinan por aprender inglés. Los especialistas consideran que ocurre una aculturación hacia mundo anglosajón. Los migrantes que han llegado a países de habla inglesa se han adaptado a la cultura receptora, lo que implica la adquisición de la lengua inglesa y sus patrones culturales, apreciados como naturales \cite{brown_principles_1994}.

Desde el decolonialismo, se cuestiona tanto la orientación moderno-colonial de la transmisión y enseñanza del inglés como lengua suprema como las políticas educativas que lo colocan como “opción natural”. Y, además, se busca alternativas a las epistemes hegemónicas \cite{jaramillo2013}. Por eso, la presente investigación toma una postura decolonial sobre la promoción y enseñanza del inglés (ELT).

Para ello, se analizan los discursos manifestados en el trabajo de entrevistas realizadas a profesores y aprendices de inglés de institutos de idiomas, así como de la carrera de traducción de universidades privadas peruanas sobre los subtítulos de películas.

El presente artículo propone un marco conceptual, una metodología, un análisis, una discusión y unas conclusiones respectivas al trabajo.


\section{Marco conceptual}

\subsection{Decolonialismo lingüístico}
El decolonialismo como movimiento busca la desconexión de los modelos, los métodos y las significaciones instituidos en la lente montada por la modernidad, cuyo análisis se concentra en examinar la legitimidad y el funcionamiento de esa modernidad. Siguiendo esta propuesta, el decolonialismo lingüístico consiste en desentramar y, sobre todo, desprenderse de las representaciones, las metodologías y los signos que se usan como fundamentos ideológicos de la superioridad lingüística y la racialización lingüística, que sirven para sostener que ciertas lenguas, ciertos hablantes y ciertos usos corresponden a la modernidad, mientras que otros no se aceptan como tales. Estas construcciones modernizantes son secuelas generalmente del colonialismo que impuso determinados significantes y significados en la vida social y en la convivencia lingüística, y que traen consecuencias ideológicas en la actual vida social, como en la pedagogía \cite{baum2019decir}, donde en algunos espacios académicos se enseña sesgadamente, verbigracia, que las lenguas indígenas son dialectos o que sus hablantes manejan lenguas incompletas o desfasadas, por lo que no sirven para la ciencia. Así, por ejemplo, los posicionamientos y las percepciones hacia el inglés, el castellano o el francés que están arraigados en legados coloniales, imperiales, mercantilistas o empresariales provocan desajustes, discriminaciones y genocidios lingüísticos de hablantes o grupos marginalizados, especialmente de los no-occidentalizados. Para \textcite{skutnabb-kangas_linguistic_2000}, las lenguas desaparecen por genocidios lingüísticos y en la escuela no se empoderan otras lenguas y culturas más allá de la pedagogía eurocéntrica. Según \textcite[p.2]{despagne__ideologias_2021}, en México, a pesar de ser un país con más de 60 lenguas indígenas, el castellano es la lengua de prestigio, como se ha impuesto en los diversos ámbitos comunicativos, lo que da paso a un genocidio lingüístico “al otorgar al español el estatus de única lengua nacional de facto y al minorizar a las lenguas indígenas”. En el Perú, y parece ser el caso de otros países de América del Sur, quien aprende inglés y no una lengua originaria se considera como ciudadano moderno o civilizado \cite{gutierrez_english_2022}, y la sociedad atribuye cualidades de superación y éxito a esta persona, mientras que los hablantes de lenguas originarias, e incluso aprendientes de estas lenguas, son vistos como incivilizados o anacrónicos, y exterminables \cite{bartolome_pobladores_2004, harambour_barbarie_2019}, por lo que terminan por olvidar o mudar su lengua. En este último caso, no se cuenta con los medios de conocimiento-difusión y promoción equitativos que posibiliten su fomento y aceptación lingüística. En Colombia, por ejemplo, se piensa que el inglés está asociado con civilización y poder, pues da ventajas económicas \cite{guerrero2010english}. Entre los políticos peruanos, las lenguas originarias se perciben como inútiles y sus hablantes se consideran desprovistos de derechos \cite{lovon-cueva_compuestos_2021}. El hecho de que las diversas informaciones y eventos socioculturales se encuentren dispuestos en inglés o se privilegie la lengua, en su traducción, uso y transmisión, no resulta normal; la cuestión está en revelar qué hay detrás de toda su producción lingüística y pensar incluso en una ruptura. Según \textcite{endo2010chapter}, en el mundo anglosajón, especialmente en los Estados Unidos, se cree que el inglés es la lengua de los ciudadanos civilizados, particularmente si emplean el inglés estándar, que permite obtener ventajas económicas y acceder a movilizaciones sociales. Al respecto de la modernidad, \textcite{quijano2000colonialidad} considera que esta se entiende como el logro del curso civilizatorio básicamente europeo u occidental. Cabe señalar que \textcite[p.406]{pedroza2006modernidad} especifican que “[l]a modernidad supone una oleada de transformaciones en el ámbito mundial: la industrialización, el desarrollo de los medios masivos de comunicación, el surgimiento de la clase empresarial, la burocratización, la secularización y la producción en masa, entre otras cosas”.

Las acciones hechas para deconolizar se conocen como decolonización y su resultado es la decolonialidad, es decir el desprendimiento y la reconstrucción de las bases que sostienen la colonialidad, que refiere a la lógica y las acciones de la civilización occidentalizante, también llamada matriz colonial del poder, aunque en un sentido lato se emplea el término decolonialidad para referir al movimiento y el pensamiento crítico que atiende la búsqueda de un nuevo orden. En este proceso, el análisis decolonial consiste en revelar los efectos de poder que traen ciertas prácticas de la colonialidad en el mundo, muchas de las cuales transgreden fronteras y se asimilan en cualquier país, a través de la globalización, la migración, los medios de comunicación y las redes sociales, para desprenderse de tales consecuencias, así como de su lógica. Al respecto, la decolonización epistémica desmonta el proyecto cultural moderno, europeo y estadounidense, que sitúa saberes lingüísticos y culturas como superiores a otras. De esta manera, pone al descubierto el encubrimiento entramado en el constructo modernidad/colonialidad. Este tipo de análisis también se preocupa especialmente por encontrar alternativas de desvinculación con la matriz colonial del poder.

Asimismo, cabe señalar que dentro de esta perspectiva puede considerarse que el lenguaje no es únicamente un instrumento para la descripción de la realidad, sino también una herramienta potente que la construye \textcite{foucault_palabras_2001}. Los lingüistas decolonizadores además sostienen que la lengua no solo es biológica o social, sino que siempre es política \cite{baum2019decir, garces__colonialidad_2009}. Al denominar, nombrar y decir se realizan hechos políticos que sitúan a las personas en lugares de conocimientos. La lengua es un constructo identitario y sutil, anclado en la configuración de la subjetividad y la estructuración de universos con sentido geopolítico. Por eso, la decolonialización busca politizar la lengua. En vez de ocultar una supuesta igualdad, lo que hay es una conflictividad lingüística que se presenta en injusticias y desigualdades, tal como sostiene \textcite{baum2019decir}, lo cual nos permite ver en este trabajo la manera en que el inglés se superpone a través de medios como el subtitulaje que impulsan empresas académicas y económicas en el mundo frente a tantas otras lenguas no hegemónicas.


\subsection{Sobre la enseñanza del inglés y el uso de subtítulos en inglés}
El aprendizaje y la enseñanza de la lengua inglesa guardan un sesgo colonial que se manifiesta en la manera de pensarla. Tras una aceptación y concepción ingenua sobre la interculturalidad, como sugiere \textcite{baum2019decir}, el aprendizaje y la enseñanza del inglés se impone con la difusión de ideas, creencias o representaciones, así como en su selección, desde un centro de poder. Las horas de aprendizaje de una lengua extranjera pueden estar inclinadas hacia el inglés. Para \textcite[p. 5]{baum2019decir}, el inglés ha devenido en lengua de prestigio, segregacionista, elitista y relacionada con el dinero; y es “una herramienta de acceso, ascenso social, promoción profesional, pertenencia académica”, y apreciada de fácil aprendizaje estructural. En relación con la tradición occidental, se crea la idea que abre la puerta hacia la modernidad. Ello se debe, asimismo, al colonialismo, al marketing y al capitalismo que están imbricados a la lengua inglesa \cite{baum2019decir}.  Su visión del mundo es la que se atribuye a los Estados Unidos, lo que muestra un caso de colonialidad, dado que se impone una lectura única válida del mundo \cite{despagne_difficulties_2010}, pero que también invita a seguir el modelo británico \cite{cardenas_cornelio_actitudes_2022}, apreciado como un éxito regional y mundial, donde su “acento” es valorado entre los británicos y aprendientes foráneos como prestigioso \cite{agha_social_2003}. Precisamente, las representaciones hacia la lengua inglesa se encuentan arraigadas en los legados coloniales \cite{despagne_modernidad_2016} e institucionales. Algunas, siguiendo a \textcite{despagne_modernidad_2016}, son las siguientes: “los ciudadanos del mundo global son los que hablan inglés”, “los estudiantes con competencia alta en inglés forman parte de la modernidad, en contraste con los aprendices de competencia más baja que son considerados como incivilizados”, “dar una conferencia en inglés tiene más valor que dar una conferencia en otra lengua”, que descansan, en síntesis, en la expresión: “si no hablas inglés, no formas parte de la modernidad”. 

La enseñanza del inglés genera una tradición aprendida, importada, anglofilizante \cite{despagne_modernidad_2016}. Algunos docentes en varias partes del mundo propagan su enseñanza-aprendizaje, aun cuando no han vivido o viajado a un país de habla inglesa naturalizan el discurso anglofílico, tanto quienes son profesores de idiomas como los que no. Los promotores del inglés crean prácticas y estructuras que asisten a su supremacía, como la difusión y uso de subtitulajes en la lengua, la rendición de exámenes internacionales, la traducción de la literatura y su empleo en redes sociales. La academia y la educación son espacios donde se hace visible la extensión o expansión de la lengua inglesa, dejando atrás otras lenguas, tanto las vernáculas como las internacionales de menor valor \cite{despagne_modernidad_2016}. La docencia exige un nivel de inglés en los estudiantes dada la producción académica en la lengua (los libros, los artículos científicos, los ensayos están escritos en inglés o se requiere el inglés en su redacción). En algunas universidades para pasar a la especialidad se requiere mostrar el dominio del inglés. La enseñanza de la lengua inglesa es presentada institucionalmente como apolítica y neutral \cite{guerrero2009english}, a pesar de que encumbre ideologías y estructuras que reproducen y consolidan una práctica de poder desigual \cite{phillipson1992linguistic}. En el África, el inglés se presenta como opción única en los departamentos de muchas universidades \cite{thiongo_decolonising_1994}. Lograr incluso que un estudiante sepa inglés requiere legitimar instrumentos de aprendizaje o llegada hacia esta lengua.

La idea de hablar inglés ha ido ocasionando que los estudiantes, los profesores y los promotores de su enseñanza encuentren medios para su adquisición. Así, se cree que por medio de la subtitulación uno se va acercando más al idioma \cite{torralba-miralles_uso_2020}. La pedagogía va recomendando el uso de los subtitulajes en películas y series. La demanda por los subtítulos va creciendo, frente a los doblajes, puesto que se cree que estos son un obstáculo para asimilar el inglés. En algunas instituciones, por ejemplo, los estudiantes sugieren su uso, pues consideran que así mejoran sus habilidades y su dominio, reclamando a los docentes que los inserten en la enseñanza \cite{jude2019subtitulacion}.

Una alternativa al distanciamiento del inglés es su cuestionamiento e incluso abandono, para convertir a hablantes únicamente en nativos monoculturales de esta lengua. La decolonización comprende una manera de desprendimiento o (des)aprendizaje \cite{palermo_para_2014} de ideas preconcebidas o establecidas como únicas. Por tanto, se busca desaprender lo impuesto y asumido por la colonización/modernidad.

Cabe señalar que el aprendizaje del inglés es un privilegio que detenta la educación privada, pues las escuelas públicas están asociadas y atribuidas con niveles bajos, sobre todo en regiones como Latinoamérica \cite{ricoy2016ensenanza}. Ser hablante inglés está en manos de quien puede más en este mundo: nacer en un país angloparlante, contar con residencia, tener familia inglesa, contar con los recursos económicos para estudiar, acceder a procesos de socialización y migración hacia zonas donde predomina el inglés. Los países de habla inglesa atraen a las personas porque la modernidad ha hecho creer que el bienestar y el desarrollo individual, como la búsqueda de una calidad de vida, se logra o es posible solo viviendo ahí. Incluso, los profesores de inglés prefieren buscar un trabajo en el sector privado, que en el público \cite{huaman_rosales_causas_2021}. Estas visiones que se presentan en ciertos contextos regionales y diversas sociedades, pueden darse en casos donde el francés o el español tienen un poder similar en regiones de África o Asia, donde aprender dichas lenguas también se ve como una posibilidad para mejorar las condiciones laborales y sociales. En Marruecos y otros países como Argelia, Ruanda y Mali, el francés es la lengua prestigiosa que se aprende. Se cree que para el 2050 África representará el 85\% de los francófonos en el mundo \cite{rfielfuturo}. El español no solo se emplea tradicionalmente en zonas como Zamboanga en Filipinas o en Israel, sino que en otras partes del Asia es un atractivo para los aprendices, con particularidad en las universidades para el establecimiento futuro de negocios o intercambios comerciales o simplemente turísticos, como son los casos en Corea del Sur, Japón o China, que cuentan con docentes de enseñanza de español como segunda lengua.
 


\section{Metodología}
Este es un estudio cualitativo de tipo interpretativo. Los datos se interpretan en relación con el marco del decolonialismo. El trabajo de \textcite{baum2019decir}, concentrado en el análisis de la pedagogía en inglés, permite extraer una serie de interpretaciones decoloniales, que, en este trabajo, se agrupan en dos: desentrame epistémico, que permite estudiar la lógica de la matriz de poder; y el desprendimiento epistémico-práctico, con el que se proponen opciones de desvinculamiento, reconstrucción, desobediencia y ruptura de la red de la inglenización, es decir, de la producción, del consumo y de la difusión del inglés.

Se seleccionan y describen los discursos manifestados en entrevistas con profesores y aprendices de inglés como lengua extranjera, y de profesores y estudiantes de la carrera de traducción de universidades privadas peruanas.

Respecto de los entrevistados, sus nombres han sido cambiados para preservar su identidad a pedido de los colaboradores. En la \Cref{tab01} se presentan los datos necesarios. Para el análisis, E significa entrevistador y C colaborador. Si bien se trabajó con cincuenta entrevistados entre noviembre de 2021 y octubre de 2022, se ha optado por elegir una muestra, dada la limitación del espacio del artículo, que ilustra la coincidencia discursiva sobre el tema en cuestión. Los entrevistados proceden de las regiones de Lima, Iquitos y Tacna. Lima se escogió por ser una ciudad centralista con diversidad de centros de idiomas, e Iquitos y Tacna, por ser ciudades transfronterizas, con presencia de turistas y emigrantes latinos o hablantes de lenguas extranjeras.

\begin{table}[h!]
\centering
\begin{threeparttable}
\caption{Entrevistados}
\begin{tabular}{llllll}
\toprule
\multicolumn{1}{p{1cm}}{Colabo\-rador} & Nombre & Condición & Institución & Edad & Nacionalidad \\
\midrule
C-1 & Ana 			Romero & Profesora 			de inglés & Centro 	de idiomas & 43 & Lima \\
C-2 & Sandro 	    Loayza & Profesora 			de inglés & Centro 	de idiomas & 35 & Tacna \\
C-3 & Pedro 		Salinas & Estudiante 		de inglés & Centro 	de idiomas & 25 & Lima \\
C-4 & Carmela 		López & Estudiante 			de inglés & Aprendizaje	en línea & 18 & Iquitos \\
C-5 & Rosa 			Pita & Traductora & Freelance & 27 & Lima \\
C-6 & Rosa 			Cueva & Profesora 			de traducción & Universidad & 38 & Tacna \\
C-7 & América 		Arias & Profesora 			de inglés & Colegio & 29 & Tacna \\
C-8 & Teo 			Salamanca & Estudiante 		de inglés & Universidad & 20 & Iquitos \\
C-9 & Caro 			Colque & Estudiante 		de inglés & Universidad & 21 & Tacna \\
C-10 & Isabel 		Tello & Profesora 			de traducción & Universidad & 40 & Lima \\
\bottomrule
\end{tabular}
\label{tab01}
\source{Elaboración propia}
\end{threeparttable}
\end{table}

Las entrevistas fueron realizadas, por un lado, a través de la aplicación WhatsApp, que posibilitó plantear preguntas por medio escrito y por audio, así como recoger respuestas por ambos medios. Se decidió optar por una metodología digital, que permite recoger los pareceres de manera espontánea, sobre todo por las restricciones de pandemia; por otro lado, se realizaron visitas a lugares, para también recoger información cara a cara con los informantes.  Como se ve, se buscó una diversidad de actores sociales relacionados con los subtitulajes de películas y series, la traducción y la enseñanza del inglés. En la transcripción se siguen los principios éticos de preservar la información y registrarla como se manifestó.


\section{Análisis decolonial}

En este apartado se presenta el análisis decolonial a partir de los comentarios ofrecidos por los entrevistados. Para ello, se recurre a analizar la lógica colonial que está detrás de los subtitulajes; de esta manera, se logra un desentrame epistémico. Luego, se proponen alternativas que encaminan el desprendimiento epistémico y práctico de subtitular películas y series en inglés.


\subsection{Desentrame epistémico}
Los difusores del inglés han hecho creer que esta lengua es la única válida para la lectura del mundo. Para los estudiantes entrevistados, como C-3 y C-8, el subtitulaje mejora sus habilidades lingüísticas y su futuro, pues como hispano y peruano en un centro público considera que su inglés es de bajo nivel, y su destino lo atribuye al idioma. Con los subtítulos se cree que los aprendices incrementan las destrezas de lectura, escritura y pronunciación. A la vez que se legitima que la lengua determina sus actividades académicas y laborales. Tanto así que en este caso se afirma que “sin inglés no se puede hacer nada”. Hay un determinismo en su visión. Y ello está respaldado por lo que promueven los profesores de idiomas. Son los docentes e investigadores que fomentan un monolingüismo del inglés a nivel global \cite{despagne__ideologias_2021}. Por otra parte, debe notarse que los doblajes son mal vistos, porque no acercan al aprendiente a captar la pronunciación inglesa. Dicho de otro modo, alejan al aprendiz de la fuente y lengua de poder. Los estudiantes piensan que escuchar en español impide su adquisición. El tener un “inglés masticado”, es decir con influencia del castellano, hace ver que el español está situado como una lengua extranjera interferente. Ser hispano y hablar inglés con acento castellano genera estereotipos que identifican y minimizan al hablante, pues repercute en su imagen negativamente, y ello se debe al prestigio del inglés y sus hablantes. \textcite{moncada_vera_hegemoniingles:_2018} señalan que los profesores de inglés perciben que la expansión de esta lengua es beneficiosa, porque otorga prestigio y acceso al capital. Así mismo, para el caso, se presupone que el aprendiz cobra mayor prestigio si habla como nativo \cite{avalos2017ser}. Colonialmente, se cree que los hablantes de inglés son más superiores que las personas que hablan otras lenguas, incluso se piensa que el inglés de unos es más legítimo que el de otros, como sucede entre los hablantes de inglés del Reino Unido frente a los de Nigeria, porque su habla registra préstamos lingüísticos de lenguas locales. Sin embargo, no es la lengua en sí misma la que trae ventajas, sino las representaciones que se hacen sobre ella. Estos cuestionamientos hacia el subtitulaje en inglés y la lengua pueden revelarse en \ref{myquote01}: \\

\begin{MyQuote}\label{myquote01}
\textbf{E:} ¿Por qué usas subtítulos en inglés cuando ves películas?

\textbf{C-3:} Me ayudan a mejorar mi inglés. Sin ellos no aprendería. Y sin inglés no se puede hacer nada. Todos lo saben, hasta los profesores del instituto lo dicen. En mi colegio nada aprendí. Los colegios públicos son malazos, más aún si no tienen buenos docentes.

\textbf{C-8:} El inglés sirve para leer, hablar, escuchar humor. Es la lengua del futuro, para trabajar, ser aceptado en el trabajo, en una beca, viajar, vivir afuera, chatear… Usar doblajes es pérdida de tiempo, no puedes aprender el inglés con doblajes. Te acuerdas más de las voces españolas, y olvidas el acento gringo, o no lo identificas. No ayuda a la pronunciación. No es bueno tampoco, porque el español cuando se te queda se te queda. Y los gringos prefieren que hables como gringo, no como hispano con inglés masticado.
\end{MyQuote}


La matriz colonial del poder, asimismo, hace creer que todos los diversos medios que lleven a la asimilación del inglés son esenciales. Otros estudiantes entrevistados, como C-4 y C-9, consideran que el subtitulaje es la fuente que le permite estar al nivel de la competencia lingüística de sus compañeros. Al considerar que todos saben inglés y ven películas y series en el idioma, también quieren aprender y verlas en la lengua meta. La socialización con la lengua hegemónica determina la existencia de grupos y la exclusión de otros. Aquellos que no hablan inglés o no lo hablan bien son tratados de forma diferencial \cite{collins2018construccion}. Hay una necesidad no solo instrumental, sino social. El desconocer el inglés puede explicarse a partir de la teoría del caos o del horror, que describe que quien no lo aprende se ve como atrasado y hasta deficitario. Dentro de ese caos, estaría aquel que aprende idiomas con doblajes, pues esta modalidad repercute en la comprensión auditiva y en el aprendizaje de patrones o conductas culturales. El doblaje se percibe como una infección o una barrera para no llegar a ser hablante de inglés ni llegar a ser estadounidense. Los medios fomentan que llegar a ser americano es el objetivo; de esta manera, se deshumaniza o subhumaniza a todos los aprendices de nacionalidades y orígenes distintos, sobre todo de los colonizados y del sur \cite{mujica2019colonialidad}. El inglés es una lengua aspiracional para muchos estudiantes peruanos. En el Perú, se piensa que, antes que viajar a Inglaterra o Australia, por ejemplo, es apremiante ir a Estados Unidos, dado que ahí se encuentra con un mundo globalizado cercano a América Latina. En EE. UU. se cree que se cumple lo que se considera el “sueño americano”. Aprender el idioma es asimilar conductas socioculturales que insertarán al hablante en una posición más aceptable llena de supuestos beneficios sociales y domésticos, entre ellos el ocio. No obstante, debe anotarse que las actitudes hacia un idioma se fundamentan más en la empatía o antipatía hacia los hablantes y la cultura que representan \cite{munoz__actitudes_2019}, no en el sistema o funcionamiento lingüístico. El norte global y sus promotores convencen a los estudiantes que por medio del inglés se alcanza la clase de vida asociada con la supuesta modernidad, como puede apreciarse en las respuestas que se transcriben a continuación.

\begin{MyQuote}\label{myquote02}
\textbf{E:} ¿Y por qué sientes la necesidad de ver películas subtituladas en inglés?

\textbf{C-4:} Para estar en sintonía con mis amigos tengo que ver películas y miniseries, y es mejor si están en audio en inglés y subtitulado en inglés. No es lo mismo si está en doblaje en español, pues no suena ni se siente como originales. Además, es malo para el oído y la cultura. Tienes que actuar como americano…

\textbf{C-9} Todos saben inglés, no me puedo quedar atrás. El sueño americano es estar en Estados Unidos, hablar inglés, comer hamburguesas, risas, hacer tu parrilla, trabajar unas cuantas horas, ver televisión, salir al parque, ir a Disney…
\end{MyQuote}


Lo que hacen los subtitulajes es empoderar la lengua y la cultura predominante. Uno de los docentes de idiomas considera que el inglés es una lengua franca, pues está presente en diversas actividades, como el comercio y la diplomacia. Incluso, considera que no solo peruanos, sino también asiáticos tienden a usar el inglés, pues es la lengua de la cultura dominante y del mundo occidental actual. Cabe señalar que las películas y las series son parte de ese mundo. La tendencia es verlas en inglés, antes que en cualquier otro idioma. En este caso, el mundo moderno ha posicionado al inglés como una lengua global, más allá de verse como la lengua materna de sus propios hablantes \cite{garrido2010lengua}. En este sentido, se imparte un conocimiento y praxis monoculturalista, cuestionable, toda vez que absorbe y borra diferencias. Estas formas de pensar se observan a continuación en C-2:

\begin{MyQuote}\label{myquote03}
\textbf{E:} ¿Por qué crees que todos los subtítulos están en inglés y no en otras lenguas?

\textbf{C-2:} Bueno. El inglés se considera lingua franca. Predomina en el comercio, la diplomacia, la “cultura” occidental. Pero también asiáticos consumen los subtitulados. Los países asiáticos tienden a occidentalizarse o a tratar de aculturarse. Más a occidente. Todavía buscan la cultura predominante.
\end{MyQuote}

La lógica occidentalizante ha determinado que el inglés sea entonces la lengua global y la que se espera en películas y series. Para una de las traductoras, C-5, el inglés subtitulado es el primero que aparece. Ella señala que incluso películas que se encuentran en otros idiomas se subtitulan y que las personas buscan ver en subtítulos en inglés. Sostiene que es la L2 sin importar la nacionalidad. Pero esa naturalización no es inmanente del inglés, sino es un poder concedido que detenta, como señala:

\begin{MyQuote}\label{myquote04}
\textbf{E:} ¿Crees que hay una tendencia a que todo el subtitulaje se realice en inglés?

\textbf{C-5:} Ah sí, en películas de otros idiomas que no son el inglés siempre sale primero el subtitulado en inglés. Porque es como el segundo idioma que todo país debe conocer por así decirlo. O mejor dicho es el segundo idioma que muchos entendemos o sabemos.
\end{MyQuote}


Dado que la modernidad es una faceta de la matriz colonial, el inglés se perpetúa como la puerta de acceso a la cultura anglosajona. De acuerdo con las entrevistadas C-6 y C-10, para los peruanos el inglés es la lengua del cine, es decir de las películas, lo que produce una mejora en su nivel lingüístico. También considera importante la enseñanza de traducción de subtítulos a nivel universitario, pues la demanda y la necesidad del inglés hace que mejoren las habilidades de los traductores. En este sentido, se interpreta que el colonialismo o imperialismo perviven en el campo de la cultura y el conocimiento. Siguiendo a Graddol, citado por \textcite{gregson2006english}, el inglés como una lengua en expansión presta atención a espacios de enseñanza en el nivel terciario, como es la universidad, más aún si hay otros idiomas extranjeros que se vuelven más populares, como el chino o el árabe, que repercuten en su auge o hegemonía. Al respecto debe apuntarse que tener una Facultad de Lenguas Modernas posiciona al inglés y lenguas extranjeras, frente a lenguas originarias, excluidas por no ser consideradas como modernas. Ahora bien, esta demanda por el subtitulado en inglés viene incluso reforzada por la Academia, donde las universidades fomentan su utilización, pues es un medio que sirve para que los estudiantes se acerquen a la cultura anglosajona. En este caso, los docentes tienen que encontrar metodologías o mecanismos para conseguir dicho objetivo. Con respecto al Perú, aun cuando el subtitulaje es demandado social y profesionalmente, todavía no hay una enseñanza específica para subtitular películas y series formalmente. Lo que se sabe es que hay iniciativas entre estudiantes que traducen el japonés o coreano al inglés. Lo que se ha encontrado es una propuesta de traducción de culturemas de películas peruanas humorísticas en el ámbito de la traducción audiovisual de subtitulado \cite{cardenas2019analisis}, por lo que hay un acercamiento inicial al trabajo con subtitulaje en el Perú. Puede señalarse al respecto que si bien la dominación, en este caso cultural, somete a las personas, al mismo tiempo, son las que dan su aquiescencia o consentimiento, como se ilustra en \ref{myquote05}.

\begin{MyQuote}\label{myquote05}
\textbf{E:} ¿Enseñan a subtitular en inglés en la universidad?

\textbf{C-6:} No hacemos traducción relacionada con películas y series en el país. Sin embargo, la traducción de subtítulos es necesaria para mucha gente. El inglés es la lengua del cine, al menos para los peruanos.

\textbf{C-10:} Hay alumnos interesados en querer traducir, otros van a escuchar películas con y sin subtítulos para mejorar su nivel. Creo que hay estudiantes que traducen al inglés y también al español, pero de forma voluntaria, espontánea, series japonesas o coreanas. En un futuro las universidades que enseñan Traducción en Perú y forman traductores, como la Facultad de Lenguas Modernas de la Universidad Ricardo Palma, pueden contemplar la posibilidad de trabajar con los subtítulos para que mejoren su inglés y traduzcan mejor otras cosas.
\end{MyQuote}
    

En este panorama, la difusión del inglés genera una necesidad violenta de pensar sobre su posición y provocar la exclusión de otras alternativas, a través de la omisión o la expulsión. En las entrevistas, es revelador encontrar que se cree que las lenguas originarias no son para el subtitulaje de películas y series. La demanda del inglés viene por parte de cinéfilos y espectadores. De acuerdo con C-1 y C-7, se cree que las lenguas peruanas originarias no son útiles en este tipo de modalidad. El subtitulaje tiende a estar en lengua extranjera inglesa. El consumo de las películas y series no se da en lenguas andinas y amazónicas como afirma, pues el espectador procede mayormente de aprendices de inglés. Las lenguas originarias están subordinadas. Como señala, se busca aprender la lengua dominante. Esto hace cuestionar que las lenguas originarias no son atractivas para su enseñanza en un centro de idiomas, sino que tampoco en espacios académicos donde prima la tecnología. Al respecto, teniendo en cuenta el concepto de el “monolenguajear”, referido a que unos poseen un lenguaje pleno, mientras que los otros no, se puede colegir que los hablantes de las lenguas indígenas no son considerables. Para \textcite{veronelli_colonialidad_2019}, el “monolenguajear” es una práctica comunicativa deshumanizante, pues construye comunicativa, mental y lingüísticamente seres no humanos. En esta superposición de lenguas, como hace notar C-7, la gente no pregunta ni se pone a pensar si la película estuvo en audio o subtitulada en lengua originaria, pues dada su exclusión en la trasmisión de películas, se asume que se vio en una lengua hegemónica que se da por natural e incuestionable, pues el cine y las series son artefactos de la cultura occidental. Así se observa en \ref{myquote06}:

\begin{MyQuote}\label{myquote06}
\textbf{E:} ¿Por qué los subtítulos no se encuentran en lenguas originarias?

\textbf{C-1:} Todos buscan el inglés. No hay un público de lenguas originarias, sea indígenas de la Amazonía o los Andes. Inglés es inglés. Las demás lenguas no existen para películas y series, apenas si aparecen es para audio de algún personaje, pero igual se pasa al inglés en el subtitulaje.

\textbf{C-7:} Las lenguas peruanas no son demandadas. Para muchos les parece que son poco servibles para este espacio, incluso cuando uno va a Netflix, tampoco encuentra subtítulos en lenguas como el aimara, el quechua o el shipibo, que son algo conocidas por la población. El inglés es lo que vende y se consume. Nadie te dice que he visto una película con subtítulos en lengua originaria tal o equis, sino se piensa naturalmente que la película la vio en inglés, sin necesidad incluso de decir que estuvo subtitulada en esta. Se busca aprender inglés, no lenguas peruanas.
\end{MyQuote}

Como se aprecia, el subtitulaje se ve como uno de los medios principales para la inglenización. El subtitulaje es considerado una ayuda en la adquisición del inglés y una estrategia de aprendizaje de idiomas \cite{jude2019subtitulacion}. Para una de las profesoras, C-1, se cree que sirve a los principiantes, al ser un aparato de ayuda, que en niveles mayores se vuelve innecesario, dado que dominar el inglés no depende de leer subtítulos. La profesora usa la metáfora de la bicicleta “montar bien” para expresar el logro que se tiene que hacer para lograr tener una buena competencia; o, dicho de otra manera, ser inglés. A los estudiantes se los empuja a ingresar a esa supuesta modernidad hegemónica, que no encuentra supuestamente en su propio país ni en su lengua materna. Sus afirmaciones se ven a continuación:

\begin{MyQuote}\label{myquote07}
\textbf{E:} ¿Crees que los subtítulos en inglés favorecen el aprendizaje del inglés?

\textbf{C-1:} Ayuda a los principiantes. Es una suerte de aparato de ayuda, que más bien sirve como bicicleta con rueditas. Después se les tiene que sacar para aprender a montar bien.
\end{MyQuote}



\subsection{Desprendimiento epistémico y práctico}
En este apartado se brindan opciones de decolonialismo frente a los subtitulajes del inglés, desde un alejamiento hasta la desobediencia y ruptura. Considerando la propuesta pedagógica decolonial de \textcite{baum2019decir}, se formulan diversas alternativas. En principio, la colonialidad no debe ser la regla, sino la excepción. Los subtítulos en inglés abundan en la cinematografía; por tanto, se debe despolitizar el espacio y permitir que otras lenguas gocen del mismo protagonismo. Así, se deconstruye la retórica moderna. Ello conlleva a no internalizar el inglés como la lengua que otorga vida. Hacer una justicia social implica que todas las lenguas sean aceptables y difundidas en los espacios de entretenimiento y académicos. Se puede convocar a activistas lingüistas para que difundan, usen y reviertan la situación de las lenguas otras, como también movilicen a hablantes para el aprendizaje y la valoración de aquellas. Al respecto, no se debe desacreditar las formas de insurgencia construidas de manera colectiva, sobre todo por movimientos populares o minorizados, ni los apoyos a traducciones y subtitulajes locales y espontáneos, como los fansub. Subtitular en lenguas originarias, asimismo, es una oportunidad para diversas comunidades, más aún cuando la UNESCO ha puesto en alerta la situación lingüística mundial. En este periodo de 2022 a 2032 ha proclamado el Decenio Internacional de las Lenguas Indígenas del Mundo para movilizar a la población en preservar, revitalizar y promocionar las lenguas indígenas, pues la mitad de los idiomas que existen en el mundo corren el peligro de extinguirse durante este siglo.

Además, debe considerarse el inglés también como una lengua otra, tan igual que cualquiera; por eso, debe desempoderarse y despojarse de su superioridad, pues sigue como lengua hegemónica perpetuando siglos de opresión, incluida su internalización. Para el caso, se puede desnaturalizar el inglés como lengua global. Empoderar el inglés genera el riesgo de que la periferia sea estigmatizada y que los hablantes de inglés nativo se perciban afortunados hasta el punto de considerarse superiores sin necesidad de aprender una segunda lengua. En este contexto, cabe reconocer las necesidades de cada colectividad y pensar en una pluridiversidad epistémica \cite{dussel_nueva_2009} desde la lingüística. Desacoplarse del inglés posibilitará sanar heridas coloniales \cite[p.6]{baum2019decir}. Privilegiar esta lengua contribuye a que se perpetúe en la enseñanza de idiomas, lo que repercute en la implementación de políticas bilingües, que impulsa tener hablantes bilingües, aunque con tendencia hacia el monolingüismo anglosajón, donde el aprendiente termina siendo el aprendiz de inglés y no aquel que puede buscar aprender alguna lengua indígena, como el quechua o el shipibo. Al respecto, para \textcite{maniglio_alisis_2021}, la imperialidad es la cara que evidencia la colonialidad, al generar nuevas formas de dependencia y de desigualdad social. La forma de dar más cabida al inglés en contra de lenguas minorizadas evidencia nuevos métodos de reducción de los rasgos culturales de las poblaciones colonizadas y racializadas. La colonialidad del lenguaje y la cultura es un proceso de racialización que continúa hasta el día de hoy \cite{veronelli_sobre_2015}. 

Otra propuesta frente a la ideologización y el avasallamiento de los subtítulos en inglés es la resistencia. Los hablantes de lenguas no inglesa y bilingües pueden resistirse al embate global, colonial, mercantilista y capitalista del inglés. No deben negociar con la inglenización, sino más bien buscar la emancipación, pues sino se seguirá posicionando de manera funcional el poder que representa. Las pedagogías de la resistencia pueden contribuir a despertar la conciencia y generar una desobediencia epistémica del posicionamiento del inglés. Para el caso \textcite{baum2019decir}, impulsa la resistencia de la formación educativa en inglés. 

Asimismo, debe lucharse desde la academia también por la difusión acrítica de los subtítulos, a pesar de las prácticas que sitúan el inglés como única opción. \textcite{baum2019decir} sugiere que la decolonización debe ser crítica. En este caso, se debe desacreditar las prácticas coloniales que pasan como prácticas culturales. Hacer ver a los docentes y alumnos de inglés que su enseñanza y aprendizaje debe ser desde una postura crítica y conocer los efectos de la globalización neoliberal, pues es necesario que las acciones lingüísticas se historicen y cuestionen. En este contexto, se puede trabajar con una interculturalidad crítica, que reconozca el aporte y los conflictos entre lenguas, que no silencie y borre las diferencias lingüístico-culturales, que desracialice a los aprendices y que desde el cuestionamiento del saber se discuta la asimilación de la lengua bajo la trampa de la modernidad. La enseñanza debe desencubrir las narraciones, las metáforas y los signos que se atribuyen al inglés como lengua racional, superior e ilustrada, lo que ocasiona epistemicidios culturales. De esta manera, se evalúa la matriz de poder. Se ha generalizado la noción de que el inglés es la única lengua enseñable, e incluso se acepta de forma incuestionable la creación y empleo de materiales y métodos de enseñanza producidos desde la occidentalidad como si fueran extraordinariamente importantes \cite{fandino-parra_decolonizing_2021}. Los docentes deben investigar la manera en que hacen las cosas y ejercen un rol en la transmisión de ideas, así como cuestionar todo proceso de lenguaje. En este contexto, se requiere postular metodologías necesarias para el desaprendizaje y reaprendizaje que conducen a la descolonización \cite{lara2015pensamiento}. 

En contraste con las prácticas de colonización, la enseñanza crítica del inglés, y sus medios de control, es una propuesta pedagógica que busca que los estudiantes cuestionen no solo las relaciones de poder del salón de clase, sino también dentro de la sociedad, con el fin desafiar su imposición \cite{despagne_modernidad_2016}; en este sentido, se requiere pensar en un paradigma otro, que permita ofrecer alternativas a la producción y los conocimientos monoculturales facilitando pensamientos críticos emergentes \cite{mignolo2005paradigma}. \textcite{walsh_pedagogias_2013} propone un pedagogizar decolonial que enfrente la colonialidad del saber y que genere un sistema educativo que rete las racionalidades occidentales, europeas y americanas, que instauran un sistema basado en el conocimiento dominante y la difusión de valores hegemónicos, en vez de pluralizar las epistemes y otras escalas de valoración, como sucede con la imposición del inglés, y sus mecanismos de difusión con los subtítulos, que se inclina por transformar seres humanos en monoculturales. Lingüistas y traductores no escapan de fomentar prácticas coloniales sobre fenómenos lingüísticos. Por tanto, es importante cuestionar la dimensión que cobra el inglés en su transmisión, como viene realizándose desde el Sur \cite{mastrella-de-andrade_abandonamos_2020}. Países como Bolivia, Guatemala, República Dominicana, México y Brasil cuestionan la supuesta modernidad colonial \cite{zapata_decolonial_2018}, donde se deconstruye las estrategias y estructuras que proceden del dominio cultural y epistémico del norte global \cite{fandino-parra_decolonizing_2021} y que se perpetúan incluso en instituciones como la universidad \cite{castro2007decolonizar}. Resulta peligroso cuando ciertas ideologías lingüísticas, al lado de las que son parte constitutiva de ciertos espacios académicos \cite{arias_loaiza_ideologias_2022}, se propalan o expanden en estos espacios y demás ocasionando el desmedro de lenguas y variedades como si fueran poco estándar o desfasadas. Téngase en cuenta que algunas ideologías sobre el lenguaje se generan en los espacios académicos, pero otras vienen de otros sectores y se asimilan en tal ámbito. Con relación a las ideologías en torno al subtitulaje, estas proceden con frecuencia de los medios y la publicidad, y se insertan entre los agentes de enseñanza y aprendizaje del inglés, por ejemplo, en centros de idiomas.

Desde el decolonialismo, se puede poner fin a la colonialidad. En palabras de \textcite{baum2019decir} se propugna un desprendimiento. La decolonialidad debe inaugurar nuevas formas de reexistir y construir el mundo, no solo pensar en prácticas de resistencia. Esto implica no legitimar los espacios de proyección y entretenimiento occidentales como prestigiosos frente a otras formas como la tradición oral o la danza. Además, los hablantes de lenguas otras, como de lenguas originarias, buscan encontrar un lugar de enunciación, dado que los subtitulajes se ven acaparados por la lengua hegemónica. A la vez, se pide desbanalizar la lengua inglesa desde sus concepciones utilitarias e instrumentales. La ruptura con el inglés y su irradiación puede generar nuevas sociedades.

\section{Discusión y conclusiones}
En suma, se puede señalar que la difusión y la demanda de subtítulos facilitan la enseñanza del inglés como práctica colonial. Sin embargo, desde el análisis decolonial se desenmascaran los propósitos de los subtítulos y se han propuesto algunas alternativas frente a su propagación.

Teniendo en cuenta los lineamientos de \textcite{baum2019decir}, se considera que la promoción y el consumo de subtítulos en inglés retienen el sesgo colonial de este. Hay una retórica moderna que busca y normaliza el uso, el aprendizaje y la solicitud del inglés a través de películas y series. Esta normalización que parece local es una maniobra en pos de un giro global. La difusión de subtítulos está mediada por procesos metalingüísticos que permiten reflexionar sobre el inglés como una lengua instrumentalmente “necesaria” y prestigiosa, que por medio de acontecimientos discursivos se configuran apreciaciones sobre el idioma, que legitiman sus aprendices. Los medios digitales, con los casos que se han podido mostrar en este estudio, producen un discurso consumista que coloca o refuerza el inglés como lengua hegemónica. Dicho de otra manera, la demanda de subtítulos de películas y series afecta a la percepción de los idiomas y sedimenta en la sociedad la funcionalidad de una lengua sobre otras. Los cibernautas reproducen discursos que la sociedad moderna viene normalizando en juicios metapragmáticos. Como señala \textcite{baum2016problematicas}, la lengua inglesa es una mercancía consumible, comerciable y cotizable. Las series de televisión con traducción generalmente dan como opción de empleo de subtítulos el inglés ante todo, con mayor frecuencia en su variante estándar y en menor frecuencia en su variante subestándar, en vez de cualquier otra lengua. Frente a ello, el decolonialismo lingüístico del inglés busca desmantelar las relaciones de poder que establece la inglenización, a la vez ofrece una respuesta crítica a su superioridad, repiensa el diálogo horizontal, crea posibilidades para una interculturalidad crítica, propicia la justicia social, da oportunidades a la dignificación humana y la diversidad, incide y modifica sobre todo al ser, fomenta movimientos de resistencia, promueve la desobediencia, despedagogiza la anglofilia, asimismo, propicia metodologías y epistemes que rompen con patrones coloniales.  

La perspectiva del decolonialismo lingüístico ha permitido entender que la colonialidad lingüística y la colonialidad en general se conceptualizan en la modernidad en tanto que es parte constitutiva y heredera de esta, por lo que frente a esta visión se levanta una postura decolonial. Siguiendo a \textcite{pennycook1998english}, la representación social de las lenguas como inferiorizaciones o dialectos parece ser resultado del propio colonialismo, pues las lenguas diferentes a la lengua extranjera dominante continúan siendo consideradas poco importantes, a causa de que no forman parte de la enseñanza formal y la vida de las personas modernas. Globalmente, el inglés se extiende y se acepta como opción natural, incluso en la redacción de trabajos académicos, por ser la lengua que es demandada en varias plataformas internacionales como en el establecimiento de relaciones comerciales. La globalización ha generado el posicionamiento de ciertos aspectos culturales y económicos en la modernidad. En contraste con las lenguas indígenas, el inglés, así como el castellano, son vistos como lenguas instrumentales \cite{despagne_modernidad_2016}, en otros términos, lenguas que otorgan un valor práctico y de cambio social. La modernidad trabaja mediante la imposición de normas lingüísticas, sociales, culturales, sociales, económicas y no solo por medio de la conquista, como acontecía en tiempos de la Colonia \cite{escobar2005alla}. Al respecto, se observa que la movilización social se realiza aprendiendo lenguas extranjeras y no lenguas indígenas. Es decir, no se concibe la movilización hacia el progreso si se aprende una lengua vernácula.

Este tipo de estudio permite revelar además la situación en la que podrían encontrarse las lenguas originarias peruanas o distintas al inglés y reclamar su lugar en la enseñanza-aprendizaje, así como su promoción. Pero también hacer ver que los hablantes de lenguas originarias se encuentran por debajo de los hablantes de lenguas extranjeras, pues sus lenguas no se aprenden, no están institucionalizadas y no se emplean para el mundo global, ni siquiera son bien valoradas en el espacio académico \cite{casas_linguistic_2019}, ni apreciadas por sus estructuras lingüísticas \cite{lovon-cueva_compuestos_2021}, por lo que sesgadamente siguen siendo considerados como sujetos inferiores. Desde la perspectiva del decolonialismo, frente a la pluridiversidad, se debe interpelar la universalidad de materializaciones sociales inclinadas por el inglés y otras lenguas dominantes, así como del despliegue de seres, saberes y poderes hegemónicos.

\section{Agradecimientos}
El autor agradece a los entrevistados que hicieron posible recoger la información con el consentimiento informado respectivo. Además, agradece los alcances brindados por la lingüista Janeth Coti Reyes Capcha en la redacción del manuscrito. Y extiende su agradecimiento a los pares ciego por sus alcances a la investigación.

\section{Financiamiento}
La presente investigación forma parte del proyecto «Ideologías lingüísticas y decolonialidad sobre las lenguas originarias y las lenguas extranjeras en el Perú» del grupo de investigación Lenguas y Filosofías del Perú, de la Facultad de Letras y Ciencias Humanas de la Universidad Nacional Mayor de San Marcos (E22031321), financiado por la Universidad Nacional Mayor de San Marcos – RR N.° 005557-2022-R/UNMSM.


\printbibliography\label{sec-bib}
% if the text is not in Portuguese, it might be necessary to use the code below instead to print the correct ABNT abbreviations [s.n.], [s.l.]
%\begin{portuguese}
%\printbibliography[title={Bibliography}]
%\end{portuguese}



\end{document}

