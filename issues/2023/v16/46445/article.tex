% !TEX TS-program = XeLaTeX
% use the following command:
% all document files must be coded in UTF-8
\documentclass[portuguese]{textolivre}
% build HTML with: make4ht -e build.lua -c textolivre.cfg -x -u article "fn-in,svg,pic-align"

\journalname{Texto Livre}
\thevolume{16}
%\thenumber{1} % old template
\theyear{2023}
\receiveddate{\DTMdisplaydate{2023}{6}{6}{-1}} % YYYY MM DD
\accepteddate{\DTMdisplaydate{2023}{7}{24}{-1}}
\publisheddate{\DTMdisplaydate{2023}{10}{3}{-1}}
\corrauthor{Claudia Barbeta}
\articledoi{10.1590/1983-3652.2023.46445}
%\articleid{NNNN} % if the article ID is not the last 5 numbers of its DOI, provide it using \articleid{} commmand 
% list of available sesscions in the journal: articles, dossier, reports, essays, reviews, interviews, editorial
\articlesessionname{articles}
\runningauthor{Barbeta} 
%\editorname{Leonardo Araújo} % old template
\sectioneditorname{Bárbara Amaral da Silva}
\layouteditorname{Thaís Coutinho}

\title{Narrativas digitais e textos multissemióticos: relato de intervenção pedagógica no ensino de língua portuguesa}
\othertitle{Digital narratives and multisemiotic texts: report on pedagogical intervention in Portuguese language teaching}
% if there is a third language title, add here:
%\othertitle{Artikelvorlage zur Einreichung beim Texto Livre Journal}

\author[1]{Claudia Barbeta~\orcid{0000-0002-2931-5933}\thanks{Email: \href{mailto:cbarbeta@gmail.com}{cbarbeta@gmail.com}}}
\affil[1]{Universidade Estadual de Londrina, Londrina, PR, Brasil.}

\addbibresource{article.bib}
% use biber instead of bibtex
% $ biber article

% used to create dummy text for the template file
\definecolor{dark-gray}{gray}{0.35} % color used to display dummy texts
\usepackage{lipsum}
\SetLipsumParListSurrounders{\colorlet{oldcolor}{.}\color{dark-gray}}{\color{oldcolor}}

% used here only to provide the XeLaTeX and BibTeX logos
\usepackage{hologo}

% if you use multirows in a table, include the multirow package
\usepackage{multirow}

% provides sidewaysfigure environment
\usepackage{rotating}

% CUSTOM EPIGRAPH - BEGIN 
%%% https://tex.stackexchange.com/questions/193178/specific-epigraph-style
\usepackage{epigraph}
\renewcommand\textflush{flushright}
\makeatletter
\newlength\epitextskip
\pretocmd{\@epitext}{\em}{}{}
\apptocmd{\@epitext}{\em}{}{}
\patchcmd{\epigraph}{\@epitext{#1}\\}{\@epitext{#1}\\[\epitextskip]}{}{}
\makeatother
\setlength\epigraphrule{0pt}
\setlength\epitextskip{0.5ex}
\setlength\epigraphwidth{.7\textwidth}
% CUSTOM EPIGRAPH - END

% LANGUAGE - BEGIN
% ARABIC
% for languages that use special fonts, you must provide the typeface that will be used
% \setotherlanguage{arabic}
% \newfontfamily\arabicfont[Script=Arabic]{Amiri}
% \newfontfamily\arabicfontsf[Script=Arabic]{Amiri}
% \newfontfamily\arabicfonttt[Script=Arabic]{Amiri}
%
% in the article, to add arabic text use: \textlang{arabic}{ ... }
%
% RUSSIAN
% for russian text we also need to define fonts with support for Cyrillic script
% \usepackage{fontspec}
% \setotherlanguage{russian}
% \newfontfamily\cyrillicfont{Times New Roman}
% \newfontfamily\cyrillicfontsf{Times New Roman}[Script=Cyrillic]
% \newfontfamily\cyrillicfonttt{Times New Roman}[Script=Cyrillic]
%
% in the text use \begin{russian} ... \end{russian}
% LANGUAGE - END

% EMOJIS - BEGIN
% to use emoticons in your manuscript
% https://stackoverflow.com/questions/190145/how-to-insert-emoticons-in-latex/57076064
% using font Symbola, which has full support
% the font may be downloaded at:
% https://dn-works.com/ufas/
% add to preamble:
% \newfontfamily\Symbola{Symbola}
% in the text use:
% {\Symbola }
% EMOJIS - END

% LABEL REFERENCE TO DESCRIPTIVE LIST - BEGIN
% reference itens in a descriptive list using their labels instead of numbers
% insert the code below in the preambule:
%\makeatletter
%\let\orgdescriptionlabel\descriptionlabel
%\renewcommand*{\descriptionlabel}[1]{%
%  \let\orglabel\label
%  \let\label\@gobble
%  \phantomsection
%  \edef\@currentlabel{#1\unskip}%
%  \let\label\orglabel
%  \orgdescriptionlabel{#1}%
%}
%\makeatother
%
% in your document, use as illustraded here:
%\begin{description}
%  \item[first\label{itm1}] this is only an example;
%  % ...  add more items
%\end{description}
% LABEL REFERENCE TO DESCRIPTIVE LIST - END


% add line numbers for submission
%\usepackage{lineno}
%\linenumbers

\begin{document}
\maketitle

\begin{polyabstract}
\begin{abstract}
O presente artigo expõe os resultados de uma intervenção pedagógica na prática docente de uma professora de Língua Portuguesa em uma instituição pública estadual situada no município de Londrina, na região norte do estado do Paraná. A ação pedagógica foi orientada para a produção de relatos multimodais e multissemióticos por meio de narrativas digitais com alunos de uma turma do primeiro ano do Ensino Médio. Partimos do pressuposto de que o trabalho com textos multissemióticos em sala de aula permite desenvolver um ensino baseado na dialogicidade e na dialética entre uso e reflexão. As observações permitem inferir que o professor de língua portuguesa, ao compreender as possibilidades de utilização das Tecnologias Digitais da Informação e Comunicação (TDIC) e ao ampliar suas competências digitais como professor, é capaz de abordar em sala de aula os novos gêneros textuais emergentes do meio digital, de maneira a explorá-los e integrá-los no processo de ensino e aprendizagem. Além disso, as observações indicam também algumas possibilidades de a escola utilizar as TDIC em favor de uma aprendizagem de língua portuguesa mais efetiva na leitura e escrita, incentivando professores a inovarem seu fazer pedagógico.

\keywords{Narrativas digitais \sep Relatos pessoais \sep TDIC \sep Práticas de ensino e aprendizagem \sep Pesquisa de intervenção}
\end{abstract}

\begin{english}
\begin{abstract}
This article presents the results of a pedagogical intervention in the teaching practice of a Portuguese language teacher at a state public institution located in the city of Londrina, in the northern region of the state of Paraná. The pedagogical action was guided towards the production of multimodal and multisemiotic accounts through digital narratives with students from a first-year High School class. We assume that working with multisemiotic texts in the classroom allows for the development of teaching based on dialogicity and dialectics between usage and reflection. Initial observations lead to the inference that the Portuguese language teacher, upon understanding the possibilities of using Digital Information and Communication Technologies (DICT) and enhancing their digital competencies as an educator, is capable of addressing the new textual genres emerging from the digital medium in the classroom, in a way that explores and integrates them into the teaching and learning process. It is expected that, at the end of this research, some possibilities for the school to use DICT in favor of more effective Portuguese language learning in reading and writing will be indicated, thus encouraging teachers to innovate in their pedagogical practices.


\keywords{Digital narratives \sep Personal accounts \sep DICT \sep Teaching and learning practices \sep Intervention research}
\end{abstract}
\end{english}
% if there is another abstract, insert it here using the same scheme
\end{polyabstract}

\section{Introdução}

Recentemente, há um crescente interesse na necessidade de as instituições de ensino de Ensino Fundamental e Ensino Médio explorarem práticas de linguagem contemporâneas, conforme evidenciado nas orientações presentes em documentos oficiais que norteiam a educação no Brasil, como os Parâmetros Curriculares Nacionais e a Base Nacional Comum Curricular. Essas práticas estão intrinsecamente relacionadas à cultura digital e aos multiletramentos. No âmbito do ensino da Língua Portuguesa, tais documentos determinam que as ações pedagógicas sejam estruturadas em torno de iniciativas que favoreçam a ampliação dos letramentos, destacando a importância da utilização de textos de natureza multissemiótica nos processos educativos. Ademais, os documentos indicam a necessidade de promover exercícios de produção textual em variados meios e semioses. Dessa forma, as estratégias pedagógicas necessitam incorporar distintas manifestações de linguagem - oralidade, leitura, escrita, visual e digital - para atingir suas metas no ambiente social. Isso implica uma revisão das abordagens de ensino nas instituições escolares.

Defendemos a proposição de oportunidades para que os alunos explorem as práticas linguísticas contemporâneas. Para isso, é imprescindível que o ensino da Língua Portuguesa inclua a análise e produção de diversos gêneros textuais, levando em consideração as diversas modalidades de uso da língua. Além disso, é fundamental estimular os alunos a refletirem criticamente sobre essas práticas, compreendendo seus potenciais, limitações e efeitos na construção de significados e na interação social. Os professores devem estar preparados para lidar com as exigências educacionais da contemporaneidade, valorizando a diversidade e as distintas formas de expressão. Conforme assinalado por \textcite[p. 19]{rojo2012}, os textos contemporâneos são “compostos de muitas linguagens (ou modos, ou semioses) e que exigem capacidades e práticas de compreensão e produção de cada uma delas (multiletramentos) para fazer significar”. Em vista disso, o ensino da Língua Portuguesa não deve se restringir às regras gramaticais ou ao aprimoramento das habilidades de leitura e escrita, mas também inserir os indivíduos na sociedade imersa nas multissemioses ou multimodalidades, propiciando condições para que se apropriem do poder semiótico do mundo contemporâneo \cite{kress2003literacy}.

O impasse com o qual os professores de Língua Portuguesa se deparam, dada essas circunstâncias, é como operacionalizar na sala de aula essas variadas semioses ou multimodalidades, especialmente considerando sua formação escolar e profissional fortemente focada no texto verbal. Com o objetivo de fomentar um leque de estratégias didáticas que possam auxiliar o professor a solucionar essa dificuldade, este artigo apresenta os resultados da intervenção nas práticas docentes de uma professora de Língua Portuguesa em uma instituição pública estadual em Londrina, no norte do Paraná. A ação pedagógica foi direcionada à produção de relatos multimodais e multissemióticos por meio de narrativas digitais, realizadas com alunos do primeiro ano do Ensino Médio. Partimos do pressuposto de que o trabalho com textos multissemióticos em sala de aula permite um ensino embasado na interação dialógica e na dialética entre uso e reflexão.

Os alunos foram incentivados a criar narrativas digitais baseadas em suas vivências pessoais. Utilizando plataformas digitais, eles produziram histórias que combinavam texto escrito, imagens, sons e elementos interativos, refletindo assim suas experiências individuais. Essa atividade mostrou-se um recurso eficaz para o aprimoramento das competências em multiletramentos, estimulando o envolvimento dos alunos e proporcionando uma visão mais aprofundada de suas identidades e vivências.

Dessa forma, lançamos um desafio aos professores de Língua Portuguesa: abordar, em suas aulas, os novos gêneros textuais emergentes do meio digital, explorando-os e integrando-os ao processo de ensino e aprendizagem. Isso visa a fomentar a proficiência na leitura desses gêneros.



\section{Multiletramentos e textos multissemióticos: implicações para a prática docente}

Vários estudos destacam a importância do desenvolvimento de competências e habilidades em leitura e escrita no contexto de um universo textual digital \cite{dioni2005generos, rojo2012, rojo2012moura, ribeiro2016textos, coscarelli2016letramento}. Dentro desse contexto, os textos multimodais, que combinam diferentes formas de linguagem, como verbal, visual e sonora, desempenham um papel essencial nas competências linguísticas exigidas no século XXI. Portanto, a implementação de estratégias didático-pedagógicas voltadas para os multiletramentos pode trazer benefícios significativos para os processos de ensino e aprendizagem da língua materna na escola.

No que se refere à produção de textos multissemióticos na sala de aula, é importante destacar o aumento da circulação desses textos impulsionado pelo surgimento das Tecnologias Digitais de Informação e Comunicação (TDIC), que estão cada vez mais presentes nas práticas sociais de linguagem. Na era digital, os textos se tornaram cada vez mais multimodais, combinando palavras escritas com imagens, sons e até mesmo interações do usuário.

Segundo \textcite{dioni2005generos}, com o surgimento de tecnologias inovadoras, tornou-se mais fácil produzir e disseminar imagens e composições criativas para um público amplo. No atual contexto sociocultural, evidencia-se uma crescente intersecção entre os domínios visual e textual. De acordo com \textcite{dioni2005generos}, esse fenômeno reflete a prevalência da visualidade na sociedade contemporânea. Contudo, a autora salienta que representações e imagens ultrapassam a função de meros mecanismos de divulgação de informações ou de representações naturais. Elas constituem, primordialmente, artefatos textuais intrincadamente construídos, os quais servem como espelhos das nossas interações sociais e das construções simbólicas produzidas por nossa sociedade. Dessa forma, o papel das imagens e representações em nossa sociedade não deve ser negligenciado, uma vez que elas atuam como agentes significantes no processo contínuo de produção e reprodução cultural.

Embora a multimodalidade seja uma característica inerente aos textos na era digital, em que elementos verbais e não-verbais interagem para criar significado \cite{dioni2005generos}, essa característica não é apenas uma consequência do desenvolvimento das TDIC, mas também reflete a vasta diversidade cultural na qual estamos inseridos. Portanto, a compreensão desses textos e a comunicação eficaz exigem o desenvolvimento de novos letramentos, ou seja, os multiletramentos.

\textcite[p. 8]{rojo2012moura} coadunam dessa perspectiva quando advogam que Trabalhar com multiletramentos pode ou não envolver (normalmente envolverá) o uso de novas tecnologias da comunicação e de informação (‘novos letramentos’), mas caracteriza-se como um trabalho que parte das culturas de referência do alunado (popular, local, de massa) e de gêneros, mídias e linguagens por eles conhecidos, para buscar um enfoque crítico, pluralista, ético e democrático - que envolva agência – de textos/discursos que ampliem o repertório cultural, na direção de outros letramentos.

A referência aos multiletramentos implica, então, a necessidade de lidar com duas formas de multiplicidade: a multiplicidade de linguagens e mídias e a multiplicidade de referências culturais, ou seja, diversidade cultural. \textcite{rojo2012} avança o conceito de "multiletramentos", o qual captura a multiplicidade complexa existente nas sociedades modernas, particularmente em ambientes urbanos. O termo encapsula dois aspectos relevantes e distintos de multiplicidade. O primeiro está relacionado à diversidade cultural das populações - um reflexo das diferenças étnicas, linguísticas, socioeconômicas, e outras variáveis que compõem a rica tapeçaria social de nossas cidades. O segundo aspecto diz respeito à multiplicidade semiótica dos textos que circulam nesses contextos multiculturais. Esses textos, que podem assumir formas visuais, escritas, e outras, são meios primordiais pelos quais as pessoas adquirem informações e se comunicam umas com as outras. Portanto, o conceito de multiletramentos enfatiza a importância de desenvolver habilidades diversas para compreender e interpretar uma variedade de textos em contextos multiculturais e multimodais. Para a autora, os multiletramentos pautados em algumas características importantes: são interativos (colaborativos); fraturam e transgridem as relações de poder estabelecidas; são híbridos, fronteiriços, mestiços (de linguagens, modos, mídias e culturas).

\textcite{rojo2012} alerta que a escola não pode ignorar essa realidade. Em vez disso, as instituições de ensino precisam reestruturar seus processos de ensino e aprendizagem para se adaptar às configurações emergentes em um mundo contemporâneo e globalizado, assumindo a responsabilidade de trabalhar com esses novos modos de perceber, sentir, agir e significar o mundo e a realidade social.

Considerando a relevância dos multiletramentos no cenário educacional atual, torna-se necessário realizar uma análise mais aprofundada de como eles podem ser implementados no ambiente de ensino. Entre as diversas possibilidades, a multimodalidade em gêneros discursivos, como o relato pessoal, apresenta-se como uma abordagem interessante. Esse gênero, caracterizado pela expressão de experiências e sentimentos pessoais, pode ser significativamente enriquecido e diversificado ao integrar diferentes modos e mídias em sua composição.

A próxima seção se dedicará a explorar essa conexão, propondo formas de aplicar a multimodalidade aos relatos pessoais e discutindo como essa abordagem pode aprimorar as competências dos alunos em multiletramentos, ao mesmo tempo em que promove um ambiente de aprendizado mais envolvente e significativo.

\section{Multimodalidade nos relatos pessoais: ampliando os horizontes dos multiletramentos na prática docente}

A multimodalidade, presente tanto nos relatos pessoais quanto nas narrativas digitais, representa um campo emergente, repleto de oportunidades expressivas para enriquecer os multiletramentos na prática docente. Essa abordagem envolve a integração de diversos modos de comunicação, tais como texto, imagem, som e movimento, permitindo uma expressão e interpretação ampla e diversificada das experiências. Ela não somente reflete a natureza integrada da nossa comunicação diária na era digital, mas também oferece maneiras inovadoras de estimular o aprendizado dos alunos. Investigar como esses elementos multimodais podem ser efetivamente incorporados à prática docente é fundamental para preparar os alunos para uma comunicação efetiva e crítica em variados contextos na sociedade digital atual.

Para aprofundar nossa compreensão da multimodalidade em gêneros discursivos, especialmente os relatos pessoais por meio de narrativas digitais, inicialmente analisaremos o relato pessoal como um gênero discursivo, realçando suas características e funções para, em seguida, apontar como esse gênero pode ser utilizado de forma inovadora com a intenção de impulsionar a competência em multiletramentos. Por fim, discutiremos as intersecções, as possibilidades e os desafios de unir narrativas digitais e relatos pessoais na era digital, ponderando sobre as implicações dessa metodologia à prática docente.



\subsection{O relato pessoal como gênero discursivo: características e função}

A configuração dos gêneros discursivos, postulada por \textcite{bakhtin}, estabelece-se como uma taxonomia textual, delimitada por conteúdo temático, estilística e construção composicional característica que direcionam à edificação e à interpretação de discursos nas várias dimensões da atividade humana. Eles são fundamentais para a comunicação, estruturando nossas interações linguísticas, possibilitando aos interlocutores prever e entender os propósitos e as estruturas discursivas.

O relato pessoal é um gênero discursivo distinto, que comporta a elucidação minuciosa de eventos autobiográficos, pontuada pela subjetividade e reflexão introspectiva. Sua função precípua é possibilitar ao autor expressar e comunicar suas vivências individuais, promovendo uma reflexão acerca de suas experiências e contribuindo para a edificação de sua identidade discursiva. As peculiaridades desse gênero incluem a presença de um narrador em primeira pessoa, a sequencialidade temporal dos eventos e as descrições detalhadas que enfatizam a perspectiva e as emoções individuais do autor. Relatar porque pertence ao domínio social da documentação e memorização das experiências humanas, embora também possua elementos típicos do narrar - como personagens, tempo, espaço e narrador, de acordo com \textcite{schneuwly2004}.

O gênero discursivo de relato pessoal manifesta-se como uma forma de expressão individual que elucida experiências vivenciadas, emoções e reflexões pessoais. Ele é frequentemente observado em ambientes acadêmicos, editoriais literários e páginas da internet e pode conter descrições detalhadas, observações pessoais, diálogos, sentimentos e reflexões do autor acerca da situação ou experiência vivenciada. Adicionalmente, pode ser complementado por elementos multimodais, como fotografias, mapas ou gráficos, com a finalidade de enriquecer a narrativa e fornecer o suporte visual aos leitores.

De acordo com \textcite{oliveira2016}, uma das principais vantagens do uso do gênero relato como proposta pedagógica em sala de aula é permitir que o aluno se torne o protagonista do texto que produz. Ademais, a produção desse gênero possibilita ao professor ter acesso ao universo dos alunos: suas experiências, alegrias e frustrações. Os autores ainda acrescentam que os relatos podem criar estratégias mais eficazes para promover uma aprendizagem significativa. No contexto educacional, portanto, o gênero relato é comumente utilizado como uma forma de expressão e reflexão sobre experiências de aprendizagem. Os alunos podem ser incentivados a redigir relatos de suas próprias vivências, sejam em forma de diários, registros de experimentos, relatórios de campo ou narrativas de eventos significativos. O engajamento com o gênero relato permite o desenvolvimento de habilidades de escrita, organização de ideias, coerência narrativa e expressão pessoal, além de promover a reflexão crítica sobre as experiências vivenciadas, auxiliando no desenvolvimento do pensamento reflexivo e da capacidade de comunicação eficaz.

À luz da importância do gênero discursivo do relato pessoal na expressão individual e na construção de identidades, torna-se imperativo traçar paralelos com as modalidades de comunicação impulsionadas pelas TDIC. A emergência da narrativa digital, conforme apontado por \textcite{ryan2009narrative}, lança um olhar inovador sobre os meios de expressão, transcendendo a linguagem escrita convencional através da incorporação de recursos textuais, visuais e sonoros em ambientes digitais. No próximo segmento, examinaremos detalhadamente a interface entre a narrativa digital e o relato pessoal, discutindo as possibilidades e desafios que essa fusão representa para a expressão individual e coletiva, e como essa conjunção pode alavancar a compreensão e o engajamento na construção da identidade discursiva.


\subsection{Narrativas digitais e relatos pessoais: intersecções, possibilidades e desafios na era digital}

Em uma era marcada pela ubiquidade da informação digital, os gêneros textuais estão passando por profundas transformações, ampliando suas potencialidades expressivas por meio da incorporação de elementos multimodais e recursos interativos \cite{rojo2012}. Com o desenvolvimento exponencial das TDIC, surgem outras modalidades de produção, divulgação e consumo, sendo a narrativa digital um destaque na esfera dos relatos pessoais.

\textcite{miller2010make} argumenta que a integração das TDIC na elaboração de relatos pessoais pode envolver os alunos de maneira mais efetiva, especialmente aqueles menos receptivos às modalidades tradicionais de produção textual. Ao estimulá-los a registrar e a compartilhar suas experiências, é possível promover uma transformação perceptiva em relação aos outros e a si mesmos.

Diferentemente do relato pessoal, em que a comunicação é predominantemente realizada por meio da fala/escrita do narrador, as narrativas digitais exploram a riqueza das múltiplas linguagens proporcionadas pelas TDIC, construídas em plataformas digitais e veiculadas na internet. As produções de narrativas digitais representam oportunidades didáticas para o manejo de tecnologias digitais e o desenvolvimento de habilidades relacionadas à redação e edição de imagens e ao áudio e vídeo.

Nesse contexto de informação digital, os gêneros textuais passam por uma metamorfose e expansão, incorporando elementos multimodais e recursos interativos oferecidos pelo ambiente digital. Entre esses novos gêneros, as narrativas digitais se destacam por sua capacidade singular de combinar texto, imagem, som e interação, resultando em experiências de leitura e escrita imersivas.

Segundo \textcite{ryan2009narrative}, as narrativas digitais se diferenciam das narrativas tradicionais por sua natureza multimídia, interativa e participativa. A hipertextualidade e a interatividade, conceitos propostos por \textcite{landow1992hypertext}, são elementos essenciais das narrativas digitais, permitindo que os usuários explorem diferentes caminhos narrativos e interajam com o conteúdo.

No cerne, as narrativas digitais são histórias veiculadas por meio das TDIC. Elas surgem com o desenvolvimento de recursos digitais que possibilitaram a animação das histórias, tornando-as mais visuais, sonoras e dinâmicas. Essas histórias podem ser denominadas de várias formas, como histórias digitais, relatos digitais, narrativas interativas, narrativas multimídia ou simplesmente narrativas digitais \cite{almeidade2012integraccao, coscarelli2016letramento}.

Da mesma forma, \textcite{almeidade2012integraccao} argumentam que as narrativas digitais podem ser multimodais, combinando diferentes modalidades, como a escrita (linear e sequencial) e a imagem (simultaneidade e espacialidade), resultando na chamada multimodalidade.

Segundo \textcite{damiani2013discutindo}, a narrativa digital pode ser conceituada como um curta-metragem resultante da combinação de uma narrativa oral e conteúdos digitais, como imagens, vídeos e sons, caracterizando-se por seu componente emocional. A complexidade de cada narrativa digital varia, desde histórias mais simples, compostas por slides acompanhados de música e narração, até produções mais elaboradas, com recursos visuais e sonoros de alta qualidade.

Como recurso pedagógico, a narrativa digital pode ser utilizada em diversas áreas do conhecimento, não exigindo que o aluno seja um especialista para aproveitar os recursos disponíveis na rede digital. Existem diversos recursos, como Flash, Movie Maker, \emph{softwares} para produção de blogues, Prezi e até mesmo aplicativos comuns para criação de apresentações, como o PowerPoint, que podem ser empregados na elaboração de narrativas digitais.

As narrativas digitais são uma forma poderosa de expressar histórias pessoais, diários e relatos de maneira cativante e envolvente. Através da utilização de recursos multimodais e instrumentos digitais disponíveis, as pessoas podem dar vida às suas experiências, emoções e pensamentos, integrando elementos visuais, sonoros e interativos. As narrativas digitais fornecem uma plataforma flexível e criativa para transmitir a singularidade de cada indivíduo, permitindo a incorporação de fotografias, vídeos, músicas e outros elementos que enriquecem a narrativa. Esse método possibilita uma conexão mais profunda com o público, uma vez que as histórias pessoais ganham uma dimensão visual e emocional que vai além do texto escrito.

\textcite{santos2016} destaca que as narrativas digitais representam um novo processo de produção textual devido ao uso de recursos tecnológicos contemporâneos. Elas permitem que o autor conte sua história por meio de uma multiplicidade mais ampla de signos. Essa é a principal característica das narrativas digitais: manifestar-se em meios que traduzem múltiplas expressões e possibilitar criação, recriação e atualização dos significados atribuídos.

Além disso, as narrativas digitais estimulam a criatividade, tanto na preparação dos conteúdos quanto na escolha e na forma de apresentação dos recursos, promovendo o pensamento crítico na seleção do tema e na adoção de uma perspectiva. É importante ressaltar também a eficácia da narrativa digital na promoção da aprendizagem colaborativa, já que, no caso de produções em grupo, o sucesso do trabalho depende do esforço conjunto.

A narrativa digital se revela uma plataforma adequada para o gênero do relato pessoal, pois oferece instrumentos e recursos que expandem a expressividade e a interatividade dessa modalidade de discurso. Através de elementos multimídia, como imagens, vídeos e áudios, os autores de relatos pessoais conseguem enriquecer suas narrativas, proporcionando uma experiência mais completa e envolvente para os leitores. Além disso, as redes sociais desempenham um papel significativo na disseminação dos relatos pessoais, permitindo que alcancem um público mais amplo e facilitem a interação e o compartilhamento de experiências.

A combinação da narrativa digital e do relato pessoal no contexto educacional traz benefícios significativos para o processo de aprendizagem. O uso da narrativa digital como recurso pedagógico estimula a criatividade, envolve os alunos e promove a reflexão crítica. O relato pessoal, por sua vez, permite aos alunos compartilharem suas experiências, desenvolver habilidades de escrita e expressão, e construir narrativas pessoais autênticas.

Na intersecção entre o relato pessoal e a narrativa digital, surge uma nova forma de expressão e construção de identidade. O gênero do relato pessoal, com seu foco nas experiências individuais e nas reflexões introspectivas, encontra um meio de expansão e amplificação nas possibilidades multimodais proporcionadas pelas TDIC. A influência crescente da tecnologia digital, especialmente das redes sociais, tem reconfigurado a forma como os relatos pessoais são apresentados e percebidos. Nessas plataformas, os relatos pessoais tornam-se mais públicos, visuais e interativos, e a escolha do que compartilhar e como compartilhar torna-se um ato performático de construção da identidade online \cite{marwick2011tweet}. Um exemplo claro dessa evolução é a popularidade de plataformas como blogues, \emph{vlogs}, Twitter, Instagram e TikTok, nas quais as pessoas compartilham suas experiências de vida por meio de texto, imagem e vídeo, criando narrativas pessoais multimodais que são simultaneamente íntimas e acessíveis ao público em geral.

Nesse contexto de mudança nas formas de contar histórias, podemos identificar um novo gênero narrativo resultante dessa combinação. Os recursos digitais proporcionam uma linguagem própria que se forma a partir dos atributos técnicos e estéticos oferecidos por cada dispositivo tecnológico. As narrativas digitais apresentam uma estrutura que enfatiza criação, interação e compartilhamento, tornando-se trilhas abertas para as construções e reconstruções do sujeito.

As narrativas digitais fora de ambientes escolares são usadas com mais frequência para criar narrativas pessoais que documentam eventos importantes na vida de alguém. No entanto, a narrativa digital também pode ser um recurso poderoso na sala de aula quando usada para produzir documentos históricos, bem como apresentações instrutivas que informam os espectadores sobre um determinado conceito ou prática \cite{robin2008}. \textcite{santos2014webquest} destacam que as tecnologias abrem novos espaços de atuação para os professores, e que seu uso potencializa a construção de conhecimentos. \textcite{robin2012} ressalta as vantagens da utilização das histórias de vida dos alunos no ensino superior, pois elas permitem que os alunos apresentem sua cultura e sua visão de mundo, além de promoverem a interação com outras pessoas. Uma estratégia de ensino sugerida é que os professores apresentem aos alunos narrativas digitais existentes e disponíveis em plataformas online com o propósito de não apenas introduzir o conteúdo, mas também inspirar esses alunos a desenvolverem novas ideias criativas \cite{robin2012}. A criação de narrativas na cibercultura demanda habilidades para criar enredos envolventes, competências técnicas no uso de recursos tecnológicos conectados à internet e capacidade expressiva e comunicativa \cite{conceiçãodos2018narrate}.

A proposta de produção de narrativas digitais reforça a importância atribuída pela BNCC, atual documento que rege o ensino e aprendizagem, aos multiletramentos e à sua relação intrínseca com o desenvolvimento das habilidades comunicativas dos alunos. As narrativas digitais desempenham um papel relevante nesse contexto, pois são uma manifestação concreta dos multiletramentos. Ao utilizar as tecnologias digitais como meio de comunicação, as narrativas digitais permitem que os alunos desenvolvam competências de leitura e produção multimodal, explorando a integração de diferentes linguagens para contar histórias e transmitir mensagens. Além disso, como observam \textcite{jewitt2003multimodal}, investigar a composição de textos multimodais pelos alunos é fundamental para entender como eles estão se adaptando e explorando as possibilidades oferecidas pela tecnologia digital.

A implementação de narrativas digitais no contexto educacional representa uma estratégia para integrar as competências comunicativas dos alunos com as tecnologias digitais. A próxima seção apresentará uma experiência de intervenção pedagógica em uma escola pública estadual em que a proposta pedagógica articulou a produção de relatos pessoais com narrativas digitais. Essa intervenção permitiu que os alunos explorassem suas habilidades narrativas e transmitissem mensagens por meio da utilização de diferentes linguagens e recursos expressivos.



\section{Narrativas digitais na sala de aula: uma intervenção pedagógica}

A presente seção expõe os resultados de uma intervenção pedagógica realizada com uma professora de Língua Portuguesa em uma instituição pública estadual localizada no município de Londrina, região norte do estado do Paraná. Uma pesquisa de intervenção escolar é uma abordagem sistemática de pesquisa que busca não apenas entender os fenômenos que ocorrem no contexto educacional, mas também implementar ações de mudança para melhorar determinadas situações ou desafios. Essa forma de pesquisa envolve o pesquisador ativamente no ambiente escolar, trabalhando em conjunto com professores, administradores e alunos. O objetivo é gerar um impacto positivo e significativo na escola por meio de intervenções baseadas em evidências.

Na pesquisa de intervenção escolar, é comum o uso de metodologias de pesquisa-ação, em que o pesquisador não é apenas um observador, mas um participante ativo no processo. Ele intervém no contexto com o objetivo de provocar uma mudança desejada, e o efeito dessa intervenção é então observado e avaliado. A intervenção pode abordar uma ampla gama de questões, desde a melhoria da qualidade do ensino e da aprendizagem, a resolução de problemas comportamentais ou a promoção do bem-estar e da saúde mental dos alunos. A intenção é sempre criar um ambiente escolar mais eficaz e inclusivo, promovendo a aprendizagem e o desenvolvimento de todos os alunos.

\textcite{damiani2012} delineia intervenções como alterações e inovações intencionais nas práticas de ensino, dirigidas por professores ou pesquisadores. Tais intervenções, moldadas e realizadas sob um quadro teórico distinto, objetivam promover avanços e refinamentos nos métodos existentes, além de provocar e aprofundar o quadro teórico, contribuindo para a expansão do conhecimento dos processos intrínsecos de ensino e aprendizagem. Estas pesquisas de intervenção, abrangendo o planejamento, a implementação e a avaliação subsequente dos efeitos das intervenções, são consideradas essencialmente aplicadas, com o propósito primordial de mitigar problemas práticos. Portanto, sublinhamos a relevância de sua categorização como pesquisas, dada a sua natureza aplicada \cite{damiani2013discutindo}. 

A intervenção descrita neste artigo teve como base o trabalho desenvolvido com um grupo de 28 alunos do primeiro ano do Ensino Médio, no período matutino. A proposta pedagógica consistiu em explorar a intersecção entre o relato pessoal e a narrativa digital, analisando o processo de produção dos alunos do Ensino Médio ao utilizar as TDIC para criar e compartilhar os relatos pessoais de forma digital.

O processo foi dividido em seis etapas principais: planejamento em conjunto com a professora da turma; aulas expositivas e dialogadas na sala de aula com os alunos do primeiro ano do Ensino Médio; formação de grupos de até três alunos; pesquisa de referências, imagens, músicas e escolha de aplicativos para a criação das narrativas digitais pelos grupos na sala de informática da escola; elaboração e edição das narrativas digitais pelos grupos e compartilhamento do produto na plataforma YouTube também na sala de informática da escola.

Para a elaboração colaborativa das narrativas digitais sobre o tema proposto com os alunos, foram planejadas diversas atividades para serem executadas na sala de informática: busca e seleção de informações relacionadas ao tema escolhido; criação de um roteiro utilizando o Google Docs; seleção dos elementos audiovisuais a serem incluídos (fotos, gravações de vídeo, trilha sonora); escolha do software para gravação; gravação da narrativa baseada no roteiro e sincronização com as imagens; montagem do vídeo, adicionando os créditos e a legenda; e compartilhamento do vídeo na plataforma YouTube.

Antes de iniciar as atividades com os alunos na sala de informática, a professora revisou os conceitos do relato pessoal que já haviam sido trabalhados e realizou uma atividade oral na qual os alunos compartilharam algumas de suas histórias pessoais. Algumas dessas histórias envolviam sonhos e expectativas, como a de um aluno que almeja ser jogador de futebol profissional ou de outro que é skatista e sonha em participar de um campeonato nacional. Outras histórias mostravam superação, como a de uma aluna que engravidou na adolescência e tem uma filha especial ou a de um adolescente que teve um relacionamento problemático com o pai alcoólatra. Gradualmente, ao ouvir essas e outras histórias, fomos confirmando o perfil desses alunos, inseridos em um contexto de vulnerabilidade social e econômica. Esses dados confirmam que o gênero de relato pessoal constitui uma forma de compartilhar aprendizagens vivenciais e pessoais, inclusive elementos do dia a dia que são frequentemente subestimados, mas que podem possuir um grau considerável de interesse ou importância para a reflexão, ainda que tais aspectos não sejam prontamente perceptíveis. Tais dados coadunam também com \textcite{aragão2016}, que postula que o gênero relato possui uma estrutura que possibilita ao aluno fortalecer sua identidade por meio de reflexões sobre seu próprio universo, permitindo que o indivíduo se volte para si mesmo em busca de compreensão sobre o seu eu. Além de abordar questões individuais do sujeito, esse gênero também promove a integração do ser por meio da troca de experiências apresentadas nas exposições escritas ou orais.

Na aula seguinte, os alunos foram orientados a respeitar os direitos autorais das imagens obtidas na internet. Nessa etapa, sugeriu-se acessar o site do Creative Commons\footnote{Creative Commons é uma organização sem fins lucrativos que fornece licenças de direitos autorais gratuitas e flexíveis para obras criativas, permitindo que os criadores especifiquem permissões para uso e compartilhamento de seu conteúdo além do tradicional modelo "todos os direitos reservados". O Creative Commons pode ser acessado por meio do seu site oficial na internet (\url{www.creativecommons.org}).} para verificar se as imagens escolhidas estavam autorizadas para publicação e divulgação. Também conversamos sobre a criação do roteiro dos relatos e incentivamos a criatividade dos alunos na escolha do foco narrativo. Discutir o foco narrativo com os alunos, além de estimular o pensamento crítico, proporciona uma oportunidade para explorar a criatividade. As narrativas digitais foram construídas a partir do roteiro dos relatos pessoais, o que auxiliou na edição dos vídeos.

A terceira aula do primeiro ano do Ensino Médio foi a primeira em que os alunos puderam acessar a plataforma do Google Drive nos computadores da escola. Os alunos foram encaminhados da sala de aula, localizada no prédio anexo, até o laboratório de informática. Devido ao número limitado de computadores em funcionamento, vários alunos ficaram sem equipamentos, e a professora organizou a turma de forma a garantir que cada grupo tivesse pelo menos um computador. Alguns poucos alunos conseguiram realizar as atividades utilizando os seus celulares.

Além dessas dificuldades, surgiram outras. Vários alunos não possuíam uma conta cadastrada no Google que lhes permitisse acessar a plataforma Google Docs e outros que já possuíam uma conta não se lembravam da senha. Foi necessário auxiliar esses alunos, orientando-os até que conseguissem acessar o Google Docs. Foi nesse momento que a professora fez uma readequação no planejamento das atividades, orientando os alunos e redistribuindo tarefas para que a turma aprendesse colaborativamente a fazer um cadastro.

Os pilares da aprendizagem colaborativa são: aprender a conhecer, aprender a fazer, aprender a viver junto e aprender a ser \cite{behrens2013}. A colaboração na aprendizagem é de grande relevância, pois há alunos que possuem mais dificuldades em compreender determinados assuntos ou matérias, enquanto outros aprendem de maneiras diferentes.

\textcite{schmitz2018social} definem aprendizagem colaborativa como uma abordagem pedagógica centrada na interação social, na qual os alunos são incentivados a desempenhar papéis ativos em uma jornada educacional orientada por objetivos compartilhados de aprendizado. Nesse paradigma, a aprendizagem é vista como um processo participativo, cooperativo e construído em conjunto, ao invés de uma transmissão unidirecional de conhecimento. Os alunos, nesse sentido, não são meros receptores passivos de informações, mas colaboradores ativos no processo de produção do conhecimento. Destaca-se, portanto, a importância da colaboração, da interação social e da participação ativa dos alunos no processo de aprendizagem.

Assim, parte-se do princípio de que o conhecimento é de natureza social e se estrutura a partir de esforços colaborativos voltados ao aprendizado, à compreensão e à resolução de problemas. Por isso, destacamos a importância da colaboração nessa etapa da aula, pois, com uns ajudando aos outros, o processo de aprendizagem torna-se mais motivador para os alunos e, consequentemente, para o professor.

Sob a orientação da professora, os alunos se familiarizaram com o editor de textos, embora alguns tenham expressado frustração por não poderem criar o vídeo de imediato. A professor explicou que a etapa inicial envolveria a redação do relato, seguida pela elaboração do roteiro da narrativa digital. Essa última fase despertou queixas entre os alunos. No cerne da questão, a roteirização é vital para construir narrativas digitais coesas, assegurando uma progressão lógica, atraindo o público e facilitando o planejamento prévio. Além disso, ela promove a harmonia entre elementos visuais, sonoros e textuais, fundamental nas narrativas digitais. Em resumo, criar um roteiro é essencial para garantir histórias envolventes, coesas e impactantes.

Por essas várias razões, a professora insistiu que roteirizar o relato é o ponto de partida para chegar à narrativa digital e exemplificou que todos os programas de televisão, séries e novelas precisam ser roteirizados primeiro. A pesquisadora também explicou que até os \emph{youtubers} famosos fazem um roteiro antes de gravar. Mesmo assim, alguns alunos demonstraram obstinação em produzir o vídeo sem precisar escrever.

A dinâmica da aula permitiu que a professora promovesse interações entre os alunos. Naquele espaço, foi possível observar uma mudança na qual os alunos trocavam informações enquanto formulavam suas dúvidas. Além disso, a professora estimulou os alunos a interagirem, mediando todo o processo. Mais uma vez, destaca-se a característica colaborativa da aprendizagem, ou seja, na aprendizagem colaborativa, o conhecimento é produzido por meio do consenso entre os membros de uma equipe. Esse consenso é alcançado mediante a troca de ideias entre pessoas dispostas a resolver um problema ou a criar um projeto em conjunto.

Após a produção do relato e do roteiro da narrativa digital, os alunos foram orientados a selecionar as imagens que utilizariam na produção dos vídeos. Vários optaram por usar imagens do acervo pessoal (fotos ou vídeos que ilustravam algum momento de suas vidas). Outros selecionaram imagens encontradas na internet, o que demandou habilidades de busca e escolha, levando em consideração os direitos autorais desses elementos ilustrativos. Alguns alunos associaram ao vídeo uma narrativa oral, gravada com suas próprias vozes. Outros escolheram uma trilha sonora encontrada em sites que disponibilizam músicas e efeitos sonoros. Assim que os vídeos eram concluídos, a professora os publicava no canal do YouTube\footnote{Exemplos de narrativas digitais produzidas pelos alunos nessa experiência podem ser visualizadas acessando os links: \url{https://www.youtube.com/watch?v=k26RIMvb5ik&ab_channel=ROSELIPIOTTO} e \url{https://www.youtube.com/watch?v=NB6lHYVpHA&t=109s&ab_channel=ROSELIPIOTTO}}.

Temos ciência de que essa única experiência não foi capaz de abranger todas as dimensões que envolvem a produção de textos multimodais. Entretanto, evidenciamos que a atividade de elaboração de narrativas digitais em sala de aula foi vista pela professora como um mecanismo de extrema importância para a renovação de sua prática docente, para a ressignificação dos papéis desempenhados (alunos e professora) e para a produção dos saberes. Elaborar um gênero discursivo (no caso desta pesquisa, o relato pessoal), pensando em uma proposta de ensino multimodal, envolveu planejamento de gravações, capacidade de síntese (no caso das legendas), edição e montagem dos vídeos. Mas, além da produção dos vídeos, todo o processo mostrou à professora que é possível proporcionar aos alunos (principalmente àqueles provenientes de uma comunidade carente) a participação em práticas de letramento digital mesmo com as várias limitações de recursos operacionais.

Os resultados dessa experiência indicam que a integração de narrativas digitais no ambiente de sala de aula pode funcionar como uma eficiente estratégia pedagógica para melhorar a competência textual dos alunos, desde que os professores recebam formação continuada adequada para o uso das TDIC. A utilização de narrativas digitais apresenta um vasto potencial e pode ser adaptada para beneficiar alunos de variados níveis educacionais. Além disso, permitir que os próprios alunos sejam os criadores de suas histórias pessoais por meio de vídeos redefine o papel do professor, que passa a atuar como um mediador no processo de ensino e aprendizagem. Essa mediação envolve fornecer a orientação necessária para que os alunos possam desenvolver a narrativa de forma eficiente, tanto do ponto de vista técnico quanto pedagógico. Dessa maneira, a postura do professor e sua interação com os alunos em espaços digitais são identificadas como elementos-chave para o sucesso de inovações pedagógicas, sublinhando a necessidade de investir na capacitação do professor para o uso efetivo das TDIC na prática educacional.

\section{Considerações finais}

O presente relato expôs os resultados de uma intervenção pedagógica na prática docente de uma professora de Língua Portuguesa em uma instituição pública estadual, região norte do estado do Paraná. A ação pedagógica teve como foco a produção de relatos pessoais multimodais e multissemióticos através de narrativas digitais. A experiência revelou que a integração de textos multimodais e multissemióticos no currículo de Língua Portuguesa pode oferecer aos alunos uma experiência de aprendizagem mais enriquecedora, ao mesmo tempo em que estimula os professores a inovarem em suas práticas pedagógicas.

A incorporação dos multiletramentos e das narrativas digitais no Ensino Médio traz uma série de benefícios para os alunos, bem como desafios para os professores. Ao incluir esses elementos no currículo, é possível estimular a criatividade dos alunos, proporcionando um ambiente propício para a expressão e o desenvolvimento de suas habilidades comunicativas. As narrativas digitais, por sua vez, despertam o engajamento dos alunos, permitindo que eles utilizem as TDIC de forma ativa na produção e interpretação de textos.

A relação entre o gênero discursivo "relato pessoal" e a narrativa digital revela um panorama rico em possibilidades de expressão e interação. As TDIC têm proporcionado outras formas de criar, compartilhar e participar de narrativas pessoais, diferentes daquelas produzidas apenas em suportes físicos de escrita, ampliando a diversidade de vozes e experiências presentes na sociedade. Ainda assim, é importante considerar os desafios relacionados à autenticidade, privacidade e qualidade das narrativas pessoais digitais. Nesse contexto, é fundamental que professores e pesquisadores explorem estratégias para promover a reflexão crítica e ética no uso dessas tecnologias, capacitando os alunos a se engajarem de maneira consciente e responsável na produção e no consumo de narrativas digitais de relato pessoal.

Além disso, a integração dos multiletramentos e das narrativas digitais amplia as possibilidades de expressão dos alunos. Eles podem utilizar uma variedade de recursos e linguagens para comunicar suas ideias e experiências, o que permite uma maior personalização e autenticidade em suas produções. Isso fortalece a autoconfiança, a capacidade de expressão e a criatividade dos alunos, incentivando-os a se tornarem protagonistas ativos em seu processo de aprendizagem.

Entretanto, os professores podem enfrentar desafios ao incorporar essas práticas no Ensino Médio. Um dos desafios é a necessidade de formação continuada, pois esses profissionais precisam adquirir competências relacionadas aos multiletramentos e ao uso das tecnologias digitais de forma pedagogicamente eficaz. Assim, ressalta-se a necessidade crucial de investimentos contínuos na formação de professores, visando à utilização das TDIC em sala de aula. Tal formação deve abranger o desenvolvimento de competências digitais, entendimento de como integrar essas tecnologias ao currículo e como elas podem aprimorar o processo de ensino e aprendizagem, promovendo um ambiente de aprendizagem mais engajador e eficaz.



\printbibliography\label{sec-bib}
% if the text is not in Portuguese, it might be necessary to use the code below instead to print the correct ABNT abbreviations [s.n.], [s.l.]
%\begin{portuguese}
%\printbibliography[title={Bibliography}]
%\end{portuguese}






\end{document}

