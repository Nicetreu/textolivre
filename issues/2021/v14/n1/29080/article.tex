% !TEX TS-program = XeLaTeX
% use the following command: 
% all document files must be coded in UTF-8
\documentclass[spanish]{textolivre}
% for anonymous submission
%\documentclass[anonymous]{textolivre}
% to create HTML use 
%\documentclass{textolivre-html}
% See more information on the repository: https://github.com/leolca/textolivre

% Metadata
\begin{filecontents*}[overwrite]{article.xmpdata}
    \Title{Contagio emocional en las redes sociales: el caso de COVID-19 en Facebook}
    \Author{Cynthia Pasquel-López \sep Pasquel y Valerio-Ureña}
    \Language{es}
    \Keywords{Contagio emocional \sep redes sociales \sep COVID-19 \sep Facebook.}
    \Journaltitle{Texto Livre}
    \Journalnumber{1983-3652}
    \Volume{14}
    \Issue{1}
    \Firstpage{1}
    \Lastpage{16}
    \Doi{10.35699/1983-3652.2021.29080}

    \setRGBcolorprofile{sRGB_IEC61966-2-1_black_scaled.icc}
            {sRGB_IEC61966-2-1_black_scaled}
            {sRGB IEC61966 v2.1 with black scaling}
            {http://www.color.org}
\end{filecontents*}

% used to create dummy text for the template file
\definecolor{dark-gray}{gray}{0.35} % color used to display dummy texts
\usepackage{lipsum}
\SetLipsumParListSurrounders{\colorlet{oldcolor}{.}\color{dark-gray}}{\color{oldcolor}}

% used here only to provide the XeLaTeX and BibTeX logos
\usepackage{hologo}

% used in this example to provide source code environment
%\crefname{lstlisting}{lista}{listas}
%\Crefname{lstlisting}{Lista}{Listas}
%\usepackage{listings}
%\renewcommand\lstlistingname{Lista}
%\lstset{language=bash,
        breaklines=true,
        basicstyle=\linespread{1}\small\ttfamily,
        numbers=none,xleftmargin=0.5cm,
        frame=none,
        framexleftmargin=0.5em,
        framexrightmargin=0.5em,
        showstringspaces=false,
        upquote=true,
        commentstyle=\color{gray},
        literate=%
           {á}{{\'a}}1 {é}{{\'e}}1 {í}{{\'i}}1 {ó}{{\'o}}1 {ú}{{\'u}}1 
           {à}{{\`a}}1 {è}{{\`e}}1 {ì}{{\`i}}1 {ò}{{\`o}}1 {ù}{{\`u}}1
           {ã}{{\~a}}1 {ẽ}{{\~e}}1 {ĩ}{{\~i}}1 {õ}{{\~o}}1 {ũ}{{\~u}}1
           {â}{{\^a}}1 {ê}{{\^e}}1 {î}{{\^i}}1 {ô}{{\^o}}1 {û}{{\^u}}1
           {ä}{{\"a}}1 {ë}{{\"e}}1 {ï}{{\"i}}1 {ö}{{\"o}}1 {ü}{{\"u}}1
           {Á}{{\'A}}1 {É}{{\'E}}1 {Í}{{\'I}}1 {Ó}{{\'O}}1 {Ú}{{\'U}}1
           {À}{{\`A}}1 {È}{{\`E}}1 {Ì}{{\`I}}1 {Ò}{{\`O}}1 {Ù}{{\`U}}1
           {Ã}{{\~A}}1 {Ẽ}{{\~E}}1 {Ũ}{{\~u}}1 {Õ}{{\~O}}1 {Ũ}{{\~U}}1
           {Â}{{\^A}}1 {Ê}{{\^E}}1 {Î}{{\^I}}1 {Ô}{{\^O}}1 {Û}{{\^U}}1
           {Ä}{{\"A}}1 {Ë}{{\"E}}1 {Ï}{{\"I}}1 {Ö}{{\"O}}1 {Ü}{{\"U}}1
           {ç}{{\c{c}}}1 {Ç}{{\c{C}}}1
}


\journalname{Texto Livre: Linguagem e Tecnologia}
\thevolume{14}
\thenumber{1}
\theyear{2021}
\receiveddate{\DTMdisplaydate{2021}{01}{19}{-1}} % YYYY MM DD
\accepteddate{\DTMdisplaydate{2021}{2}{22}{-1}}
\publisheddate{\DTMdisplaydate{2021}{4}{7}{-1}}
% Corresponding author
\corrauthor{Cynthia Pasquel-López}
% DOI
\articledoi{10.35699/1983-3652.2021.29080}
% list of available sesscions in the journal: articles, dossier, reports, essays, reviews, interviews, editorial
\articlesessionname{Comunicación y Tecnología}
% Abbreviated author list for the running footer
\runningauthor{Pasquel-López y Valerio-Ureña}
\editorname{Leonardo Araújo}

\title{Contagio emocional en las redes sociales: el caso de COVID-19 en Facebook}
\othertitle{O contágio emocional nas redes sociais: o caso do COVID-19 no Facebook}
\othertitle{Emotional contagion in social networks: the case of COVID-19 on Facebook}
% if there is a third language title, add here:
%\othertitle{Artikelvorlage zur Einreichung beim Texto Livre Journal}

\author[1]{Cynthia Pasquel-López \orcid{0000-0001-7409-3527} \thanks{Email: \url{a00609710@itesm.mx}}}
\author[1]{Gabriel Valerio-Ureña \orcid{0000-0002-4446-6801} \thanks{Email: \url{gvalerio@tec.mx}}}

\affil[1]{Tecnológico de Monterrey, Monterrey, Nuevo León, México.}

\addbibresource{article.bib}
% use biber instead of bibtex
% $ biber tl-article-template

% set language of the article
\setdefaultlanguage{spanish}
\setotherlanguage{portuguese}
\setotherlanguage{english}

% for spanish, use:
%\setdefaultlanguage{spanish}
\gappto\captionsspanish{\renewcommand{\tablename}{Tabla}} % use 'Tabla' instead of 'Cuadro'


% for languages that use special fonts, you must provide the typeface that will be used
% \setotherlanguage{arabic}
% \newfontfamily\arabicfont[Script=Arabic]{Amiri}
% \newfontfamily\arabicfontsf[Script=Arabic]{Amiri}
% \newfontfamily\arabicfonttt[Script=Arabic]{Amiri}
%
% in the article, to add arabic text use: \textlang{arabic}{ ... }

% to use emoticons in your manuscript
% https://stackoverflow.com/questions/190145/how-to-insert-emoticons-in-latex/57076064
% using font Symbola, which has full support
% the font may be downloaded at:
% https://dn-works.com/ufas/
% add to preamble:
% \newfontfamily\Symbola{Symbola}
% in the text use:
% {\Symbola ��}


\usepackage{multirow}
\newcolumntype{R}[2]{%
    >{\adjustbox{angle=#1,lap=\width-(#2)}\bgroup}%
    l%
    <{\egroup}%
}
\newcommand*\rot{\multicolumn{1}{R{90}{1em}}}% no optional argument here, please!


\begin{document}
\maketitle

\crefname{table}{tabla}{tablas}
\Crefname{table}{Tabla}{Tablas}

\begin{polyabstract}
\begin{abstract}
La estrategia de confinamiento, ante la contingencia por el COVID-19, aumentó el uso de la tecnología para continuar las actividades diarias. Sin embargo, el consumo de información proveniente de las redes sociales podría incidir en las emociones de las personas. El objetivo de esta investigación fue explorar el contagio emocional en Facebook ante una contingencia. Con un enfoque esencialmente cualitativo, se analizaron 701 publicaciones relacionados con el tema de COVID-19 y los 1872 comentarios generados a partir de dichas publicaciones. Se encontró que: A) con el fin de informar y entretener, las personas prefieren comunicarse a través de imágenes en días hábiles; B) el contagio emocional se da en Facebook, pero no en la misma proporción en todos los participantes. Esto resalta la importancia de ser conscientes de que la información que compartimos puede impactar en las emociones de otros.

\keywords{Contagio emocional \sep redes sociales \sep COVID-19 \sep Facebook.}
\end{abstract}


\begin{portuguese}
\begin{abstract}
A estratégia de confinamento, em resposta à contingência do COVID-19, causou um aumento no uso da tecnologia para dar seguimiento às atividades diárias. Entretanto, o consumo de informação nas redes sociais poderia afetar o estado emocional das pessoas. O objetivo desta pesquisa foi explorar o contágio emocional no Facebook diante de uma contingência. Com uma abordagem essencialemnte qualitativa, foram analisadas 701 publicações relacionadas ao tema da COVID-19, emitidas por quatro participantes, e 1872 comentários gerados a partir dessas publicações foram analisados também. Descobriu-se que: A) para informar e entreter, as pessoas preferem se comunicar através de imagens nos dias úteis; B) o contágio emocional ocorre no Facebook, mas não na mesma proporção em todos os participantes. Isso destaca a importância de estarmos conscientes de que as informações que compartilhamos podem impactar as emoções dos outros. 

\keywords{Contágio emocional \sep redes sociais \sep COVID-19 \sep Facebook.}
\end{abstract}
\end{portuguese}

\begin{english}
\begin{abstract}
The containment strategy, in the face of the contingency of COVID-19, 
increased the use of technology to continue daily activities. 
However, the consumption of information in social networks could affect people's
emotions. The objective of this research was to explore the emotional contagion 
on Facebook in the face of a contingency. Whit an essentially qualitative approach,
701 publications related to the topic of COVID-19 and 1872 comments generated from those publications were analyzed. It was found that: A) in order to inform and entertain, people prefer to communicate through images on weekdays; B) emotional contagion occurs on Facebook, but not in the same proportion in all participants. This highlights the importance of being aware that the information we share can impact the emotions of others.

\keywords{Emotional contagion \sep social network \sep COVID-19 \sep Facebook.}
\end{abstract}
\end{english}

% if there is another abstract, insert it here using the same scheme
\end{polyabstract}


\section{Introducción}\label{sec-introd}
Ante el escenario de contingencia de salud pública ocasionado por la epidemia del COVID-19, las instituciones de salud pública recomendaron a la población resguardarse en sus casas para evitar un contagio masivo que pudiera resultar catastrófico. En México, el Instituto Mexicano de Seguro Social promovió, desde la segunda semana de marzo del 2020, medidas de sana distancia y confinamiento. Las estadísticas oficiales del Gobierno Mexicano reportaron, al cierre del mes de abril de 2020, más de 28 mil casos de contagios y más de 3,500 muertes en todo el país.  Ante esta situación, las personas se vieron obligadas a transformar la forma en que realizan sus actividades cotidianas. En este contexto, Internet juega un papel aún más protagónico en la continuidad de las actividades, en ámbitos como: el trabajo, la educación y las relaciones personales. 

Incluso antes de la pandemia, el uso de Internet había aumentado en los últimos años. Según \textcite{kemp2020} el uso promedio diario global de Internet, es de más de ocho horas, lo cual equivale a una jornada laboral en la mayoría de los países. El desarrollo y continuidad de relaciones personales es una de las actividades en la vida más frecuentes en Internet. El mismo reporte indica que un cuarto del uso diario de Internet se usa en redes sociales, que equivale en promedio a más de dos horas diarias, en personas de entre 16 y 64 años. En el contexto virtual, las aplicaciones de redes sociales se utilizan para compartir fotos, opiniones o el sentir sobre algún evento o situación de la vida diaria, lo cual ayuda a crear, fortalecer y fomentar lazos personales. Ante el aislamiento que demanda la contingencia, el uso de las redes sociales, además de utilizarse para seguir conectados con otras personas, ha sido muy utilizado como medio de comunicación para mantenerse informado sobre la situación que se está viviendo. 

Los investigadores se han preocupado por los efectos de los medios de comunicación masivos desde los años setenta, cuando surge la Teoría de uso y gratificantes. En su trabajo, \textcite{mcquail} resaltaba la importancia estudiar los diferentes tipos de contagios o difusiones espontaneas a través de los medios. Según esta teoría, las funciones de la comunicación masiva se clasifican en cinco grupos: 
\begin{enumerate*}[label=\arabic*)] 
\item las cognitivas para el fortalecimiento de la información, conocimiento y comprensión, 
\item las afectivas y emocionales, 
\item las de credibilidad, la confianza, la estabilidad y el estatus, 
\item las conexiones sociales y 
\item las de diversión y entretenimiento.
\end{enumerate*} 
En la actualidad, las redes sociales se han convertido en una herramienta de comunicación para informar, compartir y dispersar información \cite{huttHerrera2012}. Dado el potencial de las redes sociales para difundir información masivamente e influir en el sentir de los demás, es importante explorar la dinámica del contagio emocional a través de las redes sociales; sobre todo ante una situación de emergencia global como la que se está viviendo con el COVID-19. En este estudio se revisará el involucramiento digital (\textit{digital engagement}) para explorar el contagio emocional que se da en las redes sociales. 


\section{Marco teórico}\label{sec-marco}
Paul Ekman fue de los primeros investigadores dedicados al estudio de las expresiones faciales que generan las emociones. En 1972, en su trabajo sobre las diferencias universales y culturales de las expresiones faciales de una emoción, reconoce una lista de seis emociones universales básicas: felicidad, tristeza, disgusto, sorpresa, miedo y enojo \cite{ekman1972}. Veinte años después de la primera lista, \textcite{ekman1992} en el análisis que hace de la Teoría de Ortony y Turner’s indica que hay una emoción básica que puede añadirse a la lista: desprecio. Finalmente, en años recientes, \textcite{ekman2016} confirmó que la mayoría de los científicos coincidía con cinco de sus emociones propuestas: felicidad, tristeza, disgusto, miedo y enojo; siendo solo las emociones de sorpresa y desprecio las que no coincidían. 

\textcite{ekman2011} reconocen que en su lista de emociones básicas hay poco balance entre lo negativo y lo positivo. Por otro lado, Barbara Fredrickson ha dedicado sus esfuerzos en estudiar las emociones positivas; uno de los resultados ha sido su teoría de positivismo llamada: \textit{Broaden and build theory of positive emotions}. La teoría postula que las emociones positivas pueden ampliar los repertorios de pensamiento y acción de las personas para así, construir recursos duraderos personales \cite{fredrickson2004}. Fredrickson reconoce diez emociones positivas que pueden ayudar al desarrollo de recursos personales y al bienestar: alegría, gratitud, serenidad, interés, esperanza, diversión, inspiración, asombro, orgullo y amor \cite{fredrickson2013}. 

El ser humano se distingue por su capacidad de reconocer y replicar las emociones de sus congéneres. \textcite{hatfield1993} definen estas acciones de imitación de emociones como contagio emocional y pueden ser imitaciones de expresiones faciales, vocales, posturas o comportamientos. El objetivo del contagio emocional, sostienen, es la sincronía de la conducta y el seguimiento de las emociones en cada momento, acciones que se vuelven importantes en el desarrollo de relaciones personales. Además, indican que, aunque estas imitaciones se hacen de una manera fugaz, las personas son capaces de sentir las emociones de otras, aunque no siempre sean conscientes de este efecto. El contagio emocional ofrece información sobre el sentir de la otra persona, por ejemplo, los mismos autores indicaban que los sentimientos y emociones pueden ser influenciadas por la información que se obtiene de la mímica de las emociones.  Por lo tanto, el contagio emocional puede brindar una ventaja e información invaluable para el desarrollo de las relaciones personales.  

Actualmente, el estudio del contagio emocional ha surgido también en las redes sociales. Investigaciones anteriores \cite{kramer2014} y actuales \cite{goldenberg2020,steinert2020} sostienen que el contagio emocional en las redes sociales sucede cuando las emociones de las personas se asemejan a las de otras que han sido expuestas a éstas y que las emociones de las personas pueden afectar a otros y propagarse de uno a otro.

Estudios anteriores \cite{kramer2012} han encontrado cómo las palabras de publicaciones influyen en las réplicas de otros usuarios y cómo el uso de palabras positivas amortigua contra la negatividad. Por otro lado, recientes investigaciones han encontrado que unas emociones suelen ser más acogidas que otras. Por ejemplo, algunas investigaciones señalan que las personas suelen ser más receptivas a ciertas emociones como la irá o la alegría, que a emociones como tristeza o miedo, quizás por el nivel de intensidad de la emoción \cite{wang2020,goldenberg2020}. Además, esta prevalencia durante una contingencia, como la del COVID-19, puede propiciar una propagación acelerada de las emociones negativas por medio del contagio emocional \cite{goldenberg2020}. Aunque, el contagio emocional pareciera una desventaja, en investigaciones anteriores \cite{Bazarova2015} y actuales \cite{iglesiasSanchez2020} apuntan que el intercambio y las expresiones emocionales, durante las situaciones de crisis, en los medios sociales puede ser una terapia social para regular las emociones y ayudar a las personas a mantenerse psicológicamente sanos, así como fomentar mejores relaciones con la cercanía creada. Por lo tanto, desde esa perspectiva, las redes sociales podrían ayudar a equilibrar la vida, dada la falta de interacción con otros.

Por otra parte, las redes sociales también se usan como medio de comunicación, por ello se puede estudiar el propósito de las publicaciones con base en la clasificación de \textcite{monroe1973}, quienes distinguen tres propósitos fundamentales de la comunicación: 
\begin{enumerate*}[label=\arabic*)] 
\item informar para conseguir una comprensión de algo, 
\item entretener para divertir o agradar a la audiencia y 
\item persuadir, es decir influir sobre algo o llevar a una acción. 
\end{enumerate*}

 En la actualidad, la tecnología ha transformado el proceso de comunicación masiva cambiando a un proceso más interactivo, cambiante y dinámico \cite{huttHerrera2012}. En estudios anteriores \cite{ku2013} se encontraron cinco tipos de usos de los sitios de redes sociales: 
 \begin{enumerate*}[label=\arabic*)] 
 \item mantenimiento de relaciones personales, 
 \item obtención de información, 
 \item entretenimiento, 
 \item estabilidad de estatus y 
 \item socialización.
 \end{enumerate*}
 Además, el uso se hace más relevante cuando el usuario ve un valor en la conexión sincrónica y las funcionalidades del tipo de publicaciones que el usuario considera mejor equipadas o capaces para satisfacer sus necesidades. 
 
El uso de la comunicación en las redes sociales en situaciones de contingencia sanitaria se puede resumir en: compartir y difundir información importante (brotes de la enfermedad, protocolos de diagnósticos, tratamiento y seguimiento), aprender de enfoques de otros países, aumentar la concientización, mantenerse conectado con otros, así como recibir y ofrecer apoyo social a conocidos y apoyar los estados sociopsicológicos \cite{saud2020,gonzlezPadilla2020}. Ante las nuevas tendencias tecnológicas y el uso de estas para comunicación, se puede notar la prevalencia de los tres propósitos básicos de la comunicación: informar, entretener y persuadir. 

Independientemente del propósito de una comunicación, el impacto que ésta produce en la audiencia se refleja en su nivel de involucramiento (engagement) con la misma. El involucramiento digital (digital engagement) es una colección de experiencias que tienen los lectores como resultado de las publicaciones en las redes sociales, es decir aquello que la gente entiende sobre su comportamiento y emociones \cite{davisMersey2010,pletikosaCvijikj2013}. El involucramiento digital depende del formato del mensaje (imágenes, texto plano, fotos, videos y enlaces), el tipo de contenido y el momento en el que se publica \cite{pletikosaCvijikj2013}. Por su parte \textcite{sabate2014} definen como resultado del involucramiento el número de “me gusta” y el número de comentarios. Como factores, además del formato, tipo de contenido y momento, agregan el tamaño de la publicación y el número de seguidores. Por lo tanto, para el desarrollo del estudio se tomará en cuenta el formato del mensaje, el momento en el que se publica, número de reacciones, número de comentarios y número de veces compartido. 

En el presente estudio tiene como objetivo explorar el contagio emocional a través de las redes sociales ante una contingencia sanitaria mundial. Específicamente, se plantea dar respuesta a la interrogante sobre cómo se presenta el contagio emocional en los comentarios de las publicaciones sobre COVID-19 en Facebook. 


\section{Método}
Para poder explorar el fenómeno del contagio emocional en los medios sociales se planteó una investigación de campo a través de la plataforma de Facebook; el diseño no fue experimental, por lo que no se manipularon variables ni se tuvo un muestreo estadístico.  Se analizaron un total de 701 publicaciones de Facebook y sus respectivos comentarios (1,872). Las publicaciones fueron recolectadas de cuatro participantes seleccionados utilizando un muestreo por conveniencia \cite{creswell2018}, ya que debían ser personas a los cuales los investigadores de este estudio tuvieran acceso a sus publicaciones. El perfil de los participantes de las publicaciones es el siguiente: las personas radican en México, dos en Guadalajara y dos más en Nuevo León; tres de las personas son del sexo masculino y una femenino; una persona es empresario, otro consultor independiente y dos más empleados de empresas privadas; oscilan en edades de entre 35 y 70 años y; todos cuentan con estudios superiores terminados. El criterio de elegibilidad de participantes fue que tuvieran en promedio una publicación por día en las últimas tres semanas, previo al inicio del estudio. Es importante mencionar que, para asegurar confidencialidad, en el reporte se utilizan seudónimos. Las publicaciones fueron seleccionadas según el criterio de \textcite{creswell2018} de recopilación de muestras, en el que se definieron como requisitos: A) publicaciones relacionadas sobre la COVID-19 y B) que hayan sido realizadas entre el 15 de marzo y hasta el 30 de abril del año 2020. Por lo tanto, la revisión fue selectiva, pues solo se seleccionaron las publicaciones relacionadas con el tema estudiado (COVID-19) y no todas las interacciones del participante. Además, para el análisis fueron consideradas solo las publicaciones que tuvieran texto para poder identificar la emoción reflejada en el mismo y en los comentarios generados.

Una vez recolectadas las publicaciones y sus respectivas reacciones (comentarios de sus seguidores), se realizó un análisis de contenidos usando los textos del post y los comentarios hacia ellos como las unidades de análisis. En el análisis de contenido el contenido de los datos es interpretado subjetivamente a través de un proceso sistemático de identificación de patrones y temáticas \cite{hsieh2005}. Esta metodología ha sido utilizada anteriormente en investigaciones en el contexto de redes sociales en línea \cite{woolley2010,hum2011,parsons2013,valerioUrea2015}.

Para asegurar la fiabilidad de la codificación de los evaluadores, dos investigadores realizaron la codificación de manera separada. La consistencia se midió con base en la proporción de observaciones en las cuales los codificadores coincidieron. Según \textcite{neuendorf2002}, un coeficiente de 0.7 o mayor, en este tipo de comparaciones, se considera apropiado; en este caso se tuvo un porcentaje de concordancia de 0.86. Para el caso de las divergencias, los dos investigadores analizaron juntos las unidades de análisis para llegar a un acuerdo sobre su codificación. 

Se utilizó una codificación deductiva, según \textcite{miles2013}, en esta aproximación se parte de una lista inicial de códigos previamente definida proveniente del marco conceptual. En esta investigación el marco conceptual de partida estuvo conformado por: los propósitos de comunicación de \textcite{monroe1973}, los tipos de formato de los mensajes de \textcite{pletikosaCvijikj2013} y las categorías de las emociones de \textcite{ekman1992,fredrickson2004}. El detalle de estos códigos se puede revisar en la \Cref{tab01}, así como una breve descripción de cada elemento. 

Se creó una guía para unificar los criterios de clasificación de los investigadores. La guía consistía en la definición detallada de cada código, así como algunas consideraciones, además de ejemplos para cada caso. Ambos investigadores revisaron las publicaciones para determinar la clasificación: reconociendo el tipo de propósito de comunicación y el tipo de emoción expresada, posteriormente se contabilizó la frecuencia de aparición para cada categoría. 

%
% TABELA 1
%
\begin{small}
\begin{longtable}{
    >{\raggedright\arraybackslash}
    p{0.25\textwidth}
    p{0.65\textwidth}
    }
\caption{Categorías de Propósito de comunicación, Tipo de formato del mensaje y de Emociones.}
\label{tab01}
\\
\toprule
\multicolumn{2}{c}{Categoría de Propósito de comunicación \cite{monroe1973}} \\
\midrule
Propósito de comunicación & Descripción \\
\midrule
Informar &
Publicaciones con el objetivo de brindar información para dar a conocer algo. En su mayoría se consideraron las publicaciones de canales informativos redistribuidas.  \\
Persuadir & 
Publicaciones con la finalidad de influir a las personas para motivarlas a la acción. Se consideraron en la categoría publicaciones con invitación a hacer o no algo, por ejemplo: quedarse en casa. \\
Entretener & 
Publicaciones generalmente usas para divertir a las personas. Por ejemplo, memes. \\
\toprule
\multicolumn{2}{c}{Tipo de Formato del mensaje \cite{pletikosaCvijikj2013}} \\
\midrule
Formato del Mensaje & Descripción \\
\midrule
Imagen & Se incluyeron fotografías, memes o caricaturas. \\
Texto & Publicaciones que solamente incluían texto plano directamente o bien una imagen con palabras. \\
Video & Se consideraron publicaciones con videos personales o bien de espacios informativos. \\
Enlace & Publicaciones de redistribución de información que dirigían al usuario a otro sitio. \\
\toprule
\multicolumn{2}{c}{Categoría de Emociones \cite{ekman1992,fredrickson2004}} \\
Emociones & Descripción \\
\midrule
Enojo &
La emoción que impide que nuestro objetivo se cumpla. El enojo lleva a la ira que puede implicar hacerle daño a alguien. Se considero también en esta categoría expresiones de frustración, hartazgo o cansancio. \\
Disgusto &
Es la repugnancia por la vista, olor o sabor de algo, pero también provocada por acciones de personas que puede ser ofensivas. En esta categoría también se consideraron aquellos comentarios sarcásticos.  \\
Miedo & 
Es la emoción que puede paralizar a alguien o bien querer evitar o huir de alguna situación. \\ 
Tristeza & 
Es la respuesta a una pérdida de algo o alguien. Puede implicar también resignación o angustia. Se contemplaron comentarios reflejando también resignación, decepción o empatía.\\
Desprecio &
Sucede cuando se siente superior moralmente a otra persona. Fueron contempladas también expresiones de burla o desaprobación. \\
Alegría &
Es la emoción más parecida a la felicidad, sucede cuando un evento inesperado sucede, sentimiento de disfrute o de placer ante algún evento. También fueron consideradas las muestras de reconocimiento hacia otros. \\
Sorpresa / Asombro &
Es una respuesta breve a un evento inesperado o repentino que puede ser desagradable. O bien un evento que te hace experimentar la grandeza y sentirse parte de algo superior. \\
Agradecimiento &
Es una emoción más social, se experimenta cuando algo bueno te pasa y fue gracias a que alguien más hizo algo bueno. En ocasiones se experimenta el sentimiento de regresar el acto. \\
Esperanza & 
Se genera cuando en una situación difícil se puede creer que puede mejorar o cambiar.\\
Interés &
Es el impulso de explorar, tomar nueva información y experiencias para expandir el yo. Se puede considerar como un sentimiento de curiosidad. \\
Diversión &
Es la respuesta a algo que le parece gracioso o divertido, comúnmente expresado con risas. \\
Serenidad &
Se experimenta cuando crees que tus circunstancias son muy buenas y quisieras tener más de ellas. Es la respuesta de sentir tranquilidad cuando la persona se siente segura, relajada y sostenida. \\
Orgullo &
Es la respuesta positiva hacia el logro de algún objetivo por el esfuerzo invertido, cuando esa acción es apreciada por la sociedad, cultura o entorno.  \\
Inspiración &
Es la respuesta hacia algo o alguien que motiva a la persona estimulando también la imaginación y creatividad. \\
Amor &
Es el conjunto de emociones positivas cuando se experimentan relaciones seguras y estrechas.\\
\bottomrule
\source{elaboración propia}
\end{longtable}
\end{small}


En el procedimiento para el análisis de la información de las publicaciones se identificaron, para cada participante, las siguientes variables: el día de publicación, el formato usado, el número de reacciones, el tipo de reacciones, el número de veces compartida. Asimismo, se clasificaron: el propósito de comunicación, el tema expuesto y el tipo de emoción expresada en la publicación. Cabe señalar que la reacción “me importa” aún no era integrada en la red social cuando se obtuvieron los registros. Al final se contabilizaron las ocurrencias en las variables mencionadas. 


\section{Resultados}
Para facilitar la lectura de los resultados, se presentan tres secciones: 
\begin{enumerate*}[label=\arabic*)] 
\item las características de los participantes y sus publicaciones,
\item las características de las publicaciones con más interacciones y 
\item la presencia del contagio emocional en las publicaciones. 
\end{enumerate*}

Es importante aclarar que, aunque para hacer el análisis inicial se partió de las 15 emociones presentadas en la \Cref{tab01}, durante la codificación esas emociones se agruparon en solo seis grupos (\Cref{tab02}): dos grupos de emociones negativas (malestar y preocupación), tres grupos de emociones positivas (agradecimiento, sentimiento de pertenencia y alegría) y un grupo neutro (para aquellas publicaciones cuyo texto no expresaban ninguna emoción). Para el caso de la emoción sorpresa, fue interpretada según la connotación (positiva o negativa).

%
%
% TABELA 2
%
\begin{table}[htpb]
\caption{Categorías de emociones.}
\label{tab02}
\centering
\begin{tabular}{ll}
\toprule 
Categorías & Emociones \\
\midrule 
Malestar & Enojo, disgusto, desprecio. \\
Preocupación & Tristeza y miedo. \\
Agradecimiento & Agradecimiento, serenidad y esperanza. \\
Sentido de pertenencia & Interés, inspiración y orgullo. \\
Alegría & Diversión y felicidad \\
\bottomrule
\end{tabular}
\source{elaboración propia}
\end{table}



\subsection{Características generales de los participantes y sus publicaciones.}

En la \Cref{tab03}, se puede apreciar que Mario es el participante que tiene un mayor número de contactos (5000) y publicaciones sobre el COVID-19 en la red (498), seguido de Jeny con 2056 contacto y 125 publicaciones sobre COVID-19. Además, es importante destacar que, aunque Mario tiene 10 veces más contactos y publicaciones que Fernando, ambos tienen un promedio similar del número de reacciones por publicación de COVID-19, 11.79 en promedio para Mario y 11.41 para Fernando. 


%
% TABELA 3
%
\begin{table}[htpb]
\caption{Características de los participantes.}
\label{tab03}
\small
\centering
\begin{tabular}{lllcccc}
\toprule 
Participante &
Sexo &
\multicolumn{1}{p{1cm}}{\centering Rango de edad} &
\multicolumn{1}{p{1cm}}{\centering No. de amigos} &
\multicolumn{1}{p{1.5cm}}{\centering No. de publicaciones (COVID-19)} &
\multicolumn{1}{p{1.75cm}}{\centering Promedio de publicaciones por día (COVID-19)} &
\multicolumn{1}{p{2cm}}{\centering Promedio reacciones por publicación sobre el total de amigos (COVID-19)} \\
\midrule 
Guillermo & Masculino & 46-50 & 413 & 30 & 2 & 4.97 \\
Fernando & Masculino & 66-70 & 535 & 46 & 2 & 11.41 \\
Jeny & Femenino & 36-40 & 2056 & 125 & 4 & 5.08 \\
Mario & Masculino & 46-50 & 5000 & 498 & 12 & 11.79 \\
\bottomrule
\end{tabular}
\source{elaboración propia}
\end{table}

\subsection{Publicaciones en Facebook relacionadas con COVID-19 que generaron más interacción}

Para analizar este aspecto, se tomaron solo las 10 publicaciones con más interacciones para cada participante. Es importante mencionar que el valor de la interacción es calculado sumando la cantidad de reacciones, comentarios y veces que se comparte la publicación. Asimismo, es importante aclarar que no es posible hacer comparaciones entre participantes, ya que se desconoce el alcance de las publicaciones, al no conocer la configuración del usuario y sus publicaciones (públicos o privados).

Como se puede apreciar, en general se encontró que la información sobre la pandemia que se comparte con sus contactos a través de Facebook tiene como objetivos principales informar y entretener; asimismo, se presenta primordialmente a través de imágenes y en días hábiles. En la Tabla 4 se puede observar que la mayoría de las publicaciones populares se realizaron en días hábiles (4/4), predominando el uso de imágenes para trasmitir el mensaje (2/4). Por otro lado, el propósito inferido de la publicación se divide entre informar (2/4) y entretener (2/4). La reacción más común en la mayoría de las publicaciones fue “me encanta” (3/4) y los grupos de emociones generadas son malestar (2/4). 

%
% TABELA 4
%
\begin{table}[htpb]
\caption{Características de los participantes.}
\label{tab04}
\small
\centering
\begin{tabularx}{\textwidth}{XXXXXXXXXX}
\toprule 
Partici-pante & Día & Formato & Propósito & Promedio de reacciones &
Promedio de comentarios & Promedio de compartir & Promedio de interacción &
Reacción más común & Emoción más común \\
\midrule
Guillermo & Hábil (70\%) & Imagen (60\%) & Entretener (72.73\%) & 10 & 1 & 17 & 13 & “Me divierte” (23.52\%) & Malestar (66.67\%) \\
Fernando & Hábil (50\%) & Texto (50\%) & Informar (50\%) & 21 & 16 & 1 & 38 & “Me encanta” (14.28\%) & Agradeci-miento (33.33\%) \\
Jeny & Hábil (70\%) & Imagen (70\%) & Entretener (63.63\%) & 22 & 4 & 2 & 28 & “Me encanta” (29.95\%) & Alegría (50\%) \\
Mario & Hábil (70\%) & Enlace (90\%) & Informar (81.81\%) & 39 & 22 & 824 & 885 & “Me encanta” (18.76\%) & Malestar (60\%) \\
\bottomrule
\end{tabularx}
\source{elaboración propia}
\end{table}

Para apreciar mejor el tipo de publicaciones que tuvieron más interacciones, a continuación, se describen las publicaciones con el mayor número de interacciones para cada participante del estudio. 

En el caso de Guillermo, se trata de una publicación de tipo meme, en donde se hace referencia a los doctores como súper héroes. La publicación generó 21 reacciones: “me gusta” (14) y “me encanta” (7). Entre los comentarios, se pudo apreciar un GIF de la imagen de un actor aplaudiendo el contenido. 

La publicación que tuvo mayor interacción de Fernando fue una en la cual preguntó cómo les estaba yendo a sus amigos en la contingencia (consigo mismos). Esta publicación generó diferentes reacciones, la más frecuente fue “me gusta” (18) seguida de “me divierte” (4) y “me encanta” (3), también generó “me asombra” (2) y “me enoja” (1). La publicación generó algunos comentarios positivos, indicando que se caen bien o que son “un encanto”. Otros compartieron que están tranquilos, a gusto y positivos. Algunos otros, entre líneas comparten el desafío de tener niños y convivir con ellos de manera intensa. Otros indican que algunos días están bien algunos y otros no tan bien. En este intercambio, Fernando responde a todos los comentarios al menos con un “me gusta” o incluso contesta directamente. 

En el caso de Jeny, la publicación con más reacciones se trata de una publicación de carácter personal, en donde se expone cómo celebraron el cumpleaños de su sobrino, cuidando la sana distancia. Esta publicación generó 50 reacciones de Facebook y ocho comentarios. Las reacciones generadas son de tipo “me gusta” (21), “me encanta” (18) y “me divierte” (12). Por otra parte, en los comentarios se pudieron apreciar expresiones de diversión y risas por la mayoría de sus amigos. También hay algunos que contestan a la publicación en forma de broma. Otro dato importante es que todos los comentarios presentaron la reacción “me divierte” preconfigurara de Facebook. 

Por último, la publicación de Mario con más interacciones es una noticia sobre el aumento de la productividad en las empresas durante la pandemia. La publicación generó 41 “me gusta”, 11 “me encanta”, seis “me asombra” y un “me divierte”. Por otra parte, los comentarios de sus contactos varían entre bromas sobre los requerimientos de las empresas a deshoras, las ventajas de las empresas establecidas como los ahorros en costos y, por otro lado, algunos comentarios sobre la preferencia de trabajo en “\textit{home office}”. 


\subsection{Contagio emocional en los comentarios de las publicaciones sobre COVID-19}
A partir del universo de publicaciones hechas por los cuatro participantes que contenía comentarios (n=360), se seleccionaron aquellas publicaciones que cumplieran tres criterios: 
\begin{enumerate*}[label=\arabic*)]
\item que la publicación tuviera texto, 
\item que el texto de la publicación reflejara una emoción y 
\item que los comentarios reflejaran alguna emoción.
\end{enumerate*}
Con ello, se examinaron a profundidad 139 publicaciones para identificar el posible contagio emocional; es decir, para identificar en qué medida la emoción del autor de la publicación se reflejaba en los comentarios de sus contactos.  Ya que los casos no son comparables entre sí, se presentan por separado los resultados de cada uno de los participantes. 

La \Cref{tab05} muestra el análisis realizado a las publicaciones de Fernando. En su caso, el 28.26\% (n=13) de sus publicaciones expresaban emociones, la mayoría eran relacionadas con agradecimiento (13.04\%). De estas emociones expresadas en los textos de sus publicaciones, todas excepto una (alegría) generaron principalmente la misma emoción en los comentarios de sus contactos. Es decir, que se refleja cierto grado de contagio emocional a partir de todas las emociones, excepto alegría. En la única publicación donde él expresaba alegría, los comentarios de sus contactos proyectaban preocupación.

%
% TABELA 5
%
\begin{table}[htpb]
\caption{Resumen de emociones expresadas en las publicaciones de Fernando y los comentarios de sus contactos.}
\label{tab05}
\centering
\begin{tabular}{lccccccc}
\toprule 
 & \multicolumn{6}{p{4cm}}{Emociones expresadas en los comentarios de las publicaciones} & \\
% \cline{2-7}
 & \rot{Malestar} & \rot{Preocupación} & \rot{Agradecimiento} & \rot{Sentido de pertenencia } & \rot{Alegría} & \rot{Neutro} & \rot{No. Comentarios} \\
Publicaciones & & & & & & & \\
\midrule
Malestar (1 / 2.17\%) & \cellcolor{gray!25} 4 & 0 & 0 & 0 & 0 & 0 & 4 \\
Preocupación (4 / 8.70\%) & 1 & \cellcolor{gray!25} 7 & 3 & 0 & 0 & 5 & 15 \\
Agradecimiento (6 / 13.04\%) & 0 & 6 & \cellcolor{gray!25} 13 & 1 & 0 & 11 & 31 \\
Sentido de pertenencia (1 / 2.17\%) & 0 & 0 & 0 & \cellcolor{gray!25} 4 & 0 & 5 & 9 \\
Alegría (1 / 2.17\%) & 0 & \cellcolor{gray!25} 2 & 0 & 1 & 0 & 1 & 4 \\
\bottomrule
\end{tabular}
\source{Fonte da tabela.}
\end{table}


La \Cref{tab06} muestra el análisis realizado a las publicaciones de Jeny. En su caso, solo el 4.8\% (n=6) de sus publicaciones expresaban emociones, aunque solamente tres tipos (malestar, sentido de pertenencia y alegría). Sin embargo, los tres tipos de emociones expresadas en los textos de sus publicaciones generaron principalmente la misma emoción en los comentarios de sus contactos. 

%
% TABELA 6
%
\begin{table}[htpb]
\caption{Resumen de emociones expresadas en las publicaciones de Jeny y los comentarios de sus contactos.}
\label{tab06}
\centering
\begin{tabular}{lccccc}
\toprule 
 & \multicolumn{5}{p{4cm}}{Emociones expresadas en los comentarios de las publicaciones} \\
 & \rot{Malestar} & \rot{Sentido de pertenencia} & \rot{Alegría} & \rot{Neutro} & \rot{No. Comentarios} \\
Publicaciones & & & & & \\
\midrule
Malestar (2 / 1.60\%) & \cellcolor{gray!25} 5 & 0 & 0 & 1 & 6 \\
Sentido de pertenencia (1 / 0.80\%) & 0 & \cellcolor{gray!25} 1 & 0 & 0 & 1 \\
Alegría (3 / 2.40\%) & 1 & 0 & \cellcolor{gray!25} 4 & 0 & 5 \\
\bottomrule
\end{tabular}
\source{Fonte da tabela.}
\end{table}


Por otro lado, la \Cref{tab07} muestra el análisis realizado a las publicaciones de Mario. En su caso, el 19.07\% (n=95) de sus publicaciones expresaban emociones, principalmente malestar. De estas emociones expresadas en los textos de sus publicaciones, todas excepto una (preocupación) generaron principalmente la misma emoción en los comentarios de sus contactos. Las publicaciones donde él expresaba preocupación generaron mayormente malestar. Sin embargo, ambas emociones tienen una connotación negativa. 

%
% TABELA 7
%
\begin{table}[htpb]
\caption{Resumen de emociones expresadas en las publicaciones de Mario y los comentarios.}
\label{tab07}
\centering
\begin{tabular}{lccccccc}
\toprule 
 & \multicolumn{6}{p{4cm}}{Emociones expresadas en los comentarios de las publicaciones} & \\
 & \rot{Malestar} & \rot{Preocupación} & \rot{Agradecimiento} & \rot{Sentido de pertenencia} & \rot{Alegría} & \rot{Neutro} & \rot{No. Comentarios} \\
Publicaciones & & & & & & & \\
\midrule
Malestar (37 / 7.43\%) & \cellcolor{gray!25} 44 & 9 & 2 & 0 & 8 & 51 & 114 \\
Preocupación (34 / 6.83\%) & \cellcolor{gray!25} 34 & 21 & 3 & 0 & 2 & 67 & 127 \\
Agradecimiento (6 / 1.20\%) & 1 & 1 & \cellcolor{gray!25} 3 & 0 & 0 & 5 & 10 \\
Sentido de pertenencia (2 / 0.04\%) & 0 & 0 & 0 & \cellcolor{gray!25} 3 & 0 & 4 & 7 \\
Alegría (16 / 3.21\%) & 6 & 1 & 0 & 3 & \cellcolor{gray!25} 10 & 19 & 39 \\
\bottomrule
\end{tabular}
\source{Fonte da tabela.}
\end{table}

Por último, en el caso de Guillermo, quien tuvo menos publicaciones (n=30), solamente se identificó una publicación proyectando una emoción, que obtuvo de igual forma un solo comentario; expresó una emoción de molestia, que provocó un comentario expresando la misma emoción. 


\section{Discusión y conclusiones}

En el estudio se encontró evidencia de que la interacción a través de las redes sociales, ante una situación de contingencia, puede resultar útil para mantener la comunicación y las relaciones entre las personas. En la contingencia, las redes sociales se han convertido en una herramienta para ayudar, entre otras cosas, a informar, compartir información, pero también para estar conectados y recibir y dar apoyo a familiares y amigos \cite{saud2020}. En este sentido, la investigación comprueba que los usuarios utilizan las redes sociales para informar todo lo relacionado sobre el virus y la enfermedad, las recomendaciones, los temas políticos, económicos o sociales, así como mejorar el conocimiento colectivo \cite{saud2020}. 

El estudio mostró, como otros estudios \cite{rosenbusch2019,guo2020,sasaki2021}, indicios de que el contagio emocional está presente en las publicaciones que se comparten a través de las redes sociales (en este caso Facebook). Aunque, algunas emociones expresadas en las publicaciones no coincidieron exactamente con las expresadas en los comentarios, en su mayoría sí pertenecían al mismo tipo (negativas o positivas). Por ejemplo, una publicación que expresaba preocupación podía generar emociones de malestar. Este hallazgo también es confirmado en los resultados del estudio de \textcite{goldenberg2020}, que indicaba que una prevalente expresión de emociones negativas podría conducir a una propagación acelerada de emociones negativas. Esto también coincide con los resultados de estudios anteriores, como el de \textcite{hatfield1993}, donde encontraron que las emociones negativas generaban emociones negativas.

Por otro lado, en el estudio se encontró que algunas publicaciones pueden generar emociones contradictorias. Por ejemplo, algunas publicaciones que expresaban emociones positivas, como alegría (aunque la información no fuera agradable), generaron emociones negativas, como malestar. Una explicación a esto podría ser el reflejo de un mecanismo de defensa mental para hacer frente a las tragedias \cite{dundes1987} y así regular emociones y mantenerse psicológicamente sanos \cite{Bazarova2015,iglesiasSanchez2020}. Otra explicación podría ser, que es una característica particular de la cultura mexicana, la cual presenta una amplia capacidad para ver con humor las situaciones de desgracia \cite{ruiz2018} y esto a su vez fortifica el vínculo social \cite{yus2018}. Por lo tanto, podría ser que los participantes del estudio y sus contactos tratan de amortiguar sus emociones negativas con publicaciones aparentemente divertidas. 

Tal parece que el uso de las redes sociales, ante una situación de contingencia, pueden ser una herramienta útil para mantenerse informados sobre las implicaciones y precauciones ante la situación particular. En el caso analizado, en el periodo en estudio (los primeros meses de la pandemia), podemos asumir que lo más importante era disminuir la incertidumbre que generaba el nuevo virus. Además de informar, las redes sociales se utilizaron también para entretener, sobre todo mediante imágenes en forma de memes, pero también para compartir ciertas situaciones personales relacionadas con la contingencia y su impacto en la vida diaria. En el caso de los memes, generalmente reflejaban la situación diaria de las personas y las circunstancias políticas, y económicas, del país ante los efectos de la contingencia. 

Por último, es importante notar que las características del contenido que generaron más interacciones no es el mismo en todos los participantes. Es decir, si bien se pueden identificar las características afines a aquellas publicaciones con un mayor grado de interacción, como publicaciones hechas en días hábiles con el uso de imágenes, éstas no aplican de la misma forma para todos los participantes. Un factor importante que define el nivel de interacción generado es seguramente el nivel de influencia que tiene el emisor del mensaje. En otras palabras, no todas las personas ejercen el mismo nivel de influencia en sus contactos. Sin embargo, dado que el estudio confirmó que el contagio emocional se da en alguna proporción en las redes sociales, es importante tomar consciencia que la información que compartimos puede influir en las emociones de nuestros contactos, por lo que la difusión de información falsa o errónea combinada con estados de cuarentena puede provocar ansiedad, depresión y, en el extremo, llevar al suicidio \cite{gonzlezPadilla2020}. Aunado a esto, dado que la gente responde a rumores no confirmados científicamente y por lo tanto resulta ser un reto confiar en el juicio de los receptores \cite{cato2021,wheaton2021,prikhidko2020}, es necesaria una visión crítica que, junto con los marcos interpretativos, pueda ayudar a discriminar información y minimizar los posibles efectos negativos \cite{salazar2018}. 

Es importante hacer notar que estas conclusiones están supeditadas a las limitaciones del estudio. Una de las limitaciones de este estudio fue el número y el perfil de los participantes (emisores) analizados. Ante ello, en futuras investigaciones, se recomienda ampliar el estudio a un número mayor de participantes para tener una visión más amplia. Relacionado con esto, y ya que se encontraron indicios de que el perfil del emisor de los mensajes puede ser un factor clave en el nivel de contagio emocional generado, sería importante ampliar los criterios de selección de los participantes en el estudio para poder analizar qué características de los participantes tienen mayor impacto en la influencia generada. Por último, es importante hacer notar que en un estudio como este depende en gran medida de la configuración de privacidad que los usuarios de estas herramientas tengan definidos. Esto limita el cálculo del \textit{engagement} (involucramiento) generado ya que es imposible conocer el alcance real de una publicación.


\printbibliography\label{sec-bib}
% if the text is not in Portuguese, it might be necessary to use the code below instead to print the correct ABNT abbreviations [s.n.], [s.l.] 
%\begin{portuguese}
%\printbibliography[title={Bibliography}]
%\end{portuguese}


\end{document}
