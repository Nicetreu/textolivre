% !TEX TS-program = XeLaTeX
% use the following command: 
% all document files must be coded in UTF-8
\documentclass{textolivre}
% for anonymous submission
%\documentclass[anonymous]{textolivre}
% to create HTML use 
%\documentclass{textolivre-html}
% See more information on the repository: https://github.com/leolca/textolivre

% Metadata
\begin{filecontents*}[overwrite]{article.xmpdata}
    \Title{Editorial}
    \Author{Daniervelin Pereira}
    \Language{pt-BR}
    \Journaltitle{Texto Livre}
    \Journalnumber{1983-3652}
    \Volume{14}
    \Issue{1}
    \Firstpage{1}
    \Lastpage{2}
    \setRGBcolorprofile{sRGB_IEC61966-2-1_black_scaled.icc}
            {sRGB_IEC61966-2-1_black_scaled}
            {sRGB IEC61966 v2.1 with black scaling}
            {http://www.color.org}
\end{filecontents*}

% used to create dummy text for the template file
\definecolor{dark-gray}{gray}{0.35} % color used to display dummy texts
\usepackage{lipsum}
\SetLipsumParListSurrounders{\colorlet{oldcolor}{.}\color{dark-gray}}{\color{oldcolor}}

% used here only to provide the XeLaTeX and BibTeX logos
\usepackage{hologo}

% used in this example to provide source code environment
%\crefname{lstlisting}{lista}{listas}
%\Crefname{lstlisting}{Lista}{Listas}
%\usepackage{listings}
%\renewcommand\lstlistingname{Lista}
%\lstset{language=bash,
        breaklines=true,
        basicstyle=\linespread{1}\small\ttfamily,
        numbers=none,xleftmargin=0.5cm,
        frame=none,
        framexleftmargin=0.5em,
        framexrightmargin=0.5em,
        showstringspaces=false,
        upquote=true,
        commentstyle=\color{gray},
        literate=%
           {á}{{\'a}}1 {é}{{\'e}}1 {í}{{\'i}}1 {ó}{{\'o}}1 {ú}{{\'u}}1 
           {à}{{\`a}}1 {è}{{\`e}}1 {ì}{{\`i}}1 {ò}{{\`o}}1 {ù}{{\`u}}1
           {ã}{{\~a}}1 {ẽ}{{\~e}}1 {ĩ}{{\~i}}1 {õ}{{\~o}}1 {ũ}{{\~u}}1
           {â}{{\^a}}1 {ê}{{\^e}}1 {î}{{\^i}}1 {ô}{{\^o}}1 {û}{{\^u}}1
           {ä}{{\"a}}1 {ë}{{\"e}}1 {ï}{{\"i}}1 {ö}{{\"o}}1 {ü}{{\"u}}1
           {Á}{{\'A}}1 {É}{{\'E}}1 {Í}{{\'I}}1 {Ó}{{\'O}}1 {Ú}{{\'U}}1
           {À}{{\`A}}1 {È}{{\`E}}1 {Ì}{{\`I}}1 {Ò}{{\`O}}1 {Ù}{{\`U}}1
           {Ã}{{\~A}}1 {Ẽ}{{\~E}}1 {Ũ}{{\~u}}1 {Õ}{{\~O}}1 {Ũ}{{\~U}}1
           {Â}{{\^A}}1 {Ê}{{\^E}}1 {Î}{{\^I}}1 {Ô}{{\^O}}1 {Û}{{\^U}}1
           {Ä}{{\"A}}1 {Ë}{{\"E}}1 {Ï}{{\"I}}1 {Ö}{{\"O}}1 {Ü}{{\"U}}1
           {ç}{{\c{c}}}1 {Ç}{{\c{C}}}1
}


\journalname{Texto Livre: Linguagem e Tecnologia}
\thevolume{14}
\thenumber{1}
\theyear{2021}
\receiveddate{\DTMdisplaydate{2021}{4}{30}{-1}} % YYYY MM DD
\accepteddate{\DTMdisplaydate{2021}{4}{30}{-1}}
\publisheddate{\DTMdisplaydate{2021}{4}{30}{-1}}
% Corresponding author
\corrauthor{Daniervelin Pereira}
\articledoi{33562}
\articleid{33562}
% list of available sesscions in the journal: articles, dossier, reports, essays, reviews, interviews, editorial
\articlesessionname{editorial}
% Abbreviated author list for the running footer
\runningauthor{Pereira}
\editorname{Leonardo Araújo}

\title{Editorial}
% if there is a third language title, add here:
%\othertitle{Artikelvorlage zur Einreichung beim Texto Livre Journal}

\author[1]{Daniervelin Pereira \orcid{0000-0003-1861-3609} \thanks{Email: \url{drenata@ufmg.br}}}

\affil[1]{Universidade Federal de Minas Gerais, Belo Horizonte, MG, Brasil.}
%\addbibresource{tl-article-template.bib}
% use biber instead of bibtex
% $ biber tl-article-template

% set language of the article
\setdefaultlanguage[variant=brazilian]{portuguese}
\setotherlanguage{english}

% for spanish, use:
%\setdefaultlanguage{spanish}
%\gappto\captionsspanish{\renewcommand{\tablename}{Tabla}} % use 'Tabla' instead of 'Cuadro'
%\AfterEndPreamble{\crefname{table}{tabla}{tablas}}

% for languages that use special fonts, you must provide the typeface that will be used
% \setotherlanguage{arabic}
% \newfontfamily\arabicfont[Script=Arabic]{Amiri}
% \newfontfamily\arabicfontsf[Script=Arabic]{Amiri}
% \newfontfamily\arabicfonttt[Script=Arabic]{Amiri}
%
% in the article, to add arabic text use: \textlang{arabic}{ ... }

% to use emoticons in your manuscript
% https://stackoverflow.com/questions/190145/how-to-insert-emoticons-in-latex/57076064
% using font Symbola, which has full support
% the font may be downloaded at:
% https://dn-works.com/ufas/
% add to preamble:
% \newfontfamily\Symbola{Symbola}
% in the text use:
% {\Symbola }

% reference itens in a descriptive list using their labels instead of numbers
% insert the code below in the preambule:
\makeatletter
\let\orgdescriptionlabel\descriptionlabel
\renewcommand*{\descriptionlabel}[1]{%
  \let\orglabel\label
  \let\label\@gobble
  \phantomsection
  \edef\@currentlabel{#1\unskip}%
  \let\label\orglabel
  \orgdescriptionlabel{#1}%
}
\makeatother
%
% in your document, use as illustraded here:
%\begin{description}
%  \item[first\label{itm1}] this is only an example;
%  % ...  add more items
%\end{description}
 

% custom epigraph - BEGIN 
%%% https://tex.stackexchange.com/questions/193178/specific-epigraph-style
\usepackage{epigraph}
\renewcommand\textflush{flushright}
\makeatletter
\newlength\epitextskip
\pretocmd{\@epitext}{\em}{}{}
\apptocmd{\@epitext}{\em}{}{}
\patchcmd{\epigraph}{\@epitext{#1}\\}{\@epitext{#1}\\[\epitextskip]}{}{}
\makeatother
\setlength\epigraphrule{0pt}
\setlength\epitextskip{0.5ex}
\setlength\epigraphwidth{.7\textwidth}
% custom epigraph - END


% if you use multirows in a table, include the multirow package
\usepackage{multirow}

%\usepackage{lineno}
%\linenumbers
\begin{document}
\maketitle

Os tempos de pandemia de COVID-19, vivenciados desde 2020, nos colocaram em várias situações de risco. Entre elas, professores e alunos, em diferentes níveis de ensino foram expostos, muitas vezes pela primeira vez, a contextos de ensino-aprendizagem formal mediado pelas tecnologias digitais. Outros professores e alunos já habituados ao uso delas em sala de aula se viram diante de uma rotina cansativa de exposição a essas tecnologias. Vários problemas, de naturezas diferentes, surgiram ou retornaram no ensino remoto, como a gestão do tempo, a adequação das atividades das disciplinas ao perfil dos alunos, a desigualdade no acesso aos recursos necessários para participação das atividades educativas (internet, computadores, ambiente tranquilo e silencioso para as aulas, por exemplo), depressão decorrente do conturbado momento histórico vivenciado, entre outros.

Cenários como esse demandam pesquisas que busquem soluções. Alguns artigos apresentados no número 1 do volume 14 de 2021, da revista Texto Livre: Linguagem e Tecnologia, dialogam com esse contexto de pandemia. Outros não fazem isso diretamente, mas podem contribuir para novas pesquisas, para fornecer repertório de experiências de ensino importantes para inspirar professores e promover reflexões diversas.  

O leitor encontrará no eixo Educação e Tecnologia artigos que tratam do contexto da pandemia relacionado ao TDAH (Transtorno do Déficit de Atenção com Hiperatividade), por \href{https://doi.org/10.35699/1983-3652.2021.25043}{Sineide Gonçalves e Bárbara Eduarda Barbosa Ferreira}, e às práticas e papéis na educação, por \href{https://doi.org/10.35699/1983-3652.2021.24941}{Joyce Vieira Fettermann e Annabell Dell Real Tamariz}. Encontra ainda trabalhos sobre a aprendizagem da tecnologia na área de língua espanhola e suas literaturas (\href{https://doi.org/10.35699/1983-3652.2021.26394}{Elisabeth Melguizo Moreno}), abordagem conceitual de três tipos de letramento (\href{https://doi.org/10.35699/1983-3652.2021.29513}{Julián Rodríguez López, Maricela López Ornelas, Katiuska Fernández Morales e Javier Organista Sandoval}), uso da Wikiversidade no ensino do jornalismo científico (\href{https://doi.org/10.35699/1983-3652.2021.24935}{Daniel Almeida Abrahão Dieb, João Alexandre Peschanski e Fernando Jorge da Paixão}), formação de professores de Música pelas TIC (\href{https://doi.org/10.35699/1983-3652.2021.25419}{Fernando José Sadio-Ramos, María Angustias Ortiz-Molina e María del Mar Bernabé-Villodre}), fatores que influenciam a utilização das redes sociais (\href{https://doi.org/10.35699/1983-3652.2021.25420}{Melchor Gómez-García, Moussa Boumadan, Roberto Soto-Varela e Ángeles Gutiérrez-García}), quadro Comparativo Europeu DigCompEdu e Quadro Comum para o Ensino de Competência Digital (\href{https://doi.org/10.35699/1983-3652.2021.25740}{Julio Cabero-Almenara, Juan Jesús Gutiérrez-Castillo, Antonio Palacios-Rodríguez, Julio Barroso-Osuna}), produção científica de aprendizagem invertida e sala de aula invertida (\href{https://doi.org/10.35699/1983-3652.2021.26266}{Jesús López-Belmonte, Antonio-José Moreno-Guerrero, Juan-Antonio López-Núñez e Santiago Pozo-Sánchez}) e práticas educativas ambientais na formação de educadores das infâncias (\href{	https://doi.org/10.35699/1983-3652.2021.25698}{Eliane Lima Piske, Narjara Mendes Garcia e Maria Angela Mattar Yunes}).

No eixo Produção Textual e Tecnologia, foram publicados neste número textos sobre: construção e ressignificação das práticas tecnológicas em entornos escolares (\href{https://doi.org/10.35699/1983-3652.2021.29523}{Ana Yamile Pérez Puentes e Juan Guillermo Diaz Bernal}) e competência léxica e escrita acadêmica (\href{https://doi.org/10.35699/1983-3652.2021.24560}{Gabriel Valdés-León}).

Em Linguística e Tecnologia, encontram-se reunidos artigos sobre: narrativas de evolução (\href{https://doi.org/10.35699/1983-3652.2021.26711}{Mauricio Teixeira Mendes}), principais vertentes dos estudos do letramento no Brasil (\href{https://doi.org/10.35699/1983-3652.2021.29164}{Cícero da Silva e Adair Vieira Gonçalves}), concepções de letramento para o ensino da língua portuguesa em tempos de uso de artefatos digitais (\href{https://doi.org/10.35699/1983-3652.2021.24366}{Márcia Aparecida Vergna}), letramentos e novas tecnologias no contexto da Educação do Campo (\href{https://doi.org/10.35699/1983-3652.2021.26765}{Carlos Henrique Silva de Castro}), ensino síncrono e assíncrono a distância de anáfora em línguas estrangeiras (\href{https://doi.org/10.35699/1983-3652.2021.29177}{Amanda Maraschin Bruscato e Jorge Baptista}), análise de pedidos de ajuda multimodais de um grupo de Facebook (\href{https://doi.org/10.35699/1983-3652.2021.24391}{Theodoro Casalotti Farhat e Paulo Roberto Gonçalves-Segundo}), desenvolvimento de corretor ortográfico (\href{https://doi.org/10.35699/1983-3652.2021.26469}{Leonardo Carneiro de Araujo, Aline de Lima Benevides e João Pedro Hallack Sansão}) e análise dos novos anglicismos léxicos na língua espanhola no contexto das obras e corpus acadêmico digital (\href{https://doi.org/10.35699/1983-3652.2021.24418}{David Giménez Folqués}).

No eixo Comunicação e Tecnologia, fala-se da crise sócio-ecológica e comunicação durante a Maré Vermelha de Chiloé (2016) (\href{https://doi.org/10.35699/1983-3652.2021.26231}{Jorge Valdebenito Allendes}), do contágio emocional nas redes sociais (\href{https://doi.org/10.35699/1983-3652.2021.29080}{Cynthia Pasquel-López e Gabriel Valerio-Ureña}) e de marketing digital e posicionamento web na comunicação científica (\href{	https://doi.org/10.35699/1983-3652.2021.26251}{Sara Mandiá Rubal e Maricela López Ornelas}).

No eixo Ensino Superior e Tecnologia, temos um estudo de caso no mestrado-integrado de Engenharia Eletrotécnica e de Computadores na Universidade de Trás-os-Montes e Alto Douro, de Portugal (\href{https://doi.org/10.35699/1983-3652.2021.26709}{Cleber Augusto Pereira, Paulo Moura Oliveira e Manuel José Cabral dos Santos Reis}).
	
No eixo Robótica pedagógica, os leitores encontram formas de explorar a matemática e a física com o robô seguidor de linha na perspectiva da robótica livre (\href{https://doi.org/10.35699/1983-3652.2021.24895}{Daniel da Silveira Guimarães, Élida Alves da Silva e Fernando da Costa Barbosa}).

No eixo Tradução e Tecnologia, encontramos um trabalho sobre Dicionários contextuais (\href{https://doi.org/10.35699/1983-3652.2021.26501}{Wafa Bedjaoui, Bahia Zemni, Hayfa Almalki e Marwa Elsaadany}) e um sobre pesquisa terminológica e tradução automática (\href{https://doi.org/10.35699/1983-3652.2021.26501}{Bahia Zemni, Wafa Bedjaoui e Marwa Elsaadany}).
	
Finalizamos este editorial desejando que os artigos publicados sejam produtivos e incentivem novas pesquisas nas áreas em que se inscrevem.

\end{document}
