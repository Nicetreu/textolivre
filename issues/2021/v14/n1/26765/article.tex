% !TEX TS-program = XeLaTeX
% use the following command: 
% all document files must be coded in UTF-8
\documentclass{textolivre}
% for anonymous submission
%\documentclass[anonymous]{textolivre}
% to create HTML use 
%\documentclass{textolivre-html}
% See more information on the repository: https://github.com/leolca/textolivre

% Metadata
\begin{filecontents*}[overwrite]{article.xmpdata}
    \Title{Universidade, letramentos e novas tecnologias no contexto da Educação do Campo}
    \Author{Carlos Henrique Silva de Castro}
    \Language{pt-BR}
    \Keywords{Universidade \sep Letramentos \sep Letramentos acadêmicos \sep Letramentos digitais \sep Educação do campo}
    \Journaltitle{Texto Livre}
    \Journalnumber{1983-3652}
    \Volume{14}
    \Issue{1}
    \Firstpage{1}
    \Lastpage{16}
    \Doi{10.35699/1983-3652.2021.26765}

    \setRGBcolorprofile{sRGB_IEC61966-2-1_black_scaled.icc}
            {sRGB_IEC61966-2-1_black_scaled}
            {sRGB IEC61966 v2.1 with black scaling}
            {http://www.color.org}
\end{filecontents*}

% used to create dummy text for the template file
\definecolor{dark-gray}{gray}{0.35} % color used to display dummy texts
\usepackage{lipsum}
\SetLipsumParListSurrounders{\colorlet{oldcolor}{.}\color{dark-gray}}{\color{oldcolor}}

% used here only to provide the XeLaTeX and BibTeX logos
\usepackage{hologo}

% used in this example to provide source code environment
%\crefname{lstlisting}{lista}{listas}
%\Crefname{lstlisting}{Lista}{Listas}
%\usepackage{listings}
%\renewcommand\lstlistingname{Lista}
%\lstset{language=bash,
        breaklines=true,
        basicstyle=\linespread{1}\small\ttfamily,
        numbers=none,xleftmargin=0.5cm,
        frame=none,
        framexleftmargin=0.5em,
        framexrightmargin=0.5em,
        showstringspaces=false,
        upquote=true,
        commentstyle=\color{gray},
        literate=%
           {á}{{\'a}}1 {é}{{\'e}}1 {í}{{\'i}}1 {ó}{{\'o}}1 {ú}{{\'u}}1 
           {à}{{\`a}}1 {è}{{\`e}}1 {ì}{{\`i}}1 {ò}{{\`o}}1 {ù}{{\`u}}1
           {ã}{{\~a}}1 {ẽ}{{\~e}}1 {ĩ}{{\~i}}1 {õ}{{\~o}}1 {ũ}{{\~u}}1
           {â}{{\^a}}1 {ê}{{\^e}}1 {î}{{\^i}}1 {ô}{{\^o}}1 {û}{{\^u}}1
           {ä}{{\"a}}1 {ë}{{\"e}}1 {ï}{{\"i}}1 {ö}{{\"o}}1 {ü}{{\"u}}1
           {Á}{{\'A}}1 {É}{{\'E}}1 {Í}{{\'I}}1 {Ó}{{\'O}}1 {Ú}{{\'U}}1
           {À}{{\`A}}1 {È}{{\`E}}1 {Ì}{{\`I}}1 {Ò}{{\`O}}1 {Ù}{{\`U}}1
           {Ã}{{\~A}}1 {Ẽ}{{\~E}}1 {Ũ}{{\~u}}1 {Õ}{{\~O}}1 {Ũ}{{\~U}}1
           {Â}{{\^A}}1 {Ê}{{\^E}}1 {Î}{{\^I}}1 {Ô}{{\^O}}1 {Û}{{\^U}}1
           {Ä}{{\"A}}1 {Ë}{{\"E}}1 {Ï}{{\"I}}1 {Ö}{{\"O}}1 {Ü}{{\"U}}1
           {ç}{{\c{c}}}1 {Ç}{{\c{C}}}1
}


\journalname{Texto Livre: Linguagem e Tecnologia}
\thevolume{14}
\thenumber{1}
\theyear{2021}
\receiveddate{\DTMdisplaydate{2020}{12}{19}{-1}} % YYYY MM DD
\accepteddate{\DTMdisplaydate{2021}{2}{5}{-1}}
\publisheddate{\today}
% Corresponding author
\corrauthor{Carlos Henrique Silva de Castro}
% DOI
\articledoi{10.35699/1983-3652.2021.26765}
% list of available sesscions in the journal: articles, dossier, reports, essays, reviews, interviews, editorial
\articlesessionname{Linguística e Tecnologia}
% Abbreviated author list for the running footer
\runningauthor{Castro}
\editorname{Daniervelin Pereira}

\title{Universidade, letramentos e novas tecnologias no contexto da Educação do Campo}
\othertitle{University, literacies and new technologies in rural education contexts}
% if there is a third language title, add here:
%\othertitle{Artikelvorlage zur Einreichung beim Texto Livre Journal}

\author[1]{Carlos Henrique Silva de Castro \orcid{0000-0001-8593-060X} \thanks{Email: \url{carlos.castro@ufvjm.edu.br}}}

\affil[1]{Universidade Federal dos Vales do Jequitinhonha e Mucuri, Diamantina, MG, Brasil.}

\addbibresource{article.bib}
% use biber instead of bibtex
% $ biber tl-article-template

% set language of the article
\setdefaultlanguage[variant=brazilian]{portuguese}
\setotherlanguage{english}

% for spanish, use:
%\setdefaultlanguage{spanish}
%\gappto\captionsspanish{\renewcommand{\tablename}{Tabla}} % use 'Tabla' instead of 'Cuadro'

% for languages that use special fonts, you must provide the typeface that will be used
% \setotherlanguage{arabic}
% \newfontfamily\arabicfont[Script=Arabic]{Amiri}
% \newfontfamily\arabicfontsf[Script=Arabic]{Amiri}
% \newfontfamily\arabicfonttt[Script=Arabic]{Amiri}
%
% in the article, to add arabic text use: \textlang{arabic}{ ... }


\begin{document}
\maketitle

\begin{polyabstract}
\begin{abstract}
Este trabalho traz reflexões sobre os letramentos ligados às esferas acadêmica e digital de graduandos de um curso denominado Licenciatura em Educação do Campo, ofertado pela Universidade Federal dos Vales do Jequitinhonha e Mucuri - UFVJM. Como objetivo principal, será apresentado como, no discurso desses estudantes, a universidade oportuniza eventos de letramento e suas relações com as novas tecnologias. A metodologia de pesquisa conta com a análise de tecnobiografias que, entendidas como narrativas biográficas das relações do estudante com as novas tecnologias, podem trazer informações de relevância para se entender a relação dos futuros professores não apenas com os letramentos acadêmicos, mas também a relação dos letramentos com as novas tecnologias. Nota-se que muitos têm suas primeiras experiências com gêneros das esferas acadêmica e digital já na universidade por meio de metodologias que se mostram bastante produtivas. Como resultado, conseguem ampliar não somente as habilidades necessárias às práticas letradas de suas novas esferas de interação, mas também suas leituras de todo um conjunto de significados envolvido nos novos contextos.

\keywords{Universidade \sep Letramentos \sep Letramentos acadêmicos \sep Letramentos digitais \sep Educação do campo.}
\end{abstract}

\begin{english}
\begin{abstract}
This work brings reflections about the literacies connected to the academic and digital spheres from a group of undergraduate students of a course for language teachers in rural contexts in Brazil. As a main objective, it will be presented how, in the students' discourse, the university provides literacy events and their relations with new technologies. The research methodology relies on the analysis of technobiographies which, understood as biographical narratives of the student's relations with new technologies, can bring relevant information in order to understand the relationship of the future teachers not only with academic literacies, but also its relationship with technologies. It’s noted that many of them have their first experiences with genres from the academic and digital spheres once they come to the university, through methodologies that prove to be quite productive. As a result, they expand not only their skills for the literate practices needed in their new spheres of interaction, but their readings of a set of meanings involved in these new contexts.

\keywords{University \sep Literacies \sep Academic literacies \sep Digital literacies \sep Rural education.}
\end{abstract}
\end{english}

% if there is another abstract, insert it here using the same scheme
\end{polyabstract}


\section{Para início de conversa}\label{sec-intro}
A preocupação com as técnicas que envolvem os processos de leitura e escrita justifica-se por algumas razões amplamente debatidas. Primeiramente, estamos em uma sociedade altamente letrada, na qual a escrita e a leitura são essenciais para uma enormidade de práticas. Outro fator de preocupação real são os índices de analfabetismo funcional que, no Brasil, atingiram 30\% da população em 2018\footnote{Dados preliminares do INAF 2018 disponível em: \url{http://acaoeducativa.org.br/wp-content/uploads/2018/0}}. Além disso, a visão de língua que predomina,  no âmbito social, também merece destaque, tendo em vista que hierarquiza conhecimentos, saberes e, consequentemente, usos linguísticos distintos. No caso da Educação do Campo, soma-se ao quadro geral o preconceito representado pela dicotomia campo-cidade e os estereótipos dele resultante. A expectativa de qualidade da educação no campo costuma ser baixa, a exemplo de outros territórios periféricos, devido a muitas questões, como a falta de escolas, a formação inadequada de professores e as metodologias de ensino descontextualizadas. Apesar dos avanços sociais e educacionais que ocorreram a partir de 2002, em um país onde cerca de 40\% dos docentes não têm formação adequada e 12\% sequer têm ensino superior\footnote{Segundo a Agência de notícias Brasil disponível em: \url{http://agenciabrasil.ebc.com.br/educacao/noticia/2016-03/quase-40-dos-professores-no-brasil-nao-tem-formacao-adequada}. Acesso em 12/12/2020.}, é evidente que a educação básica carece de mais empenho político, o que é contextual a essa discussão.

Os estudos de letramentos foram iniciados tardiamente no Brasil, já na década de 1980 com Mary Kato \cite{soares_letramento_1998}. Antes disso, a discussão mais ampla sobre leitura que tivemos foi fomentada por \textcite{freire_importancia_2011}, seus trabalhos de alfabetização no Nordeste brasileiro e o conceito de leitura de mundo que envolve toda a questão contextual e de percepção das relações sociais para além das letras. No entanto, suas ideias lhe renderam o cárcere imposto pela ditadura militar, como é amplamente sabido. Com os anos, a crescente industrialização do país, o crescimento das cidades e a abertura democrática levaram à necessidade de se entender a leitura e a escrita para além do código escrito. Assim, os estudos de letramento apontam para a necessidade do estudo de uma língua real, em contextos reais de uso \cite{marcuschi_da_2001, marcuschi_producao_2008, soares_letramento_1998, street_letramentos_2014}, diversa e mutável, a exemplo da nossa sociedade que organiza de diferentes formas seus gêneros textuais, em diferentes estilos e construções composicionais, sobre diferentes temas, de acordo com as esferas de atividade humana em que constrói seus discursos, seus textos \cite{bakhtin_speech_1986, bakhtin_marxismo_2006}.

A concepção de língua vem naturalmente se alterando ao longo dos anos de estudos linguísticos que, se considerarmos apenas o período em que a área é reconhecida como ciência, datam de pouco mais de 100 anos. \textcite{marcuschi_producao_2008} sintetiza quatro posições existentes sobre a noção de língua:
\begin{enumerate*}[label={(\alph*)}] 
\item como forma ou estrutura; 
\item como instrumento; 
\item como atividade cognitiva; 
\item como atividade sociointerativa situada. 
\end{enumerate*} 
Na primeira posição, a língua é tida como entidade abstrata e, descolada dos contextos, é analisada autonomamente e em níveis estruturais, como na gramática tradicional: fonológico, morfológico, sintático e semântico. A perspectiva que vê a língua como instrumento afasta seus estudos das questões cognitivas e sociais e, assim, discute questões superficiais e que pouco colaboram para o entendimento do processo de compartilhamento de informação como um processo natural. Já a terceira visão de língua como atividade cognitiva mostra-se radical na medida em que desconsidera qualquer questão social que envolva os processos de construção de sentido e, consequentemente, de língua. Para \textcite[p. 60]{marcuschi_producao_2008}, com essa concepção, “(\ldots) teríamos dificuldades de entender como é que a cultura, a experiência e nossa realidade cotidiana passam pela língua.” De uma amplitude maior, a noção de língua como atividade sociointerativa situada considera os fenômenos cognitivos e sociais – tais como tempos, espaços, identidades – que envolvem os processos de significação. Para \textcite[p. 60]{marcuschi_producao_2008}, essa concepção:

\begin{quote}
  Na realidade, contempla a língua em seu aspecto sistemático, mas observa-a em seu funcionamento social, cognitivo e histórico, predominando a ideia de que o sentido se produz situadamente e que a língua é um fenômeno encorpado e não abstrato e autônomo.
\end{quote}

Na mesma direção, \textcite{antunes_aula_2003} aponta que, ao longo dos estudos linguísticos, a percepção dos fatos de linguagem se divide em duas tendências: (a) uma desvinculada das condições de realização, os contextos, e focada nas regras; e (b) outra preocupada em explicar a língua como fenômeno social, enquanto atividade e interação entre sujeitos contextualmente situados. Assim como \textcite{antunes_aula_2003} e \textcite{marcuschi_producao_2008}, a visão de língua adotada neste trabalho é 

\begin{quote}
[...] interacionista, funcional e discursiva da língua, da qual deriva o princípio geral de que a língua só se atualiza a serviço da comunicação intersubjetiva, em situações de atuação social e através de práticas discursivas, materializadas em textos orais e escritos. \cite[p. 42]{antunes_aula_2003}  
\end{quote}

O trabalho de ensino e pesquisa conduzido por mim em minha universidade, do qual trago um conjunto de práticas para nossas reflexões, adota uma concepção interacionista da língua, na qual a prática com a leitura e a escrita se dá como objetivo primeiro dos letramentos. Em se tratando das práticas letradas em contexto acadêmico, as especificidades teóricas que conduzirão as análises serão explicitadas na segunda seção. Por ora, contextualizo um pouco mais as experiências. Este texto relata e reflete sobre práticas de letramentos no âmbito acadêmico e no contexto de interação mediado por novas tecnologias no curso de graduação denominado Licenciatura em Educação do Campo (LEC) ofertado pela Universidade Federal dos Vales do Jequitinhonha e Mucuri (UFVJM).

O curso forma professores para a educação básica em duas habilitações: (a) Linguagens e Códigos, que forma professores de português e inglês; e (b) Ciências da Natureza, que forma professores de biologia, física e química. O curso é ofertado em regime de alternância que, como esclarecem \textcite{magnani_da_2018}, refere-se a uma metodologia que acontece em dois espaços e em tempos diferentes. Há o tempo em que os estudantes saem de suas comunidades e vão até a universidade, o qual chamamos de Tempo Universidade (TU), e lá dão os retornos das pesquisas e trabalhos práticos desenvolvidos nas comunidades, além de cursar cerca de 80 por cento de toda a carga horária do curso ao longo de seis semanas por semestre. Após o TU, os estudantes retornam às suas comunidades, onde vivem o Tempo Comunidade (TC), no qual desenvolvem pesquisas para as disciplinas do semestre, preparam e aplicam práticas de ensino nas comunidades e escolas locais. A partir dessa realidade, a construção do Projeto Pedagógico de Curso (PPC) privilegia o diálogo de saberes entre os espaços envolvidos e, assim, dialoga diretamente com os pressupostos que aqui apresento. Em trabalho sobre a construção do curso e suas diretrizes epistemológicas, os autores pontuam que o esperado:

\begin{quote}
    [...] é um olhar que deve dialogar com as discussões de uma educação que se quer popular, a partir da necessidade de contextualização, do respeito a identidades e diferenças, do desenvolvimento de autonomia, entre outros pressupostos. \cite[p. 67]{magnani_da_2018}
\end{quote}

Este artigo baseia-se, sobretudo, nas experiências proporcionadas nas disciplinas e nos projetos que conduzo, sempre articulando ensino e pesquisa, e em narrativas autobiográficas produzidas pelos estudantes. Essas narrativas foram produzidas em dois formatos e quatro momentos diferentes, sendo que aquelas que serão utilizadas aqui serão as duas últimas. Em julho de 2016 e janeiro de 2017,  produziram uma coletânea de narrativas sobre suas práticas com a leitura e a escrita ao longo de suas vidas. O material foi editado e transformado, inicialmente, em um livro artesanal e, posteriormente, em um \textit{e-book}. Em janeiro e julho de 2018, produzimos uma nova coletânea de narrativas de letramentos, mas dessa vez consideramos as relações diretas com as tecnologias de informação e comunicação (TICs), que aqui chamaremos de novas tecnologias por acreditar que essa expressão delimita mais o termo no campo dos novos letramentos em discussão. Essa produção compõe um banco de dados \textit{on-line} e aberto\footnote{Tecnobiografias disponíveis em: \url{http://veramenezes.com/tecnocarlos.html}. Acesso em 08/02/2021.} para pesquisa ligada ao meu pós-doutoramento, finalizado no segundo semestre de 2019, tendo em vista que seu conteúdo está diretamente ligado ao objetivo deste trabalho. Mais detalhes sobre as questões metodológicas serão apresentados na terceira seção.

O objetivo principal deste trabalho é verificar como, no discurso dos estudantes, a universidade oportuniza eventos de letramento e suas relações com as novas tecnologias. Por se tratar do estudo de eventos de letramentos, práticas letradas e suas condições de realização que, conceitualmente, são localizáveis histórica e espacialmente e, portanto, é contextual, os letramentos da esfera acadêmica podem variar de acordo com outras questões, como a profissão que se busca, a região onde se pretende atuar, as tecnologias que medeiam as interações dos grupos envolvidos, ou seja, as práticas situadas de uso da língua escrita. Sendo assim, uma educação voltada para os letramentos deve trabalhar o ensino dos gêneros textuais e discursivos tendo em vista que eles representam as construções discursivas contextualizadas em esferas de atividade humana específicas. Há quem defenda que apenas o conceito de gêneros já dê conta da contextualização desejada; concordo, desde que não sejam ensinados de forma instrumental, mas como parte da construção de sentidos, em consonância com uma visão sociointeracionista de língua \cite{marcuschi_producao_2008, antunes_aula_2003}, em um ensino ideologicamente orientado como postula \textcite{street_letramentos_2014}, para atendimento aos propósitos dos sujeitos envolvidos.

A escolha por se estudar eventos, e não apenas as práticas, se dá pelas questões epistemológicas que envolvem a escolha dos termos. Enquanto práticas não pressupõem os contextos, os eventos de letramento ampliam o olhar da prática para todo o conjunto de valores que as cercam. \textcite{fischer_construcao_2007}, em revisão teórica para sua tese, diferencia os termos da seguinte forma:

\begin{quote}
    Se as práticas, no sentido de maneiras culturais de utilização do letramento, são unidades que não podem ser observadas, na sua totalidade, em pequenas atividades e tarefas, os \textbf{eventos de letramento} representam episódios observáveis, os quais se formam e se constituem através dessas práticas. Eventos são atividades onde o letramento tem uma função, são ocasiões em que os textos fazem parte da natureza das interações dos participantes e de seus processos interpretativos. \cite[p. 27, grifos da autora]{fischer_construcao_2007}
\end{quote}

Trata-se de um desafio grande criar oportunidades de letramentos, a partir de eventos amplamente contextualizados como sugere o conceito, sobretudo se não conhecemos os estudantes, pois, nesse caso, podemos incorrer no erro de criar oportunidades com gêneros textuais que não lhes são relevantes socialmente e que, consequentemente, não resultam em letramentos. Levar adiante o objetivo do letramento é um ato político na medida em que o acesso às letras dá voz a quem pouco se manifesta, tira da invisibilidade um grupo de pessoas que pouco participam de práticas letradas, que têm pouco acesso à informação e, portanto, têm limitações impostas pelo desconhecimento.

\section{A academia e suas novas esferas de interação}\label{sec-academia}
De acordo com \textcite{takaki_letramentos_2008}, é a partir da Segunda Guerra Mundial que a história dos letramentos adquire mais teor. Em um contexto de terra arrasada e de inovações tecnológicas, como a TV, a preocupação com a leitura e a escrita é diretamente associada a acesso e a crescimento econômico. Ainda segundo a autora, nos países subdesenvolvidos, o analfabetismo chegava a 80\%, o que gerou uma série de programas de alfabetização em países como os da América Latina. No Brasil, logo no início do uso do conceito, a visão era unicamente utilitária, no sentido de o conceito limitar-se a passar informações facilmente adaptáveis a diferentes situações, tendo em vista, sobretudo, o \textit{modus operandi} neoliberal que passa a comandar o mundo. Nas palavras de \textcite{street_letramentos_2014}, esse seria o letramento autônomo.

Contudo, letramento é um fenômeno social e, como tal, não deve ser hierarquizado com atitudes como a seleção de gêneros mais e menos relevantes a partir de uma visão urbana e disciplinar. Por se tratar de um curso interdisciplinar, com profissionais de várias áreas, \textcite[p. 67]{magnani_da_2018} pontuam que:

\begin{quote}
    Dentro dessa heterogeneidade, não necessariamente os pressupostos que são trazidos das diversas áreas possuem uma conceituação clara a respeito das práticas de leitura e escrita (...). No entanto, as demandas acontecem de formas diferentes e as atividades de leitura e escrita atendem a pressupostos interacionistas.
\end{quote}

Ainda como pontuam os autores:

\begin{quote}
    (\ldots) é possível encontrar nos discursos públicos sobre formação de professores para o Campo -- ainda que nem sempre de modo explícito ou fundamentado teoricamente com estudos do letramento -- vertentes, práticas e sujeitos alinhados com concepções dominantes de letramento, tal como a ideia do “letramento autônomo” \cite{street_letramentos_2014} e sua crença subsequente em uma relação direta entre aperfeiçoamento pessoal ou cognitivo e domínio de certa escrita. E, nesse sentido, deve-se atentar para que a particularidade da academia e do discurso teórico enquanto prática letrada não se oponha à proposta de resgatar, respeitar e articular saberes locais com os conhecimentos legitimados em nível global. \cite[p. 68]{magnani_da_2018}
\end{quote}

Os esforços das práticas aqui trazidas para esse momento de reflexão vão no sentido do diálogo entre saberes, como alertam os autores. Em sentido contrário, uma visão dominante nas sociedades ocidentais prioriza certos padrões que não dão conta da variedade de letramentos relevantes para o estudante. Para \textcite[p. 68]{magnani_da_2018}:

\begin{quote}
    Assim, vê-se que a “monocultura do saber acadêmico” \cite{santos_para_2005} – mantida e propagada pelo pensamento científico dominante – é também viabilizada por essa espécie de “monocultura do letramento acadêmico”. Esse processo, ao legitimar um feixe muito restrito de gêneros discursivos como práticas válidas de escrita ou de expressão pública, pode entrar em conflito com um dos pilares da Educação do Campo, que é um entendimento mais aprofundado do papel de saberes já existentes nos espaços comunitários rurais.
\end{quote}

Na busca da quebra dessa “monocultura do saber acadêmico”, as práticas que aqui apresento propõem, antes de tudo, uma reflexão sobre os letramentos diversos do graduando, sobre o seu contato com a leitura e a escrita, sobre as tecnologias que medeiam suas interações e sobre o ensino nas aulas de português. Em consonância com as reflexões iniciais, conduzo trabalhos nos quais os estudantes são orientados a construírem artefatos culturais diretamente ligados aos letramentos demandados em seus grupos sociais, como livros, artigos acadêmicos, sequências didáticas, apresentações etc.. Muitos deles apresentam novas reflexões sobre seus processos de aprendizagem e sobre suas práticas de letramento a exemplos de artigos e comunicações sobre estágios, práticas de ensino, pesquisas nas comunidades e escolas que atuam etc.. Considerando-se que letramentos são sociais e, portanto, contextuais, uma análise dos letramentos de quaisquer grupos sociais depende de se analisar questões como que práticas letradas são relevantes para esses grupos e como criar eventos de letramentos em cadeias dialógicas reais, para que sejam realmente significativas. Se não podem ser reais, que sejam ao menos simuladas. O contexto da academia demanda determinados letramentos que ora estão diretamente ligados a ela, como um artigo científico, ora são importantes em outras esferas sociais, como o uso de teclado e mouse. \textcite{lea_academic_2006} nos apresentam uma abordagem de letramento acadêmico composta por três modelos de referência para o enfoque da escrita em contextos de instrução, mas não necessariamente produtivo em termos de produção de sentidos, no sentido crítico. \textcite{silva_letramento_2013} sintetizaram os três modelos de referência \cite{lea_academic_2006} que não são excludentes, porém complementares no quadro expositivo que nos auxilia melhor na compreensão da abordagem adotada a seguir.

\begin{table}[htpb]
\caption{Quadro expositivo de \textcite{lea_academic_2006} por \textcite[p. 41]{silva_letramento_2013}.}
\label{tbl01}
\centering
\begin{tabular}{lp{0.7\textwidth}}
\toprule
\textbf{Tipologia} & \textbf{Síntese} \\ 
\midrule
\begin{tabular}[c]{@{}l@{}}
Modelo de Habilidades \\ Estudadas:
\end{tabular} & 
a escrita e o letramento originalmente como uma  habilidade individual e cognitiva. Esta abordagem focaliza aspectos superficiais da forma linguística, por exemplo, aspectos ortográficos e gramaticais. Presume que os estudantes podem transferir sem problema seus conhecimentos da escrita e do letramento de um  contexto para outro. \\ 
\begin{tabular}[c]{@{}l@{}}
Modelo de Socialização \\ Acadêmica:
\end{tabular} &
  Focaliza a aculturação dos estudantes em discursos e  gêneros característicos das disciplinas. Os estudantes adquirem maneiras de falar, escrever, pensar e usar o  letramento, características dos membros atuantes em uma disciplina ou integrantes de uma comunidade. Presume que os discursos e os gêneros característicos das disciplinas são relativamente estáveis; logo, uma vez  aprendidas as regras de funcionamento do discurso  acadêmico, os estudantes são capazes de reproduzi-las sem problema. \\
\begin{tabular}[c]{@{}l@{}}
Modelo de Letramento \\ Acadêmico:
\end{tabular} &
  Focaliza a construção de significado, identidade,  poder e autoridade. Coloca em primeiro plano a natureza institucional que conta como conhecimento em qualquer contexto acadêmico.\\ 
\bottomrule
\end{tabular}
\source{elaboração própria.}
\end{table}

Na medida em que o primeiro modelo, de habilidades estudadas, entende a escrita como habilidade exclusivamente cognitiva, a ser aprendida nos primeiros anos de contato com a escola e que desconsidera a relevância dos contextos, como a universidade, ele não nos interessa. O segundo modelo, de Socialização Acadêmica, por sua vez, tem um viés instrumental e não abrange os objetivos de formação de cidadãos críticos, pois não é voltado para uma leitura significativa do mundo, mesmo que os sujeitos atuem com eficiência nos seus contextos de interação. Já o Modelo de Letramento Acadêmico dialoga diretamente com o entendimento de língua que este trabalho assume, focado na importância dos processos de construção de sentidos para uma leitura mais ampla dos eventos entendidos em seus contextos e em diálogo com saberes mais amplos de mundo. Consequentemente, a esse modelo importa o contexto acadêmico.

Nas esferas de atividade da UFVJM, mais especificamente no âmbito da LEC e da atuação e participação dos professores em formação, há gêneros e tecnologias de relevância para os letramentos dos licenciandos que certamente não serão relevantes em outros contextos. Ensinar os gêneros de maneira instrumental, por exemplo, para um futuro professor, não é eficiente. Tal afirmação se dá por observar que as tecnologias se alteram rapidamente e a nossa interação também. Os gêneros textuais, em consequência das alterações dos contextos, e ecológicos que são, renovam-se \cite{bakhtin_speech_1986}.

Como a mudança é a norma, na sala de aula, o que se deve fazer, antes de tudo, é provocar reflexões sobre o quê, como e sob quais condições se aprende nas relações com a língua, com os gêneros textuais, com a escrita e todas as tecnologias de acesso ao mundo letrado. As reflexões devem ser registradas pensando-se sempre no compartilhamento e na continuidade do diálogo. Se os produtos dessas reflexões adentrarem novos ciclos dialógicos, um tanto melhor. A partir das reflexões, os estudantes devem, ainda, produzir gêneros que sejam relevantes em suas esferas de atuação, sempre se observando os processos envolvidos na estabilização e/ou mutação dos gêneros, nas relações de poder que lhes dão relevância, nas formas, nos temas e nas demais questões contextuais. No campo dos letramentos acadêmicos, há um grupo de gêneros que atualmente têm maior importância, como artigos científicos, resumos, resenhas, esquemas, textos informativos, sequências didáticas, planos de aula, \textit{e-mails}, apresentações digitais, apresentações orais como seminários e comunicações, dentre outros. Raramente chega-se à universidade conhecendo esses gêneros e suas tecnologias, sobretudo em contextos periféricos. A reflexão se dá a partir da pesquisa sobre as práticas discursivas do estudante e da academia, buscando-se o entendimento dos processos de letramentos que o sujeito já experimentou e em quais novas esferas de atuação deve interagir, com quais gêneros textuais, sob quais condições etc. Assim, o foco é tornar o discente autônomo na aprendizagem de novos gêneros.

Nesse sentido, reflexões e eventos de letramento são oportunizados. Um exemplo é a produção do livro \textit{Memórias de letramentos: vozes do campo}\footnote{Disponível em: \url{http://acervo.ufvjm.edu.br/jspui/bitstream/1/1586/1/memorias_letramento_vozes_campo.pdf}. Acesso em 08/02/2021.}, que foi feito a partir das memórias com a escrita e a leitura de 68 estudantes em duas disciplinas diferentes. A primeira delas denomina-se Gêneros Discursivos e Textuais e nela foram produzidos 23 textos no segundo semestre de 2016. A segunda é a disciplina Leitura e Produção de Textos, ofertada em janeiro de 2017, em parceria com o professor Luiz Henrique Magnani, com quem organizei o referido livro. As tecnobiografias, por sua vez, foram produzidas na quantidade de 20 na disciplina Linguagens e Códigos: Ensino e Novas Tecnologias no primeiro semestre de 2018, e outras 17 no segundo semestre de 2018 na disciplina Metodologia de Ensino de Língua Portuguesa. Não é objetivo deste texto descrever cada uma dessas disciplinas, mas, de maneira geral, apresentar os eventos de letramento que proporcionaram e suas relações com as novas tecnologias.

Para a confecção das memórias de letramentos, por exemplo, nas duas disciplinas citadas, os estudantes dedicaram-se às seguintes práticas: 
\begin{enumerate*}[label=(\arabic*)] 
\item a partir de provocações dos professores com questões sobre as trajetórias de letramentos dos estudantes, produção de um esquema de ideias sobre o que seria um texto memorialístico autobiográfico; 
\item escrita do texto em formato digital; 
\item duas revisões do material, uma a partir das observações de um colega e outra partir das observações do professor; 
\item reescrita do texto; 
\item oficina de produção de livros artesanais. 
\end{enumerate*} 
Adicionalmente, os estudantes da disciplina Gêneros textuais e discursivos tiveram quatro eventos com práticas extras a fim de produzirem reflexões teoricamente embasadas: 
\begin{enumerate*}[label=(\arabic*)] 
\item estudos teóricos sobre gêneros textuais e letramentos; 
\item escrita de artigo acadêmico em meio digital com a temática gêneros textuais e letramentos; 
\item revisão e edição final do material; 
\item participação em evento acadêmico \textit{on-line} com apresentação do artigo produzido. 
\end{enumerate*} 
A apresentação compreendeu o diálogo em espaço digital com os leitores do texto por meio de comentários. Em análise específica sobre tais eventos de letramento, \textcite[p. 37-38]{castro_reflexao_2017} pontua que:

\begin{quote}
    O trabalho buscou promover uma reflexão teoricamente embasada sobre as práticas de letramentos dos próprios estudantes ao longo de suas vidas, para que entendessem os processos sociais de construção de sentidos pelos quais passaram, bem como suas relações com os diferentes gêneros textuais, diferentemente organizados de acordo com as especificidades contextuais.
\end{quote}

Trata-se, ainda, de um processo de autoria que traz os sujeitos e suas vozes e, portanto, é um exercício de empoderamento. É, ainda, um exercício para a autonomia e a horizontalidade de relações, visto que a reflexão leva o estudante a aprender a aprender e a criticar. Somente nesse breve exemplo de atividades, podemos citar inúmeras habilidades ligadas às esferas de atividade da academia que compõem os eventos de letramentos aqui descritos. Nesse processo de edição, os estudantes tiveram contato com gêneros acadêmicos, como resumos, esquemas, artigos científicos e o debate público em congresso. Precisaram usar ferramentas como os computadores com seus teclados, mouses, telas, câmeras ou os celulares; trabalhar com editores de texto e suas funções de formatação e revisão; usar editores de imagens e buscadores \textit{on-line}; ler ícones, leiautes, imagens diversas e palavras para clicar e acessar os locais exatos para interação em ambiente \textit{on-line}; dentre outras habilidades. Além disso, como veremos na análise de narrativas, muitas vezes não tinham experiência nem com um teclado ou mouse.

No caso da confecção das tecnobiografias utilizadas nas análises que aqui apresentarei, o trabalho foi reduzido, uma vez que não houve confecção de livros e de artigos com reflexões a respeito. Nesse caso, os eventos de letramentos oportunizados nas disciplinas que serviram de base para atividade foram outros, como a confecção de objetos de aprendizagem e de recursos educacionais abertos.

No âmbito da oralidade, os estudantes têm a oportunidade de participar dos seminários de estágio (que são eventos de extensão abertos às comunidades e contam com professores e convidados externos); da colocação comum das práticas de ensino no início de todos os semestres; da Noite Cultural que também é semestral; e de outras atividades diversas que variam de acordo com o semestre e disciplinas ofertadas.

Perguntados sobre letramentos e suas tecnologias\footnote{Questionário de pesquisa aplicado em 29 de julho de 2016 com divulgação consentida sob condição de anonimato.}, 10 de 17 estudantes da disciplina Gêneros Discursivos e Textuais disseram que adquiriram computador de mesa ou pessoal quando ingressaram na universidade. Isso significa que esses discentes apresentavam, à época, baixos níveis de letramentos digitais. É importante ainda observar que, no contexto acadêmico, há algum tempo que as novas tecnologias substituíram tecnologias ainda comuns na educação básica, como copiar e responder atividades em cadernos. Nesse sentido, a expressão \textit{novas tecnologias} para este texto não se restringe às tecnologias digitais, como os \textit{smartphones}, a internet e os computadores de última geração, mas àquelas que são novidades para os sujeitos envolvidos na pesquisa e que passam a fazer parte de suas realidades nessas novas esferas de atividade interativa. No que se refere aos gêneros comuns ao contexto acadêmico, na mesma pesquisa, apenas 2 estudantes dentre os 17 afirmaram que já liam artigos científicos antes de se tornarem universitários. Sobre outros trabalhos acadêmicos, como monografias, dissertações e teses, todos começaram a ter contato em decorrência da universidade. Para dar voz a esses futuros professores do campo, a próxima seção explica a metodologia de análise narrativa adotada para, então, a quarta seção apresentar e analisar os dados produzidos nas tecnobiografias.

\section{Biografias para análises narrativas}\label{sec-biografias}
Historicamente, a pesquisa narrativa apareceu na psicologia e na sociologia há cerca de um século e, atualmente, ela chama a atenção das ciências sociais, como afirmam \textcite{barkhuizen_narrative_2014}. Os autores esclarecem que há dois modelos de pensamento, cada um traz a sequência de eventos e experiências à sua maneira: o paradigmático traz as experiências em forma de argumentos e como provas de certas teses; o narrativo convence pela verossimilhança que a sequência de eventos e experiências pode apresentar. Nas palavras dos autores:

\begin{quote}
    Os argumentos convencem de sua “verdade” apelando aos procedimentos para estabelecer provas formais e empíricas; as histórias convencem de sua “vitalidade” apelando mais aos critérios de verossimilhança.\footnote{Em tradução livre realizada por este autor, assim como as demais citações de referências em inglês ao longo do texto.} \cite[p. 1]{barkhuizen_narrative_2014}
\end{quote}

Relatam ainda que a pesquisa narrativa paradigmática é recente e aparece em um cenário fértil ao pensamento racional, mas já recebe críticas por descontextualizar as experiências com os formatos engessados de escrita argumentativa. No entanto, os modelos podem ser complementares, bem como podem dialogar com outras metodologias, a exemplo desta proposta de trabalho que traz observações empíricas, pesquisa narrativa e breve entrevista. Sobre sua importância,

\begin{quote}
    [e]m resumo, a principal força da investigação narrativa está em seu foco em como as pessoas usam histórias para dar sentido a suas experiências em áreas de investigação, onde é importante entender os fenômenos a partir das perspectivas daqueles que os experimentam. \cite[p. 2]{barkhuizen_narrative_2014}
\end{quote}

A perspectiva dialoga diretamente com a concepção de língua e ensino a que me proponho trabalhar, o sociointeracionismo situado, e com o modelo de letramento social e ideológico \cite{street_letramentos_2014}, que é oposto à ideia de “monocultura do saber acadêmico” \cite{santos_para_2005}. Adicionalmente, projetos de escrita e reflexão de narrativas dão voz, sobretudo, no caso da Educação do Campo, àqueles campesinos que pouco têm seus contextos representados e respeitados no espaço acadêmico e escolar. Dentre as diversas possibilidades de pesquisas narrativas, o quadro abaixo apresenta as diferenças enumeradas por \textcite[p. 10]{barkhuizen_narrative_2014}.

\begin{table}[htpb]
\caption{Tipos de pesquisa narrativa na pesquisa e ensino de línguas.}
\label{tbl02}
\begin{tabular}{lllll}
\toprule
 & \textit{Análise narrativa} & \textit{Análise de narrativa} & \textit{Autobiografia} & \textit{Biografia} \\ 
\midrule
Memórias de linguagem                                                       & Sim & Não & Sim & Não \\ 
\begin{tabular}[c]{@{}l@{}}Estudos de memórias de \\ linguagem\end{tabular} & Não & Sim & Não & Sim \\ 
\begin{tabular}[c]{@{}l@{}}Estudos de caso \\ autobiográficos\end{tabular}  & Sim & Sim & Sim & Não \\ 
Estudos de caso biográficos                                                 & Sim & Sim & Não & Sim \\ 
\begin{tabular}[c]{@{}l@{}}Estudos de múltiplas \\ narrativas\end{tabular}  & Não & Sim & Não & Sim \\ 
\bottomrule
\end{tabular}
\source{elaboração própria.}
\end{table}

A diferença entre análise narrativa e análise de narrativa é que a primeira não analisa, necessariamente, narrativas, mas vem em forma de narrativas, a exemplo de qualquer relato. O único tipo de pesquisa narrativa do quadro que não apresenta análises é as memórias de linguagem escritas, normalmente, para fins pessoais e sem nenhuma conexão com o pesquisador ou a academia. Trabalho aqui com um estudo de múltiplas narrativas em relação às tecnologias que se enquadram no modelo Histórias de Aprendizagem de Línguas, apesar de os contextos de usos das referências citadas sejam o ensino de segunda língua ou língua adicional, e não na formação de professores de língua materna. Para \textcite[p. 37]{barkhuizen_narrative_2014},

\begin{quote}
    (...) \textit{histórias de aprendizagem de línguas} (em inglês: LLHs) são relatos retrospectivos da aprendizagem passada. São as histórias escritas de experiências de aprendizagem de línguas. \textcite[p. 548]{benson_language_2011} aponta que “a palavra história sugere uma conta de longo prazo, embora os períodos de tempo cobertos por LLHs possam variar muito, variando de todo o período em que uma pessoa aprendeu uma língua para períodos muito mais curtos, como um ano ou semestre de estudo ou um episódio que não dura mais do que alguns minutos”.
\end{quote}

As instruções para a escrita contaram com uma série de questões que sugeriram uma temporalidade no contato com ambientes, pessoas e tecnologias de leitura e escrita, tal como sugere \apud[p. 86]{murphey_learners_2004}[p. 38]{barkhuizen_narrative_2014}:

\begin{quote}
    As intenções deste conjunto de questões são claras. Eles evocam dados narrativos que são temporais (refletindo sobre o passado e olhando para o futuro), emotivos (experiências positivas e negativas, surpresas), reflexivos (crenças, expectativas e práticas), estratégicos (planos e objetivos) e instrutivos (conselhos).
\end{quote}

As tecnobiografias podem trazer informações de relevância para se entender a relação dos futuros professores não apenas com os letramentos acadêmicos, mas também a relação dos letramentos com as tecnologias, tendo em vista que tecnobiografia se refere a uma narrativa biográfica sobre as relações do sujeito com as tecnologias. É o que pontua \textcite{kennedy_technobiography:_2003} ao afirmar que o termo foi usado pela primeira vez como uma coletânea de histórias de mulheres e suas relações cotidianas com as tecnologias em estudo em coautoria intitulado \textit{Cyborg lives? Women’s Technobiographies}. Para \textcite[p. 120]{kennedy_technobiography:_2003}, “(...) histórias vividas, ou tecnobiografias, são significantes por causa das diferenças sutis e nuances em cada experiência individual com a tecnologia.” Contudo, neste artigo, elas não serão usadas em sua potencialidade de se contar histórias individuais com relação às tecnologias. Outrossim, servirão para a construção de dados acerca de uma temática específica ligada aos objetivos deste trabalho. O modelo de análise de dados refere-se a uma análise temática em um estudo de múltiplas histórias de aprendizagens de línguas. Para \textcite[p. 77]{barkhuizen_narrative_2014},

\begin{quote}
    [\ldots] a análise temática é provavelmente mais adequada para estudos de casos múltiplos, porque abre a possibilidade de comparar as narrativas em um conjunto de dados, de estabelecer temas compartilhados, bem como destacar as diferenças individuais. Polkinghorne (1995: 15), por exemplo, argumenta que “a força dos procedimentos paradigmáticos é sua capacidade de desenvolver conhecimentos gerais sobre uma coleção de histórias”.
\end{quote}

Nesse sentido, a metodologia adotada contou com momentos de leitura de todas as narrativas, seleção e categorização de todos os trechos que apontavam ligação direta entre determinados eventos de letramentos, universidade e novas tecnologias. As limitações de pesquisas desse tipo, múltiplas narrativas, são, como apontado por \textcite{barkhuizen_narrative_2014}, de despersonalização dos sujeitos, tendo em vista que buscam características gerais. O ganho, como também apontam \textcite{barkhuizen_narrative_2014}, é termos uma visão panorâmica do grupo.

O banco de dados é composto por 37 tecnobiografias, escritas por 5 estudantes do sexo masculino e 32 do sexo feminino, com idades que variavam quando da escrita de 19 a 34 anos. Nelas foram encontrados 32 excertos sobre a temática em estudo. No que se refere às questões éticas, todos os estudantes consentiram na pesquisa que aqui apresento. No momento da escrita, foram encorajados a escrever sobre suas trajetórias de vida com as tecnologias a partir dos pressupostos já citados por \apud{murphey_learners_2004}{barkhuizen_narrative_2014} com as seguintes questões:

\begin{itemize}
\item \textbf{Como tudo começou}: Onde foi seu primeiro contato com tecnologia digital? Como foi esse contato? O que você já fez com tecnologia e o que não faz mais? Você se lembra de quando usou pela primeira vez, um mouse, enviou uma mensagem, fez uma busca na Wikipédia, entrou em uma rede social? Que pessoa(s) foi/foram importantes no seu processo de aprendizagem?
\item \textbf{Práticas atuais}: Quais são as páginas web/blogs que você mais visita? Você contribui com algum deles? Há diferenças no uso diário de tecnologia em sua vida de estudante, profissional, ativismo político, atividade religiosa, esportes etc.? Você já vivenciou alguma proibição em relação ao uso de alguma tecnologia?
\item \textbf{Participação}: Você participa de redes sociais? Você posta comentários em notícias ou anúncios de produtos? Você participa de votações na web? Se sim, que tipo(s) de votação? Você já compartilhou imagens e vídeos para receber comentários? Se sim, onde?
\item \textbf{Um dia em sua vida}: Pense no dia de ontem, qual ou quais tecnologia(s) você usou logo depois de acordar? Que tecnologia(s) você usou ao longo do dia?
\item  \textbf{Transições}: Que práticas sociais você mudou em função da tecnologia? Ex. Catalogar endereços de pessoas, marcar encontros, usar mapas etc. O que você ainda não fez, mas pretende fazer?
\item \textbf{Comparações}: que diferenças no uso de tecnologia você percebe em relação às gerações mais velhas e mais novas? Você consegue identificar diferenças entre culturas, amigos estrangeiros e entre gêneros?
\item  \textbf{Avaliação}: Quais são os seus sentimentos em relação às tecnologias? Quais foram as experiências mais positivas e mais negativas? O que você usa ou usaria como professor?
\end{itemize}

%\begin{description}
%\item[Como tudo começou]: Onde foi seu primeiro contato com tecnologia digital? Como foi esse contato? O que você já fez com tecnologia e o que não faz mais? Você se lembra de quando usou pela primeira vez, um mouse, enviou uma mensagem, fez uma busca na Wikipédia, entrou em uma rede social? Que pessoa(s) foi/foram importantes no seu processo de aprendizagem?
%\item[Práticas atuais]: Quais são as páginas web/blogs que você mais visita? Você contribui com algum deles? Há diferenças no uso diário de tecnologia em sua vida de estudante, profissional, ativismo político, atividade religiosa, esportes etc.? Você já vivenciou alguma proibição em relação ao uso de alguma tecnologia?
%\item[Participação]: Você participa de redes sociais? Você posta comentários em notícias ou anúncios de produtos? Você participa de votações na web? Se sim, que tipo(s) de votação? Você já compartilhou imagens e vídeos para receber comentários? Se sim, onde?
%\item[Um dia em sua vida]: Pense no dia de ontem, qual ou quais tecnologia(s) você usou logo depois de acordar? Que tecnologia(s) você usou ao longo do dia?
%\item[Transições]: Que práticas sociais você mudou em função da tecnologia? Ex. Catalogar endereços de pessoas, marcar encontros, usar mapas etc. O que você ainda não fez, mas pretende fazer?
%\item[Comparações]: que diferenças no uso de tecnologia você percebe em relação às gerações mais velhas e mais novas? Você consegue identificar diferenças entre culturas, amigos estrangeiros e entre gêneros?
%\item[Avaliação]: Quais são os seus sentimentos em relação às tecnologias? Quais foram as experiências mais positivas e mais negativas? O que você usa ou usaria como professor?
%\end{description}

O recorte temático em análise, como já explicitado, compreende o contexto universitário e, portanto, não usa de todas as experiências relatadas. O resultado é apresentado na próxima seção.

\section{Análise de narrativas}\label{sec-analise}
Com a entrada na universidade, as esferas de atuação dialógica dos estudantes se ampliaram. Tendo em vista referências urbanas, certamente as alterações notadas vão além das expectativas. A universidade, atrelada a um conjunto de programas que a sustentam, proporcionou, inclusive, a inserção de novas tecnologias na vida de muitos desses sujeitos. É o que se constata no excerto de narrativa \ref{exc01}, de uma  de assentamento que, no Brasil, refere-se aos povoamentos rurais constituídos a partir de lutas sociais como as empreendidas pelo Movimento dos Trabalhadores Rurais Sem Terra (MST)\footnote{\url{https://mst.org.br/quem-somos}}. Em outros casos, não há uma ligação direta citada pelos estudantes com benefícios voltados para a permanência na universidade, como em \ref{exc02}, \ref{exc03} e \ref{exc04}, mas, mesmo assim, a academia é o divisor de águas.

% \begin{table}[htpb]
% \centering
% \begin{tabular}{lp{0.9\textwidth}}
% \toprule
% (1) &
%   Atualmente, como estudante da Licenciatura em Educação do Campo, acessei o programa de auxílio da universidade chamado Bolsa Permanência e isso possibilitou a compra do meu primeiro \textit{notebook}.\\ 
% (2) &
%   Em questão a utilidades de uso pessoal, tenho um \textit{notebook} que adquiri em 2016  com a necessidade de realizar atividades da faculdade. \\ 
% (3) &
%   Esse foi o período que ingressei pra faculdade, então precisei comprar um novo telefone mais avançado e com mais funções, pois eu precisei para auxiliar na  questão de troca de \textit{e-mails}, postagem de trabalhos etc. \\ 
% (4) &
%   (\ldots) utilizava o computador do meu namorado e com a prática fui aprendendo rapidamente a formatar os meus trabalhos. \\ 
% \bottomrule
% \end{tabular}
% \end{table}

\begin{enumerate}[label={(\arabic*)},ref={\arabic*},topsep=1ex,partopsep=1ex]
    \item\label{exc01} Atualmente, como estudante da Licenciatura em Educação do Campo, acessei o programa de auxílio da universidade chamado Bolsa Permanência e isso possibilitou a compra do meu primeiro \textit{notebook}.
    \item\label{exc02} Em questão a utilidades de uso pessoal, tenho um \textit{notebook} que adquiri em 2016  com a necessidade de realizar atividades da faculdade.
    \item\label{exc03} Esse foi o período que ingressei pra faculdade, então precisei comprar um novo telefone mais avançado e com mais funções, pois eu precisei para auxiliar na  questão de troca de \textit{e-mails}, postagem de trabalhos etc.
    \item\label{exc04} (\ldots) utilizava o computador do meu namorado e com a prática fui aprendendo rapidamente a formatar os meus trabalhos.
\end{enumerate}


Como vemos no excerto \ref{exc01}, a questão financeira parecia ser o primeiro grande obstáculo para se acessar possibilidades interativas digitais, até a entrada na universidade; o que não aparece claramente em \ref{exc02} e em \ref{exc03} que, assim como \ref{exc01}, também não tinham um computador pessoal até entrarem na universidade. No caso de \ref{exc03}, não havia acesso a e-mails nem por celular. Em \ref{exc04}, o que podemos ter certeza é de que a discente não tinha e não comprou um computador, mas usará um que tem acesso esporádico, o que pode limitar suas práticas letradas em ambientes digitais.

Nesse mesmo sentido, alguns discentes não falam sobre compra de computadores, no entanto de acesso à internet de forma geral, como em \ref{exc05} e \ref{exc06}. Em \ref{exc05}, podemos observar que há um esforço coletivo da comunidade na aquisição de uma antena rural para a internet. Em \ref{exc06}, não há uma relação clara da compra da nova antena como consequência da entrada na universidade, diferentemente do contato com gêneros digitais.

% \begin{table}[htpb]
% \begin{tabular}{ll}
% \hline
% (5) &
%   \begin{tabular}[c]{@{}l@{}}Em 2015 entrei pra universidade. Juntamente a alguns outros jovens da \\ comunidade, e devido a necessidade do uso constante da internet para os estudos, \\ cada um comprou um roteador que ligado a uma antena rural funcionaram muito \\ bem e assim chegou a internet na comunidade.\end{tabular} \\ \hline
% (6) &
%   \begin{tabular}[c]{@{}l@{}}Eu estava adentrando na Universidade e em minha comunidade foi instalada uma \\ antena de telecomunicação. Foi aí que vários moradores adquiriram aparelhos \\ celulares tipo \textit{smartphones} (...) eu começava a utilizar novos meios de \\ comunicação e outras plataformas da Universidade.\end{tabular} \\ \hline
% \end{tabular}
% \end{table}

\begin{enumerate}[resume,label={(\arabic*)},ref={\arabic*},topsep=1ex,partopsep=1ex]
    \item\label{exc05} Em 2015 entrei pra universidade. Juntamente a alguns outros jovens da comunidade, e devido a necessidade do uso constante da internet para os estudos, cada um comprou um roteador que ligado a uma antena rural funcionaram muito bem e assim chegou a internet na comunidade.
    \item\label{exc06} Eu estava adentrando na Universidade e em minha comunidade foi instalada uma antena de telecomunicação. Foi aí que vários moradores adquiriram aparelhos celulares tipo \textit{smartphones} (\ldots) eu começava a utilizar novos meios de comunicação e outras plataformas da Universidade.
\end{enumerate}

Se para muitos universitários o contato com gêneros digitais, como o \textit{e-mail} ou sites de busca, é algo corrente desde o ensino fundamental, diante do cenário que se apresenta, em que mais de 50\% dos sujeitos não tinham computador antes de entrar na faculdade como explicitado na segunda seção deste trabalho, o \textit{e-mail} atrela-se à experiência universitária como se vê nos excertos \ref{exc07}, \ref{exc08}, \ref{exc09} e \ref{exc10}, a seguir.

% \begin{table}[htpb]
% \begin{tabular}{ll}
% \hline
% (7) &
%   \begin{tabular}[c]{@{}l@{}}O primeiro \textit{e-mail} que eu mesma abri já foi na faculdade, onde aprendi também a \\ fazer pesquisas.\end{tabular} \\ \hline
% (8) &
%   \begin{tabular}[c]{@{}l@{}}(...) me inscrevi para o vestibular do curso Licenciatura em Educação do campo e \\ a diretora fez todo processo da inscrição, porém utilizou (...) o \textit{e-mail} da minha irmã. \\ Passei no vestibular e (...) o professor que ministrava a disciplina \\ Metodologia de Ensino Científico ressaltou sobre o que se tratava um \textit{e-mail} e \\ que, como universitários, o usaríamos com frequência.\end{tabular} \\ \hline
% (9) &
%   \begin{tabular}[c]{@{}l@{}}Para minha alegria, a internet em Capivari chegou junto com minha aprovação no \\ vestibular. Quanta alegria! Pude olhar o resultado (do vestibular) na minha casa, \\ criei \textit{e-mail}, fiz cursos on-line e comprei uma impressora (...)\end{tabular} \\
% (10) &
%   \begin{tabular}[c]{@{}l@{}}Só abri uma conta no \textit{Gmail} e passei a manusear um computador quando passei no \\ vestibular do curso de Licenciatura em Educação do Campo. Foi a partir do curso \\ que passei a fazer mais parte desse mundo virtual, não só para comunicação, mas \\ para estudar.\end{tabular}
% \end{tabular}
% \end{table}

\begin{enumerate}[resume,label={(\arabic*)},ref={\arabic*},topsep=1ex,partopsep=1ex]
\item\label{exc07} O primeiro \textit{e-mail} que eu mesma abri já foi na faculdade, onde aprendi também a fazer pesquisas.
\item\label{exc08} (\ldots) me inscrevi para o vestibular do curso Licenciatura em Educação do campo e a diretora fez todo processo da inscrição, porém utilizou (\ldots) o \textit{e-mail} da minha irmã. Passei no vestibular e (\ldots) o professor que ministrava a disciplina Metodologia de Ensino Científico ressaltou sobre o que se tratava um \textit{e-mail} e que, como universitários, o usaríamos com frequência.
\item\label{exc09} Para minha alegria, a internet em Capivari chegou junto com minha aprovação no vestibular. Quanta alegria! Pude olhar o resultado (do vestibular) na minha casa,  criei \textit{e-mail}, fiz cursos on-line e comprei uma impressora (\ldots)
\item\label{exc10} Só abri uma conta no \textit{Gmail} e passei a manusear um computador quando passei no vestibular do curso de Licenciatura em Educação do Campo. Foi a partir do curso que passei a fazer mais parte desse mundo virtual, não só para comunicação, mas para estudar.
\end{enumerate}


Complementarmente, em \ref{exc08}, a estudante detalha uma intervenção de um docente em prol do uso de novas tecnologias e do gênero \textit{e-mail}, em \ref{exc09} há ainda uma referência a cursos \textit{on-line} e em \ref{exc10} há nova citação aos serviços de \textit{e-mail}, o que ilustra um tanto mais os caminhos dos letramentos desses estudantes. Há de se lembrar que o curso funciona em regime de alternância, em dois tempos, na Universidade (TU), e nas comunidades (TC). No TC, o contato com os professores orientadores é via internet, por comunicadores instantâneos, \textit{e-mails} e o ambiente virtual Moodle; e presencialmente em dois momentos por semestre, de 2 dias e meio cada, nos quais estudantes e professor se dedicam a atividades de práticas de ensino. Diante do quadro, percebemos que as novas tecnologias são essenciais para que tudo aconteça a contento, com uma interação produtiva e resultados em aprendizagem. Contando que há muitos discentes com poucas experiências com as tecnologias digitais, a colaboração torna-se uma necessidade e, não sem esforço, uma realidade, como se vê em \ref{exc11} e \ref{exc12}.

% \begin{table}[htpb]
% \begin{tabular}{ll}
% \hline
% (11) &
%   \begin{tabular}[c]{@{}l@{}}Além das pessoas na faculdade, minhas amigas mais letradas nas tecnologias foram \\ essenciais para o meu aprendizado com as novas tecnologias.\end{tabular} \\ \hline
% (12) &
%   \begin{tabular}[c]{@{}l@{}}Com isso (ajuda), aprendi a baixar fotos, documentos, a enviar e a receber \textit{e-mails}, \\ postar trabalhos no Drive do \textit{Google} e a me manter informada sobre minha vida \\ acadêmica no sistema \textit{Ecampus}.\end{tabular} \\ \hline
% \end{tabular}
% \end{table}

\begin{enumerate}[resume,label={(\arabic*)},ref={\arabic*},topsep=1ex,partopsep=1ex]
\item\label{exc11} Além das pessoas na faculdade, minhas amigas mais letradas nas tecnologias foram essenciais para o meu aprendizado com as novas tecnologias.
\item\label{exc12} Com isso (ajuda), aprendi a baixar fotos, documentos, a enviar e a receber \textit{e-mails},  postar trabalhos no Drive do \textit{Google} e a me manter informada sobre minha vida acadêmica no sistema \textit{Ecampus}.
\end{enumerate}

As habilidades requeridas e os usos começam a aparecer como o “baixar fotos, documentos” e as ferramentas também como o Drive\footnote{\url{http://drive.google.com/}} e o sistema de notas Ecampus\footnote{\url{https://ecampus.ufvjm.edu.br/}} citados em \ref{exc12}. Em novos excertos, são relatadas também experiências interativas em novos ambientes, além de pesquisa e \textit{e-mails}, ligados à universidade como os “sites educacionais” citados e os espaços para “publicar trabalhos realizados na faculdade” citadas em \ref{exc13}; além das atividades letradas ligadas à profissão como os concursos citados em \ref{exc14} e \ref{exc15} ou a tarefa de atualizar o currículo \textit{Lattes}\footnote{\url{http://lattes.cnpq.br/}}, ligada à profissão e à academia.

% \begin{table}[htpb]
% \begin{tabular}{ll}
% \hline
% (13) &
%   \begin{tabular}[c]{@{}l@{}}Atualmente, navego com frequência por sites educacionais, pesquiso editais de \\ concursos, faço pesquisas acadêmicas (...). A partir do ano de 2015 que me tornei \\ mais internauta, pois surgiu a necessidade de estar conectada com o mundo virtual \\ para auxiliar nos trabalhos de faculdade. (...) também já usei (a internet) como para \\ publicar trabalhos realizados na faculdade.\end{tabular} \\ \hline
% (14) &
%   \begin{tabular}[c]{@{}l@{}}O computador eu uso bastante para fazer meus trabalhos de faculdade e pesquisas. \\ Também o utilizo para (...) para olhar minhas classificações nos \\ concursos, olhar nota do Enem para amigos, atualizar o \textit{currículo lattes}.\end{tabular} \\ \hline
% (15) &
%   \begin{tabular}[c]{@{}l@{}}O computador eu uso bastante para fazer meus trabalhos de faculdade e pesquisas. \\ Também o utilizo para acessar o \textit{Facebook}, para olhar minhas classificações nos \\ concursos, olhar nota do Enem para amigos, atualizar o \textit{currículo lattes}.\end{tabular}
% \end{tabular}
% \end{table}

\begin{enumerate}[resume,label={(\arabic*)},ref={\arabic*},topsep=1ex,partopsep=1ex]
\item\label{exc13} Atualmente, navego com frequência por sites educacionais, pesquiso editais de concursos, faço pesquisas acadêmicas (\ldots). A partir do ano de 2015 que me tornei mais internauta, pois surgiu a necessidade de estar conectada com o mundo virtual para auxiliar nos trabalhos de faculdade. (\ldots) também já usei (a internet) como para publicar trabalhos realizados na faculdade.
\item\label{exc14} O computador eu uso bastante para fazer meus trabalhos de faculdade e pesquisas. Também o utilizo para (\ldots) para olhar minhas classificações nos concursos, olhar nota do Enem para amigos, atualizar o \textit{currículo lattes}.
\item\label{exc15} O computador eu uso bastante para fazer meus trabalhos de faculdade e pesquisas. Também o utilizo para acessar o \textit{Facebook}, para olhar minhas classificações nos concursos, olhar nota do Enem para amigos, atualizar o \textit{currículo lattes}.
\end{enumerate}

As pesquisas parecem ter sido lembradas por todos, mas, além do \textit{Google}\footnote{\url{http://google.com/}} e do YouTube\footnote{\url{https://www.youtube.com/}}, o único site específico para pesquisa acadêmica que aparece é o \textit{Google} Acadêmico\footnote{\url{https://scholar.google.com.br}}, citado em três situações, como pode ser visto em \ref{exc16}, \ref{exc17} e \ref{exc18}. A \textit{Wikipédia}\footnote{\url{https://pt.wikipedia.org}} também teve uma citação, como exposto em \ref{exc19}. Destaco o trecho de \ref{exc17} que relata as melhorias que a estudante percebeu em suas habilidades de pesquisa.

% \begin{table}[htpb]
% \begin{tabular}{ll}
% \hline
% (16) &
%   \begin{tabular}[c]{@{}l@{}}(...) com mais prática, explorando e descobrindo mais o uso de ferramentas e outros, \\ a exemplo disso os e-mails do Hotmail e Gmail que uso constantemente para edições \\ e produções de documentos e pesquisas O \textit{Moodle, Google, Yahoo, YouTube} são as \\ páginas de concursos, \textit{Facebook} e o \textit{Google+} são as páginas que mais visito (...)\end{tabular} \\ \hline
% (17) &
%   \begin{tabular}[c]{@{}l@{}}Depois que eu entrei pra universidade passei a visitar \textit{sites} que eu não visitava, como \\ \textit{Google Acadêmico} que apesar de ser um pouco complexo, são muito bons os artigos. \\ Aprendi a pesquisar mais as coisas que eu leio, questionar se realmente são confiáveis.\end{tabular} \\ \hline
% (18) &
%   \begin{tabular}[c]{@{}l@{}}Tenho também um canal pessoal, mas não uso muito, faço uso no \textit{WhatsApp, Gmail}, \\ \hline \textit{Google+, Hangouts, Maps, Google, Google Acadêmico, Instagram} etc. todos \\ contribuem na minha vida pessoal e profissional.\end{tabular} \\ \hline
% (19) &
%   \begin{tabular}[c]{@{}l@{}}Como estudante, utilizo o \textit{Google}, a \textit{Wikipédia} para pesquisas acadêmicas, YouTube \\ para assistir videoaulas e para outros estudos.\end{tabular} \\ \hline
% \end{tabular}
% \end{table}

\begin{enumerate}[resume,label={(\arabic*)},ref={\arabic*},topsep=1ex,partopsep=1ex]
\item\label{exc16} (\ldots) com mais prática, explorando e descobrindo mais o uso de ferramentas e outros, a exemplo disso os e-mails do Hotmail e Gmail que uso constantemente para edições e produções de documentos e pesquisas O \textit{Moodle, Google, Yahoo, YouTube} são as páginas de concursos, \textit{Facebook} e o \textit{Google+} são as páginas que mais visito (\ldots)
\item\label{exc17} Depois que eu entrei pra universidade passei a visitar \textit{sites} que eu não visitava, como \textit{Google Acadêmico} que apesar de ser um pouco complexo, são muito bons os artigos. Aprendi a pesquisar mais as coisas que eu leio, questionar se realmente são confiáveis.
\item\label{exc18} Tenho também um canal pessoal, mas não uso muito, faço uso no \textit{WhatsApp, Gmail}, \textit{Google+, Hangouts, Maps, Google, Google Acadêmico, Instagram} etc. todos contribuem na minha vida pessoal e profissional.
\item\label{exc19} Como estudante, utilizo o \textit{Google}, a \textit{Wikipédia} para pesquisas acadêmicas, YouTube  para assistir videoaulas e para outros estudos.
\end{enumerate}


Com as novas práticas nas novas esferas de atividade, os estudantes têm novos desafios em termos de autoria. Na universidade, há a necessidade de se letrarem em ferramentas para confeccionar, editar e disponibilizar seus materiais. Para a edição de textos, a ferramenta \textit{Word} aparece nos recortes \ref{exc20} e \ref{exc21}. Para a postagem dos trabalhos, o ambiente \textit{Moodle} é citado em \ref{exc22} e \ref{exc23}. Ainda em \ref{exc21}, há a citação dos dicionários \textit{on-line}, importantes ferramentas de consulta para quem trabalha com textos, como os profissionais em formação na área de Linguagens e Códigos. Em \ref{exc23}, a estudante detalha as habilidades que aprendeu até letrar-se em algum nível nas suas novas esferas de atuação.

% \begin{table}[htpb]
% \begin{tabular}{ll}
% \hline
% (20) &
%   \begin{tabular}[c]{@{}l@{}}Também nesse período tive que utilizar o \textit{Word}, porém eu não fazia ideia de como \\ criar um trabalho por meio dessa ferramenta e precisei da ajuda de colegas.\end{tabular} \\ \hline
% (21) &
%   \begin{tabular}[c]{@{}l@{}}Depois que me ingressei na universidade, minha relação com as novas tecnologias \\ mudou. Hoje as utilizo mais para atividades acadêmicas como pesquisas no \\ aplicativo \textit{Google} e dicionários \textit{on-line}, digitação e edição no \textit{Word} etc.\end{tabular} \\ \hline
% (22) &
%   \begin{tabular}[c]{@{}l@{}}Nesse mesmo tempo (entrada na faculdade) já aprendi a acessar, por meio do \\ computador e celular, a plataforma de envio de trabalho da faculdade, o \textit{Moodle}.\end{tabular} \\ \hline
% (23) &
%   \begin{tabular}[c]{@{}l@{}}Era (na outra faculdade em EaD) um procedimento similar ao que fazemos hoje no \\ curso da LEC que é referente as postagens no \textit{Moodle}. Antes de postar as atividades \\ nesse ambiente, primeiro passamos pelo processo de digitar, formatar, renomear, \\ salvar, etc. Depois busca-se o arquivo onde ele foi salvo. Este tipo de letramento \\ hoje me parece natural, mas foram nesses processos simples que me perdi várias \\ vezes nas primeiras vezes.\end{tabular} \\ \hline
% \end{tabular}
% \end{table}

\begin{enumerate}[resume,label={(\arabic*)},ref={\arabic*},topsep=1ex,partopsep=1ex]
\item\label{exc20} Também nesse período tive que utilizar o \textit{Word}, porém eu não fazia ideia de como criar um trabalho por meio dessa ferramenta e precisei da ajuda de colegas.
\item\label{exc21} Depois que me ingressei na universidade, minha relação com as novas tecnologias mudou. Hoje as utilizo mais para atividades acadêmicas como pesquisas no  aplicativo \textit{Google} e dicionários \textit{on-line}, digitação e edição no \textit{Word} etc.
\item\label{exc22} Nesse mesmo tempo (entrada na faculdade) já aprendi a acessar, por meio do computador e celular, a plataforma de envio de trabalho da faculdade, o \textit{Moodle}.
\item\label{exc23} Era (na outra faculdade em EaD) um procedimento similar ao que fazemos hoje no curso da LEC que é referente as postagens no \textit{Moodle}. Antes de postar as atividades nesse ambiente, primeiro passamos pelo processo de digitar, formatar, renomear, salvar, etc. Depois busca-se o arquivo onde ele foi salvo. Este tipo de letramento hoje me parece natural, mas foram nesses processos simples que me perdi várias vezes nas primeiras vezes.
\end{enumerate}

Mesmo que apenas uma vez, outras ferramentas e gêneros aparecem nas narrativas como o formato .PDF de arquivos de texto, vídeos e jogos e um gênero que se é também comum em papel, certamente é mais acessível digitalmente, os artigos, como relatado em \ref{exc24}.

% \begin{table}[htpb]
% \begin{tabular}{ll}
% \hline
% (24) &
%   \begin{tabular}[c]{@{}l@{}}Todo esse avanço tecnológico, em minha opinião, contribuiu e contribui muito com \\ a formação, pois disponibiliza conteúdos para estudo, como por exemplo, livros em \\ PDF, artigos, vídeos, jogos educativos, e informações a todo o momento.\end{tabular} \\ \hline
% \end{tabular}
% \end{table}

\begin{enumerate}[resume,label={(\arabic*)},ref={\arabic*},topsep=1ex,partopsep=1ex]
\item\label{exc24} Todo esse avanço tecnológico, em minha opinião, contribuiu e contribui muito com a formação, pois disponibiliza conteúdos para estudo, como por exemplo, livros em PDF, artigos, vídeos, jogos educativos, e informações a todo o momento.
\end{enumerate}

Uma prática interativa muito comum na atualidade é a de bate-papos e discussões via comunicadores instantâneos. Esses ambientes \textit{on-line} são extremamente caóticos, uma vez que não há uma ordenação de tópicos de discussão, mas apenas uma organização temporal. Para a finalidade de estudos, tendo em vista que há a necessidade de se recuperar dados e informações, em um primeiro instante, parece algo um tanto complicado. No entanto, a ferramenta mostra-se útil no contexto em que os celulares e suas conexões 3G são mais acessíveis que os computadores e a internet \textit{wi-fi} ou a cabo. Nesse sentido, estudantes relatam fazer grupos de estudo funcionarem via \textit{WhatsApp}\footnote{\url{https://www.whatsapp.com/}}, como nos relatos \ref{exc25} e \ref{exc26}.

% \begin{table}[htpb]
% \begin{tabular}{ll}
% \hline
% (25) &
%   \begin{tabular}[c]{@{}l@{}}Após começar estudar tive mais acesso ao \textit{e-mail} e ao \textit{WhatsApp} com grupos de \\ estudo, de turma e curso.\end{tabular} \\ \hline
% (26) &
%   \begin{tabular}[c]{@{}l@{}}Todos os dias assim que acordo, o meio tecnológico que uso é o aplicativo \\ \textit{WhatsApp}, pois a maioria dos assuntos relacionados à faculdade, trabalho e muitas \\ questões pessoais são discutidos por meio deste.\end{tabular} \\ \hline
% \end{tabular}
% \end{table}

\begin{enumerate}[resume,label={(\arabic*)},ref={\arabic*},topsep=1ex,partopsep=1ex]
\item\label{exc25} Após começar estudar tive mais acesso ao \textit{e-mail} e ao \textit{WhatsApp} com grupos de  estudo, de turma e curso.
\item\label{exc26} Todos os dias assim que acordo, o meio tecnológico que uso é o aplicativo \textit{WhatsApp}, pois a maioria do questões pessoais são discutidos por meio deste.
\end{enumerate}


As discussões \textit{on-line} estão cada vez mais comuns em diversas esferas de nossas atividades interativas, mesmo em algumas que anteriormente eram 100\% realizadas presencialmente. Na academia, mesmo em cursos presenciais, a presença da mediação \textit{on-line} é constante. Uma alternativa à distância dos estudantes no TC e aos poucos recursos das instituições e famílias para viagens são os congressos \textit{on-line}. Como citado na segunda seção deste trabalho, os estudantes foram, na disciplina Gêneros Textuais e Discursivos, orientados a produzirem artigos com reflexões teóricas acerca de seus próprios letramentos ou de outros sujeitos das comunidades. \textcite{magnani_da_2018, castro_reflexao_2017}, citados ao longo do presente texto, relatam os trabalhos e resultados. Nas tecnobiografias em análise, três de nossos sujeitos mencionam a experiência conforme excertos a seguir.

% \begin{table}[htpb]
% \begin{tabular}{ll}
% \hline
% (27) &
%   \begin{tabular}[c]{@{}l@{}}Nos dias atuais, sou administrador de um site, tenho página no \textit{Facebook}, dois \\ canais no \textit{Youtube}, dois \textit{e-mails}, (um para estudo, e outro para trabalho), dois blogs, \\ e participo de projetos e congressos \textit{on-line}.\end{tabular} \\ \hline
% (28) &
%   \begin{tabular}[c]{@{}l@{}}Por meio das facilidades tecnológicas já participei de congresso \textit{on-line}, apresentei \\ e publiquei trabalho em site tudo feito virtualmente, ou seja, sem as tecnologias \\ digitais eu talvez não tivesse a oportunidade de estudar.\end{tabular} \\ \hline
% (29) &
%   \begin{tabular}[c]{@{}l@{}}Tive uma grande satisfação de participar de um projeto, executado pelo professor da \\ LEC (...) na escrita de um artigo, \textit{Práticas de gênero discursivo e letramentos em} \\ \textit{São Gonçalo do Rio das Pedras}, para o UEADSL, com resultado de menção \\ honrosa pela escrita do artigo.\end{tabular} \\ \hline
% \end{tabular}
% \end{table}

\begin{enumerate}[resume,label={(\arabic*)},ref={\arabic*},topsep=1ex,partopsep=1ex]
\item\label{exc27} Nos dias atuais, sou administrador de um site, tenho página no \textit{Facebook}, dois canais no \textit{Youtube}, dois \textit{e-mails}, (um para estudo, e outro para trabalho), dois blogs, e participo de projetos e congressos \textit{on-line}.
\item\label{exc28} Por meio das facilidades tecnológicas já participei de congresso \textit{on-line}, apresentei e publiquei trabalho em site tudo feito virtualmente, ou seja, sem as tecnologias digitais eu talvez não tivesse a oportunidade de estudar.
\item\label{exc29} Tive uma grande satisfação de participar de um projeto, executado pelo professor da LEC (\ldots) na escrita de um artigo, \textit{Práticas de gênero discursivo e letramentos em} \textit{São Gonçalo do Rio das Pedras}, para o UEADSL, com resultado de menção honrosa pela escrita do artigo.
\end{enumerate}

Em \ref{exc27}, o estudante mostra-se bastante engajado em práticas letradas em meio digital, como administrar site, página de rede social e canais de vídeos \textit{on-line}, além da participação de outras práticas típicas da academia, e também em meio digital, como o congresso citado em \ref{exc29}: UEADSL\footnote{\url{www.ueadsl.textolivre.pro.br}}, que é o Congresso Nacional Universidade, Educação a Distância e Software Livre. O estudante do excerto \ref{exc28} credita às novas tecnologias a oportunidade da prática: “sem as tecnologias digitais eu talvez não tivesse a oportunidade de estudar.” E em \ref{exc29} há o contentamento expresso em “Tive uma grande satisfação” e “com resultado de menção honrosa pela escrita do artigo.”

As mudanças são citadas conscientemente por todos. Até discentes que já tinham acesso a certas tecnologias digitais, como computadores, no ensino médio apontam diferenças, como a questão de direitos autorais, ponto de observação de uma das estudantes, como pode ser visto em \ref{exc30}.

% \begin{table}[htpb]
% \begin{tabular}{ll}
% \hline
% (30) &
%   \begin{tabular}[c]{@{}l@{}}Até o ensino médio, as produções como estudante exigiam o acesso à internet, para \\ pesquisas e também para que tivéssemos contato com o mundo digital; porém, na \\ faculdade essa exigência é maior, pois temos responsabilidades maiores, com \\ cuidados com direitos autorais e a organização de arquivos segundo algumas regras.\end{tabular} \\ \hline
% \end{tabular}
% \end{table}

\begin{enumerate}[resume,label={(\arabic*)},ref={\arabic*},topsep=1ex,partopsep=1ex]
\item\label{exc30} Até o ensino médio, as produções como estudante exigiam o acesso à internet, para pesquisas e também para que tivéssemos contato com o mundo digital; porém, na faculdade essa exigência é maior, pois temos responsabilidades maiores, com cuidados com direitos autorais e a organização de arquivos segundo algumas regras.
\end{enumerate}

Como todos os dados apontam, a universidade apresentou a esses sujeitos grandes oportunidades de acesso a práticas de letramentos, em gêneros acadêmicos, e também em gêneros digitais não exclusivos da academia, como o \textit{e-mail} e até mesmo as redes sociais, que não tiveram excertos ilustrativos neste trabalho, tendo em vista que o tópico extrapola a investigação temática proposta, mas aparecem em todos os relatos. No entanto, mesmo diante dos novos letramentos que se apresentam, ainda há aqueles que confessam que fazem somente o mínimo, como em \ref{exc31}.

% \begin{table}[htpb]
% \begin{tabular}{ll}
% \hline
% (31) &
%   \begin{tabular}[c]{@{}l@{}}Sempre que pude busquei aprender sobre os usos dessas tecnologias, porém não \\ amplamente e sim apenas para suprir minhas necessidades no que diz respeito a \\ acessar um \textit{site}, fazer um curso, criar uma página no Facebook, fazer um trabalho \\ da faculdade e enviar.\end{tabular} \\ \hline
% \end{tabular}
% \end{table}

\begin{enumerate}[resume,label={(\arabic*)},ref={\arabic*},topsep=1ex,partopsep=1ex]
\item\label{exc31} Sempre que pude busquei aprender sobre os usos dessas tecnologias, porém não amplamente e sim apenas para suprir minhas necessidades no que diz respeito a  acessar um \textit{site}, fazer um curso, criar uma página no Facebook, fazer um trabalho da faculdade e enviar.
\end{enumerate}

Nas narrativas em análise, há também uma preocupação com a letra no papel, como no excerto \ref{exc32}, típica das gerações que mais digitam do que escrevem manualmente. A expectativa e meu empenho é que, nas experiências dessa estudante, também haja espaço para a autoria em, e para acesso a, diferentes plataformas, para novos letramentos, tão necessários nessa época de informação abundante e análises escassas.

% \begin{table}[htpb]
% \begin{tabular}{ll}
% \hline
% (32) &
%   \begin{tabular}[c]{@{}l@{}}Acredito que o uso de tecnologia digital tem se tornado constante em minha vida. \\ Isso de certa forma não é tão bom, pois para mim que estudo a habilitação de \\ Linguagens é preciso exercitar e explorar minha escrita no papel e a leitura em \\ livros impressos para que o uso dessas tecnologias não tome espaço geral em \\ minhas atividades.\end{tabular} \\ \hline
% \end{tabular}
% \end{table}

\begin{enumerate}[resume,label={(\arabic*)},ref={\arabic*},topsep=1ex,partopsep=1ex]
\item\label{exc32} Acredito que o uso de tecnologia digital tem se tornado constante em minha vida. Isso de certa forma não é tão bom, pois para mim que estudo a habilitação de Linguagens é preciso exercitar e explorar minha escrita no papel e a leitura em livros impressos para que o uso dessas tecnologias não tome espaço geral em minhas atividades.
\end{enumerate}

Por último, trago o excerto de uma estudante para ilustrar o sucesso como os novos letramentos que os futuros educadores do campo dos Vales do Jequitinhonha e Mucuri, e adjacências, alcançam.

% \begin{table}[htpb]
% \begin{tabular}{ll}
% \hline
% (33) &
%   \begin{tabular}[c]{@{}l@{}}A partir de quando entrei na faculdade, tive grandes oportunidades de uso dos meios \\ tecnológicos que são usados diariamente, por exemplo, o notebook. Agora vejo que é \\ bem mais fácil e simples de ser usado.\end{tabular} \\ \hline
% \end{tabular}
% \end{table}

\begin{enumerate}[resume,label={(\arabic*)},ref={\arabic*},topsep=1ex,partopsep=1ex]
\item A partir de quando entrei na faculdade, tive grandes oportunidades de uso dos meios tecnológicos que são usados diariamente, por exemplo, o notebook. Agora vejo que é bem mais fácil e simples de ser usado.
\end{enumerate}

Esse último excerto fala de uma máquina, o computador, que é primordial aos letramentos digitais e acadêmicos dos sujeitos de pesquisa e outros acadêmicos na atualidade. Com ele e tudo que a universidade lhes proporcionou, os estudantes já foram muito além, sendo que a maioria da turma da citada disciplina Gêneros Discursivos e Textuais já se formou, muitos deles já estão atuando como professores e alçando diferentes voos com atuações em associações, sindicatos, cursos e escolas informais e na pós-graduação com pesquisas sobre linguagens, culturas e educação do campo.

\subsection{Considerações para a continuidade de um diálogo}\label{sec-consideracoes}
Este texto trata apenas de uma parte dos enunciados que tenho produzido com meus estudantes, como outros exemplos de trabalhos citados e referenciados ao longo dessas páginas. Como busquei mostrar, temos pavimentado nosso caminho com construções sólidas para a promoção de uma educação crítica, com a valorização dos contextos para reflexão e promoção de eventos e práticas de letramentos diversos em diferentes esferas de atuação. Os desafios são enormes, como instituições de educação totalmente disciplinares e descontextualizadas que segregam ainda mais os sujeitos que já estão à margem de diversos processos sociais. Como exemplo simples e corriqueiro, a própria inscrição para o ENEM e o SISU, sistemas que dão acesso à grande maioria das universidades brasileiras, excluem aqueles com poucos letramentos digitais. No início da \cref{sec-analise}, por exemplo, os excertos enumerados de \ref{exc01} a \ref{exc06} referem-se a narrativas de estudantes que não tinham acesso à internet ou até mesmo não tinham computador \textit{off-line} antes de entrar na universidade. Já os excertos \ref{exc07}, \ref{exc08}, \ref{exc09} e \ref{exc10} dizem respeito a narrativas de estudantes que conheceram o gênero \textit{e-mail} apenas em decorrência de suas entradas na universidade. Em um grupo de 33 excertos, sendo 1 de cada estudante, quase um terço poderia não ter entrado na universidade por uma questão de acesso e práticas de letramento esperadas no ensino médio. A pergunta que fica é: Quantos outros estudantes de ensino médio no país desistem ou nem pensam em fazer um curso superior por falta dos letramentos adequados?

Se a porta de entrada é tão estreita para o nosso público, o caminho é virtuoso de acordo com as nuances presentes nas narrativas mostradas ao longo do texto. Nas novas esferas de atuação, os discentes são estimulados e demandados de diversas formas. Mesmo em um curso presencial, em regime de alternância, as ferramentas digitais para a interação são essenciais, sobretudo quando estão distantes da universidade, no TC. Da mesma forma, em um curso de formação de professores da área de humanas, arquivos digitais também se tornam grandes aliados por várias questões, como a quantidade de textos a serem lidos e debatidos, a circulação fácil de textos digitais, os preços, a facilidade de edição das produções, dentre outras. Além desses fatores, a partir de metodologias dialógicas, como mostrado, a universidade e os professores, com atenção especial às experiências apresentadas, promovem diversos eventos de letramentos que têm importante papel na promoção de reflexão crítica sobre letramentos e de letramentos diversos na esfera acadêmica e digital. Letramento é condição essencial para o acesso aos eventos e práticas sociais de quaisquer grupos letrados e, então, essencial para a construção de um docente que seja pesquisador, que tenha consciência crítica, que saiba aprender ao ensinar.

A variedade de habilidades que apareceram nas narrativas não é diferente das de outros estudantes em contextos urbanos. Mesmo vindo por caminhos diferentes, os estudantes do campo que me cedem suas narrativas como corpus de pesquisa mostram-se capazes de estabelecerem novos diálogos produtivos nas cadeias dialógicas da universidade a exemplo dos citados. Na voz dos próprios sujeitos, apresentadas ao longo do texto, a universidade lhes apresentou muitos gêneros digitais e/ou acadêmicos e suas novas práticas. E, enquanto futuros professores, as novas esferas de atividades não lhes permitirão que haja retrocesso nesse sentido. Com isso, conseguiram ampliar não somente as habilidades necessárias às práticas letradas dos novos contextos, mas suas leituras de todo um conjunto significativo envolvido na construção dos gêneros acadêmicos e digitais. Não menos importante, entendo que os letramentos acadêmicos e digitais que emergiram vieram em diálogo direto com os saberes que os estudantes trouxeram na medida em que foram construídos a partir de autobiografias e com reflexões dos próprios sujeitos sobre seus próprios processos de letramentos.

O objetivo apresentado, de verificar como, no discurso dos estudantes, a universidade oportuniza eventos de letramento e suas relações com as novas tecnologias, foi atingido, pois foram apresentados diversos eventos de letramento, com as reflexões consideradas adequadas às práticas e às várias questões contextuais nesses processos. No que se refere a generalizações dos resultados, a pesquisa narrativa tem suas observações, como as demais investigações qualitativas, principalmente no que se refere aos contextos de pesquisa. Para \textcite[p. 92]{barkhuizen_narrative_2014}, “[e]studos de investigação narrativa tipicamente limitam suas afirmações nesses aspectos (generalizações), enfatizando seu foco no particular e no indivíduo”. A expectativa, para a continuidade desse diálogo, é que essas experiências nos auxiliem no desafio permanente de ampliar os letramentos de toda a sociedade, para que outros espaços sejam cada vez mais acessíveis a todos, de diversas formas e, especialmente, por meio da formação de professores, foco do nosso trabalho.

\printbibliography\label{sec-bib}

\end{document}
