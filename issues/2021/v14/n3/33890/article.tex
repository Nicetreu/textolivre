% !TEX TS-program = XeLaTeX
% use the following command: 
% all document files must be coded in UTF-8
\documentclass[english]{textolivre}
% for anonymous submission
%\documentclass[anonymous]{textolivre}
% to create HTML use 
%\documentclass{textolivre-html}
% See more information on the repository: https://github.com/leolca/textolivre

% Metadata
\begin{filecontents*}[overwrite]{article.xmpdata}
    \Title{An analysis of social communicative acts among MMORPG players}
    \Author{Ricardo Casañ-Pitarch}
    \Language{en}
    \Keywords{Socializing \sep MMORPG \sep Video-games \sep Discourse Analysis \sep Language}
    \Journaltitle{Texto Livre}
    \Journalnumber{1983-3652}
    \Volume{14}
    \Issue{3}
    \Firstpage{1}
    \Lastpage{16}
    \Doi{10.35699/1983-3652.2021.33890}

    \setRGBcolorprofile{sRGB_IEC61966-2-1_black_scaled.icc}
            {sRGB_IEC61966-2-1_black_scaled}
            {sRGB IEC61966 v2.1 with black scaling}
            {http://www.color.org}
\end{filecontents*}

\journalname{Texto Livre}
\thevolume{14}
\thenumber{3}
\theyear{2021}
\receiveddate{\DTMdisplaydate{2021}{5}{13}{-1}} % YYYY MM DD
\accepteddate{\DTMdisplaydate{2021}{09}{21}{-1}}
\publisheddate{\DTMdisplaydate{2021}{09}{29}{-1}}
% Corresponding author
\corrauthor{Ricardo Casañ-Pitarch}
% DOI
\articledoi{10.35699/1983-3652.2021.33890}
%\articleid{NNNN} % if the article ID is not the last 5 numbers of its DOI, provide it using \articleid{} commmand
% list of available sesscions in the journal: articles, dossier, reports, essays, reviews, interviews, editorial
\articlesessionname{articles}
% Abbreviated author list for the running footer
\runningauthor{Casañ-Pitarch}
\sectioneditorname{Daniervelin Pereira}
\layouteditorname{Anna Izabella M. Pereira}


\title{An analysis of social communicative acts among MMORPG players}
\othertitle{Uma análise dos atos de comunicação social entre jogadores de MMORPG}
% if there is a third language title, add here:
%\othertitle{Artikelvorlage zur Einreichung beim Texto Livre Journal}

\author[1]{Ricardo Casañ-Pitarch \orcid{0000-0002-1689-7954} \thanks{Email: \url{ricapi@upv.es}}}

\affil[1]{Universitat Politècnica de València, Department of Applied Linguistics, Valencia, Spain.}

\addbibresource{article.bib}
% use biber instead of bibtex
% $ biber tl-article-template


% reference itens in a descriptive list using their labels instead of numbers
% insert the code below in the preambule:
\makeatletter
\let\orgdescriptionlabel\descriptionlabel
\renewcommand*{\descriptionlabel}[1]{%
  \let\orglabel\label
  \let\label\@gobble
  \phantomsection
  \edef\@currentlabel{#1\unskip}%
  \let\label\orglabel
  \orgdescriptionlabel{#1}%
}
\makeatother
%
% in your document, use as illustraded here:
%\begin{description}
%  \item[first\label{itm1}] this is only an example;
%  % ...  add more items
%\end{description}
 

% custom epigraph - BEGIN 
%%% https://tex.stackexchange.com/questions/193178/specific-epigraph-style
\usepackage{epigraph}
\renewcommand\textflush{flushright}
\makeatletter
\newlength\epitextskip
\pretocmd{\@epitext}{\em}{}{}
\apptocmd{\@epitext}{\em}{}{}
\patchcmd{\epigraph}{\@epitext{#1}\\}{\@epitext{#1}\\[\epitextskip]}{}{}
\makeatother
\setlength\epigraphrule{0pt}
\setlength\epitextskip{0.5ex}
\setlength\epigraphwidth{.7\textwidth}
% custom epigraph - END


% if you use multirows in a table, include the multirow package
\usepackage{multirow}

% add line numbers for submission
%\usepackage{lineno}
%\linenumbers

\begin{document}
\maketitle

\begin{polyabstract}
\begin{abstract}
This research aims to analyze the communicative acts used by video-game players to socialize with other ones in MMORPG. The method to analyze these communicative acts is based on the results obtained from a questionnaire in which MMORPG players participated by responding to a series of closed questions. This questionnaire focused on analyzing some of the participants’ socializing communication habits following the classification of \textcite[p. 26]{guntermann1982}, and it aims to determine how a series of variables interfere in the socializing process with other players. These variables determining how they interfere in the socializing process among MMORPG players are age, extroversion, experience, dedication, and languages spoken. Thus, our results show the correlation between the uses of different communicative acts to socialize with other virtual players according to the previous variables. 

\keywords{Socializing \sep MMORPG \sep Video-games \sep Discourse analysis \sep Language}
\end{abstract}

\begin{portuguese}
\begin{abstract}
Esta pesquisa tem como objetivo analisar os atos comunicativos utilizados por jogadores de videogame para se socializarem em MMORPG. O método de análise desses atos comunicativos é baseado nos resultados obtidos a partir de um questionário no qual os jogadores de MMORPG participaram respondendo a uma série de questões fechadas. Esse questionário teve como foco a análise de alguns hábitos de comunicação socializante dos participantes segundo a classificação de \textcite[p. 26]{guntermann1982}, e visa determinar como uma série de variáveis interfere no processo de socialização com outros jogadores. Essas variáveis que determinam como interferem no processo de socialização entre os jogadores de MMORPG são idade, extroversão, experiência, dedicação e línguas faladas. Assim, nossos resultados mostram a correlação entre os usos de diferentes atos comunicativos para socializar com outros jogadores virtuais de acordo com as variáveis anteriores.

\keywords{Socialização \sep MMORPG \sep Videogames \sep Análise do discurso \sep Linguagem}
\end{abstract}
\end{portuguese}

% if there is another abstract, insert it here using the same scheme
\end{polyabstract}


\section{Introduction}\label{sec-intro}
The popularity of Massively Multiplayer Online Role-Playing Games (MMORPG) has been growing since the beginning of the XXI century. Advances in technology during this period have permitted that individuals can play online role games with other players worldwide. Nowadays, most homes have computers and internet access; and the improvements in these two in the last years have been noticeable. Present computers are increasingly more powerful and Internet connection is faster and more stable and reliable than it used to be only a few years ago. In this sense, MMORPG has become a profitable market that involves millions of players and provides more benefits than other more traditional entertaining sectors. According to \textcite{siwek2007}, the video-game industry generated more than \$ 21,000 million in 2012 only in the US, whereas worldwide sales had reached \$ 93,000 million in 2013. 

These data suggest that there is a huge community of users; and consequently, the formation and development of online gaming communities is already a fact, being this a relatively new concept that is characterized by both traditional and real-world features as well as components related to virtual worlds. Thus, these new online arenas with virtual communities or tribes are part of our present daily lives and, as result, they are worth studying. 

Some previous research has focused on different aspects concerning social interaction in MMORPG. In this sense, \textcite{chen2007} explained how players socialize in the video-game World of Warcraft; the result was a report on the factors influencing social interaction in the video-game such as rules of conduct, the attitude of the player towards other players or collectives, or contextual factors among others. \textcite{yee2006} also carried out research and found there are new forms of identity and interaction in virtual worlds. Another example is \textcite{cole2007}; they explored the social interaction among MMOPRG players and their results revealed findings on the players’ behavior and how these interfere in their relations. Later on, \textcite{mccreery2015} performed an experiment in which some actions that required communication among the players were evaluated: interactivity, affective association, community cohesion, interaction intensity, and knowledge and experience sharing. For the aim of this research, these previous studies on MMORPG players have been especially useful since they introduced communicative issues related to social interaction.

This research aims at analyzing how MMORPG players interact with other players and how often they use communicative acts to socialize according to a series of variables that may interfere in their acts of exchange: age, experience with video games, degrees of interaction and dedication, languages spoken and players’ extroversion (personality). This study hypothesizes that there is a connection among these variables and the way MMORPG players interact with other users depending on the type of relationship they have with them: the participant knows the other player from the real world, the participant knows the other player from the virtual world, or the participant does not know the other player. 

To fulfill the purpose of this study, an experiment has been performed with 201 participants through an online survey that was shared in specialized gaming forums, social networks, and it was also distributed to students. The only conditions to take part in this experiment were that the participants had some experience at playing MMORPG and that they had ever interacted with other players in virtual worlds. In this survey, the participants had to respond to a series of closed questions based on a \emph{Likert} scale. Having this stage completed, the data collected has been analyzed for later discussion. 

\section{Social Discourse in Massively Multiplayer Online Role-Playing Games}
This section covers a review of previous work on topics related to this research. Firstly, the main principles of social discourse are introduced. Definitions of anthropological linguistics and social discourse are provided. Then, Guntermann and Phillips’ taxonomy on the most frequent social communicative acts is shown \cite*[p. 26]{guntermann1982}. Secondly, the main characteristics around MMORPG are introduced and their effects on communication among players are considered to understand the nature of this research. 

\subsection{A Brief Introduction to Social Discourse}
In linguistics, language is studied from different perspectives and, consequently, it is also divided into different specialized fields. This research refers to anthropological linguistics, a broad field within linguistics, and more specifically to social discourse. \textcite{foley1997} defined anthropological linguistics as the study of languages considering the influence of their social and cultural contexts. Thus, anthropological linguistics focuses on the study of any specific linguistic community and how their social rules, cultural identify and ideologies influence the way their members speak and interact. 

Among the different sub-fields within anthropological linguistics, this paper focuses on the study of social discourse; in other words, it aims at studying a series of social linguistic strategies used by the members of a particular community when they interact with other members from the same community. To this purpose, \textcite[p. 4]{farrell1998} defined social discourse as “[linguistic] practices that constitute and are constituted by, the economic, cultural, and political contexts in which they occur" and they may represent the different status of power, value, and identity in the communicative acts of exchange. The use of these communicative acts may vary depending on the context in which they occur, the features of each community, and the nature of every single player. 

Thus, the framework of this research is based on the analysis of social-communicative acts used by MMORPG players when they interact with other players in their virtual communities considering the taxonomy introduced by \textcite{guntermann1982}, whose communicative acts are listed below: 

\begin{enumerate}\label{list1}
\item Greeting others
\item Taking leave; planning to meet again
\item Introducing and meeting people
\item Making small talk
\item Getting to know others
\item Issuing invitations, and accepting or declining them
\item Paying visits and receiving visitors
\item Offering things, and accepting or declining them
\item Sharing leisure activities
\item Joking
\item Flirting
\item Telling or listening to stories, recounting events
\item Gossiping
\item Expressing feelings
\item Apologizing and reacting to apologies
\item Displaying accomplishments
\item Making Social Plans
\item Responding to Social Plans
\end{enumerate}

\subsection{Defining Massively Multiplayer Online Role-Playing Games}
To understand this research, it is also necessary to introduce and define the main characteristics of MMORPG. To start with, these games involve different individuals playing at the same time and in the same virtual space. This type of video game requires an internet connection and thousands of players control their avatars and interact with other ones \cite{zhong2011}. Thus, MMORPG are online games that involve interaction with multiple players around the world in real-time \cite{barker2012} and they can enter or leave the virtual world at any time \cite{bell2008}. Therefore, the functioning and action in this world are continuous; and this is similar to the real world when people are either awake or asleep.

Another characteristic of these video games is their playability, which may vary from one game to another ones. Within this genre of video games, there are two main broad categories concerning the environment or scenario where they occur: they can be scripted and unscripted \cite{alemi2007}. On the one hand, scripted environments are those games in which the player has to accomplish a series of tasks and goals to get through the different levels of the game; in other words, they follow a script designed by the game developer (i.e.: World of Warcraft, Guild Wars 2). On the other hand, unscripted environments refer to those video games in which the player can do any action or task as they please, despite it is necessary to consider the possibilities and limitations of the video game. In this case, there are no levels or pre-established tasks (i.e.: \emph{Minecraft}, \emph{Second Life}). 

In addition to these features, these video games are also characterized by the great number of participants involved, as its name suggests. The number of players can be fewer than ten (i.e.: \emph{Diablo 2}) but they can also involve millions of them (i.e.: \emph{World of Warcraft}). This massive participation together with the fact that players can interact in real-time is the main reason to hypothesize that MMORPG players form virtual communities have their own rules and unique characteristics \cite{bradley2004, gordon2008}. In this sense, MMORPG players can meet people for the first time, but they can also find allies in the virtual world or the real one and bring them to the game \cite{dickey2007, smith2005}. These huge virtual communities can be inhabited by millions of players \cite{nakamura2009}; therefore, knowledge on the use of foreign languages can be a great help to interact with players with different nationalities and get their support \cite{cornillie2012, reinders2009}.

Besides, as virtual worlds never sleep, dedication to these video games is fundamental to achieve certain status and recognition and get through new levels \cite{nagy2016, stone2016}. Similarly, group strategy and creating or joining a community can sometimes be necessary to accomplish some missions \cite{whang2004}. Thus, personality can be a relevant factor to determine the success, or not, in certain missions \cite{collins2012, mancini2017}. As it happens in the real world, people can approach or avoid certain people due to psychological or emotional reasons based on their character or attitude \cite{mccreery2015}. Thus, in video games, players can also feel attracted by other players for different reasons and purposes. 

\section{Method}
This research has been based on an experiment with 201 samples of MMORPG players. The participants completed an online questionnaire that was circulated through gaming forums as well as to university students. Among the participants, 147 were male (\emph{n} = 73.13\%) and 54 female (\emph{n} = 26.87\%). The age range was set between 15 and 60, whereas the mean was 24.80. There were 32 nationalities represented among the participants, being Spain the most popular (115; n=57.21\%). Among these participants, 181 resided in their country of origin (\emph{n} = 90.05\%). 

The material used for this research was an online survey that was created with Google forms and whose results were later analyzed. Questions were closed and most of them were Likert-scale based. The questionnaire was divided into five sections. The first one compiled data about the participants (name/nickname, gender, age, nationality, country of residence) and checked if they were players of MMORPG. In case the participants were not players of this type of video game, their answers were disregarded. The second group aimed at enquiring about the players' knowledge of languages and their level according to the Common European Framework of Reference for Languages (CEFRL). The third one focused on the players’ experience with video games, their degree of interaction, and their weekly dedication in hours. In the fourth group, the personality of the player was determined using the Revised NEO Personality Inventory (NEO PI-R) \cite{costa1992}; in this case, our survey only included those elements concerning extroversion (12 items). The last group performed the analysis of the communicative social acts following the taxonomy introduced by \textcite{guntermann1982}; our adapted version included 20 items. 

The procedure of this experiment was divided into three stages. Firstly, the questionnaire was designed and tested with 42 students, of whom, 24 confirmed that they played or had ever played MMORPG video games (\emph{n} = 59.50\%). After testing the survey, it was circulated to different MMORPG forums, published and shared in social networks, and submitted to students at Universitat Jaume I as well as to colleagues and to departments from other universities to achieve higher participation. As it has been previously explained, only the participants who confirmed that they were MMORPG players were considered in this research. Once there was a considerable number of participants who had taken part in the experiment, the results were analyzed considering six variables (1. age, 2. extroversion, 3. experience with MMORPG, 4. degree of interaction with other players, 5. dedication at playing MMORPG, 6. languages spoken) and three specific contexts (1. players mutually known in the real world, 2. players mutually known in the virtual world, 3. unknown players). 

\section{Results}
\subsection{Variables: Demographic Profile of Participants}
\begin{itemize}
\item \emph{Variable 1: Age}. The mean age of our participants was 22.6 years; the age of male players was 24.42, whereas in females it was 25.94. 
\item \emph{Variable 2: Extroversion (Personality)}. Using a 1-5 Likert Scale, the mean level of extroversion was 3.41 among the participants, being 3.40 for men and 3.44 for women. Concerning age ranges, those under 20 seemed to be more extroverted (3.53) than those aged 20-30 (3.36) as well as those older than 30 years (3.35). In this sense, 52 respondents (\emph{n} = 25.87\%) were tagged as extrovert (4-5), 99 respondents (\emph{n} = 49.25\%) were neutral (3-3.99), whereas the remaining 50 (\emph{n} = 24.88\%) could be considered introvert (1-2.99).
\item \emph{Variable 3: Experience in Playing video games and MMORPG}. Using a \emph{Likert} scale from 1 to 5, the meaningful experience of players with video games was 3.98, being this correspondent to three and six years but very close to seven and ten years. In this case, 77 respondents (\emph{n} = 38.31\%) were tagged as experts (4-5), 50 respondents (\emph{n} = 24.88\%) were neutral (3-3.99), whereas the remaining 74 (\emph{n} = 36.81\%) could be considered novice (1-2.99). Our survey also focused on their experience at MMORPG, which resulted in 2.97, corresponding to the range between one and three years but again being near to the next degree, between three and six years. Results also showed that participants aged between 20 and 30 were the most experienced at MMORPG (3.29) in comparison to those participants older than 30 (3.23) and younger than 20 (2.38). Concerning gender, males were more experienced (3.03) than women (2.80). Similarly, players tagged as introverts were more expertise (3.28) than those considered extroverts (3.21), and neutral ones had the least experienced in this type of video game (2.69). 
\item \emph{Variable 4: Current dedication to playing video games and MMORPG}. The mean dedication of weekly hours playing video games was (3.11), of which between one and five hours were used at playing MMORPG (2.68). In this sense, 48 respondents (\emph{n} = 23.88\%) were tagged as devoted players (4-5), 54 respondents (\emph{n} = 26.87\%) were neutral (3-3.99), whereas the remaining 99 (\emph{n} = 49.25\%) could be considered occasional (1-2.99). Results also showed that introvert players were the ones with more weekly hours playing MMORPG (3.12), in comparison to extrovert (2.79) and neutral players (2.39). Concerning age, players aged 20 or older seem to devote more weekly hours to MMORPG (2.81) than younger ones (2.38). Male participants also tend to play more often (2.71) than females (2.59).
\item Variable 5: Level of interaction in virtual worlds. Results showed that the mean frequency of interactions with other players was 3.35 using a 1-5 \emph{Likert} Scale. In this sense, the mean frequency of interactions with players they know from the real world was 3.69, with a player they know from the virtual world was 3.12, and with a player that they did not know was 3.25. As result, considering the mean level of interaction of players, 68 (\emph{n} = 33.83\%) were tagged as players with higher (4-5), 89 (\emph{n} = 44.28\%) with intermediate (3-3.99), and 44 (\emph{n} = 21.89\%) with lower levels of interactions (1-2.99).
\item Variable 6: Languages spoken. Concerning the number of languages spoken by the participants, results showed that the mean of languages spoken by the participants, with at least an elementary level, was 3.28; of which 1.41 were their mother tongues and 1.87 were foreign languages. In this sense, 193 participants (\emph{n} = 96.02\%) had an elementary or higher level in at least a foreign language, 162 (\emph{n} = 80.60\%) had an intermediate or higher level in at least a foreign language, 82 (\emph{n} = 41.79\%) had an upper-intermediate or higher level in at least a foreign language, and 75 (\emph{n} = 37.31\%) had an advanced level in at least a foreign language. Among all the participants and using a 1-5 \emph{Likert} Scale, the most usual languages in MMORPG among our participants were English (4.06) and Spanish (3.43). In addition, the ones speaking more languages with at least an intermediate level among the participants were the ones tagged as extroverted (2.81); in contrast, introverted ones were the ones with fewer languages spoken with at least an intermediate level (2.12). Results also showed that female participants were able to speak more languages with at least an intermediate level (2.52) than males (2.44). Concerning ages, the older ones were the ones with a higher command of languages with at least an intermediate level (2.63), whereas those under 20 could speak more languages with at least an intermediate level (2.48) than the ones in the age range 20-30 (2.36). 
\end{itemize}

\subsection{Players’ Social Communicative Acts}
In reference to the communicative acts analyzed, results showed that the most common communicative acts were ‘saying goodbye’, ‘thanking’, ‘congratulating’, ‘joking’ and ‘sympathizing', whereas the least frequent were 'flirting', 'gossiping', 'making plans in the real world', 'using small talks' and 'inviting others to take actions in the virtual world'. However, this list could vary if it concerns interactions with a player they know from the real or virtual worlds, or if they are unknown players. 

As it can be observed in \Cref{figure1},  the most common acts among players that they know each other from the real world are 'saying goodbye', 'thanking', 'joking', 'sympathizing’, and 'congratulating’. Whereas, among players that they know each other from the virtual world, the most common acts were 'saying goodbye', 'thanking', 'joking', 'congratulating', and 'showing understanding'. At last, among players that did not know each other, the most usual acts are 'thanking', 'saying goodbye', 'congratulating', 'apologizing', and 'showing understanding'.

\begin{table}[htpb]
\caption{Frequency of Players’ Social Communicative Acts.}
\label{figure1}
\centering
\begin{tabular}{p{0.03\textwidth}p{0.3\textwidth}p{0.1\textwidth}p{0.1\textwidth}p{0.12\textwidth}p{0.1\textwidth}}
\toprule 
\# & Act & Mean & Real W. & Virtual W. & Unknown
\\ 
\midrule
1 & Saying Goodbye & 3.95 & 4.31 & 4.02 & 3.52
\\
2 & Thanking & 3.93 & 4.22 & 3.89 & 3.68
\\
3 & Congratulating & 3.62 & 3.98 & 3.60 & 3.28
\\
4 & Joking & 3.55 & 4.10 & 3.63 & 2.93
\\
5 & Sympathizing & 3.51 & 3.99 & 3.47 & 3.06
\\
6 & Showing Understanding & 3.51 & 3.90 & 3.51 & 3.11
\\
7 & Apologizing & 3.44 & 3.76 & 3.41 & 3.17
\\
8 & Greeting & 3.32 & 3.96 & 3.39 & 2.60
\\
9 & Complimenting & 3.19 & 3.60 & 3.19 & 2.78
\\
10 & Introducing & 3.09 & 3.60 & 3.16 & 2.51
\\
11 & Praising & 3.00 & 3.38 & 2.99 & 2.61
\\
12 & Introducing others & 2.87 & 3.30 & 2.90 & 2.39
\\
13 & Making Plans in the VW & 2.79 & 3.69 & 2.75 & 1.94
\\
14 & Interrupting & 2.73 & 3.00 & 2.67 & 2.52
\\
15 & Taking Time to consider & 2.68 & 3.00 & 2.74 & 2.30
\\
16 & Inviting others & 2.66 & 3.46 & 2.61 & 1.92
\\
17 & Using Small Talks & 2.56 & 2.91 & 2.57 & 2.19
\\
18 & Making Plans in the RW & 2.32 & 3.20 & 2.27 & 1.48
\\
19 & Gossiping & 2.03 & 2.52 & 2.00 & 1.57
\\
20 & Flirting & 1.63 & 1.79 & 1.56 & 1.55
\\ 
\midrule
& Mean Frequency & 3.02 & 3.48 & 3.02 & 2.56
\\
\bottomrule
\end{tabular}
\source{own elaboration.}
\end{table}

In addition, as it can be observed at the bottom of \Cref{figure2}, MMORPG players tend to interact more with players they know, firstly from the real world (3.48) and secondly from virtual ones (3.02), than with those who have never met before (2.56). This makes a percent variation of 15.23\% between the first and the second groups, 35.94\% between the first and the third groups, and 17.97\% between the second and the third groups. It shall also be noticed that this fact happens in all the acts analyzed. \Cref{figure2} shows the results obtained when enquiring our participants on the use of the target particular social-communicative acts. 

Next, this research continues by introducing the results obtained from the analysis on the effects of the selected variables in each of the different social-communicative acts shown in \Cref{figure2} and aims at determining their frequency according to the influence of the six variables. The following tables (\ref{figure3}, \ref{figure4}, \ref{figure5}, \ref{figure6}, \ref{figure7}) show the influence of the six variables on the social-communicative acts using a 1-5 \emph{Likert} Scale.

\begin{table}[htpb]
\caption{Variable of Age}
\label{figure2}
\centering
\begin{tabular}{lllllllllll}
\toprule
A. & 1 & 2 & 3 & 4 & 5 & 6 & 7 & 8 & 9 & 10
\\ 
\midrule
V1.1 & 3.91 & 3.72 & 3.43 & 3.46 & 3.53 & 3.36 & 3.08 & 3.12 & 2.86 & 2.91
\\
V1.2 & 3.98 & 4.10 & 3.70 & 3.66 & 3.54 & 3.58 & 3.71 & 3.42 & 3.40 & 3.08
\\
V1.3 & 3.96 & 3.91 & 3.76 & 3.42 & 3.33 & 3.47 & 3.54 & 3.41 & 3.24 & 3.39
\\
\midrule
A. & 11 & 12 & 13 & 14 & 15 & 16 & 17 & 18 & 19 & 20
\\
\midrule
V1.1 & 2.78 & 3.05 & 2.77 & 2.59 & 2.63 & 2.63 & 2.58 & 2.19 & 1.88 & 1.58
\\
V1.2 & 3.09 & 3.30 & 2.78 & 2.82 & 2.70 & 2.73 & 2.46 & 2.38 & 2.19 & 1.67
\\
V1.3 & 3.07 & 3.67 & 2.85 & 2.71 & 2.64 & 2.54 & 2.71 & 2.30 & 1.97 & 1.57
\\ 
\midrule
\multicolumn{11}{c}{Variables: V1.1. Age: <20; V1.2. Age: 20-30; V1.3. Age: >30.}
\\
\bottomrule
\end{tabular}
\source{own elaboration.}
\end{table}

\begin{table}[htpb]
\caption{Variable of Personality}
\label{figure3}
\centering
\begin{tabular}{lllllllllll}
\toprule
A. & 1 & 2 & 3 & 4 & 5 & 6 & 7 & 8 & 9 & 10
\\ 
\midrule
V2.1 & 3.71 & 3.80 & 3.51 & 3.2 & 3.31 & 3.40 & 3.35 & 3.08 & 3.30 & 2.85
\\
V2.2 & 4.14 & 4.05 & 3.63 & 3.60 & 3.61 & 3.57 & 3.46 & 3.37 & 3.21 & 3.07
\\
V2.3 & 4.28 & 4.26 & 3.84 & 3.85 & 3.80 & 3.65 & 3.70 & 3.66 & 3.30 & 3.57
\\
\midrule
A. & 11 & 12 & 13 & 14 & 15 & 16 & 17 & 18 & 19 & 20
\\
\midrule
V2.1 & 2.99 & 3.04 & 2.67 & 2.67 & 2.47 & 2.54 & 2.29 & 2.30 & 1.93 & 1.64
\\
V2.2 & 2.93 & 3.26 & 2.87 & 2.57 & 2.60 & 2.69 & 2.67 & 2.23 & 1.97 & 1.65
\\
V2.3 & 3.15 & 3.42 & 2.79 & 2.93 & 2.85 & 2.74 & 2.94 & 2.27 & 2.24 & 1.53
\\ 
\midrule
\multicolumn{11}{c}{V2.1. Personality: Introvert; V2.2. Personality: Neutral; V2.3. Personality: Extrovert.}
\\
\bottomrule
\end{tabular}
\source{own elaboration.}
\centering
\end{table}

\begin{table}[htpb]
\caption{Variable of Experience}
\label{figure4}
\centering
\begin{tabular}{lllllllllll}
\toprule
A. & 1 & 2 & 3 & 4 & 5 & 6 & 7 & 8 & 9 & 10
\\ 
\midrule
V3.1 & 3.62 & 3.31 & 2.98 & 3.06 & 3.14 & 3.01 & 2.81 & 2.85 & 2.64 & 2.57 
\\
V3.2 & 4.14 & 4.05 & 3.63 & 3.60 & 3.61 & 3.57 & 3.46 & 3.37 & 3.21 & 3.07
\\
V3.3 & 4.15 & 4.44 & 4.23 & 3.99 & 3.79 & 3.94 & 4.04 & 3.72 & 3.70 & 3.6
\\
\midrule
A. & 11 & 12 & 13 & 14 & 15 & 16 & 17 & 18 & 19 & 20
\\
\midrule
V3.1 & 2.52 & 3.03 & 2.37 & 2.47 & 2.32 & 2.39 & 2.34 & 1.98 & 1.79 & 1.43
\\
V3.2 & 2.93 & 3.26 & 2.87 & 2.57 & 2.60 & 2.69 & 2.67 & 2.23 & 1.97 & 1.65
\\
V.3.3 & 3.49 & 3.60 & 3.14 & 3.09 & 3.08 & 2.90 & 2.70 & 2.69 & 2.29 & 1.82
\\ 
\midrule
\multicolumn{11}{c}{V3.1. Experience: Novice; V3.2. Experience: Intermediate; V3.3. Experience: Master.}
\\
\bottomrule
\end{tabular}
\source{own elaboration.}
\end{table}

\begin{table}[htpb]
\caption{Variable of Dedication}
\label{figure5}
\centering
\begin{tabular}{lllllllllll}
\toprule
A. & 1 & 2 & 3 & 4 & 5 & 6 & 7 & 8 & 9 & 10
\\ 
\midrule
V4.1 & 3.77 & 3.45 & 3.16 & 3.23 & 3.16 & 3.11 & 2.96 & 3.02 & 2.74 & 2.83
\\
V4.2 & 4.19 & 4.35 & 3.98 & 3.73 & 3.72 & 3.70 & 3.98 & 3.52 & 3.43 & 3.26
\\
V.4.3 & 4.09 & 4.46 & 4.15 & 3.97 & 3.90 & 4.10 & 3.84 & 3.68 & 3.81 & 3.38
\\
\midrule
A. & 11 & 12 & 13 & 14 & 15 & 16 & 17 & 18 & 19 & 20
\\
\midrule
V4.1 & 2.62 & 3.00 & 2.59 & 2.45 & 2.36 & 2.48 & 2.53 & 2.13 & 1.89 & 1.64
\\
V4.2 & 3.29 & 3.45 & 2.88 & 2.89 & 2.84 & 2.71 & 2.65 & 2.34 & 2.10 & 1.57
\\
V4.3 & 3.40 & 3.71 & 3.07 & 3.09 & 3.15 & 2.89 & 2.43 & 2.60 & 2.20 & 1.68
\\ 
\midrule
\multicolumn{11}{c}{V4.1. Dedication: Low; V4.2. Dedication: Intermediate; V4.3. Dedication: High.}
\\
\bottomrule
\end{tabular}
\source{own elaboration.}
\end{table}

\begin{table}[htpb]
\caption{Variable of Interaction}
\label{figure6}
\centering
\begin{tabular}{lllllllllll}
\toprule
A. & 1 & 2 & 3 & 4 & 5 & 6 & 7 & 8 & 9 & 10
\\ 
\midrule
V5.1 & 3.65 & 3.49 & 3.36 & 3.34 & 3.27 & 3.22 & 2.99 & 2.99 & 2.86 & 2.68
\\
V5.2 & 3.73 & 3.58 & 3.10 & 3.18 & 3.34 & 3.12 & 2.94 & 2.94 & 2.64 & 2.88
\\
V5.3 & 4.30 & 4.47 & 4.16 & 3.96 & 3.78 & 3.97 & 4.10 & 3.79 & 3.79 & 3.51
\\
\midrule
A. & 11 & 12 & 13 & 14 & 15 & 16 & 17 & 18 & 19 & 20
\\
\midrule
V5.1 & 2.73 & 2.93 & 2.62 & 2.40 & 2.47 & 2.58 & 2.40 & 2.18 & 1.95 & 1.57
\\
V5.2 & 2.67 & 2.9 & 2.46 & 2.54 & 2.45 & 2.45 & 2.55 & 2.07 & 1.92 & 1.60
\\
V5.3 & 3.4 & 3.84 & 3.14 & 3.09 & 3.01 & 2.86 & 2.66 & 2.59 & 2.16 & 1.69
\\ 
\midrule
\multicolumn{11}{c}{V5.1. Interactions: Low; V5.2. Interactions: Intermediate; V5.3. Interactions: High.}
\\
\bottomrule
\end{tabular}
\source{own elaboration.}
%\centering
%\notes{V5.1. Interactions: Low; V5.2. Interactions: Intermediate; V5.3. Interactions: High.}
\end{table}

\begin{table}[htpb]
\caption{Variable of Languages Spoken}
\label{figure7}
\centering
\begin{tabular}{lllllllllll}
\toprule
A. & 1 & 2 & 3 & 4 & 5 & 6 & 7 & 8 & 9 & 10
\\ 
\midrule
V6.1 & 4.03 & 4.28 & 3.99 & 3.72 & 3.93 & 3.89 & 3.90 & 3.27 & 3.49 & 2.95
\\
V6.2 & 3.78 & 3.79 & 3.50 & 3.47 & 3.48 & 3.48 & 3.40 & 3.26 & 3.12 & 3.00
\\
V6.3 & 4.06 & 3.95 & 3.62 & 3.57 & 3.43 & 3.43 & 3.37 & 3.37 & 3.16 & 3.19
\\
\midrule
A. & 11 & 12 & 13 & 14 & 15 & 16 & 17 & 18 & 19 & 20
\\
\midrule
V6.1 & 3.47 & 3.38 & 2.89 & 2.58 & 2.71 & 2.65 & 2.74 & 2.54 & 2.46 & 2.04
\\
V6.2 & 2.93 & 3.17 & 2.77 & 2.80 & 2.68 & 2.70 & 2.49 & 2.36 & 1.93 & 1.69
\\
V6.3 & 2.93 & 3.39 & 2.78 & 2.71 & 2.70 & 2.63 & 2.56 & 2.23 & 1.99 & 1.49
\\ 
\midrule
\multicolumn{11}{c}{V6.1. Languages Spoken: 1; V6.2. Languages Spoken: 2; V6.3. Languages Spoken: 3.}
\\
\bottomrule
\end{tabular}
\source{own elaboration.}
%\notes{V6.1. Languages Spoken: 1; V6.2. Languages Spoken: 2; V6.3. Languages Spoken: 3.}
\end{table}

\begin{enumerate}
\item \emph{Saying Goodbye}. This social activity is the most popular among the ones analyzed. However, results show that its use may vary depending on the player. In this sense, \emph{saying goodbye} is more common among those participants with higher levels of interactions (4.30) rather than those with intermediate (3.73) or lower levels (3.65); the percent variations are 15.28\% and 17.81\% respectively. This difference is similar among players whose personality is extrovert (4.28) and neutral (4.14) in comparison to introvert ones (3.71); in this case, the variations are 15.36\% and 11.59\%. There is also a significant difference among players whose experience is mastered (4.15) or intermediate (4.14) in comparison to more novice players (3.62); the variations are 14.36\% and 14.69\% respectively. In the remaining cases, there are no relevant variations or clear connections among them.
\item \emph{Thanking}. The second most popular act in the list was thanking; however, results have shown that the participants who were more experienced in MMORPG (4.44), had a higher degree of interaction with other players (4.47), or spent more time playing (4.46) were more likely to use this communicative act in comparison to novice players (3.31), those who interacted less often (3.49), or those with fewer weekly hours of gaming (3.45). The percent variations were 34.14\% (experience), 29.28\% (dedication) and 28.08\% (interaction). Less significant was the variable of age, whose participants aged 20-30 (4.10) said thanks more often than participants older than 30 (3.91) and under 20 (3.72). Their variations were 4.86\% and 10.22\% respectively. 
\item \emph{Congratulating}. The following action in the list is congratulating; and, according to our results, it is more frequently used by MMORPG masters (4.23), those with a high degree of interactions (4.16), or the ones who spend more hours playing these video games (4.15). Its use was the most significant and the percent variation changed up to 41.95\% when comparing the degrees of expertise, 34.19\% on interactions, and 31.33\% on dedication. Personality was also relevant, but the variation between extrovert (3.84) and introvert (3.51) players was only 9.40\%, 
\item \emph{Joking}. This communicative act is the fourth most commonly used. Regarding the percent variation when introducing variables, results have shown that experienced users joke more often (3.99) than less experienced (3.06), the variation is 30.39\%. If the players interact more often with other players, then they are also more likely to joke (3.94) than if they are considered intermediate (3.18) or low (3.34); in this sense, the difference can be up to 22.91\%. Players who devote more time to play are also more likely to joke (3.97) than those who play fewer hours (3.23); in this case, the difference is 22.91\%. At last, extroversion can also contribute to a higher or lower degree in the use of this act. The most introvert (3.85) joke more often than the least ones (3.20), being the variation 20.31%. 
\item \emph{Sympathizing}. This act is more likely to happen among players with a higher degree of dedication (3.90) and experience (3.79) than with those who spend less time playing (3.16) or are a novice (3.14). The percent variations of these items are 23.42\% and 20.70\% respectively. On a secondary level, players with a higher degree of interaction (3.78) or extrovert ones (3.80) used to sympathize with other players more often than those with a lower degree of interaction (3.27) or extroversion (3.31). In this case, the percent variations were 15.60\% and 14.80\% respectively. In this sense, it is also significant that players speaking at least one language with at least an intermediate level (3.93) sympathized more often than those who spoke two (3.43) or three languages (3.43). The percent variations were 12.93\% and 14.58\%. In addition, players younger than 30 (<20: 3.53; 20-30: 3.54) showed that they sympathize more often than those over 30 (3.33). The difference between them was up to 6.31%. 
\item \emph{Showing Understanding}. Results have revealed that there are differences when the higher degrees of dedication (4.1), experience (3.94), or interaction (3.97) are compared with lower ones (dedication: 3.11; experience: 3.01; interaction: 3.22). In this sense, the percent variations are 31.83\%, 30.90\%, and 23.29\% respectively. In this case, the fewer languages spoken (3.89) is also connected to more frequent use of this act if these participants are compared with those speaking two (3.48) or three or more languages with at least an intermediate level (3.43). The variation among participants speaking one and two languages is 11.78\% and among those speaking one and three languages is 13.41\%. At last, extroversion and age do not seem to interfere as much as the other variables. The percent variations between the highest and lowest degrees of each item are 7.35\% and 6.55\% respectively. 
\item \emph{Apologizing}. The following item in the list is ‘apologizing’, and experience seems to be fundamental to understand its use; the more experienced (4.04) are 43.77\% more frequent than novice ones (2.81). Similarly, this act happens more often among players with higher degrees of interaction (4.1) than among players with lower (2.99) or intermediate degrees (2.94); the variation may be up to 39.46\%. It is also significant the difference among players with intermediate (3.98) or high degrees (3.83) of dedication to gaming and low ones (2.96). In this sense, the main percent variation is 34.46\%. At last, other variables are less relevant but no less important; they are age, extroversion, and languages spoken. Concerning age, the groups of 20-30-year-old (3.71) and older than 30 (3.58) were more likely to apologize than younger ones (3.08), being the maximum difference of 20.45\%. In reference to languages spoken with at least an intermediate level, the ones with fewer languages (3.90) were also more likely to apologize than the ones with more languages (two languages: 3.40; 3 three languages: 3.37), whose maximum percent variation was 15.73\%. At last, the percent variation between extrovert (3.70) and introvert (3.35) was 10.45%.
\item \emph{Greeting}. Similar to the previous acts, the degree of interactions, experience, dedication, and personality seems to be the most relevant variables. In this case, players with more experience (3.72) stated that they use this act 30.49\% more often than those with little experience (2.85). Concerning dedication, those with a higher degree (3.68) were more likely to use greetings than those with a low degree (3.01); the percent variation was 28.77\%. When it comes to the degree of interaction, players who tend to interact more often (3.79) acknowledged that they greet more often than those with a low (2.99) and intermediate degree (2.94), being the maximum percent variation of 28.77\%. The degree of extroversion was also a relevant variable since the percentage of variation between extrovert (3.66) and introvert players (3.08) was 18.83\%. On the contrary, it seems that the variations between the degrees of age and language spoken were less significant, 9.51\% and 3.58\% respectively. 
\item \emph{Complimenting}. Concerning the variables that most influenced were interaction, experience, and dedication. In this sense, players with a higher degree (interaction: 3.79; experience: 3.7; dedication: 3.81) happened more often than those with a low degree (interaction: 2.86; experience: 2.64; dedication: 2.74), as result their percent variation seemed to be significant (interaction: 43.56\%; experience: 40.15\%; dedication: 39.05\%). On a secondary level, players aged 20-30 (3.40) were more likely to compliment than players older than 30 (3.24) and younger than 20 (2.86), establishing a percent variation of 4.94\% and 18.88\% respectively. Similarly, the number of languages spoken meant a maximum percent variation of 10.44\% between those who spoke one language (3.49) and three or more languages (3.16). At last, the degree of extroversion had little relevance on possible variations (2.80\%).
\item \emph{Introducing}. This item has been ranked in the tenth position of our classification and it is more commonly used by players who are considered experts (3.6), extrovert (3.57), interact more often (3.51), have a high degree of dedication to playing MMORPG (3.38), the oldest (3.39), or can speak three or more languages (3.19). In this sense, the opposite happens when the players involved were previously tagged as a novice (2.57), introvert (2.85), interact less often (2.68), have a low degree of dedication to playing MMORPG (2.83), the youngest (2.91), or can speak only one language (2.95). Thus, results on percent variation are the following comparing the highest and the lowest degree of each item: experience (40.08\%), interaction (30.97\%), extroversion (25.26\%), dedication (19.43\%), age (16.49\%), languages spoken (8.14\%). 
\item \emph{Praising}. In this case, the degrees of experience, dedication, or interaction were the most significant variables. In the first case, players with more experience (3.49) were praised more than those more novice (2.52); as result, the percent variation was 38.49\%. Concerning dedication, players with a higher degree (3.40) seem to praise more frequently than those with a lower degree (2.62); the percent variation was fixed at 29.77\%. Regarding interaction, players who considered that they interact with other players more often (3.40) used this act more often than players who interact less (2.73) or were tagged as intermediate (2.67); the percent variations were 24.54\% and 27.34\% respectively. Relevant could also be considered the fact that players with fewer languages (3.47) used this act more often than those with two or more (2.93), the variations were 18.43\%. In this case, the variables of age (11.15\%) and extroversion (5.35\%) seemed to be less significant.
\item \emph{Introducing others}. This act was ranked twelfth in our classification. The most relevant variable was the degree of interaction; in this sense, players who interacted with other players more often were more likely to introduce their companions to other players (3.71) than those who showed a lower degree of interaction, either intermediate (3.45) or low (3.00); the variations were 31.06\% and 32.41\% respectively. This variation was followed by dedication; in this case, players who spend more hours playing MMORPG seem to be more willing (3.71) to introduce other players than the ones who play shorter (3.00); the variation is 23.67\%. In the case of experience, this act also occurred more often with master players (3.60) than with novice ones (3.06), being the variation 18.81\%. Similarly, older players (3.67) also seem to use this activity more often than younger ones (3.05), variation: 20.33\%. Concerning extroversion, introvert players also tend to use this activity more often (3.42) than extrovert ones (3.04); the variation is 12.50\%. At last, it seems there was not a clear correlation among the number of languages spoken since players speaking one (3.38) or three (3.39) languages used this act more often than those speaking two languages (3.17). 
\item \emph{Making Plans in the Virtual World}. Following the results obtained, experience is the most significant variable; in this case, expert players tend to be more likely to make plans in the virtual world (3.14) than novice players (2.37), being the variation 32.90\%. Next, players with a high degree of interaction tend to use this activity more often (3.14) than those with low (2.62) or intermediate degrees (2.46); the percent variations are 19.85\% and 27.64\% respectively. Concerning players' dedication, those who devoted more hours to the game are more likely to use this act (3.07) than those who spend less time playing (2.59), the variation is 18.53\%. Regarding extroversion, this act occurred more often among players with a neutral personality (2.87), than those with extrovert (2.79) or introvert ones (2.67), being the variations 2.87\% and 7.49\% respectively. The case of language and age seems to be irrelevant since they both were below 5\%. 
\item \emph{Interrupting politely}. The following action in our list concerns interruptions and results suggest that interaction, dedication, and experience are the most significant variables. In this sense, the players tagged as the ones interacting the most seem to use this act more (3.09) than those tagged as the least (2.40), the percent variation is 28.75\%. Similarly, those who devote more hours to the game (3.09) are also more likely to use this act than those who play fewer hours (2.45), the percent variation of these two groups is 26.12\%. In addition, those who are more experience also interrupt politely with more frequency (3.09) than novice players (2.47), the variation is 25.10\%. In the other three variables, there is no correlation with the variables, but results have shown that the group aged 20-30 use this activity more often (2.82) than those over 30 (2.71) or under 20 (2.59). Players tagged as extroverts also seem to use this activity with a higher frequency (2.93) than those tagged as an introvert (2.67) or neutral (2.57). At last, players speaking 2 languages tend to be the ones who interrupt politely more often (2.80) in comparison with the other two groups. 
\item \emph{Taking time to consider}. This act seems to occur more often when it comes to players with more experience and with those spending more time playing. Taking time to consider happens more frequently when players are experts (3.08) rather than novices (2.32); in this case, the variable is 32.76\%. Similarly, players spend more time playing MMORPG the use of this act is more likely to occur (3.15) than with players who devote fewer hours (2.36), the percent variation is 33.47\%. Concerning interaction, results show that those who interact more often (3.01) also take more time to consider in comparison to players who interact less (2.45) or are tagged as intermediate (2.47), the variations are 21.86\% and 22.86\% respectively. In this case, language and age do not seem to interfere significantly. 
\item \emph{Inviting others to participate in actions}. Players with more experience (2.90) are more likely to use this act than novice ones (2.39), the variation is 21.34\%. The results concerning the variables of dedication and interaction are similar; in this case, players with a higher degree of dedication make use of this act (2.89) more often than those with a lower degree (2.48); the same happens with players with a higher degree of interaction (2.86) in comparison to those who interact less (2.58) or neutral ones (2.45). The variables are 16.53\% for dedication, and 10.85\% and 16.73\% respectively for interaction. Considering extroversion, extrovert players tend to use this activity more often (2.74) than introvert ones (2.54), being the variation 7.87\%. Regarding age, the group 20-30 were more likely to use this act (2.73) than those aged under 20 (2.63) or over 30 (2.54), the variations are 3.80\% and 7.48\%. 
\item \emph{Using Small Talks}. This act is ranked in the position seventeenth. Results have revealed that the most significant variable, in this case, was extroversion; the most extrovert players (2.94) tend to use small talks more often than introvert ones (2.29), being the variable 28.38\%. The following relevant variable was experience; master players (2.70) seem to be more likely to use this act in comparison to novice ones (2.34), the variation is 15.38\%. The degree of interactions was slightly significant; in this case, players who tend to interact more often seem to use this act more (2.66) than those who interact less (2.40), the variation is 10.83\%. Concerning languages spoken and age, it does not seem to be a clear correlation; results have shown that over 30-year-old participants were more likely to use small talks (2.71) than those aged under 20 (2.58) or those aged 20-30 (2.46); whereas in the case of languages spoken, results suggest that those speaking one language (2.74) were more likely to use this act than those speaking two (2.56) or three or more (2.49). At last, dedication to gaming has not clearly influenced the use of this act, results show that those with an intermediate degree of dedication were more likely to use this act (2.65) than those with low (2.53) or high degrees (2.43).
\item \emph{Making Plans in the Real World}. In this case, the most experimented players were the ones who seem to tend to make plans with virtual players in the real world more often (2.69) in comparison to novice players (1.98), being the variation 35.86\%. The degrees of interaction and dedication were also significant. On the one hand, players who affirmed to interact more often (2.59) were more likely to use this act than those with a low (2.18) or intermediate degree (2.07), variations: 18.81\% and 25.12\%. On the other hand, results also showed that the ones who play MMORPG more often (2.60) use this act more than the ones who play less (2.13), variation: 22.07\%. On a secondary level, results showed that players speaking one language tend to make plans more often (2.54) than the ones speaking two (2.36) or three or more languages (2.23); the variations were 5.83\% and 13.90\% respectively. Differences in terms of age seemed to be irrelevant and concerning extroversion near inexistent. 
\item \emph{Gossiping}. Results have revealed that this act is quite uncommon among MMORPG players. Results have also shown that experience is the most significant variable, in which expertise players use it more often (2.29) than novice ones (1.79), being the variation 27.93\%. Similarly, players speaking one language (2.46) gossiped more frequently than those speaking two languages (1.93) or three or more (1.99), being the variations 27.46\% and 23.62\%. On a secondary level, other variables that also seemed to be relevant were dedication, extroversion, and interaction. In this sense, players who spend more hours playing acknowledged that they gossip more often (2.20) than the ones playing fewer hours (1.89), being the variation 16.40\%. Concerning extroversion, extrovert players (2.24) also use this activity with higher frequency than introvert ones (1.93), percent variation: 16.06\%. Group ages also were significant, and differences are noticeable among the three different groups; participants aged 20-30 were the ones who used this act more often (2.19), followed by those over 30 (1.97) and those under 20 (1.88), being the variations 11.17\% and 16.49\%. At last, players who interact often seem to gossip more (2.16) than the ones with an intermediated degree of interaction (1.92) and a low one (1.95), percent variations: 12.50\% and 10.77\% respectively.
\item \emph{Flirting}. At last, this act seems to be the least used among MMORPG players. In this case, the most significant variable was the number of languages spoken; participants speaking one language flirted more often (2.04) than those speaking two (1.69) or three or more (1.49). The percent variations were 20.71\% and 36.91\% respectively. The other significant variable was experience; results have shown that master players use this act more frequently (1.82) than novice ones (1.43); being the variation 27.27\%. The other variations were below 10\% and consequently less significant. As opposed to the tendency of this research, it should be remarkable that neutral (1.65) and introvert (1.64) players flirted more often than extrovert ones (1.53), being the variations 7.84\% and 7.19\% respectively. 
\end{enumerate}

To finish this section, the following table (\ref{figure8}) shows the mean percent variation among the groups tagged for the analysis of the players' social-communicative acts to determine the most significant variables in this research. To accomplish this, it has been necessary to calculate the percent variation between the group with the highest score and the one with the lowest within the same category. Then, the group with the highest score has also been compared with the second highest; and the second-highest with the lowest as well. Results suggest that the most significant variables have been experience (30.04\%), interaction (24.00\%), and dedication (22.64\%). The ones which seem to be less significant are extroversion (11.58\%), languages spoken (8.70\%), and age (7.24\%). 

\begin{table}[htpb]
\caption{Percent variation among tagged groups.}
\label{figure8}
\centering
\begin{tabular}{p{0.2\textwidth}p{0.1\textwidth}p{0.1\textwidth}p{0.1\textwidth}p{0.08\textwidth}p{0.08\textwidth}p{0.08\textwidth}}
\toprule
& High Deg. (3) & Neutral Deg. (2) & Low Deg. (1) & Dif. 1st-3rd & Dif. 1st-2nd & Dif. 2nd-3rd
\\ 
\midrule
V1. Age & 3.07** & 3.11* & 2.90*** & 7.24\% & 1.30\% & 5.86\%
\\
V2. Extroversion & 3.24* & 3.06** & 2.90*** & 11.72\% & 5.52\% & 5.88\% 
\\
V3. Experience & 3.42* & 3.06** & 2.63*** & 30.04\% & 16.35\% & 11.76\%
\\
V4. Dedication & 3.38* & 3.23** & 2.76*** & 22.46\% & 17.03\% & 4.64\%
\\
V5. Interaction & 3.41* & 2.75*** & 2.78** & 24.00\% & 22.66\% & 1.09\%
\\
V6. Languages & 3.03** & 2.99*** & 3.25* & 8.70\% & 7.26\% & 1.34\%
\\
\bottomrule
\end{tabular}
\vspace{1ex}
\notes{*FIRST IN USE; **SECOND IN USE; ***THIRD IN USE}
\source{own elaboration.}
\end{table}


\section{Discussion}
This research has focused on analyzing the use of social-communicative acts among 201 MMORPG Players who participated in our experiment. As previously stated, this research has aimed at explaining how a series of variables interfere in the socializing process among MMORPG players. In this sense, six variables and three contexts have been considered to explain the use of 20 communicative acts.

Our six variables were age, extroversion (personality), experience and current dedication at playing MMORPG, degree of interactions with other players, and languages spoken; whereas the three contexts were interactions between players who know each other from the real world, between players who know each other from the virtual world, and between players who know each other neither from the real nor the virtual worlds. Thus, considering the results obtained from the analysis of the communicative acts by taking into account these six variables and the three possible scenarios, the main findings of this research should be discussed.

To start with, our demographic research has revealed some data that suggest a possible profile of the usual MMORPG players. In this sense, male and introverted participants aged 20-30 were the most experienced at playing MMORPG and the ones who devoted more weekly hours to the game as well. The mean of languages spoken among participants was approximately 3, of which 1 was the player’s mother tongue and the other 2 were foreign languages whose level was at least intermediate (B). In this case, the most popular foreign language spoken was English, which could be considered the lingua franca of this genre of video games. In addition, results have also shown that extrovert participants could speak more languages (2.81) than introvert ones (2.12), and older player and female participants also seemed to be slightly more skillful at speaking foreign languages in comparison to men and younger participants. 

Concerning the most common acts, \Cref{figure1} ranked the communicative social acts that were analyzed in this research. As it can be observed, they can vary depending on the context where they occur. In this sense, players’ attitudes and behavior usually change when they know the other players from either the real or the virtual worlds or if they are unknown. MMORPG seems to be a social place where there are continuous interactions among players with different profiles; and, similarly to the real world, players can feel different degrees of affinity or attraction to other players according to several variables which should be considered in further research. 

Results in this research also suggest the possible social behavior of the players in MMORPG video games; thus, we can imagine that \emph{World of Warcraft} or similar games are not the most suitable place for flirting, gossiping, or trying to make new friends meet in the real world. Instead, it seems there are some non-written rules about behavior which involve politeness and some good manners among players in which ‘greeting’, ‘thanking’, ‘congratulating’, ‘apologizing’ and ‘sympathizing’, among others, happen very often. Thus, it seems that these non-written rules require that players behave tactfully and with a certain degree of modesty and sympathy towards the other players. In addition to this, ‘joking’ is another frequent social act in MMORPG; therefore, humor and fun among players seem to be usual attitudes in virtual worlds which could turn to mock tones. 

Next, concerning the variables analyzed, results suggest that that three of the six variables introduced seemed to be significant on the target communicative social acts in this research. In this sense, experience, interaction, and dedication have been the most significant variables, whereas extroversion, languages are spoken, and age the least ones. However, it shall be acknowledged that the influence of these variables can vary among the selected acts. 

This fact means that despite there is significant overall relevance of the three acts, the other three can also be significantly influential on concrete occasions. For example, results have shown that introverted people joke more often (20.31\%) than introverted ones, whereas it is the opposite when it comes to greeting (18.31\%). Similarly, players aged over 20 apologize more frequently (20.45\%) than those under that age; or participants speaking more languages seemed to flirt with higher frequency (36.91\%) than those speaking fewer. Thus, their use and percent variation should be assessed individually.

At last, results have also revealed that participants tend to speak foreign languages more often in the virtual world than in the real one. This fact suggests that the virtual world is a place where there are more opportunities to practice the use of foreign language entertainingly. On the one hand, students are exposed to the use of foreign language in a virtual context where they can experience problem-solving tasks and meaningful negotiation opportunities with other players. On the other hand, MMORPG are video games whose purpose is to entertain the player; thus, if this was applied to foreign language teaching, the result is that players are experimenting with the language in an entertaining environment. 

However, it shall also be acknowledged that MMORPG players interact more often with players they know from the real world than those they know from the virtual one or with players who have never met before. This should not be considered a limitation but a real challenge for teaching professionals who are willing to integrate foreign language learning and (MMORPG) video games. 

\section{Conclusion}
This paper has analyzed and discussed the frequency of the use of some social-communicative acts based on a series of variables. According to the results obtained, this research concludes that the selected variables have had a certain effect on the use of the social-communicative acts, despite some can be more influential than others. In this sense, this research has fulfilled the initial objective that consisted of determining the frequency of the aforementioned acts and then analyzing the influence of the variables in an attempt at explaining the behavior of MMORPG players in terms of socialization. 

The main limitation of this research was to find suitable participants who were willing to take our survey. The number of our students who participated was limited, whereas emailing students with the survey was little effective. On the other hand, posting messages on pages on social networks was not always welcomed by some administrators, who often reported our posts despite the survey was accompanied by a text which was written with a formal and polite tone, avoiding any commercial tone, and always inviting individuals to participate anonymously. 

In addition, this research could have included further details such as the influence of the three contexts of participants' familiarity with other players within the variables, but this data has not been used to avoid exceeding the word limit. In the future, this research could be extended by adding these or new variables (i.e.: gender), including more participants, or focusing exclusively on single acts, which could be further explained. Besides, this research could also be addressed for educational purposes to teach FL students how to use the most common social-communicative acts before using MMORPG as a pedagogical resource. 


\printbibliography\label{sec-bib}
% if the text is not in Portuguese, it might be necessary to use the code below instead to print the correct ABNT abbreviations [s.n.], [s.l.] 
%\begin{portuguese}
%\printbibliography[title={Bibliography}]
%\end{portuguese}

\end{document}
