% !TEX TS-program = XeLaTeX
% use the following command: 
% all document files must be coded in UTF-8
\documentclass{textolivre}
% for anonymous submission
%\documentclass[anonymous]{textolivre}
% to create HTML use 
%\documentclass{textolivre-html}
% See more information on the repository: https://github.com/leolca/textolivre

% Metadata
\begin{filecontents*}[overwrite]{article.xmpdata}
    \Title{TIC y ocio familiar durante el confinamiento: agentes involucrados}
    \Author{Maria Luisa Belmonte \sep Jose Santiago Alvarez Muñoz \sep Maria Angeles Hernandez-Prados}
    \Language{es}
    \Keywords{Familia \sep Nuevas tecnologías \sep Ocio \sep Confinamiento}
    \Journaltitle{Texto Livre}
    \Journalnumber{1983-3652}
    \Volume{14}
    \Issue{2}
    \Firstpage{1}
    \Lastpage{16}
    \Doi{10.35699/1983-3652.2021.33938}

    \setRGBcolorprofile{sRGB_IEC61966-2-1_black_scaled.icc}
            {sRGB_IEC61966-2-1_black_scaled}
            {sRGB IEC61966 v2.1 with black scaling}
            {http://www.color.org}
\end{filecontents*}

% used to create dummy text for the template file
\definecolor{dark-gray}{gray}{0.35} % color used to display dummy texts
\usepackage{lipsum}
\SetLipsumParListSurrounders{\colorlet{oldcolor}{.}\color{dark-gray}}{\color{oldcolor}}

% used here only to provide the XeLaTeX and BibTeX logos
\usepackage{hologo}

% used in this example to provide source code environment
%\crefname{lstlisting}{lista}{listas}
%\Crefname{lstlisting}{Lista}{Listas}
%\usepackage{listings}
%\renewcommand\lstlistingname{Lista}
%\lstset{language=bash,
        breaklines=true,
        basicstyle=\linespread{1}\small\ttfamily,
        numbers=none,xleftmargin=0.5cm,
        frame=none,
        framexleftmargin=0.5em,
        framexrightmargin=0.5em,
        showstringspaces=false,
        upquote=true,
        commentstyle=\color{gray},
        literate=%
           {á}{{\'a}}1 {é}{{\'e}}1 {í}{{\'i}}1 {ó}{{\'o}}1 {ú}{{\'u}}1 
           {à}{{\`a}}1 {è}{{\`e}}1 {ì}{{\`i}}1 {ò}{{\`o}}1 {ù}{{\`u}}1
           {ã}{{\~a}}1 {ẽ}{{\~e}}1 {ĩ}{{\~i}}1 {õ}{{\~o}}1 {ũ}{{\~u}}1
           {â}{{\^a}}1 {ê}{{\^e}}1 {î}{{\^i}}1 {ô}{{\^o}}1 {û}{{\^u}}1
           {ä}{{\"a}}1 {ë}{{\"e}}1 {ï}{{\"i}}1 {ö}{{\"o}}1 {ü}{{\"u}}1
           {Á}{{\'A}}1 {É}{{\'E}}1 {Í}{{\'I}}1 {Ó}{{\'O}}1 {Ú}{{\'U}}1
           {À}{{\`A}}1 {È}{{\`E}}1 {Ì}{{\`I}}1 {Ò}{{\`O}}1 {Ù}{{\`U}}1
           {Ã}{{\~A}}1 {Ẽ}{{\~E}}1 {Ũ}{{\~u}}1 {Õ}{{\~O}}1 {Ũ}{{\~U}}1
           {Â}{{\^A}}1 {Ê}{{\^E}}1 {Î}{{\^I}}1 {Ô}{{\^O}}1 {Û}{{\^U}}1
           {Ä}{{\"A}}1 {Ë}{{\"E}}1 {Ï}{{\"I}}1 {Ö}{{\"O}}1 {Ü}{{\"U}}1
           {ç}{{\c{c}}}1 {Ç}{{\c{C}}}1
}


\journalname{Texto Livre}
\thevolume{14}
\thenumber{2}
\theyear{2021}
\receiveddate{\DTMdisplaydate{2020}{12}{11}{-1}} % YYYY MM DD
\accepteddate{\DTMdisplaydate{2021}{02}{28}{-1}}
\publisheddate{\today}
% Corresponding author
\corrauthor{Maria Luisa Belmonte}
% DOI
\articledoi{10.35699/1983-3652.2021.33938}
% Abbreviated author list for the running footer
\runningauthor{Belmonte et al.}
\editorname{Anna Izabella M. Pereira}

\title{TIC y ocio familiar durante el confinamiento: agentes involucrados}
\othertitle{TIC e lazer familiar durante o confinamento: agentes envolvidos}
\othertitle{ITC and family leisure during confinement: agents involved}
% if there is a third language title, add here:
%\othertitle{Artikelvorlage zur Einreichung beim Texto Livre Journal}

\author[1]{Maria Luisa Belmonte \orcid{0000-0002-1475-3690} \thanks{Email: \url{marialuisa.belmonte@um.es}}}
\author[1]{Jose Santiago Alvarez Muñoz \orcid{0000-0002-9740-6175} \thanks{Email: \url{josesantiago.alvarez@um.es}}}
\author[2]{Maria Angeles Hernandez-Prados \orcid{0000-0002-3617-215X} \thanks{Email: \url{mangeles@um.es}}}

\affil[1]{Universidad de Murcia, Facultad de Educación, Departamento de Métodos de Investigación y Diagnóstico en Educación, Murcia, España.}
\affil[2]{Universidad de Murcia, Facultad de Educación, Departamento de Teoría e Historia de la Educación, Murcia, España.}

\addbibresource{article.bib}
% use biber instead of bibtex
% $ biber tl-article-template

% set language of the article
\setdefaultlanguage{spanish}
\setotherlanguage{portuguese}
\setotherlanguage{english}

% for spanish, use:
%\setdefaultlanguage{spanish}
\gappto\captionsspanish{\renewcommand{\tablename}{Tabla}} % use 'Tabla' instead of 'Cuadro'
\AfterEndPreamble{\crefname{table}{tabla}{tablas}\Crefname{table}{Tabla}{Tablas}}

% for languages that use special fonts, you must provide the typeface that will be used
% \setotherlanguage{arabic}
% \newfontfamily\arabicfont[Script=Arabic]{Amiri}
% \newfontfamily\arabicfontsf[Script=Arabic]{Amiri}
% \newfontfamily\arabicfonttt[Script=Arabic]{Amiri}
%
% in the article, to add arabic text use: \textlang{arabic}{ ... }

% to use emoticons in your manuscript
% https://stackoverflow.com/questions/190145/how-to-insert-emoticons-in-latex/57076064
% using font Symbola, which has full support
% the font may be downloaded at:
% https://dn-works.com/ufas/
% add to preamble:
% \newfontfamily\Symbola{Symbola}
% in the text use:
% {\Symbola }

% reference itens in a descriptive list using their labels instead of numbers
% insert the code below in the preambule:
\makeatletter
\let\orgdescriptionlabel\descriptionlabel
\renewcommand*{\descriptionlabel}[1]{%
  \let\orglabel\label
  \let\label\@gobble
  \phantomsection
  \edef\@currentlabel{#1\unskip}%
  \let\label\orglabel
  \orgdescriptionlabel{#1}%
}
\makeatother
%
% in your document, use as illustraded here:
%\begin{description}
%  \item[first\label{itm1}] this is only an example;
%  % ...  add more items
%\end{description}
 

% custom epigraph - BEGIN 
%%% https://tex.stackexchange.com/questions/193178/specific-epigraph-style
\usepackage{epigraph}
\renewcommand\textflush{flushright}
\makeatletter
\newlength\epitextskip
\pretocmd{\@epitext}{\em}{}{}
\apptocmd{\@epitext}{\em}{}{}
\patchcmd{\epigraph}{\@epitext{#1}\\}{\@epitext{#1}\\[\epitextskip]}{}{}
\makeatother
\setlength\epigraphrule{0pt}
\setlength\epitextskip{0.5ex}
\setlength\epigraphwidth{.7\textwidth}
% custom epigraph - END


% if you use multirows in a table, include the multirow package
\usepackage{multirow}

% add line numbers for submission
%\usepackage{lineno}
%\linenumbers

\begin{document}
\maketitle

\begin{polyabstract}
\begin{abstract}
Con el propósito de esclarecer el papel de diferentes agentes en relación con el ocio familiar desde la perspectiva de padres e hijos, se llevó a cabo el presente estudio en una muestra total de 763 padres y 286 hijos residentes en el territorio español, los cuales cumplimentaron un cuestionario tipo Likert de 1 a 5 valorando el rol desarrollado por los siguientes agentes: nuevas tecnologías, escuela, administración pública, asociaciones, medios de comunicación y la propia familia, además de indicar la edad y el número de miembros en el núcleo familiar. Los resultados muestran que los padres de menor edad tienen una visión más optimista acerca de los agentes y, especialmente, respecto a las nuevas tecnologías, a diferencia de aquellos padres de mayor edad que se muestras escépticos al respecto. En el caso de los hijos, peor percepción tienen conforme mayor es su edad, siendo una visión más positiva la de los adolescentes. Las viviendas donde hay más miembros familiares aprecian el papel desarrollado por los agentes y contemplan las nuevas tecnologías como un recurso de apoyo útil para el ocio familiar. De esta manera, se ha de potenciar políticas e intervenciones de reeducación que destierre la visión única de estos medios como ente de ruptura entre los miembros familiares. 

\keywords{Familia \sep Nuevas tecnologías \sep Ocio \sep Confinamiento}
\end{abstract}

\begin{portuguese}
\begin{abstract}
A fim de esclarecer o papel dos diferentes agentes em relação ao lazer familiar a partir da perspectiva dos pais e das crianças, este estudo foi realizado em uma amostra total de 763 pais e 286 crianças residentes na Espanha, que preencheram um questionário do tipo Likert-de 1 a 5, avaliando o papel desempenhado pelos seguintes agentes: novas tecnologias, escola, administração pública, associações, mídia e a própria família, além de indicar a idade e o número de membros da família. Os resultados mostram que os pais mais jovens têm uma visão mais otimista dos agentes e, especialmente, das novas tecnologias, em contraste com os pais mais velhos, que são céticos em relação a eles. No caso das crianças, sua percepção é pior à medida que envelhecem, com uma visão mais positiva entre os adolescentes. Os moradores com mais membros da família apreciam o papel desempenhado pelos agentes e veem as novas tecnologias como um recurso de apoio útil para o lazer familiar. Dessa forma, políticas e intervenções de reeducação devem ser promovidas para banir a visão única desses meios de comunicação como fonte de ruptura entre os membros da família.

\keywords{Família \sep Novas tecnologias \sep Lazer \sep Contenção}
\end{abstract}
\end{portuguese}

\begin{english}
\begin{abstract}
In order to clarify the role of different agents in relation to family leisure from the perspective of parents and children, this study was carried out on a total sample of 763 parents and 286 children living in Spain, who completed a Likert-type questionnaire from 1 to 5, assessing the role played by the following agents: new technologies, school, public administration, associations, the media and the family itself, as well as indicating age and number of family members. The results show that younger parents have a more optimistic view of the agents and, especially, of new technologies, in contrast to older parents who are sceptical about them. In the case of children, their perception is worse as they get older, with a more positive view among adolescents. Dwellings with more family members appreciate the role played by agents and see new technologies as a useful support resource for family leisure. In this way, re-education policies and interventions should be promoted to banish the single vision of these media as a source of rupture between family members. 

\keywords{Family \sep New technologies \sep Leisure \sep Containment}
\end{abstract}
\end{english}

% if there is another abstract, insert it here using the same scheme
\end{polyabstract}


\section{Introdução}\label{sec-intro}
La preocupación por cómo invertimos productiva y saludablemente nuestro tiempo ha sido una constante en la existencia humana. Como consecuencia, desde la pedagogía de la productividad, se incrementan los libros de autoayuda que consolidan en los lectores la fantasía de que el tiempo puede ser controlado, con la finalidad de conseguir la eficiencia empresarial, pero también personal y cotidiana, cuando en realidad se acrecienta el volumen de horas laborales y se disminuye la compensación salarial en relación con las ganancias corporativas \cite{gregg2018}. Esta preocupación se ha visto agudizada en las últimas décadas por los cambios en los estilos de vida de los más jóvenes, asociadas al sedentarismo del consumo masivo de pantallas \cite{wang2019}, y más recientemente, por la pandemia mundial, la COVID-19, que ha tenido un efecto paralizador de las actividades de ocio exteriores \cite{cencerradomalmierca2020}, y en contraposición un aumento del tiempo en los hogares. De hecho, el aumento del tiempo libre y la reducción de las oportunidades de ocio ha obligado a las personas al consumo y a la producción de medios alternativos, generando un ocio improvisado en base a las oportunidades que les brinda el entorno de confinamiento \cite{bond2020}.

Las políticas de confinamiento en España son acciones tomadas por el gobierno ante la actual crisis sanitaria. Aunque, la COVID-19 es un síndrome respiratorio severo altamente infeccioso, menos mortal que otras epidemias (SARS, ÉBOLA, etc.), su peligrosidad radica en la facilidad de propagación \cite{callaway2020}, lo que ha empujado a los dirigentes al estado de alarma y aplicación de medidas restrictivas y de confinamiento absoluto, limitado o parcial. Esta nueva realidad social ha supuesto una auténtica revolución en la organización vital del ser humano, que ha derivado en un incremento del teletrabajo \cite{santillan2020}, de la mayor responsabilidad parental de la educación de sus hijos \cite{moreno2020}, del cuidado y la asistencia de los familiares vulnerables a la pandemia \cite{rodriguezescanciano2020}, así como la necesidad de buscar espacios y tiempos de distensión para alejar a la persona del estrés o la ansiedad que puede generar este tipo de “encierro” \cite{ozamiz-etxebarria2020}. 

Esta hiperfuncionalidad de los hogares COVID también ha generado estragos, tensiones, estrés, conflictos, añoranza, entre otras emociones desestabilizadoras de la acostumbrada individualidad de los miembros familiares \cite{belmonte2020, molina2020, robinet-serrano2020}. Las familias se enfrentaban a lo desconocido, un confinamiento impuesto, inmediato y urgente, totalmente desinformadas, desprovistas de las herramientas para poder sobrellevar el aislamiento continuado en sus casas. Nunca hijos y padres habían estado expuestos a una convivencia tan continuada con la que lidiar \cite{pedreira2020}, lo que puede considerarse un desafío al que hacer frente o, por el contrario, una excelente oportunidad para estrechar lazos familiares \cite{vestettekal2020}. En cualquier caso, es una realidad que no deja indiferente a nadie, y suscita múltiples interrogantes. La inabarcabilidad de todos de forma conjunta, obliga a tener que delimitar y seleccionar uno de ellos, centrando el foco de interés en esta ocasión en la gestión que realizan los miembros familiares, concretamente padres e hijos, del tiempo de ocio, pues en tiempos de crisis, disponer de momentos de disfrute y entretenimiento resulta eficaz.

El ocio como actividades realizadas fuera del tiempo de trabajo obligado, laboral o escolar, se define como un constructo multidimensional, que abarca tanto aspectos estructurales como subjetivos, para medir el compromiso con el ocio entendido como el grado en que los individuos participan en él \cite{newman2014}. Este carácter multidimensional ilustra que el ocio no es simplemente una lista de diversión y actividades relajantes, pero es, de hecho, algo cosa que varía en forma y tiempo a los ojos de diferentes personas, o incluso de la misma persona en diferentes etapas de su vida \cite{chen2018}. Así mismo, el carácter no impositivo de las actividades de ocio implica, por un lado, la diversidad de alternativas en la que este puede materializarse, y por otro, el protagonismo de la libertad en los procesos de toma de decisiones que determinan la elección de la misma. Por tanto, los elementos constitutivos del ocio son la determinación del tiempo a gastar, la elección libre de la actividad a realizar y el desarrollo subjetivo de la experiencia \cite{alvarez2020, kelly2012}. La diversidad de manifestaciones del ocio es tal, que para diferenciarlos se denominan con el calificativo pertinente en función de la actividad elegida o el contexto en el que se desarrolla: ocio deportivo, tecnológico, saludable, cultural o familiar, entre otros. De todos ellos nos centramos en el último. 

En la investigación sobre la temática, el énfasis se fija en variables como el tipo de actividad realizada, la cantidad de tiempo invertida, las limitaciones culturales y ambientales sociales sobre el ocio, así como los efectos del mismo en diversas facetas del desarrollo cognitivo y emocional de las personas \cite{newman2014}. De ahí que generalmente el ocio sea tratado como medio para el desarrollo o consecución de otros menesteres, y no como fin propiamente dicho de las intervenciones educativas \cite{chen2018}. Afortunadamente, el ocio entendido como una plataforma para el compromiso social con la familia, los amigos y la sociedad \cite{kelly2012}, lo sitúa en el epicentro del interés investigativo, favoreciendo que emergen nuevas variables. 

Sin menospreciar las posibilidades de creación de alternativas de ocio familiar, desde la propia inventiva de los miembros que la componen, aspecto este que debería cultivarse mucho más, la oferta exterior liderada por otros agentes se ha visto incrementada y diversificada en las últimas décadas. De este modo, el ocio familiar se adentra en las reglas del mercado, viéndose delimitado por la oferta y la demanda. A pesar de las fluctuaciones posibles, el estudio sobre la percepción que tanto padres como hijos pueden tener sobre los agentes involucrados en el ocio familiar resulta pertinente para determinar el tipo de ocio demandado. 

Tomando como punto de partida la clasificación que propone \textcite{alvarez2020}, los agentes dinamizadores del ocio familiar son: la propia familia, el centro educativo del hijo o hija, las administraciones públicas, los clubes deportivos, las asociaciones, los centros comerciales, las empresas privadas, la propia familia y otras familias. De esta manera se acota los entes más tradicionales con otros de carácter más innovador que están adquiriendo más protagonismo en su ejercicio con las familias. 

Entre los agentes que actúan como referentes en el desarrollo de las prácticas de ocio familiar destacan, en primer lugar, la propia familia, lo cual resulta lógico pues, los padres son aquellos que deciden las prácticas de ocio que se van a realizar o no \cite{buswell2012}, independientemente de la crisis que habitualmente se da entre adolescentes y familia \cite{gomezcobos2008}. En lo que respecta al contexto escolar, el tiempo dentro de los centros educativos se asemeja al laboral. Un tiempo impuesto, obligado, gestionado y organizado jerárquicamente por otros, en la que el estudiante, independientemente de su edad y capacidad, debe desempeñar un comportamiento esperado y la realización de tareas marcados por otros, quedando ausente, generalmente, de la toma de decisiones. De este modo, los adolescentes también perciben la escasa dedicación al tiempo de ocio al sufrirlo de primera mano, protagonistas de una infancia y adolescencia en la que prima la estancia en servicios complementarios, con el fin de solventar los problemas de conciliación laboral-familiar \cite{martinezpampliega2019, varela2016}. Pero los centros educativos, si van más allá de lo curricular, también constituyen uno de los principales referentes y una excelente oportunidad para la didáctica del ocio \cite{gairin2004}, haciendo posible la inclusión en otros contenidos de diversa naturaleza (deporte, tecnología, medio ambiente…), así como la colaboración con las familias en comunidades de aprendizaje \cite{parra2020}. De este modo la propia escuela se convierte en un medio posibilitador del tiempo en familia.

Por otro lado, estudios previos evidencian que ni las administraciones públicas, a través de la instauración de políticas y creación de servicios concretos, ni las empresas privadas, son fuente de alimentación para el tiempo compartido en familia \cite{mchale2014}. No obstante, las administraciones suelen actuar a través de las asociaciones, entendidas como espacios de articulación y extensión de diferentes políticas, específicas o genéricas, como la de deporte o salud, que proyectan orientaciones para condicionar las formas de ocio de las familias \cite{king2015}. Al respecto, \textcite{freeman2003} señalan las asociaciones para la población discapacitada como unas de las que más contribuyen a crear un ocio familiar íntegro para todos los miembros, mientras que \textcite{hernandezprados2019} enfatizan las asociaciones deportivas como aquellas que, desde la promoción de actividades físicas en familia, dotan de una vertiente saludable a las formas de ocio que éstos desempeñan en familia.

Finalmente, otro de los grandes agentes del ocio familiar son los medios de comunicación y las nuevas tecnologías. Además de gozar de gran reconocimiento como agente social del nuevo milenio para los adolescentes \cite{gomes2014}, su influencia se extiende al nexo familiar hasta el punto de cambiar la forma en la que nos comunicamos \cite{orchard2010}. También ofrecen una oferta de actividades más novedosa que comporta diferentes maneras de dedicar el tiempo en familia, lo cual lo hace más atractivo y motivador para los pequeños \cite{godfrey2009}. Otros investigadores califican las TIC como una barrera u obstáculo, fruto de las fuertes diferencias intergeneracionales respecto a su uso, más que un enlace entre miembros \cite{fernandezmontalvo2015}. Recientes estudios también sostienen la introducción de las nuevas tecnologías en el hogar como causa del empobrecimiento de las relaciones familiares, pues inducen a un uso individual excesivo \cite{carvalho2015}. Así pues, este agente se representa como una fortaleza o una amenaza dependiendo del uso que se le dé al mismo, señalando la necesidad de formación al respecto en los padres y madres para su correcta aplicación \cite{wang2015}. 

Durante la COVID-19, el uso y la potenciación de las mismas se han visto incrementado con creces, no solo como recursos con el que invertir el tiempo de ocio, sino también como visualización de las actividades de ocio en las redes. De este modo, rara vez el ocio familiar constituye un acto aislado, incluso cuando parece serlo, pues detrás de estos se encuentran estructuras de redes complejas que ubican este acto en un ámbito social más amplio, de las redes sociales y el capitalismo del ocio \cite{bond2020}. Lo que lleva a identificarlo como uno de los elementos más sensibles a la cadencia social, que se halla en continúo cambio, viéndose, irrevocablemente, influenciado por el papel que tienen las nuevas tecnologías como recurso de entretenimiento, aunque también se da una revalorización de aquellas formas de ocio familiar más tradicionales \cite{vanleeuwen2020}. 

Atendiendo a lo expuesto anteriormente, el presente estudio pretende analizar la percepción de los padres y de los hijos sobre los agentes involucrados en el ocio familiar durante el confinamiento, en función de la edad de cada colectivo y del número de miembros con los que conviva. La doble percepción es novedosa, al incluir tanto a los padres como a los hijos, lo que permite adquirir una cobertura de las dinámicas intrafamiliares muy amplia. Pero sin lugar a dudas, el hecho de analizar los agentes involucrados en el ocio familiar permite complementar esta línea de investigación que, con mayor predominio, considera el ocio externo por encima del intrafamiliar, aspecto este que puede verse alterado por la COVID, modificando el tipo de agente habitual, por otros menos comunes. En definitiva, nos interrogamos si el predominio de las TIC como agentes involucrados en la promoción del ocio familiar continua vigente durante el confinamiento.

\section{Metodología}
\subsection{Diseño de la investigación}
Para obtener información respecto a la situación del ocio familiar durante la etapa de confinamiento por la pandemia COVID-19 se ha diseñado una investigación de corte descriptivo-correlacional, no experimental y transversal que dota de descripciones precisas y cuidadosas respecto a un fenómeno socioeducativo concreto, otorgando hechos y datos específicos que sirven de soporte para la obtención de hallazgos \cite{bisquerra2004}. 

\subsection{Participantes}
Para poder participar en el estudio, los participantes debían ser adultos, españoles y tener un hijo con un rango de edad entre los 3 y 18 años participantes fueron reclutados a través de un muestreo de bolas de nieve mediante medios regulares como comunicación por correo electrónico a los equipos directivos de 54 centros escolares, las federaciones y asociaciones de padres y madres, y utilizando diversas redes sociales  el uso de Twitter personal, cuentas de Facebook y web de centros educativos. Este estudio siguió los parámetros éticos establecidos por la Universidad de Murcia en su web, de manera que todos los participantes fueron debidamente informados del proceso de investigación que garantiza el anonimato de los mismos, de la protección y custodia de la base de datos, y proporcionaron su consentimiento informado en línea en el propio cuestionario.

Un total de 895 participantes padres y 356 participantes hijos realizaron el cuestionario por vía telemática. No obstante, 132 y 70, respectivamente, se hallaban incompletos, por lo cual, se tuvieron que eliminar de los datos obtenidos. Finalmente, la muestra de participantes final fue de 763 progenitores y 286 hijos residentes en el territorio español. En la siguiente \Cref{tab1}, se muestran los datos sociodemográficos respecto a las principales variables tomadas en consideración: 

\begin{table}[htpb]
\caption{Distribución de la muestra de familias participantes en el estudio}
\label{tab1}
\centering
\begin{tabular}{p{0.3\textwidth}p{0.14\textwidth}p{0.3\textwidth}p{0.12\textwidth}}
\toprule 
Características participantes & Padres N (\%) 763 (72,73)	& Características participantes	& Hijos N (\%) 286 (27,27)
\\
\midrule
Edad & & Edad &
\\ 
\midrule
Menos de 30	& 161 (21,10) & Menos de 17 & 127 (44,45)
\\
30-40 & 444 (58,16) & De 17 a 20 & 79 (27,65)
\\
Más de 40 & 158 (20,69) & De 21 a 25 & 30 (10,48)
\\
& & Más de 25 & 50 (17,50)
\\
\midrule
Número de miembros & & Número de miembros &
\\
\midrule
2 o 3 & 266 (34,84)	& 2 o 3 & 77 (26,95)
\\
4 o 5 & 433 (56,72)	& 4 o 5 & 185 (64,75)
\\
6 o más	& 64 (8,38)	& 6 o más & 24 (8,4)
\\
\bottomrule
\end{tabular}
\centering
\source{Elaboración propia.}
\end{table}

\subsection{Procedimiento}
Una vez fue elaborado el instrumento de recogida de información definitivo, se procedió a su distribución la cual no siguió un procedimiento aleatorio. Inicialmente se realizó un rastreo documental por medio del cual obtener el contacto telemático de 54 centros educativos a los cuales se les enviaría un correo electrónico con la información de la investigación junto el enlace que lleva a la cumplimentación del cuestionario. Todo ello con el propósito de que los colegios e institutos dieran difusión a partir de un mensaje junto al enlace que permitiera acceder al cuestionario. Además, se hizo una importante propagación por redes sociales (Instagram, Twitter y Facebook) distribuyendo la comunicación elabora en cuentas de índole educativa, social y familiar. Se estableció un mes de difusión continuada para poner fin a la recogida de datos, una vez pasó dicho tiempo se vaciaron los datos para su tratamiento.  

\subsection{Instrumento de recogida de información}
En esta investigación, que forma parte de un estudio más amplio, se va a utilizar la dimensión de los agentes que contempla de cuestión 38 a la 43. Parte del constructo total de un cuestionario que comporta 61 ítems acerca del ocio familiar en tiempo de confinamiento. Se obtuve una fiabilidad global del cuestionario bastante alta (,887) y, de forma específica, la dimensión de los agentes obtuvo un valor inferior al global (,727), aunque sigue estando comprendido en el rango de fiabilidad aceptable \cite{celina2005}. Para la cumplimentación del cuestionario se ha de seleccionar un valor dentro de una escala Likert de 1 a 5 que comprende la siguiente relación de valores: 1 (nada), 2 (poco), 3 (algo), 4 (bastante) y 5 (mucho). De esta manera, se cuestiona acerca del papel que ocupa en el fomento del ocio familiar los siguientes agentes: las nuevas tecnologías, la escuela, las administraciones públicas, la propia familia, los medios de comunicación y las asociaciones. Destacar que, previamente a los ítems de las tres dimensiones, se incluían las siguientes cuestiones sociodemográficas como la edad y el número de miembros en el núcleo familiar. 

\subsection{Técnicas de análisis de los datos}
Para contrastar los datos cuantitativos procedentes de la aplicación del cuestionario facilitado a las familias, se empleó el paquete estadístico SPSS, versión 24. 

Se recurrió tanto a la estadística descriptiva como a la inferencial paramétrica, al asumirse que se podía emplear esta última, por varias razones \cite{siegel1991}, principalmente tras calcular que se cumplen los supuestos de normalidad (\emph{Kolmogorov-Smirnov}), porque el n es mayor de 30 y porque se verifica que hay igualdad de las varianzas u homocedasticidad (\emph{Levene}). En todos los casos se ha utilizado un nivel de significación estadística de $\alpha$ =.05. 

Además, dado que la significación estadística no ahonda en la fuerza de la diferencia o de la relación, se procedió a estimar la magnitud de la diferencia entre variables a partir del tamaño del efecto \cite{cohen1988}, a través del cálculo de la prueba \emph{d de Cohen}.

\section{Resultados}
A continuación, se muestra el número de sujetos (n) y los estadísticos descriptivos de las variables de la investigación, concretamente las puntuaciones medias ($\tilde{X}$̃), desviaciones típicas ($\sigma$), valor crítico de Snedecor (F), significación estadística (p) y tamaño del efecto (d), por cada uno de los objetivos planteados en el estudio.

\subsection{Percepción de los padres y de los hijos sobre los agentes involucrados en el ocio familiar durante el confinamiento}
El periodo de confinamiento sufrido a nivel comunitario y social, entre otras muchas repercusiones, ha supuesto alteraciones (que no necesariamente han sido negativas) en el ocio familiar. Concretamente, tal y como se refleja en la \Cref{tab2}, durante este espacio de tiempo, los participantes, tanto padres como hijos, afirman que la propia familia en sí misma ha potenciado bastante el ocio familiar ($\tilde{X}_{P/41}=4,09$; $\tilde{X}_{H/41}=4,08$), junto con las nuevas tecnologías ($\tilde{X}_{P/38}=3,46$; $\tilde{X}_{H/38}=3,94$). Por otro lado, agentes como los medios de comunicación ($\tilde{X}_{P/42}=2,91$; $\tilde{X}_{H/42}=3,03$) o la escuela ($\tilde{X}_{P/39}=2,97$; $\tilde{X}_{H/39}=2,52$), también han contribuido algo al fomento de dicho ocio. Los medios que apenas han promovido el ocio familiar han sido las asociaciones ($\tilde{X}_{P/43}=2,14$; $\tilde{X}_{H/43}=1,82$) y las administraciones públicas ($\tilde{X}_{P/40}=1,95$; $\tilde{X}_{H/40}=1,90$). 

\begin{table}[htpb]
\caption{Estadísticos descriptivos de padres e hijos sobre los agentes que fomentan el ocio}
\label{tab2}
\centering
\begin{tabular}{lllllll}
\toprule
\multirow{2}{*}{Fomento del ocio familiar según:} & \multicolumn{2}{c}{n} & \multicolumn{2}{c}{$\tilde{X}$} & \multicolumn{2}{c}{\begin{math}\sigma\end{math}}
\\
& Padres & Hijos & Padres & Hijos & Padres & Hijos
\\
\midrule
P38... Las nuevas tecnologías. & 763 & 286 & 3,46 & 3,94 & 1,102 & 1,135
\\
P39... La escuela. & 763 & 286 & 2,97 & 2,52 & 1,173 & 1,342
\\
P40... Las Administraciones públicas. & 763 & 286 & 1,95 & 1,90 & ,948 & 1,000
\\
P41... La propia familia. & 763 & 286 & 4,09 & 4,08 & ,878 & 1,012
\\
P42... Los medios de comunicación. & 763 & 286 & 2,91 & 3,03 & 1,085 & 1,255
\\
P43... Las asociaciones. & 763 & 286 & 2,14 & 1,82 & 1,153 & 1,077
\\
\bottomrule
\end{tabular}
\centering
\source{Elaboración propia.}
\end{table}

\subsection{Influencia de la edad de los integrantes de la familia acerca de la percepción sobre los agentes implicados en el ocio familiar durante el confinamiento}
Analizando la percepción de los padres sobre los agentes involucrados en el ocio familiar durante el confinamiento a nivel general, tomando en consideración la edad de los mismos, tras el cálculo de la prueba \emph{Anova de un factor} y la aplicación de las pruebas post-hoc, concretamente el \emph{test de comparaciones múltiples de Bonferroni}, en la \Cref{tab3} se observa que existe significación estadística (\emph{p}G=,044), concretamente respecto al criterio de los progenitores que tienen entre 30 y 40 años y los más mayores (\emph{p}=,038), a favor de los primeros ($\tilde{X}_{PG/30-40}=2,95$). Tras analizar tales diferencias con la estimación de la prueba d de Cohen, se observa que la magnitud obtenida muestra que estas no presentan un índice cercano al típico establecido al tratarse de diferencias (\emph{d}=,5), sino más bien por debajo (\emph{d}=0,226), no siendo, por tanto, demasiado potentes. 

Haciendo especial hincapié en la opinión concreta sobre las nuevas tecnologías, se hallan también diferencias (\emph{p}T=,040), entre los padres de 30 y 40 y los de más de 40 años (\emph{p}=,036), siendo los progenitores de mayor edad los que menos de acuerdo se posicionan en cuanto al carácter fomentador del ocio, de los medios digitales ($\tilde{X}_{PT/30-40}=3,25$). En este caso, la magnitud de las mencionadas diferencias tampoco es demasiado grande, al no superar el valor típico (\emph{d}=0,244).

\begin{table}[htpb]
\caption{Descriptivos, F de Snedecor y significación estadística de los padres, según su edad}
\label{tab3}
\centering
\begin{tabular}{llllllllll}
\toprule
& \multirow{2}{*}{n} & \multicolumn{2}{c}{$\tilde{X}$̃} & \multicolumn{2}{c}{\begin{math}\sigma\end{math}} & \multicolumn{2}{c}{F} & \multicolumn{2}{c}{\emph{p}}
\\
& & Global & TIC & Global & TIC & Global & TIC & Global & TIC
\\
\midrule
Menos de 30	& 161 & 2,93 & 3,47 & ,669 & 1,199 & 3,148 & 3,230 & ,044 & ,040
\\
30-40 & 444 & 2,95 & 3,51 & ,598 & 1,068 & & & &	 	 
\\
Más de 40 & 158 & 2,81 & 3,25 & ,641 & 1,058 & & & &
\\
\bottomrule
\end{tabular}
\centering
\source{Elaboración propia.}
\end{table}

Por otro lado, tomando en consideración la percepción de los hijos a nivel global, según la edad que tengan estos, sobre los agentes implicados en el ocio familiar, tal y como se visualiza en la \Cref{tab4}, se halla significación estadística (\emph{p}G=,000), entre el parecer de los más pequeños y los de edades restantes (\emph{p}=,000; \emph{p}=,000; \emph{p}=,002), a favor de los más jóvenes ($\tilde{X}_{PG/-17}=3,20$). El tamaño del efecto en todos los casos, supera el típico establecido (\emph{d}=1,035; \emph{d}=0,973; \emph{d}=0,679) y por tanto se corrobora que dichas diferencias poseen una magnitud considerablemente alta. 

Hablando concretamente sobre las nuevas tecnologías, de nuevo se encuentra significación estadística (\emph{p}T=,000) entre los mismos grupos de edad (\emph{p}=,000; p=,000; p=,000), y a favor de los más jóvenes ($\tilde{X}_{PT/-17}=4,65$), siendo esta vez el tamaño de las diferencias incluso mayor que en el supuesto anterior (\emph{d}=1.471; \emph{d}=1,653; \emph{d}=1,243).

\begin{table}[htpb]
\caption{Descriptivos, F de Snedecor y significación estadística de los hijos, según su edad}
\label{tab4}
\centering
\begin{tabular}{llllllllll}
\toprule
& \multirow{2}{*}{n} & \multicolumn{2}{c}{$\tilde{X}$̃} & \multicolumn{2}{c}{\begin{math}\sigma\end{math}} & \multicolumn{2}{c}{F} & \multicolumn{2}{c}{\emph{p}}
\\
& & Global & TIC & Global & TIC & Global & TIC & Global & TIC
\\
\midrule
Menos de 17 & 127 & 3,20 & 4,65 & ,604 & ,510 & 22,634 & 48,479 & ,000 & ,000
\\
17 a 20 & 79 & 2,52 & 3,25 & ,706 & 1,245 & & & &	 	 	 
\\
De 21 a 25  & 30 & 2,55 & 3,13 & ,72 & 1,196	 	 	 	 
\\
Más de 25 & 50 & 2,82 & 3,68 & ,510 & ,978	 	 	 	 
\\
\bottomrule
\end{tabular}
\centering
\source{Elaboración propia.}
\end{table}

\subsection{Influencia del número de miembros en el hogar, acerca de la percepción sobre los agentes involucrados en el ocio familiar durante el confinamiento}
Valorando ahora la opinión de los padres sobre los agentes implicados en el ocio familiar durante el confinamiento, en función del número concreto de familiares en el hogar, se observa en la \Cref{tab5} que existe significación estadística (\emph{p}G=,048), entre el juicio de los padres de familias compuestas por 4 o 5 miembros y las familias que cuentan con 2 o 3 personas (\emph{p}=,047), a favor de los primeros ($\tilde{X}_{PG/4-5}=2,96$). Aunque se observa que el índice arrojado está por debajo al típico establecido (\emph{d}=0,194), no siendo, por tanto, la magnitud de estas diferencias, demasiado grande.

Específicamente, haciendo relación a la percepción sobre la tecnología como agente potenciador del ocio, aparecen diferencias estadísticamente significativas (\emph{p}T=,040) entre los padres de familias con menos miembros y las que más tienen (\emph{p}=,036), a favor de los primeros ($\tilde{X}_{PT/4-5}=3,47$), aunque sin alcanzar el tamaño del efecto un valor relevante (\emph{d}=0,196).

\begin{table}[htpb]
\caption{Descriptivos, F de Snedecor y significación estadística de los padres, según nº miembros}
\label{tab5}
\centering
\begin{tabular}{llllllllll}
\toprule
& \multirow{2}{*}{n} & \multicolumn{2}{c}{$\tilde{X}$} & \multicolumn{2}{c}{\begin{math}\sigma\end{math}} & \multicolumn{2}{c}{F} & \multicolumn{2}{c}{\emph{p}}
\\
& & Global & TIC & Global & TIC & Global & TIC & Global & TIC
\\
\midrule
2 o 3 & 266 & 2,84 & 3,47 & ,633 & 1,198 & 3,014 & 3,230 & ,048 & ,040
\\
4 o 5 & 433 & 2,96 & 3,51 & ,602 & 1,065 & & & &
\\
6 o más & 64 & 2,95 & 3,25 & ,711 & 1,052 & & & &
\\
\bottomrule
\end{tabular}
\centering
\source{Elaboración propia.}
\end{table}

Si se fija la atención en los hijos, en la tabla 6 se encuentran diferencias significativas (\emph{p}G=,017) entre las percepciones de los que pertenecen a familias con 6 o más miembros y las que tienen 2 o 3 (\emph{p}=,021), junto con las que tienen 4 o 5 convivientes (\emph{p}=,018), a favor en ambos casos del parecer de los hijos de las familias con mayor número de integrantes ($\tilde{X}_{PG/6+}=3,27$), con un tamaño del efecto superior al típico en ambos casos (\emph{d}=0,656 y \emph{d}=0,632).

Por último, al evaluar la visión de los hijos sobre si los agentes tecnológicos han fomento el ocio durante el confinamiento, se hallan también estas diferencias estadísticamente significativas (\emph{p}T=,049). Tal y como se observa (\Cref{tab6}), entre los hijos integrantes de las familias más grandes, frente a los hijos de familias compuestas por 2 o 3 miembros (\emph{p}=,046), siendo los primeros los que mejor percepción tienen sobre esto ($\tilde{X}_{PT/6+}=4,33$). La magnitud de las diferencias se encuentra en torno al valor medio establecido (\emph{d}=0,514).

\begin{table}[htpb]
\caption{Descriptivos, F de Snedecor y significación estadística de los hijos, según nº de miembros}
\label{tab6}
\centering
\begin{tabular}{llllllllll}
\toprule
& \multirow{2}{*}{n} & \multicolumn{2}{c}{$\tilde{X}$̃} & \multicolumn{2}{c}{\begin{math}\sigma\end{math}} & \multicolumn{2}{c}{F} & \multicolumn{2}{c}{\emph{p}}
\\
& & Global & TIC & Global & TIC & Global & TIC & Global & TIC
\\
\midrule
2 o 3 & 77 & 2,83 & 3,73 & ,706 & 1,166 & 4,125 & 2,911 & ,017 & ,049
\\
4 o 5 & 185 & 2,85 & 3,97 & ,695 & 1,105 & & & &
\\
6 o más & 24 & 3,27 & 4,33 & ,632 & 1,167 & & & &
\\
\bottomrule
\end{tabular}
\centering
\source{Elaboración propia.}
\end{table}

\section{Conclusiones}
Durante el periodo de tiempo del confinamiento, los participantes en el estudio afirman que la propia familia en sí misma ha potenciado bastante el ocio familiar, resultados similares a los obtenidos en el estudio de \textcite{medina2021} donde el 72\% prefieren compartir su tiempo de ocio en familia en lugar de practicarlo de forma individual, postura totalmente contraria a aquellas investigaciones que manifiestan la vivencia compartida como una experiencia desencadenante de estrés, ansiedad y desesperación \cite{vazquezsoto2020, zayas-fajardo2021}. Las nuevas tecnologías también se identifican como agentes de fomento del ocio familiar. \textcite{vaziri2020} señalan el potencial que tienen para una mejor conciliación laboral mientras que algunos padres identifican estos recursos como el causante del aislamiento familiar de sus hijos \cite{drouin2020}. Por otro lado, agentes como los medios de comunicación o la escuela, también han contribuido algo al fomento de dicho ocio, aunque en menor medida. Se sostiene que los centros escolares, a pesar de ser punto de unión entre familias y alumnos, presentan obstáculos a la participación derivados de la dificultad de conciliación que tienen los tiempos familiares con los escolares, manteniendo, en ocasiones, una relación de tensiones y discrepancias \cite{moreno2020}.

En cambio, las asociaciones y las administraciones públicas, bajo su perspectiva, apenas han promovido el ocio familiar , siendo similar esto a la postura que sostenían las familias en la situación previa a la pandemia en la que los adolescentes declaran, por un lado, que los esfuerzos realizados por las instituciones y las políticas ejercidas son insuficientes y, por otro lado, una menor tendencia de participación familias en las asociaciones, más identificado como entorno para el ejercicio del ocio individual \cite{alvarez2020}.

Analizando la percepción de los padres sobre los agentes involucrados en el ocio familiar durante el confinamiento, en función a la edad de los mismos, se halla significación estadística, aunque de pequeñas magnitudes, de modo que los progenitores de entre 30 y 40 años y los padres más mayores son quienes se muestran menos optimistas con respecto a este supuesto, la juventud parental otorga de una mayor vitalidad y voluntad por compartir tiempo en familia \cite{roeters2016}. Tomando en consideración la percepción de los hijos, también aparecen diferencias significativas entre el parecer de los más pequeños y los de edades restantes, a favor de los más jóvenes, la extensión del periodo de desarrollo de la adolescencia dificulta la maduración y, por lo tanto, extiende y posterga ese sentimiento de desinterés por lo familiar tan característico de esta época \cite{hernandez2017}. 

En el tránsito de la vida adolescente a la vida adulta se mantiene el protagonismo formas de ocio en las que la familia no es considerada para su ejercicio, tales como el ocio nocturno \cite{lavielle-sotomayor2014} o el uso de los dispositivos móviles \cite{alfarogonzales2015}. En esta ocasión, dado que el tamaño del efecto en todos los casos supera el típico establecido, queda patente que la fuerza de dichas diferencias es muy alta. Es decir, los hijos más jóvenes no tienen ninguna duda de la influencia que poseen algunos agentes a la hora de impulsar el ocio familiar, los menores de edad identifican su ocio familiar ajeno a las amenazas familiares, estudiantiles o laborales al hallarse inserto en un contexto social que actúa como pasarela a la revaloración positiva de estas prácticas \cite{sanzarazuri2018}. Tales resultados también ponen de manifiesto que los crecientes conflictos parentales \cite{cummings2008} junto a un mayor protagonismo de los iguales en la vida inicial adulta \cite{caneroperez2019} y una preferencia por el ocio individual \cite{urgilesleon2018}, lleva a que los jóvenes cada vez aprecien menos el papel que otorgan al tiempo compartido agentes externos al núcleo familiar. 

Haciendo especial hincapié en la opinión concreta sobre las nuevas tecnologías, se hallan también diferencias de baja magnitud entre los padres de 30 y 40 y los de más de 40 años, siendo los progenitores de mayor edad los que menos de acuerdo se posicionan en cuanto al carácter de los medios digitales para fomentar el ocio, a mayor edad incrementa el desconocimiento de uso, el poco alcance y el descontrol parental respecto a estos recursos, su poca consideración se muestra en la escasa frecuencia de uso de los mismos \cite{nikken2015, sanchez-valle2017}. En el caso de los hijos, de nuevo se encuentra significación estadística entre la opinión de los más jóvenes, con tamaños de la diferencia muy elevados, imagen alejada de la idea fundada de las nuevas tecnologías como producto de contaminación y ruptura del entorno familiar ante un excesivo uso de las tecnologías \cite{gamito2019}. Así, los padres jóvenes, en la franja de los 30 a los 40, y los hijos de edades más tempranas, confían más en el potencial de la tecnología como agente potenciador del ocio familiar, rangos de edad que coinciden con los resultados de \textcite{gonzalezfernandez2020} al identificar los padres jóvenes y los hijos adolescentes como los que mayor dominio tienen de los dispositivos electrónicos.

Valorando la influencia del número de miembros en el hogar, sobre la percepción de los padres en cuanto a los agentes implicados en el ocio familiar durante el confinamiento, aparece significación estadística, con magnitudes no demasiado grandes, entre el juicio de los padres de familias compuestas por 4 o 5 miembros y las familias que cuentan con 2 o 3 personas, a favor de los primeros, dado que las madres o padres solteros carecen de tiempo para poder dedicar al ocio familiar y los padres con un solo hijo sobreprotegen a su único hijo hasta el punto de considerar múltiples amenazas en el contexto \cite{townsend2017}. Si se fija la atención en los hijos, se encuentran diferencias significativas con un tamaño del efecto superior al típico, entre las percepciones de los que pertenecen a familias con 6 o más miembros y las que tienen únicamente 2 o 3, junto con las que tienen 4 o 5 convivientes, a favor en ambos casos del parecer de los hijos de las familias con mayor número de integrantes, un mayor número de hermanos junto a unas relaciones saludables aumenta el sentimiento de pertenencia familiar y, por consiguiente, una mejor consideración del ocio familiar \cite{oberg2017}.

Por último, haciendo relación a si los agentes tecnológicos han fomento el ocio durante el confinamiento, aparecen diferencias estadísticamente significativas, no de gran tamaño, entre la percepción de los padres de familias con menos miembros y las que más tienen, a favor de los primeros. Al evaluar la visión de los hijos, se hallan también diferencias estadísticamente significativas, esta vez sí, con magnitudes en torno al típico, entre los hijos integrantes de las familias más grandes, frente a los hijos de familias compuestas por 2 o 3 miembros. Los hijos de familias más numerosas valoran más positivamente la tecnología como potenciador del ocio familiar, rompiendo con la idea popularizada del uso nocivo de estos soportes, y abriendo su percepción a un uso de las nuevas tecnologías como vía de unificación, cohesión y entretenimiento en el núcleo familiar \cite{doss2017}. 

A raíz de los resultados obtenidos, se expone un rol de las nuevas tecnologías que destierra todos los discursos que etiquetan estos medios como signos de ruptura o desintegración familiar. La gestión y valor que se les dé será primordial de cara a extraer beneficios a nivel grupal o individual, sin embargo, el desconocimiento o mal uso puede desembocar en numerosos perjuicios que no son a causa del medio sino de los padres. De esta forma, se hace necesaria la implementación de formaciones a fin de acercar estos medios y hacer conocedores a los padres, especialmente a los mayores, de las posibilidades que puede otorgar este tipo de recursos a partir de un uso con criterio y reflexión. Para ello, también resulta prioritario una articulación coordinada de políticas familias en relación a las nuevas tecnologías, visible a través de instituciones como los centros educativos, asociaciones o medios de comunicación contribuyendo a reeducar a la población respecto a la visión y uso de estos recursos en el entorno familiar. 

Aunque el presente estudio ha aportado importantes hallazgos, esta no ha estado al margen de ciertas limitaciones que han influenciado en el desarrollo de la investigación. La más patente ha sido la insuficiente participación obtenida, debiendo llevar a cabo numerosas difusiones para poder llegar a un número considerable, situación que se hacía más ardua ante unas instituciones, como centros escolares, en colapso de trabajo por la situación de trabajo causada por la pandemia COVID-19. De cara a futuras intervenciones, otorgaría un plus de calidad contemplar la opinión de otros familiares al margen de padres e hijos y, además, especificar aquellos descriptores que llevan a considerar o no las nuevas tecnologías como un agente de influencia del ocio familiar. 





\printbibliography\label{sec-bib}
% if the text is not in Portuguese, it might be necessary to use the code below instead to print the correct ABNT abbreviations [s.n.], [s.l.] 
%\begin{portuguese}
%\printbibliography[title={Bibliography}]
%\end{portuguese}


\end{document}
