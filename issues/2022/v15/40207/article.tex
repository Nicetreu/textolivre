% !TEX TS-program = XeLaTeX
% use the following command:
% all document files must be coded in UTF-8
\documentclass[portuguese]{textolivre}
% build HTML with: make4ht -e build.lua -c textolivre.cfg -x -u article "fn-in,svg,pic-align"

\journalname{Texto Livre}
\thevolume{15}
%\thenumber{1} % old template
\theyear{2022}
\receiveddate{\DTMdisplaydate{2022}{6}{21}{-1}} % YYYY MM DD
\accepteddate{\DTMdisplaydate{2022}{7}{28}{-1}}
\publisheddate{\DTMdisplaydate{2022}{10}{13}{-1}}
\corrauthor{Newton Lima Neto}
\articledoi{10.35699/1983-3652.2022.40207}
%\articleid{NNNN} % if the article ID is not the last 5 numbers of its DOI, provide it using \articleid{} commmand 
% list of available sesscions in the journal: articles, dossier, reports, essays, reviews, interviews, editorial
\articlesessionname{articles}
\runningauthor{Lima Neto e Carvalho} 
%\editorname{Leonardo Araújo} % old template
\sectioneditorname{Bárbara Amaral da Silva}
\layouteditorname{Carolina Garcia}

\title{Letramento digital: breve revisão bibliográfica do limiar entre conceitos e concepções de professoras e de professores}
\othertitle{Digital literacy: a brief literature review of in-between concepts and of teachers’ conceptions}
% if there is a third language title, add here:
%\othertitle{Artikelvorlage zur Einreichung beim Texto Livre Journal}

\author[1,2]{Newton Vieira Lima Neto  \orcid{0000-0002-2824-8337} \thanks{Email: \href{mailto:newtonneto@ufmg.br}{newtonneto@ufmg.br}}}
\author[1]{Alexandra Bittencourt de Carvalho  \orcid{0000-0003-3159-2021} \thanks{Email: \href{mailto:alexandraportugues@yahoo.com.br}{alexandraportugues@yahoo.com.br}}}
\affil[1]{Universidade Federal de Minas Gerais, Faculdade de Letras, Belo Horizonte, MG, Brasil.}
\affil[2]{Instituto Federal de Brasília, Riacho Fundo I, Brasília, DF, Brasil.}

\addbibresource{article.bib}
% use biber instead of bibtex
% $ biber article

% used to create dummy text for the template file
\definecolor{dark-gray}{gray}{0.35} % color used to display dummy texts
\usepackage{lipsum}
\SetLipsumParListSurrounders{\colorlet{oldcolor}{.}\color{dark-gray}}{\color{oldcolor}}

% used here only to provide the XeLaTeX and BibTeX logos
\usepackage{hologo}

% if you use multirows in a table, include the multirow package
\usepackage{multirow}

% provides sidewaysfigure environment
\usepackage{rotating}

% CUSTOM EPIGRAPH - BEGIN 
%%% https://tex.stackexchange.com/questions/193178/specific-epigraph-style
\usepackage{epigraph}
\renewcommand\textflush{flushright}
\makeatletter
\newlength\epitextskip
\pretocmd{\@epitext}{\em}{}{}
\apptocmd{\@epitext}{\em}{}{}
\patchcmd{\epigraph}{\@epitext{#1}\\}{\@epitext{#1}\\[\epitextskip]}{}{}
\makeatother
\setlength\epigraphrule{0pt}
\setlength\epitextskip{0.5ex}
\setlength\epigraphwidth{.7\textwidth}
% CUSTOM EPIGRAPH - END

% LANGUAGE - BEGIN
% ARABIC
% for languages that use special fonts, you must provide the typeface that will be used
% \setotherlanguage{arabic}
% \newfontfamily\arabicfont[Script=Arabic]{Amiri}
% \newfontfamily\arabicfontsf[Script=Arabic]{Amiri}
% \newfontfamily\arabicfonttt[Script=Arabic]{Amiri}
%
% in the article, to add arabic text use: \textlang{arabic}{ ... }
%
% RUSSIAN
% for russian text we also need to define fonts with support for Cyrillic script
% \usepackage{fontspec}
% \setotherlanguage{russian}
% \newfontfamily\cyrillicfont{Times New Roman}
% \newfontfamily\cyrillicfontsf{Times New Roman}[Script=Cyrillic]
% \newfontfamily\cyrillicfonttt{Times New Roman}[Script=Cyrillic]
%
% in the text use \begin{russian} ... \end{russian}
% LANGUAGE - END

% EMOJIS - BEGIN
% to use emoticons in your manuscript
% https://stackoverflow.com/questions/190145/how-to-insert-emoticons-in-latex/57076064
% using font Symbola, which has full support
% the font may be downloaded at:
% https://dn-works.com/ufas/
% add to preamble:
% \newfontfamily\Symbola{Symbola}
% in the text use:
% {\Symbola }
% EMOJIS - END

% LABEL REFERENCE TO DESCRIPTIVE LIST - BEGIN
% reference itens in a descriptive list using their labels instead of numbers
% insert the code below in the preambule:
%\makeatletter
%\let\orgdescriptionlabel\descriptionlabel
%\renewcommand*{\descriptionlabel}[1]{%
%  \let\orglabel\label
%  \let\label\@gobble
%  \phantomsection
%  \edef\@currentlabel{#1\unskip}%
%  \let\label\orglabel
%  \orgdescriptionlabel{#1}%
%}
%\makeatother
%
% in your document, use as illustraded here:
%\begin{description}
%  \item[first\label{itm1}] this is only an example;
%  % ...  add more items
%\end{description}
% LABEL REFERENCE TO DESCRIPTIVE LIST - END


% add line numbers for submission
%\usepackage{lineno}
%\linenumbers

\begin{document}
\maketitle

\begin{polyabstract}
\begin{abstract}
Apesar de o termo Letramento Digital estar em voga e ser recorrentemente citado nas mais diversas práticas sociais, muito ainda há para se discutir sobre o conceito, já que ele próprio se modifica na fluidez em que se configuram os ambientes digitais. Este artigo tem como objetivo, pois, dialogar sobre as definições acerta de tal termo, tanto as produzidas no campo acadêmico, por teóricas e teóricos, quanto no espaço socioescolar, por professoras e por professores em formação e em exercício, de diferentes realidades geopolíticas – Estados Unidos, Indonésia e Brasil –, a fim de evidenciar aproximações e afastamentos entre o teórico e o prático. O estudo fornece uma breve revisão bibliográfica sobre o tema. A análise é realizada de maneira qualitativo-interpretativista. Os dados foram extraídos de artigos teóricos e cruzados entre si para interpretação e discussão. Os resultados demonstraram que as definições sobre Letramento Digital produzidas por professoras e por professores são pouco complexas, principalmente no Brasil, sintetizadas, muitas vezes, pelas ferramentas digitais, e não pelo seu uso consciente e crítico, bem como pelo seu potencial socioemocional.

\keywords{Letramento digital \sep Práticas acadêmicas \sep Práticas socioescolares}
\end{abstract}

\begin{english}
\begin{abstract}
Even though Digital Literacy is currently a widely used term, which is frequently cited in many social practices, there is yet a lot to discuss when it comes to its concept, since it modifies itself as it flows through the many digital environments. In this article, the main objective is to establish dialogues between posited definitions both in academic and educational domains (by teachers and preservice teachers), within different geopolitical realities (in the United States, in Indonesia, and in Brazil). In doing so, we intend to demonstrate to what extent theory and practice might or might not converge. The study offers a brief literature review regarding the topic. The analysis is of qualitative and interpretative nature. The data was extracted from peer-reviewed articles and cross-checked for interpretation and discussion. The results have demonstrated that the definitions regarding Digital Literacy offered by teachers are not very complex, especially in Brazil. Oftentimes, the definition is simply linked to digital tools, whereas conscious and critical use, as well as potential socioemotional aspects are disregarded.

\keywords{Digital literacy \sep Academic practice \sep Socio-educational practice}
\end{abstract}
\end{english}
% if there is another abstract, insert it here using the same scheme
\end{polyabstract}

\section{Introdução}

Em um mundo mediado textualmente \cite{chouliaraki_discourse_1999}, sobretudo por textos multimodais em ambientes digitais, surge a necessidade de se pensar como sujeitos produzem e consomem os diversos textos, nas mais variadas práticas sociais, que circulam na rede. Muito se tem pesquisado nas últimas décadas: pesquisas sobre a hipertextualidade, com o pontapé inicial, no Brasil, de \textcite{marcuschi_o_2001,marcushi_hipertexto_2005}; sobre funcionamento e estrutura de textos digitais \cite{lima_forma_2017,xavier_reflexoes_2005}; e sobre ciberativismo \cite{alcantara_ciberativismo_2015,araujo_ciberativismo:_2011}; para citar apenas algumas temáticas possíveis da produção de saberes acadêmicos acerca do mundo digital.

Apesar dessa heterogeneidade temática, acreditamos que um ponto perpassa por ela, podendo, assim, ser considerado um eixo importante para reflexão. Como os sujeitos utilizam as ferramentas digitais? Há um espaço no qual eles aprendem isso? Se sim, o que se aprende? Se não, o que os impedem? Quais são os sentidos dados por sujeitos educacionais sobre o ensino-aprendizagem delas? Todos esses questionamentos são discutidos sob este eixo: o conceito de Letramento Digital.

É a partir do último questionamento que o presente artigo se desenha. Em nossa análise, o objetivo geral é discutir a relação entre o conceito de Letramento Digital produzido academicamente e as concepções de professoras e de professores sobre ele, a partir de dados já fornecidos por estudos anteriores. Assim sendo, temos como objetivos específicos: (i) conceituar Letramento Digital; (ii) trazer concepções sobre Letramento Digital de professoras e de professores; e (iii) contrastar os sentidos de (i) e de (ii), de modo a refletirmos sobre aproximações e afastamentos entre as teorias e as concepções fornecidas.

O artigo é, dessa forma, organizado: primeiramente, discutiremos o conceito de Letramento Digital, a partir de artigos previamente selecionados por discentes de uma disciplina em nível de pós-graduação intitulada “Letramento Digital”, para daí localizarmos como a academia tem pensado esse construto. Após essa discussão, dialogaremos sobre como professoras e professores, do Brasil e fora dele, concebem o conceito para, por fim, realizarmos uma discussão dessa relação entre teorias e concepções.

\section{O que é Letramento Digital?}

O conceito de Letramento Digital\footnote{A própria designação do que aqui se entende por “Letramento Digital” é múltipla e não consensual. Entre argumentações que os tornam intercambiáveis, aproximados ou distantes, é possível encontrar termos como “Competência Digital”, “Letramento Informacional” e “Alfabetização Informacional”, por exemplo. Cada um desses termos é mais ou menos adotado em diferentes países, contextos e literaturas. Como nos mostram \textcite{list_framework_2020}, pedir para um docente definir o que é “\textit{digital literacy}” nos Estados Unidos, por exemplo, e “\textit{digital competence}”, na Suécia, – termos respectivamente adotados em maior escala nesses países – já pode trazer resultados discrepantes, dada a ênfase da designação. Para fins deste artigo, portanto, atemo-nos a dados que analisam as definições de “Letramento Digital” (ou “\textit{Digital Literacy}”) fornecidas por professoras/es.} é um terreno teórico abrangente e aberto, o que implica em posições distintas sobre o que é essa prática social. \textcite{herring_internet_2004} já apontava essa falta de consenso e, portanto, não temos como pretensa apresentar um conceito fechado sobre o termo, mas, sobretudo, dialogar autores e autoras que se debruçaram sobre o tema.

\textcite{aillerie_teenagers_2019} aponta que a definição de Letramento Digital é postulada na década de 1990, por \textcite{glister_digital_1997}. Segunda a autora, \textcite{glister_digital_1997} a percebe como uma forma de habilidades de produção, de compreensão e de uso crítico das informações, quando estas são circuladas a partir de computadores. \textcite{herring_internet_2004}, entretanto, remonta à origem do termo para o séc. XIX, tendo se fortalecido principalmente a partir de teóricos do Construtivismo, como Piaget, que defendiam conceitos de manipuladores (digitais) dentro de um contexto educacional guiado e crítico. O que podemos inferir dessa origem é que, apesar de serem datadas diferentemente, ambas possuem a ideia de que o uso das tecnologias deve ser crítico, ou seja, demanda reflexividade dos sujeitos em práticas digitais.

Estudos mais recentes têm avançado sobre a conceituação do termo. Na introdução do livro \textit{Digital literacies}: \textit{concepts}, \textit{policies and pratices}, \textcite{lankshear_digital_2008}, por exemplo, apontam a importância de se admitir a pluralidade do conceito, deslocando-o para “letramentos digitais”, o que tem se tornado consenso entre múltiplas abordagens sobre o tema. Nessa obra, os autores apresentam duas perspectivas gerais sobre definições acerca dos letramentos digitais: de um lado, teóricos e teóricas que salientam o desenvolvimento de habilidades tecnológicas individuais; de outro, os/as que evidenciam os aspectos cognitivos e sociais do termo.

\textcite{gourlay_textual_2014} discorrem sobre a relação entre estudos de letramento e tecnologias de aprendizagem, colocando as duas perspectivas citadas por \textcite{lankshear_digital_2008} não em posição binária, mas em complementação. Segundo as pesquisadoras, as tecnologias são instrumentos poderosos, que proporcionam alternativas relacionadas aos ambientes de provisão educacional e ao modo em que ocorrem os processos de aprendizagem; os letramentos, por sua vez, constituem-se como espaços para a aprendizagem prática e simbólica nos ambientes digitais \cite[p.~2]{gourlay_textual_2014}. Dada uma aparente tênue fronteira entre esses conceitos, as autoras apontam, ainda, que

\begin{quote}
[u]ma resposta a essa confusão terminológica tem sido a tentativa dentro da tecnologia de aprendizagem de se estabelecer definições taxonômicas de letramentos digitais, dividindo o conceito em elementos constitutivos [, tais como:] 'habilidades com teclado, uso de tecnologias de captura, uso de ferramentas de análise, uso de ferramentas de apresentação, habilidades gerais de navegação/IU, adaptabilidade, agilidade, confiança/exploração' \cite[p.~8]{gourlay_textual_2014}\footnote{Tradução nossa do original: \textit{One response to this messiness of terminology has been the attempt within learning technology to establish taxonomic definitions of digital literacies, breaking the concept into constituent elements [, such as:] ‘keyboard skills, use of capture technologies, use of analysis tools, use of presentation tools, general navigation/UI skills, adaptivity, agility, confidence/exploration’}.}.  
\end{quote}

Entretanto, argumentam as autoras, nessa solução, o enfoque recai sobre as habilidades de aprendizagem envolvidas na familiarização com determinadas ferramentas, o que coloca os/as aprendizes apenas como sujeitos deficitários/as em direção ao que seria um ideal de “usuário”, em lugar de se pensar as práticas em que, de fato, podem se engajar. Embora as autoras critiquem a aparente simples definição feita por tecnólogos da aprendizagem acerca dos letramentos digitais, elas não a excluem. Em vez disso, compreendem-na como uma das partes dos letramentos digitais, estes vistos como prática social situada, que mobilizam habilidades em função do contexto digital em que estão inseridos.

Neste ponto, é importante elucidar que \textcite{gourlay_textual_2014} cunham o termo “\textit{New Literacy Studies}” (NLS) para alicerçar as concepções de letramentos digitais às quais recorrem. Os NLS referem-se a abordagens que há quase quarenta anos, contrariando a orientação positivista de aquisição universal de habilidades do sujeito, reafirmam a natureza situada, e, portanto, contingencial, das práticas digitais, além de reivindicar seu caráter interdisciplinar, colocando os letramentos digitais em constante diálogo com outros campos do saber, como a história, os estudos de mídia e a comunicação. Outro ponto levantado é a natureza semiótica da aprendizagem, evidenciando-se as relações de poder que textos em ambientes digitais produzem. Nesse aspecto, no entanto, as autoras apontam críticas tecidas quanto à supervalorização dos aspectos textuais (principalmente verbais), em detrimento dos sócio-históricos que os NLS produziram.

Ainda sobre os aspectos semióticos que subjazem essa prática social, \textcite[p.~5]{lankshear_digital_2008} afirmam que os letramentos digitais não são apenas um mero exercício de se codificar e decodificar textos – tanto nos processos de leitura como nos de produção –, mas que se deve entendê-los como práticas particulares de se ler e de se escrever, que demandam um olhar sociocultural para as linguagens, em geral, na medida em que permeiam e que produzem os ambientes digitais. Para isso, sujeitos se engajam na construção de sentidos nos textos que produzem, consomem, distribuem e trocam. Nesse enquadre, portanto, há a necessidade de se incluir, para além de preocupações puramente linguísticas, aspectos sociais, culturais e discursivos nos processos de aprendizagem. Dessa forma, \textcite{lankshear_digital_2008} admitem que a agência dos sujeitos e os significados produzidos por eles são fundamentais para tais letramentos.

Em direção semelhante, ao evidenciar tanto os aspectos das habilidades quanto os cognitivos e os sociais, \textcite[p.~17]{dudeney_letramentos_2016} definem letramentos digitais como “habilidades individuais e sociais necessárias para interpretar, administrar, compartilhar e criar sentido eficazmente no âmbito crescente dos canais de comunicação digital”. Os autores assumem, pois, a posição de que tais letramentos acontecem em práticas nas quais sujeitos mobilizam habilidades individuais – de cunho cognitivo – e sociais – de compartilhamento.

Avançando nessa discussão, \textcite{list_framework_2020} entendem que o conceito de Letramento Digital mobiliza não só as dimensões técnica e cognitiva da aprendizagem, mas também uma outra ainda pouco considerada: a socioemocional. Assim, nessa interseção tríplice, a dimensão técnica relaciona-se às habilidades operacionais de utilização de ferramentas digitais tanto em contexto educacional como no cotidiano; a cognitiva, às habilidades para analisar criticamente as informações que derivam de ambientes digitais; e a socioemocional, por sua vez, à capacidade de os sujeitos utilizarem as Tecnologias Digitais da Informação e Comunicação (TDICs) de maneira ética, responsável e colaborativa.

Em diálogo com a dimensão socioemocional, \textcite{tagata_rethinking_2021} propõem a necessidade de se pensar o Letramento Digital também como um Letramento Emocional, já que a utilização das TDICs, principalmente a partir dos novos contextos remotos derivados da pandemia da COVID-19, mobiliza técnicas, posicionamento crítico e emoções, marcadas fortemente pelos contextos nos quais os sujeitos se inserem. Em outras palavras, as habilidades e as competências socioemocionais não são apenas as capacidades éticas no uso das ferramentas, mas também as emoções que delas surgem, positivas e negativas. É por isso que os autores reiteram a necessidade de se decolonizar os estudos sobre Letramento Digital, admitindo estruturas e relações de poder, já que, por exemplo, emoções negativas são mais facilmente relatadas em contextos de “minorias”, como em relatos de pessoas negras e pobres.

A partir das proposições apresentadas até aqui, coloca-se, pois, a seguinte pergunta: como se dá o Letramento Digital? Para \textcite[p.~13]{rezende_o_2016}, tal promoção só é possível por meio de uma educação digital que deva proporcionar “oportunidade para utilizar os meios digitais com autonomia e participação individual e cooperativa”. Além disso, acreditamos que tal promoção só será possível, de fato, quando professoras/es\footnote{Em todos os estudos analisados nesta breve revisão bibliográfica, uma relevante maioria de sujeitos investigados se declarou como sendo do gênero feminino, razão pela qual priorizamos aqui o termo “professoras”.} e educadoras/es, em geral, tiverem passado por processos de desenvolvimento de seu próprio Letramento Digital e construído uma concepção multifacetada do termo. Por essa razão, analisamos nas seções subsequentes de que forma as concepções de professoras/es sobre Letramento Digital dialogam (ou não) com aquelas fornecidas em nosso aporte teórico.

\section{Metodologia}

Este artigo é uma análise qualitativa, ou seja, uma interpretação de parte do mundo capaz de desenhar uma formulação teórico-metodológica \cite{ramalho_alise_2011} sobre questões acerca do Letramento Digital. A pesquisa é uma revisão bibliográfica realizada em duas etapas: 1) seleção de textos teóricos, com extração de dados e conclusões; e 2) cruzamento das informações encontradas, interpretação e discussão.

A seleção de textos teóricos foi previamente feita de forma colaborativa, na disciplina “Letramento Digital”, ministrada no segundo semestre de 2021, pela professora Carla Viana Coscarelli, no âmbito do Programa de Pós-Graduação em Estudos Linguísticos, da Universidade Federal de Minas Gerais. Numa revisão bibliográfica colaborativa, cada discente da disciplina ficou responsável pela análise e extração de informações de artigos em língua portuguesa e estrangeira. A sintetização das informações foi registrada numa planilha com o nome do artigo, o link da publicação, conceitos depreendidos de Letramento Digital, habilidades envolvidas, teorias mobilizadas, outros conceitos subjacentes, sujeitos, metodologia e resultados.

Dos dados da planilha, escolhemos aqueles relacionados aos conceitos de Letramento Digital disponíveis na literatura específica e a partir de concepções de professoras/es para a elaboração desta breve revisão bibliográfica. Dos 40 (quarenta) artigos disponibilizados, fizemos o primeiro filtro: selecionamos aqueles que conceituavam “Letramento Digital” de forma explícita e sem outra denominação, além de terem como sujeitos de pesquisa professoras/es já em exercício ou em formação. Nessa fase, reduzimos os artigos do corpus a 13 (treze). Desses, fizemos uma saturação temática: elencamos 4 (quatro) que em alguma fase da pesquisa solicitasse às/aos professoras/es que definissem, em suas palavras, “Letramento Digital”. Além disso, cuidamos para que os 4 (quatro) artigos selecionados utilizassem conceitos da literatura presentes nos outros 9 (nove), de modo a elaborar o aporte teórico da seção anterior. Dos artigos selecionados, 2 (dois) dizem respeito a pesquisas conduzidas no Brasil e 2 (dois) no exterior, sendo um nos Estados Unidos e outro na Indonésia. Salientamos a atualidade das pesquisas selecionadas: todas foram conduzidas e publicadas nos últimos cinco anos.

Selecionados os textos, procedemos à fase de organização, sintetização e interpretação das informações depreendidas a partir dos excertos disponibilizados pelos/as autores/as, bem como de suas próprias conclusões, para, finalmente, realizarmos uma analítica que os dialoga. De antemão, é preciso ressaltar que os estudos selecionados têm objetivos de pesquisa e percursos metodológicos bastante distintos, sendo aquilo que os aproxima: (i) os sujeitos investigados (professoras/es); e (ii) a análise, em algum ponto das pesquisas, sobre a concepção de Letramento Digital desses sujeitos.

\section{“Letramento Digital” por professoras e professores: estudos em três países}

Nesta seção, compartilhamos alguns dos resultados de quatro investigações realizadas em três países. Inicialmente, apresentamos as concepções sobre Letramento Digital de professoras e de professores em formação inicial em uma universidade dos Estados Unidos, a partir do estudo de \textcite{list_framework_2020}. Na sequência, demonstramos os resultados de um estudo similar conduzido por \textcite{nabhan_pre-service_2021} com professoras/es em formação inicial de uma universidade na Indonésia. Por fim, passamos às duas pesquisas que investigaram as concepções de professoras/es brasileiras/os em exercício numa escola municipal de Teresina, Piauí \cite{santos_o_2017} e em formação inicial numa universidade no interior da Bahia \cite{_santos_letramento_2018}.

\subsection{Estados Unidos \cite{list_framework_2020}}

O primeiro estudo investigando a concepção de professoras/es em formação sobre Letramento Digital é o de \textcite{list_framework_2020}. Os dados dos quais nos referimos na sequência foram coletados numa universidade dos Estados Unidos, com 188 estudantes matriculadas/os em cursos relacionados à preparação para a docência em algum segmento do sistema K-12, o que se aproximaria, no Brasil, a cursos de licenciatura de formação inicial para atuação na educação básica. A pesquisa se utilizou de uma metodologia mista, com etapas qualitativa e quantitativa.
A Etapa 1, de natureza qualitativa, é a que discutimos nesta seção. Na Etapa 2, quantitativa, as autoras aplicaram um questionário fechado, que pedia às/aos participantes que elencassem, entre 24 habilidades previamente selecionadas, aquelas que tinham mais diretamente a ver com “Letramento Digital”. Na Etapa 1, por sua vez, as/os participantes deveriam responder previamente, e de forma aberta, à pergunta que aqui nos interessa:

\begin{quote}
   Recentemente, tem havido um impulso para ensinar Letramento Digital a estudantes ou para desenvolver em estudantes habilidades de letramento do século XXI. Como você define Letramento Digital? Que habilidades você considera necessárias para o Letramento Digital? \cite[p.~4]{list_framework_2020}\footnote{Tradução nossa do original: \textit{Recently, there has been a push to teach students digital literacy or to develop students’ 21st century literacy skills. How do you define digital literacy? What skills do you consider to be necessary for digital literacy?}}. 
\end{quote}

Na \Cref{Tabela01}, nota-se que as respostas fornecidas pelas/os professoras/es foram alocadas pelas pesquisadoras em quatro categorias, seguidas de um dado que se apresenta como exemplo representativo daquele grupo de respostas:


\begin{table}[h!]
\centering
\begin{threeparttable}
\caption{“Letramento Digital” por professoras/es em formação de uma universidade nos Estados Unidos.}
\label{Tabela01}
\begin{tabular}{p{0.3\textwidth}cp{0.55\textwidth}}
\toprule
Categoria da definição & \% & Exemplo de definição \\
\midrule
%\arrayrulecolor[gray]{.7}
Enfoque tecnológico\protect\footnotemark & 39,36 & Eu considero que o letramento digital é ter um entendimento sobre tecnologia e como ela funciona. As habilidades que considero necessárias seriam digitar, usar um computador com programas do tipo Google e Microsoft Word e conseguir usar um Iphone corretamente. \\
%\midrule
Leitura digital\protect\footnotemark & 28,19 & Eu definiria letramento digital como as leituras e informações disponíveis on-line. Você pode não só acessar uma informação on-line, como também acessar múltiplas fontes de uma vez. Habilidades que eu acho necessárias são saber como usar e navegar em um computador, habilidades de Leitura e foco. \\
%\midrule
Orientada ao cumprimento de objetivos\protect\footnotemark & 17,55 & ...eu definiria [Letramento Digital] como descobrir ideias e conceitos que você precisa através de um uso mais digital e ler aquilo que você precisa de modo a conseguir isso.\\
%\midrule
Uso crítico\protect\footnotemark & 12,23 & Eu definiria letramento digital como a habilidade de ter uma compreensão e a habilidade de ser um conhecedor digital, seja ao saber qual recurso tecnológico usar e quando ou entender as implicações da era digital. Eu considero [ter] uma mente aberta e maturidade coisas necessárias para o letramento digital.\\
%\midrule
Outra & 2,66 & Não fornecido. \\
%\arrayrulecolor{black}
\bottomrule
\end{tabular}
\source{Adaptação e tradução nossa de \textcite[p.~7]{list_framework_2020}.}
\end{threeparttable}
\end{table}
\addtocounter{footnote}{-4} 
\stepcounter{footnote}\footnotetext{Tradução nossa do original: \textit{Technology Focused. I consider digital literacy to be having an understanding of technology and how it works. The skills I consider necessary would be typing, using a computer like programs such as Google and Microsoft Word, and being able to use an iPhone correctly}.}
\stepcounter{footnote}\footnotetext{Tradução nossa do original:\textit{ Digital Reading. I would define digital literacy as readings and information being available online. You can not only access one piece of information online, you can access multiple sources all at once. Skills I think are necessary are knowing how to use and navigate a computer, Reading skills, and focus}.}
\stepcounter{footnote}\footnotetext{Tradução nossa do original: \textit{Goal Directed. \ldots I would describe it as figuring out ideas and concepts that you need to through a more digital use and reading what you need in order to complete that}.}
\stepcounter{footnote}\footnotetext{Tradução nossa do original: \textit{Critical Use. I would define digital literacy as the ability to have an understanding and the ability to be digitally savvy, whether that be knowing what technological resource to use and when or understanding the implications of the digital age. I consider an open mind and maturity necessary to be necessary [sic] for digital literacy}.}

Já à primeira vista, é possível constatar que as definições que dão um “enfoque tecnológico” ao conceito são as de maior predominância. As que enfocam a criticidade, por sua vez, são pelo menos três vezes menos frequentes no universo dos dados. De antemão, é possível depreender que a maior parte das concepções fornecidas no estudo de \textcite{list_framework_2020} dialogam mais diretamente com concepções de letramento(s) digital(is), cujo enfoque recai no domínio de ferramentas digitais, e não na familiarização com determinadas práticas sociais em ambiente digital. Outras implicações no delineamento da \Cref{Tabela01} e nos resultados obtidos em outros estudos serão, contudo, apresentadas mais adiante.

\subsection{Indonésia \cite{nabhan_pre-service_2021}}

Na sequência, passamos a um estudo de caso conduzido na Indonésia \cite{nabhan_pre-service_2021}, que também envolveu uma metodologia de natureza mista – quantitativa e qualitativa – com 107 professoras/es em formação inicial, cursando disciplinas de escrita no Departamento de Ensino de Língua Inglesa de uma universidade naquele país. O que se constatou no estudo, a priori, é que as concepções de Letramento Digital dos sujeitos investigados inicialmente foram “[...] associadas com a proficiência, em sentido estrito, de se utilizar ferramentas digitais e dispositivos tecnológicos, deixando de lado uma perspectiva crítica”\footnote{Tradução nossa do original: \textit{“[…] associated with the narrow proficiency of utilizing online tools and technological devices and set aside a critical mindset”}.} \cite[p.~187]{nabhan_pre-service_2021}.

Em uma das etapas qualitativas da pesquisa, solicitou-se às/aos professoras/es que respondessem, em um questionário aberto, entre outras perguntas, à seguinte: “em sua opinião, como você define Letramento Digital?”\footnote{Tradução nossa do original: \textit{“from your own opinion, how do you define digital literacy?”}.} \cite[p.~190]{nabhan_pre-service_2021}. Após análise exaustiva, os dados foram compilados em três categorias, que acabaram se relacionando a habilidades necessárias a essa prática, quais sejam: a) utilização de softwares; b) encontrar informações; e c) comunicação. Para ilustrar o enquadre em cada categoria, o autor fornece exemplos de respostas:

\begin{table}[h!]
\centering
\begin{threeparttable}
\caption{“Letramento Digital” por professoras/es em formação de uma universidade na Indonésia.}
\label{Tabela02}
\begin{tabular}{lp{0.7\textwidth}}
\toprule
Categoria de definição & Exemplo de definição \\
\midrule
Utilização de softwares\protect\footnotemark & Letramento Digital na minha opinião é quando podemos usar muito bem todas as plataformas como Canvas e Google Classroom. \\
%\hline
Encontrar informações\protect\footnotemark & É útil hoje em dia, para nós, buscarmos algumas informações necessárias relacionadas ao acadêmico e a outras coisas de que precisamos. \\
%\hline
Comunicação\protect\footnotemark & Um indivíduo que pode utilizar mídia digital para acessá-la ou operá-la de modo a estabelecer um relacionamento com outros. \\
\bottomrule
\end{tabular}
\source{Síntese e tradução nossa a partir de \textcite[p.~190-191]{nabhan_pre-service_2021}.}
\end{threeparttable}
\end{table}
\addtocounter{footnote}{-3} 
\stepcounter{footnote}\footnotetext{Tradução nossa do original: \textit{Using software. Digital literacy in my opinion is when we can use all of platforms like Canvas and Google Classroom very well}.}
\stepcounter{footnote}\footnotetext{Tradução nossa do original: \textit{Finding information. It’s helpful for us nowadays to gather some necessary information related to academic and the other things for our need}.}
\stepcounter{footnote}\footnotetext{Tradução nossa do original: \textit{Communication. An individual who can utilize digital media for accessing or operating to make a relationship with others}.}

De acordo com \textcite[p.~191]{nabhan_pre-service_2021}, o que se depreende é que, nas três categorias encontradas,

\begin{quote}
[...] a concepção de Letramento Digital não foi compreensivamente entendida no sentido de que Letramento Digital integra todos os aspectos da criatividade, e-segurança e pensamento crítico [nos processos de] avaliação, análise da informação e outros aspectos substanciais de Letramento Digital, não somente como operar plataformas digitais\footnote{Tradução nossa do original: \textit{[…] the conception of digital literacy was not comprehensively understood in the sense that digital literacy integrates all the aspects of creativity, e-safety, and critical thinking in [the processes of] evaluating, analyzing the information, and other substantial aspects of digital literacy, not only how to operate the digital platforms}.}.
\end{quote}

A frequência de respostas em cada categoria nesta etapa não foi divulgada, uma vez que a pesquisa prosseguia com uma etapa quantitativa, elencando competências pré-definidas relacionadas ao Letramento Digital; essas, sim, com frequência disponível e analisadas em maior detalhe. A título de ilustração, nessa etapa foram elencadas sete dimensões de Letramento Digital, da maior à menor, seguidas da média ponderada: comunicação (3,95), habilidades de segurança on-line (3,87), encontrar informação (3,79), pensamento crítico (3,77), habilidades funcionais (3,75), colaboração e criatividade (3,43) e dimensão cultural (3,40).

\subsection{Brasil \cite{santos_o_2017,_santos_letramento_2018}}

Diferentemente dos dados coletados nos dois estudos anteriores, contextos de maior escala e com etapas quantitativas, os dois artigos brasileiros selecionados tratam de pesquisas exclusivamente qualitativas, com grupos focais reduzidos. Em ambos, as concepções de professoras/es sobre Letramento Digital não são objeto central da investigação, senão uma etapa que subsidia outros instrumentos e perguntas de pesquisa. Apresentamos, na sequência, algumas contribuições dos estudos de \textcite{santos_o_2017} e de \textcite{_santos_letramento_2018}.

No estudo de \textcite{santos_o_2017}, investigou-se o grau de Letramento Digital de 3 (três) docentes de língua portuguesa, já em exercício, em três diferentes escolas de ensino fundamental de Teresina, no Piauí. Uma das perguntas do instrumento utilizado, um roteiro de entrevista estruturada, consistia em responder “o que é Letramento Digital?” \cite[p.~85]{santos_o_2017}. As respostas fornecidas pelos participantes encontram-se na \Cref{Tabela03}:

\begin{table}[h!]
\centering
\begin{threeparttable}
\caption{“Letramento Digital” por professoras/es em exercício em Teresina, Piauí.}
\label{Tabela03}
\begin{tabular}{lp{0.8\textwidth}}
\toprule
Participante & Definição \\
\midrule
P1 & É o processo pelo qual se desenvolve o Letramento Digital com o auxílio das mídias digitais. \\
%\hline
P2 & É o conhecimento que o indivíduo tem ou adquire sobre os recursos tecnológicos e manuseio.\\
%\hline
P3 & É a capacidade que o indivíduo tem de utilizar recursos tecnológicos como computador, tablet e outros para ampliar, facilitar e consolidar seus saberes.\\
\bottomrule
\end{tabular}
\source{\textcite[p.~85]{santos_o_2017}.}
\end{threeparttable}
\end{table} 

Numa breve análise, as autoras concluem que, embora as três respostas apontem para direções distintas, sendo P3 a que mais se aproxima de um direcionamento crítico do conceito, todas se encontram na interseção da utilização de “recursos tecnológicos”, o que, de certa forma, corrobora resultados obtidos nos dois estudos anteriormente apresentados.

A pesquisa de \textcite{_santos_letramento_2018}, por sua vez, realizou-se com 6 (seis) professoras/es em formação inicial, estudantes de licenciatura em língua inglesa, de uma universidade no interior da Bahia. O estudo teve como objetivo central analisar a contribuição da utilização das TDICs no processo de desenvolvimento e de aprendizagem de professoras/es de inglês em formação inicial nas habilidades de \textit{speaking} e \textit{writing}, sendo que o conceito de Letramento Digital foi uma das perguntas de um questionário aberto aplicado. Na \Cref{Tabela04}, na sequência, apresentamos as respostas fornecidas.

\begin{table}[h!]
\centering
\begin{threeparttable}
\caption{“Letramento digital” por professoras/es em formação inicial numa universidade do interior da Bahia.}
\label{Tabela04}
\begin{tabular}{lp{0.8\textwidth}}
\toprule
Participante\footnotemark & Definição\\
\midrule
George & É o desenvolvimento do indivíduo para as novas mídias desse tempo pós moderno; sendo elas as diversas redes sociais, TVs por assinatura, etc.\\
%\hline
Eugênio & Eu entendo como sendo um tipo de alfabetização digital. Aprender a lidar com as TIC para fazer uso das mesmas na sala de aula.\\
%\hline
Suzy & Acredito que seja a forma que o professor se atenta a essas tecnologias e busque utiliza-las em sala de aula.\\
%\hline
Franciele & Acredito que seja quando o professor busca estudar e aprender a utilizar as TIC. \\
%\hline
Maria & É o professor buscar aprender a usar as TIC para um melhor desenvolvimento da sua metodologia em sala de aula. \\
%\hline
Lazia & Seria estar apto a usar todo tipo de recurso digital que existem hoje, saber usar.\\
\bottomrule
\end{tabular}
\source{\textcite[p.~446]{_santos_letramento_2018}.}
\end{threeparttable}
\end{table}

\footnotetext{Os nomes utilizados nos dados são fictícios e foram atribuídos pelos próprios participantes da pesquisa.}

De acordo com \textcite[p.~446]{_santos_letramento_2018}, na definição de “Letramento Digital” dada pelos participantes, percebeu-se uma “[...] íntima relação estabelecida entre Letramento Digital e a utilização das TDIC unicamente à escola ou à [sic.] sala de aula”. De fato, como se depreende a partir das respostas disponíveis na \Cref{Tabela04}, as definições recorrem primordialmente ao uso de tecnologias (“novas mídias”, “recursos digitais”, “TIC”, “recurso digital” etc.) associadas a um campo semântico da prática escolar (“professor”, “sala de aula” e “metodologia”, por exemplo).

Apresentadas as contribuições dos quatro estudos, passamos agora à discussão que os relaciona.

\section{Discussão}

Como já apresentado, os estudos aqui revisados não podem ser fielmente comparáveis, uma vez que têm objetivos de pesquisa distintos, além de percorrerem metodologias consideravelmente diferentes. Tampouco seria possível dizer que quaisquer dos artigos analisados desvelam dados exaustivos para os países e para os contextos investigados. Reiteramos, portanto, aquilo que lhes é comum: pesquisas com fase de análise qualitativa em que os/as pesquisadores/as solicitam a um grupo de professoras/es – em larga ou microescala – que definam, em suas palavras, o que entendem por “Letramento Digital”.

No caso analisado com professoras/es dos Estados Unidos \cite{list_framework_2020}, a alocação de respostas em quatro categorias revelou uma predominância relevante daquelas que dão um “enfoque tecnológico” ao conceito. Ao longo do artigo, as autoras explicam que foi possível identificar um \textit{continuum} de complexidade, começando pelo nível do “enfoque tecnológico” até chegar a conceitos que enfatizam o “uso crítico”. Como se vê na \Cref{Tabela01}, quanto menor a frequência de respondentes numa dada categoria, portanto, mais refinada é a definição do conceito. As autoras argumentam que o ranking é justificável: os Estados Unidos são um país desenvolvido, com uma “[...] abundância de tecnologia” \cite[p.~2]{list_framework_2020} disponível a essas/es professoras/es ainda quando são estudantes de educação primária e secundária. No entanto, não há um comprometimento explícito com o “uso crítico” dessas tecnologias, mesmo nos currículos e nas políticas educacionais estadunidenses, o que tornaria a probabilidade de familiarização com esse viés do conceito bastante remota.

Em \textcite{nabhan_pre-service_2021}, vimos que a categorização das respostas com professoras/es na Indonésia segue uma lógica similar à do estudo anterior: “utilização de softwares”, “encontrar informações” e “comunicação”. No campo semântico em que se localizam as definições de Letramento Digital fornecidas por docentes, saltam aos olhos os nomes de plataformas e softwares, tais como o \textit{Google Classroom} e o \textit{Canvas}. Embora o estudo também tivesse o objetivo mais pormenorizado de entender a que dimensões do Letramento Digital as/os professoras/es investigadas/os se filiariam (Cf. 4.2), no que concerne às definições do conceito de Letramento Digital fornecidas por docentes, “[...] demonstrou-se que [...] fo[ram] meramente associada[s] à habilidade de usar tecnologia para a escrita \cite[p.~196]{nabhan_pre-service_2021}\footnote{Tradução nossa do original: \textit{“[...] It was [\ldots] shown that [they were] merely associated with the ability of using technology for writing”}.}”.

Nos dois artigos brasileiros \cite{santos_o_2017,_santos_letramento_2018}, analisando contextos em microescala, as conclusões também engrossam esse mesmo coro: as/os docentes investigados primordialmente referem-se à noção do uso de “mídias”, “TICs”, “redes sociais” e outros termos do campo semântico da tecnologia, sem se debruçarem, porém, sobre “como” esse uso se daria, legando o papel de desenvolvimento de “Letramento Digital” às comunidades escolares. Note-se que, embora a Base Nacional Comum Curricular \cite{brasil_ministerio_2018} trate das questões de “Letramento Digital” – sem, contudo, mencionar o termo \textit{ipsis litteris}, associando-o aos “multiletramentos” e às “novas tecnologias” – de maneira mais criticamente orientada, os dados dos estudos foram coletados antes da homologação desse marco legal. Ainda que já estivesse vigente, porém, é preciso ressaltar que essa é uma discussão relativamente nova, que adentra lentamente os espaços acadêmicos de formação de professoras/es, galgando suas primeiras incursões nos espaços escolares da educação básica.

Nesse sentido, mesmo em contextos de investigação bastante distintos, o que aqui se pode depreender é que, nos estudos analisados, os sujeitos investigados – professoras/es em formação inicial ou em exercício – ainda recorrem, primordialmente, a um campo semântico que associa “Letramento Digital” ao mero domínio de práticas sociais ligadas à “tecnologia” e que apenas uma minoria fornece definições que enfocam substancialmente o primeiro termo (letramento), em lugar do termo que o adjetiva (digital).

Ademais, na mobilização da tripla interseção sugerida a partir dos estudos de \textcite{list_framework_2020} – técnica, cognitiva e socioemocional – os dados dos estudos aqui revisados demonstram que a maior parte dos respondentes se filia à primeira – técnica –, com respostas mais complexas, sendo categorizadas pelos/as pesquisadores/as em dimensões associadas à criticidade – de natureza mais cognitiva. Definições que relacionem o Letramento Digital ao desenvolvimento de habilidades e competências socioemocionais não foram fornecidas no escopo dos artigos investigados. Note-se, por exemplo, que no estudo de \textcite{list_framework_2020}, apenas pouco mais de 2\% das respostas figurou na categoria “outro”, sem exemplo fornecido, o que inviabiliza a alocação de possíveis respostas no âmbito socioemocional a uma categoria bem definida. No estudo de \textcite{nabhan_pre-service_2021}, todas as respostas foram alocadas em três categorias macro e exaustivas, o que não necessariamente sugere que competências ligadas à dimensão socioemocional não tenham sido mencionadas. Nos artigos brasileiros de \textcite{santos_o_2017} e de \textcite{_santos_letramento_2018}, há, de fato, definições curtas e sem qualquer aproximação a um campo semântico de competências relacionadas às dimensões cognitiva ou socioemocional.

\section{Considerações Finais}

Como discutimos ao longo do artigo, as definições teóricas de Letramento Digital são amplas e em muito ultrapassam o domínio técnico de ferramentas. Em nosso aporte teórico, demonstramos de que forma diferentes autores/as associam o desenvolvimento do Letramento Digital não só a competências imediatas e mensuráveis do que se sabe fazer nos ambientes digitais (dimensão técnica), mas também do localizar, questionar e criar (dimensão cognitiva e crítica). Demonstramos, ainda, como num estudo mais recente, \textcite{list_framework_2020} acenam à necessidade de se incorporar uma dimensão socioemocional no delineamento de competências necessárias ao desenvolvimento do Letramento Digital.

A partir do cruzamento dos resultados obtidos em quatro pesquisas conduzidas em três países sobre as concepções de Letramento Digital de professoras/es, no entanto, é possível concluir que o corpus aqui analisado aponta para a predominância de concepções que \textcite{list_framework_2020} definiriam como pouco complexas. Isso porque, primordialmente, estabelecem uma conexão imediata entre o termo “digital” e plataformas, softwares e aplicativos mais relacionados ao \textit{lócus} – ao “onde” – e menos ao “quem”, ao “para quê”, e ao “para onde” nas práticas de letramento(s) digital(is).

No contraste que apresentamos entre estudos no Brasil e em outros dois países, o que fica evidente é que não é o mero acesso ou o uso de tecnologias digitais o que molda as concepções sobre Letramento Digital de professoras/es. Acreditamos que a familiarização com uma definição complexa, crítica e humana – daí a dimensão socioemocional – perpassa a necessidade de que esse termo seja revisitado nos marcos legais, projetos pedagógicos de cursos de licenciatura e nos currículos de educação básica.

Nesse sentido, fica a futuras pesquisas o desafio de mapear e de analisar, em outros contextos de micro e de macroescala, no Brasil e em outros países, as concepções de professoras/es e outros agentes das comunidades escolares acerca dos letramentos digitais. Métodos qualitativos como a Análise de Conteúdo, a Análise do Discurso ou Análise de Discurso Crítica podem ser empregados, em conjunto com outros métodos quantitativos, de modo a lançar luzes sobre as similitudes e as diferenças em relação às questões aqui levantadas. Além disso, em contextos de investigação acadêmica acerca de concepções de professoras e de professores sobre letramento digital, acreditamos que dados que não se enquadrem em categorias tradicionalmente emergentes – como as de “tecnologia” ou a de “escrita digital” – devem ser discutidos de maneira mais pormenorizada e rigorosa, uma vez que outras potenciais categorias – de natureza crítica e socioemocional, por exemplo – podem fornecer valiosas contribuições que retroalimentem a construção de documentos norteadores e de novas práticas de letramento digital.


\printbibliography\label{sec-bib}
% if the text is not in Portuguese, it might be necessary to use the code below instead to print the correct ABNT abbreviations [s.n.], [s.l.]
%\begin{portuguese}
%\printbibliography[title={Bibliography}]
%\end{portuguese}


%full list: conceptualization,datacuration,formalanalysis,funding,investigation,methodology,projadm,resources,software,supervision,validation,visualization,writing,review
\begin{contributors}[sec-contributors]
\authorcontribution{Newton Vieira Lima Neto}[conceptualization,datacuration,formalanalysis,investigation,methodology,visualization,writing,review]
\authorcontribution{Alexandra Bittencourt de Carvalho}[conceptualization,formalanalysis,investigation,methodology,writing,review]
\end{contributors}


\end{document}

