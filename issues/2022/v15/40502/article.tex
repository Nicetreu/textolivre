% !TEX TS-program = XeLaTeX
% use the following command:
% all document files must be coded in UTF-8
\documentclass[english]{textolivre}
% build HTML with: make4ht -e build.lua -c textolivre.cfg -x -u article "fn-in,svg,pic-align"

\journalname{Texto Livre}
\thevolume{15}
%\thenumber{1} % old template
\theyear{2022}
\receiveddate{\DTMdisplaydate{2022}{7}{17}{-1}} % YYYY MM DD
\accepteddate{\DTMdisplaydate{2022}{8}{23}{-1}}
\publisheddate{\DTMdisplaydate{2022}{9}{29}{-1}}
\corrauthor{Daniel Álvarez Ferrándiz}
\articledoi{10.35699/1983-3652.2022.40502}
%\articleid{NNNN} % if the article ID is not the last 5 numbers of its DOI, provide it using \articleid{} commmand 
% list of available sesscions in the journal: articles, dossier, reports, essays, reviews, interviews, editorial
\articlesessionname{dossier}
\runningauthor{Álvarez Ferrándiz et al.} 
%\editorname{Leonardo Araújo} % old template
\sectioneditorname{Daniervelin Pereira}
\layouteditorname{Leonado Araújo}

\title{Neurodidactic factors in the prediction of academic dropout in Andalusian university students: preventive actions based on ICT}
\othertitle{Fatores neurodidáticos na previsão de abandono acadêmico em estudantes universitários andaluzes: ações preventivas baseadas em TIC}
% if there is a third language title, add here:
%\othertitle{Artikelvorlage zur Einreichung beim Texto Livre Journal}

\author[1]{Daniel Álvarez Ferrándiz \orcid{0000-0003-4924-1334} \thanks{Email: \href{mailto:dalferrandiz@ugr.es}{dalferrandiz@ugr.es}}}
\author[1]{María Arias Corona \orcid{0000-0002-6290-5636} \thanks{Email: \href{mailto:mawi46@gmail.com}{mawi46@gmail.com}}}
\author[2]{Esther González Castellón \orcid{0000-0002-8129-5693} \thanks{Email: \href{mailto:esthergonzalez@ugr.es}{esthergonzalez@ugr.es}}}
\author[1]{Manuel Fernández Cruz \orcid{0000-0002-6873-7186} \thanks{Email: \href{mailto:manuelfernandezcruz@ugr.es}{manuelfernandezcruz@ugr.es}}}
\affil[1]{Universidad de Granada, Facultad de Ciencias de la Educación, Departamento de Didáctica y Organización Escolar, Granada, España.}
\affil[2]{Universidad de Granada, Facultad de Ciencias de la Educación y del Deporte Melilla, Departamento de Psicología Experimental, Melilla, España.}

\addbibresource{article.bib}
% use biber instead of bibtex
% $ biber article

% used to create dummy text for the template file
\definecolor{dark-gray}{gray}{0.35} % color used to display dummy texts
\usepackage{lipsum}
\SetLipsumParListSurrounders{\colorlet{oldcolor}{.}\color{dark-gray}}{\color{oldcolor}}

% used here only to provide the XeLaTeX and BibTeX logos
\usepackage{hologo}

% if you use multirows in a table, include the multirow package
\usepackage{multirow}

% provides sidewaysfigure environment
\usepackage{rotating}

% CUSTOM EPIGRAPH - BEGIN 
%%% https://tex.stackexchange.com/questions/193178/specific-epigraph-style
\usepackage{epigraph}
\renewcommand\textflush{flushright}
\makeatletter
\newlength\epitextskip
\pretocmd{\@epitext}{\em}{}{}
\apptocmd{\@epitext}{\em}{}{}
\patchcmd{\epigraph}{\@epitext{#1}\\}{\@epitext{#1}\\[\epitextskip]}{}{}
\makeatother
\setlength\epigraphrule{0pt}
\setlength\epitextskip{0.5ex}
\setlength\epigraphwidth{.7\textwidth}
% CUSTOM EPIGRAPH - END

% LANGUAGE - BEGIN
% ARABIC
% for languages that use special fonts, you must provide the typeface that will be used
% \setotherlanguage{arabic}
% \newfontfamily\arabicfont[Script=Arabic]{Amiri}
% \newfontfamily\arabicfontsf[Script=Arabic]{Amiri}
% \newfontfamily\arabicfonttt[Script=Arabic]{Amiri}
%
% in the article, to add arabic text use: \textlang{arabic}{ ... }
%
% RUSSIAN
% for russian text we also need to define fonts with support for Cyrillic script
% \usepackage{fontspec}
% \setotherlanguage{russian}
% \newfontfamily\cyrillicfont{Times New Roman}
% \newfontfamily\cyrillicfontsf{Times New Roman}[Script=Cyrillic]
% \newfontfamily\cyrillicfonttt{Times New Roman}[Script=Cyrillic]
%
% in the text use \begin{russian} ... \end{russian}
% LANGUAGE - END

% EMOJIS - BEGIN
% to use emoticons in your manuscript
% https://stackoverflow.com/questions/190145/how-to-insert-emoticons-in-latex/57076064
% using font Symbola, which has full support
% the font may be downloaded at:
% https://dn-works.com/ufas/
% add to preamble:
% \newfontfamily\Symbola{Symbola}
% in the text use:
% {\Symbola }
% EMOJIS - END

% LABEL REFERENCE TO DESCRIPTIVE LIST - BEGIN
% reference itens in a descriptive list using their labels instead of numbers
% insert the code below in the preambule:
%\makeatletter
%\let\orgdescriptionlabel\descriptionlabel
%\renewcommand*{\descriptionlabel}[1]{%
%  \let\orglabel\label
%  \let\label\@gobble
%  \phantomsection
%  \edef\@currentlabel{#1\unskip}%
%  \let\label\orglabel
%  \orgdescriptionlabel{#1}%
%}
%\makeatother
%
% in your document, use as illustraded here:
%\begin{description}
%  \item[first\label{itm1}] this is only an example;
%  % ...  add more items
%\end{description}
% LABEL REFERENCE TO DESCRIPTIVE LIST - END


% add line numbers for submission
%\usepackage{lineno}
%\linenumbers

\begin{document}
\maketitle

\begin{polyabstract}
\begin{abstract}
Persistence and dropout are two sides of the same coin. Together with personal and social factors, associated with the quality of teaching provided by universities determine students' ability to persist in achieving their academic degree or, on the contrary, to drop out of university studies. Our working hypothesis is that the impact on improving the quality of teaching by considering neurodidactic factors, which are currently being researched and experimented with, can improve the overall persistence rate by reducing the dropout rate. In this preliminary study, we intend to make a first approach to the phenomenon of academic failure in Andalusian universities from the prediction and diagnosis of risk groups and the recommendation of preventive measures. Among the measures we proposed for prevention, we highligh those that have an impact on neurodidactic factors. Our work consisted of applying an instrument to diagnose the risk of dropping out of university studies to a sample of first-year university students in Andalusian universities. The instrument was applied at the beginning of the second semester. Out of the 976 students surveyed, we have established a risk group of 34 students. We will propose measures oriented towards the neurodidactics factors that predict dropout and call for the use of ICT in the implementation of preventive measures.

\keywords{Neuroscience \sep Student dropout \sep ICT \sep Preventive actions \sep Neurodidactics}
\end{abstract}

\begin{portuguese}
\begin{abstract}
Persistência e desistência são dois lados da mesma moeda. Juntamente com fatores pessoais e sociais, associados à qualidade do ensino fornecido pelas universidades, determinam a capacidade dos estudantes de persistir na obtenção de seu diploma acadêmico ou, ao contrário, de abandonar os estudos universitários. Nossa hipótese de trabalho é que o impacto na melhoria da qualidade do ensino, considerando fatores neurodidáticos, que estão sendo pesquisados e experimentados atualmente, pode melhorar a taxa geral de persistência, reduzindo a taxa de abandono escolar. Neste estudo preliminar, fazemos uma primeira abordagem do fenômeno do fracasso acadêmico nas universidades andaluzas a partir da previsão e diagnóstico dos grupos de risco e da recomendação de medidas preventivas. Entre as medidas propostas para a prevenção, destacamos aquelas que têm impacto sobre os fatores neurodidáticos. Nosso trabalho consistiu em aplicar um instrumento para diagnosticar o risco de abandonar os estudos universitários a uma amostra de estudantes universitários do primeiro ano nas universidades andaluzas. O instrumento foi aplicado no início do segundo semestre. Dos 976 estudantes pesquisados, estabelecemos um grupo de risco de 34 estudantes. Propomos medidas orientadas para os fatores neurodidáticos que preveem a evasão escolar e que exigem o uso das TIC na implementação de medidas preventivas. 

\keywords{Neurociência \sep Abandono de alunos \sep TIC \sep Ações preventivas \sep Neurodidática}
\end{abstract}
\end{portuguese}
% if there is another abstract, insert it here using the same scheme
\end{polyabstract}

\section{Persistence vs. Dropout in Higher Education}\label{sec-intro}
Persistence and dropout are two sides of the same coin in the quality of university systems. Together with personal and social aspects, the factors associated with the quality of the teaching provided by universities determine students' ability to persist in achieving their academic degree or, on the contrary, to drop out of their university studies \cite{henriquez_construccion_2007}. Our working hypothesis is that the impact on improving the quality of teaching by taking neurodidactic factors into account can improve the overall persistence rate by decreasing the dropout rate. This is currently being researched and experimented with in practice.

Dropping out of studies is a problem of a global nature with pernicious personal, social and economic effects which we have investigated from different perspectives in recent years, including the perspective of the COVID pandemic effect in the student population \cite{fernandez_cruz_evaluation_2020,lizarte_simon_determinantes_2020,lizarte_simon_caracterizacion_2019}. The different university systems in developed countries, in their national reports, account for the seriousness of the issue. This is done in Spain, for example, through the annual reports of Data and Figures of the Spanish University System published by the government, which indicates that, in the 2021 academic year, the phenomenon affected 21.2\% of new undergraduate students \cite{ministerio_de_universidades_datos_2021}. Successive reports by the OECD and the EU through Eurostat deal with the situation of Spain in the group of developed countries with a comparative perspective and the global dimension of the problem. All the reports coincide in pointing out that, in general, the problem of dropout affects almost 20\% of the student population in Higher Education. In view of this, scholars are calling for the pooling of academic and scientific efforts to find solutions \cite{feixas_condom_hacia_2015}. Studies such as that by \textcite{cabrera_evolucion_2014} report the concern of Spanish universities about the increase in the number of students who drop out of their studies. These studies coincide with international studies, such as that by \textcite{tinto_reflections_2017}, which corroborates that approximately one third of first-year university students in advanced countries consider dropping out.

In Spain, three different rates are used to measure university dropout: (a) the system dropout rate; (b) the degree dropout rate; and (c) the change of university studies rate. The system dropout rate is defined as the percentage of students from a cohort of new enrolments in year X who do not enrol in any university degree at any university in the Spanish University System -SUE- for two consecutive years and have not graduated. The degree dropout rate is defined as the percentage ratio between the total number of students in a new enrolment cohort who should have obtained the degree in the previous academic year and who have not enrolled either in that academic year or in the previous one, and the total number of students in that degree in the reference academic year. The degree change rate is established from the difference between the sum of the dropout rates of each degree and the SUE dropout rate. Our work focuses on the former, i.e., the system dropout rate.

We work on dropout as a definitive cause of absence or weakness in the factors that encourage persistence. Therefore, we understand that weakness in factors whose strength causes persistence leads to dropout. From this perspective, the prediction of dropout and delimitation of risk groups focuses on the diagnosis of the weakness of persistence factors. At the same time, the preventive or corrective measures to be applied in the at-risk population will focus on reversing the orientation of those same factors which, we have already proven, generate persistence.

\subsection{Factors involved in persistence or dropping out of education}\label{sec-normas}
\textcite{triado_ivern_satisfaccion_2015,lizarte_alisis_2017} have analysed psychoeducational, biographical, socio-economic, pedagogical, and vocational causes in the persistence or dropout of university studies. Other authors \cite{lizarte_simon_determinantes_2020} analyse these same causes, separating and extending them, pointing out factors of a social, family, economic, psychological, academic, organisational, pedagogical, didactic, and even neurodidactic nature. All studies agree on the multi-causal origin of these phenomena. No single cause appears in isolation as the origin of dropout, but rather several converge in the individual who is forced to abandon his or her studies.

We have already studied how the attitude of persistence in studies, a factor that counteracts the possibility of dropping out, is related to academic satisfaction \cite{urbina_cardenas_abandono_2016} or the mastery of basic skills levels \cite{fernandez_cruz_formacion_2012} and psychological resilience \cite{lightsey_positive_2011}, which explains that persistent university students possibly show greater empowerment to successfully cope with difficulties that occur during university studies or student optimism that comes into play to cope with negative situations that can counteract academic failure \cite{sohail_stress_2013,avila_quinones_estilos_2014}.

Continuous study monitoring strategies improve the performance of university students in most cases \cite{carpenter_student_2015,lizarte_simon_determinantes_2020}. We have evidence that Spanish university students who continue and complete their studies at university, compared to those who drop out, differ more in the performance of a continuous, prolonged, and up-to-date study activity than in the use of specific techniques.  

\textcite{cabrera_evolucion_2014} explain that more than 60\% of Spanish university students undertake studies that initially do not match the ones they wanted to study when they entered university. Studies such as those by \textcite{gairin_student_2014,torrado_contexto_2013} confirm the relationship between the academic experience prior to university and the academic integration of first-year Spanish university students with the dropout or persistence of their studies. It seems clear that better orientation of students in pre-university stages is necessary to adapt the enrolment profiles to the requirements of the degree programme \cite{fernandez_cruz_nuevas_2011,fernandez_cruz_formacion_2015}. Effective vocational guidance programmes should be implemented in pre-university education so that students can carry out a vocational assessment according to their personal profile (level of education, vocational suitability, motivation, among others) and accompanying guidance programmes offered by faculties and schools during the study of the degree. Studies show that the implementation of guidance programmes at the university itself prolongs the retention rates of students at risk of dropping out by one year \cite{louise_mathematics_2015,blundell_descriptive_2015}.

\section{Neurodidactic factors involved in academic dropout}\label{sec-conduta}
However, in addition to all the above, the implication of neurodidactic factors in success or, on the contrary, in academic failure and dropout in Higher Education, is emerging more strongly \cite{jorda_neurodidactica_2018}. To focus on these, dropout or the attitude of persistence depend on successful or unsuccessful strategies for coping with study, the effectiveness of the academic guidance received for the choice of studies, the mastery or lack of command of basic study skills, self-concept and optimism versus a pessimistic attitude, the degree of motivation or social adaptation in the university institution, or others related to emotional regulation and the styles of coping with the difficulties that the individual adopts \cite{muchiut_neurodidactica_2018}.

Neurodidactics is an educational discipline that studies the application of advances in neuroscience to teaching \cite{hernandez_fernandez_relation_2020}. The optimisation of learning processes in higher education requires effective management of information and academic knowledge, the regulation of emotions and the acquisition of the competences that configure the professional profile of a degree. University didactics must offer new teaching models that integrate these three learning requirements (cognitive, emotional, and executive) and adapt to the new social realities in which successive generations of students develop and to the changing social needs for which they are being trained.

Advances in neuroscience provide university didactics with a theoretical basis from which to experiment with teaching models that favour metacognitive processes (motivation, attention, memory) in learning, the development of emotional intelligence and the facilitation of brain reorganisation applied to the resolution of academic and professional problems specific to higher education \cite{marina_neurociencia_2012}. These teaching models should be based on activities that promote social participation, creativity, and the transfer of knowledge between subjects and disciplines and between classrooms, laboratories or workshops and companies and institutions in the world of practice.

With its modest advances, neurodidactics is providing the basis for innovative teaching projects \cite{hnoievska_formation_2021} whose results can be transferred between classrooms, training and university institutions, to achieve effective learning models that reduce academic failure and its most dreaded effect: dropping out of university studies.


\section{Method}\label{sec-fmt-manuscrito}
\subsection{Objectives}
The study we present is part of the research project on academic dropout in Andalusian universities, financed by FEDER funds.

In this preliminary study, we propose to make a first approach to the phenomenon of academic failure in Andalusian universities from a predictive and diagnostic perspective of risk groups and the recommendation of preventive measures for avoiding dropout. Among the measures proposed for prevention, we will highlight those that have an impact on neurodidactics factors.

\subsection{Process}\label{sec-formato}
Our work consisted of applying an instrument to diagnose the risk of dropping out of university studies to a sample of first-year university students at universities in Andalusia. The instrument was applied at the beginning of the second semester so that students had six months of experience. In order to obtain respondents, the questionnaire was applied to whole groups of students whose teacher showed interest in collaborating with our research. The distribution was done by convenience with a non-probabilistic sampling. 


\subsection{Participants}\label{sec-modelo}
We have applied the diagnostic instrument to first-year students from three universities in Andalusia. A total of 976 subjects participated in the study. Out of the 970 who have acknowledged their gender, 755 are female (77.35\%) and 215 are male (22.02\%). Out of the total, 642 studied at the University of Granada (65.77\%), 260 at the University of Jaén (26.63\%) and 73 at the Pablo de Olavide University in Seville (7.47\%), with one student not answering to which university he/she belonged.

\subsection{Instrument}\label{sec-organizacao}
For the diagnosis of individuals at risk of dropout, we have used the "Survey on successful student retention" by \textcite{velazquez_narvaez_factores_2017}, which the authors applied to a population of nursing students at the Matamoros Multidisciplinary Academic Unit of the Autonomous University of Tamaulipas. A small modification was made to the wording of the items in this survey to adapt it to the Spanish student reality, and two of the initial 73 items were removed as they were not considered applicable in our context. In its final application format, the survey consisted of 71 items and 6 questions for the socio-demographic identification of the student. The survey is in the form of a 5-degree Likert-type scale on which the student expresses his or her degree of agreement or disagreement with the opinion expressed.

The usefulness of a persistence survey in studies to diagnose at-risk groups is evident: to the extent that factors shown to be effective for persistence do not appear in a student, he or she belongs to the at-risk group and is a candidate for remedial measures that enhance the persistence factors and lessen the factors predictive of dropout.

The original survey establishes four factors and twelve categories that explain persistence in university studies: motivation, commitment, attitude and behaviour, and socio-economic conditions, as presented in \Cref{tab01}.

\begin{table}[h!]
\centering
\begin{threeparttable}
\caption{Factors and categories in the university studies persistence survey.}
\label{tab01}
\begin{tabular}{l l}
\toprule
Factor & Categories \\
\midrule
\arrayrulecolor[gray]{.7}
\multirow{2}{*}{Motivation} & Internal  \\
& External \\
\midrule
\multirow{5}{*}{Commitment} & Self-efficacy \\
& Performance \\
& Perception of difficulty \\
& Career quality \\
& Academic services \\
\midrule
\multirow{3}{*}{Attitude and behaviour} & Sense of belonging \\
& Relationship with academic authorities \\
& Relationship with peers \\
\midrule
\multirow{2}{*}{Socio-economic conditions} & Social and family interaction \\
& Economic conditions \\
\arrayrulecolor{black}
\bottomrule
\end{tabular}
\source{\textcite{velazquez_narvaez_factores_2017}.}
\end{threeparttable}
\end{table}

In light of the results and the use we make of the survey, which is to predict non-persistence, i.e., dropout, we have preferred to call the factors differently in order to better target the preventive measures to be applied to the risk group. This will be discussed in the results.

\section{Results}\label{sec-organizacao-latex}
\subsection{Elements of risk of dropout}
The overall mean score reached by the entire set of all items in the whole sample is 3.83, i.e., the surveyed student population seems to be oriented towards persistence in studies rather than dropping out. 

However, there are seven items that do not reach the mean score of 3.00, i.e., they indicate a certain level of dissatisfaction with the personal situation in the degree programme. These items are presented in the \Cref{tab02}.

\begin{table}[htbp]
\begin{threeparttable}
\caption{Lowest scoring items in the overall sample.}
\label{tab02}
\centering
\begin{tabular}{p{1cm} p{8cm} p{2cm} p{2cm}}
\toprule
Nº & Item & Factor & Mean \\
\midrule
37 & The degree coordinator takes action to ensure that there are no free hours between classes. & Organisational & 2,47 \\
%\midrule
2 & My instructors use assessment strategies that favour my creativity. & Neurodidactic & 2,77  \\
%\midrule
26 & I consider my degree programme not to be excessively difficult. & Neurodidactic & 2,80 \\
%\midrule
4 & My instructors are concerned about my work in class. & Neurodidactic & 2,82 \\
%\midrule
7 & In general, I feel motivated by my instructors. & Neurodidactic & 2,88 \\
%\midrule
6 & I feel that my effort is recognised by my instructors. & Neurodidactic & 2,92 \\
%\midrule
19 & I actively participate in class. & Neurodidactic & 2,97 \\
\bottomrule
\end{tabular}
\source{FEDER B-SEJ-516-UGR18.}
\end{threeparttable}
\end{table}

Out of the seven non-persistence-oriented items, one of them, item 37, refers to an organisational factor and the remaining six to what we have called neurodidactic factors.

\subsection{Risk group}\label{sec-autores}
To establish the risk group, we have established that students who do not reach an average score of 3.00 in all of their responses to the 71 items are considered to be included.

There are 34 students who do not reach this average score of 3.00, which represents 3.48\% of the average. Out of these 34 students, 26 are female and 8 are male; 20 belong to UGR, 8 to UJA and 5 to UPO. These characteristics of the risk group are shown in the \Cref{tab03}.

\begin{table}[h!]
\centering
\begin{threeparttable}
\caption{Risk group characteristics.}
\label{tab03}
\begin{tabular}{l l l}
\toprule
 & N & Percentage \\
\midrule
\arrayrulecolor[gray]{.7}
Risk group & 34 & 3,48\% \\
\midrule
\multirow{2}{*}{Gender} & 26 women & 76,47\%  \\
& 8 men & 23,52\% \\
\midrule
\multirow{3}{*}{University} & 20 UGR & 58,82\% \\
& 8 UJA & 23,52\% \\
& 5 UPO & 14,70\% \\
\midrule
Average score reached & \multicolumn{2}{c}{2,21} \\
\arrayrulecolor{black}
\bottomrule
\end{tabular}
\source{FEDER B-SEJ-516-UGR18.}
\end{threeparttable}
\end{table}

In \Cref{tab04} we present a comparison between the characteristics of the total sample and the characteristics of the risk group about which we make the following comments:

\begin{enumerate}
    \item as already mentioned, the overall mean of the risk group drops to 2.21 from the 3.48 points it reaches in the total sample.
    \item Females are over-represented by 3.12\% in the risk group. Males, in turn, are under-represented by 1.5\%.
    \item Students from the University of Granada are under-represented in the risk group by 6.95\%.
    \item Students from the University of Jaén are also under-represented in the risk group. In this case by 3.11\%.
    \item Students from the Pablo de Olavide University in Seville are over-represented in the risk group by almost twice as much as in the sample, from 7.47\% to 14.70\%, i.e., 7.23\% more.
\end{enumerate}

\begin{table}[htbp]
\centering
\begin{threeparttable}
\caption{Comparison between the characteristics of the risk group and the overall sample.}
\label{tab04}
\begin{tabular}{l l l l l}
\toprule
& \multicolumn{2}{c}{Overall sample} & \multicolumn{2}{c}{Risk group} \\
\midrule
Mean & \multicolumn{2}{c}{3,83} & \multicolumn{2}{c}{2,21}  \\
%\midrule
Participants & 976 & 100\% & 34 & 3,48\%  \\
%\midrule
Women & 755 & 73,35\% & & 76,47\% \\
%\midrule
Males & 215 & 22,02\% & & 23,52\% \\
%\midrule
UGR & 642 & 65,77\% & & 58,82\% \\
%\midrule
UJA & 260 & 26,63\% & & 23,52\% \\
%\midrule
UPO & 73 & 7,47\% & & 14,70\% \\
\bottomrule
\end{tabular}
\source{FEDER B-SEJ-516-UGR18.}
\end{threeparttable}
\end{table}

In \Cref{tab05} we present the items with the highest scores for at-risk students, i.e., the items that are persistence oriented as they score above 3.00, i.e., 11 out of the total of 71 items presented. In other words, in 60 items, at-risk students scored below the mean of 3.00.

\begin{table}[htbp]
\centering
\begin{threeparttable}
\caption{Top-scoring items by risk group.}
\label{tab05}
\begin{tabular}{p{1cm} p{7cm} p{2cm} p{2cm}}
\toprule
Nº & Item & Dimension & Mean \\
\midrule
9 & I want to graduate  & Professional & 4,11   \\
%\midrule
11 & Being a good professional is a personal goal & Professional & 3,88 \\
%\midrule
12 & I wish to practice my profession after finishing my studies & Professional & 3,82 \\
%\midrule
10 & Completing my studies on time is important to me & Academic & 3,55 \\
%\midrule
8 & I am interested in obtaining an outstanding grade in my subjects & Pedagogical & 3,44 \\
18 & I fulfil the tasks that I am given in the different subjects & Pedagogical & 3,41 \\
%\midrule
14 & I consider myself an intelligent and capable person  & Psychological & 3,23 \\
%\midrule
17 & I see myself as a person with the necessary skills to be successful in my professional life & Psychological & 3,20 \\
%\midrule
13 & I see myself as a successful professional & Psychological & 3,05 \\
%\midrule
20 & I give priority to fulfilling my obligations as a student & Pedagogical & 3,02 \\
%\midrule
27 & I have to spend time every day studying or doing academic work & Pedagogical & 3,00 \\
\bottomrule
\end{tabular}
\source{FEDER B-SEJ-516-UGR18.}
\end{threeparttable}
\end{table}

These persistence-oriented items refer to goal achievement, task accomplishment and self-concept.

In \Cref{tab06} we present the items with the lowest scores for students at risk and which should guide the corrective measures to be proposed. There are 28 items corresponding to the social (10), organisational (5), academic (5), pedagogical (1), economic (1) and neurodidactic (4) dimensions.

\begin{table}[htbp]
\centering
\small
\begin{threeparttable}
\caption{Lowest scored items by at-risk students: below 2.00 (1.02 to 1.97).}
\label{tab06}
\begin{tabular}{p{1cm} p{7cm} p{2cm} p{2cm}}
\toprule
Nº & Item & Dimension & Mean \\
\midrule
64 & The means of transport I use to get to the faculty is not a problem for me to attend my classes on time	 & Social & 1,17   \\
%\midrule
38 & I feel that there is a commitment on the part of the academic authorities to attend to my needs as a student    & Organisational & 1,17  \\
%\midrule
65 & I have never interrupted the continuity of my studies for one semester or more & Academic & 1,17 \\
%\midrule
68 & I currently have no subjects pending from previous semesters & Academic & 1,23 \\
%\midrule
41 & I consider the tutorial activity to have had a positive impact on my academic performance & Neurodidactic & 1,32 \\
%\midrule
66 & I have never considered interrupting my university studies either temporarily or permanently	& Academic & 1,38 \\
%\midrule
39 & The library timetable is in line with my academic schedule & Organizational & 1,40 \\
%\midrule
43 & I have no family problems that affect my concentration or performance & Social & 1,47 \\
%\midrule
70 & I have never failed one or more subjects for not complying with the compulsory attendance percentage & Pedagogical & 1,50 \\
%\midrule
48 & I feel morally supported by the members of my family  & Social & 1,50 \\
%\midrule
46 & At home, domestic activities are shared by all members of the family  & Social & 1,50 \\
%\midrule
36 & The coordinator of my programme facilitates the execution of my academic activities & Organisational & 1,50 \\
%\midrule
35 & The administrative procedures that I have requested from the corresponding area have been resolved satisfactorily & Organisational & 1,50 \\
%\midrule
37 & The degree co-ordinator takes action to ensure that there are no free hours between classes & Organisational & 1,50 \\
%\midrule
45 & The communication between my family members is positive and open & Social & 1,55 \\
%\midrule
69 & I am keeping up to date with my English levels
 & Academic & 1,58 \\
% \midrule
40 & I have a tutor in my faculty & Neurodidactic & 1,61 \\
44 & The values of study and effort are encouraged and practised in my home & Social & 1,64 \\
%\midrule
63 & I do not need to work to pay for my university education & Economic & 1,70 \\
%\midrule
47 & I identify my parents as authority figures & Social & 1,73 \\
%\midrule
62 & I have external financial support such as family income, funding, or scholarships for my studies & Economic
& 1,76 \\
%\midrule
51 & I feel proud of the degree I am studying & Social & 1,82 \\
%\midrule
49 & I feel fully integrated in my group of fellow students at the faculty. & Neurodidactic & 1,89 \\
%\midrule
56 & I feel accepted and valued by my classmates & Neurodidactic & 1,94 \\
%\midrule
60 & I consider myself a productive and socially accepted person & Social & 1,94 \\
%\midrule
61 & In my home I have adequate spaces, services, and equipment to carry out my university tasks. & Economic & 1,94 \\
%\midrule
67 & I have studied all my subjects in the face-to-face modality at the University & Academic & 1,94 \\
%\midrule
42 & My relationship with my family is friendly and respectful & Social & 1,97 \\
\bottomrule
\end{tabular}
\source{FEDER B-SEJ-516-UGR18}
\end{threeparttable}
\end{table}

The four neurodidactic aspects of the teaching received by students at risk refer to the non-appointment of a tutor to guide them and the non-existence of a tutorial action plan in their institution, the non-perception of the teacher's concern for personal progress and the lack of motivation as an element of teaching.

\section{Dropout prevention measures}\label{sec-idioma}
Over the last 15 years, various international funding agencies (World Bank, Inter-American Development Bank, OECD, UNESCO and others) together with the most prestigious universities in the world have co-financed initiatives of Technological Training Centres, Professional Institutes, Higher Education and the Universities themselves, to design, experiment and, where appropriate, implement some type of preventive, even remedial, training that manages to avoid students leaving the system by influencing the factors that improve persistence in their studies.

The co-financed projects have been aimed at developing detailed strategic diagnoses, setting up curriculum articulation commissions, defining matrices and creating competence assessment descriptors, designing instruments, carrying out pilots, and building systems for monitoring and assessing competence achievement, training teachers to carry out preventive actions and remedial training, and trying to implement and institutionalise the training designed in such a way that it is supported by curricular, administrative, financial, organisational and didactic modifications that make it possible to achieve the desirable objectives and ensure the maintenance, sustainability and replicability of these actions over time \cite{fernandez_cruz_formacion_2015}. As an example, the projects that stand out among them are those that have been called "Transition Programmes between Secondary and Higher Education" in Europe and "Remedial Training Programmes" in the United States and the rest of America. Whatever they are called, they aim at the levelling of competences through the design and delivery of specific courses aimed at students diagnosed with low levels of basic skills on arrival at higher education institutions. It is generally considered that almost 40\% of students entering higher education would need some form of remedial intervention.

Reviews on the effectiveness of dropout prevention have allowed us to find factors associated with the success of remedial or competency-based programme designs \cite{rienties_longitudinal_2008}:

\begin{enumerate}[label=\alph*.]
    \item Introduction of guidance and counselling actions for students.
    \item Clarification of the objectives pursued in each course, workshop, or training action.
    \item Implementation of good adult learning models.
    \item Strengthening of good didactic structuring of the programme.
    \item Increasing and strengthening academic coordination mechanisms.
    \item Combination of sanctioning assessment strategies with elements of supplementary instruction for those students who fail the assessment.
    \item Consideration of social and emotional aspects of student development and learning.
    \item Inclusion of vocational components.
    \item ICT support for online remedial training.
\end{enumerate}

Now that these types of actions have been institutionalised and the situation of students who lack sufficient academic resources to successfully pursue university education has been overcome, we are left with the intervention in neurodidactic factors that would lessen the causes of dropout and strengthen the conditions for successful persistence in studies.

As reflected in the results of our study, the neurodidactic factors that should be integrated into preventive actions for the entire population of students in Andalusia are the following ten (\Cref{tab02}):

\begin{enumerate}[label=\alph*.]
    \item My instructors use assessment strategies that favour my creativity (\Cref{tab02});
    \item I do not consider my degree programme to be excessively difficult (\Cref{tab02});
    \item My instructors are concerned about my work in class (\Cref{tab02});
    \item In general, I feel motivated by my instructors (\Cref{tab02});
    \item I feel that my effort is recognised by my instructors (\Cref{tab02});
    \item I participate actively in class (\Cref{tab02});
    \item I feel that my effort is recognised by my instructors (\Cref{tab02});
    \item I actively participate in class (\Cref{tab02});
    \item I consider that the tutorial activity has had a positive impact on my academic performance (\Cref{tab06});
    \item I have a tutor in my faculty (\Cref{tab06});
    \item I feel fully integrated in my group of fellow students in the faculty (\Cref{tab06});
    \item I feel accepted and valued by my peers (\Cref{tab06}).
\end{enumerate}

In other words, preventive measures should focus on the transformation of the teaching methodology that favours creative (a) and participative (f) activities that favour social integration in the classroom (i) and total acceptance by classmates (j). In addition, teachers should seek to work on motivation for learning as an integral part of teaching (d), express their concern for the personal work of each student (c) and the effort dedicated to training (e), as well as strengthening the mechanisms for tutorial action (g) and personal monitoring of specifically assigned students (h). In other words: improving teaching methodology and tutorial action from the perspective of the contributions of neurodidactics. Together with the above, helping the student to understand the real degree of difficulty of the studies (b) and the real possibilities of overcoming the difficulty with the help of the teacher and the provision of the necessary training resources, is another neurodidactic function that university teaching must integrate.

The experience gained in recent years in supporting teaching with technological devices has shown us that there are many and varied ways of achieving a methodological renovation of higher education that includes tutorial action and student support and monitoring as an integral part of it. In any case, the explosion in the availability of technological devices as a teaching resource, the technification of today's society and the ever-increasing support of online and networked relational and communicative processes generate new learning scenarios from which the higher education teacher cannot escape. In terms of \textcite{barroso-osuna_visiones_2020}, it is a matter of achieving ubiquitous, decontextualized, instantaneous learning, close to the learner, contextualized and globally connected, providing temporal and spatial responses adapted to the characteristics and needs of the university student. The technological devices available, as well as those that are appearing successively and rapidly over time, not only provide new, more powerful, and more adjusted ways of accessing knowledge, but also new means of representing knowledge and interacting with one's own thinking and that of others, which favour different cognitive, emotional, and executive functions. What is certain is that teachers cannot remain immovable in the face of this unstoppable wave of continuous change but must equip themselves with broad methodological patterns that allow them to incorporate all the richness that ICTs are adding to the didactic processes \cite{chavez_neurodidactica_2020}. In concrete terms, teachers must be able to generate new learning environments that allow for innovation, personalization, student monitoring and continuous support for ongoing training.

The prevention of academic dropout entails the use of tools for predicting and diagnosing risk groups; it requires the transformation of teaching from the perspective of advances in neurodidactics; it requires the integration of all the new technological devices linked to teaching innovation for all students; and the planning of tutorial action plans and specific corrective activities for students at risk which, together with the use of general improvements in teaching, allow for an in-depth knowledge of the rest of the personal, family, social and economic causes involved in the difficulty in studying and the provision of teaching advice or referral to specialised services to overcome the disadvantages detected.


\section{Acknowledgment and clarification}\label{sec-conclusao}
This article comes from the research with reference B-SEJ-516-UGR18 approved in the call for projects I+D+i FEDER Andalucía 2014-20.

This research has been approved by the Research Ethics Committee of the University of Granada with registration number 2778/CEIH/2022.


\printbibliography\label{sec-bib}
% if the text is not in Portuguese, it might be necessary to use the code below instead to print the correct ABNT abbreviations [s.n.], [s.l.]
%\begin{portuguese}
%\printbibliography[title={Bibliography}]
%\end{portuguese}


%full list: conceptualization,datacuration,formalanalysis,funding,investigation,methodology,projadm,resources,software,supervision,validation,visualization,writing,review
\begin{contributors}[sec-contributors]
\authorcontribution{Daniel Álvarez Ferrándiz}[formalanalysis,methodology,writing]
\authorcontribution{María Arias Corona}[formalanalysis,methodology,writing]
\authorcontribution{Esther González Castellón}[conceptualization,investigation,writing]
\authorcontribution{Manuel Fernández Cruz}[supervision,review]
\end{contributors}


\end{document}

